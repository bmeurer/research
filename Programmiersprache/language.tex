\documentclass[12pt,a4paper]{report}

\usepackage{amsmath}
\usepackage{amssymb}
\usepackage{amstext}
\usepackage{array}
\usepackage[american]{babel}
\usepackage{color}
\usepackage{enumerate}
%\usepackage[T1]{fontenc}
%\usepackage{german}
\usepackage[a4paper,%
            colorlinks=false,%
            final,%
            pdfkeywords={},%
            pdftitle={},%
            pdfauthor={Benedikt Meurer},%
            pdfsubject={},%
            pdfdisplaydoctitle=true]{hyperref}
\usepackage[latin1]{inputenc}
\usepackage{latexsym}
\usepackage[final]{listings}
\usepackage{makeidx}
%\usepackage{ngerman}
\usepackage[standard,thmmarks]{ntheorem}
%\usepackage{scrpage2}
\usepackage{stmaryrd}
\usepackage{url}

%% LaTeX macros
%%
%% macros.tex - Useful LaTeX macros
%%
%% Copyright (c) 2006-2011 Benedikt Meurer <benedikt.meurer@googlemail.com>
%% 


%%
%% Styles
%%

\newcommand{\nstyle}[1]{\ensuremath{\mathsf{#1}}}
\newcommand{\sstyle}[1]{\ensuremath{\mathit{#1}}}


%%
%% Misc
%%

\newcommand{\abort}{\ensuremath{\mathbf{abort}}}
\newcommand{\pto}{\rightharpoonup}
\newcommand{\step}{\ensuremath{\rightsquigarrow}}

\newcommand{\sem}[1]{\ensuremath{[\![#1]\!]}}


%%
%% Names
%%

\newcommand{\arity}{\ensuremath{\mathit{arity}}}
\newcommand{\cl}{\ensuremath{\mathit{cl}}}
\newcommand{\fr}{\ensuremath{\mathit{fr}}}
\newcommand{\free}{\ensuremath{\mathit{free}}}
\newcommand{\graph}{\ensuremath{\mathit{graph}}}
\newcommand{\id}{\ensuremath{\mathit{id}}}


%%
%% Sets
%%

\newcommand{\I}{\ensuremath{\mathcal I}}
\newcommand{\N}{\ensuremath{\mathbb N}}
\renewcommand{\O}{\ensuremath{\mathcal O}}
\newcommand{\Z}{\ensuremath{\mathbb Z}}
\newcommand{\Cl}{\sstyle{Cl}}
\newcommand{\Env}{\sstyle{Env}}
\newcommand{\Exp}{\sstyle{Exp}}
\newcommand{\Frame}{\sstyle{Frame}}
\newcommand{\Id}{\sstyle{Id}}
\newcommand{\Node}{\sstyle{Node}}
\newcommand{\Val}{\sstyle{Val}}


%%
%% Expressions
%%

\newcommand{\app}[2]{{#1}\,{#2}}
\newcommand{\abstr}[2]{\lambda{#1}.\,{#2}}
\newcommand{\ifte}[3]{\mathbf{if}\,{#1}\,\mathbf{then}\,{#2}\,\mathbf{else}\,{#3}}


%%
%% Values
%%

\newcommand{\clov}[2]{\langle{#1},{#2}\rangle}
\newcommand{\false}{\mathbf{false}}
\newcommand{\true}{\mathbf{true}}


%%
%% Grammars
%%

\newenvironment{grammar}{\begin{array}{rrlll}}{\end{array}}

\newcommand{\is}{& ::= &}
\newcommand{\al}{\\ & \mid &}
\newcommand{\nl}{\vspace{2mm}\\}


%%
%% Other environments
%%

\newenvironment{case}{\left\{\!\!\!\begin{array}{ll}}{\end{array}\right.}


%%% Local Variables: 
%%% mode: latex
%%% TeX-master: "compiler"
%%% End: 


% TODO
\newcommand{\CExp}{\nstyle{CExp}}
\newcommand{\CVal}{\nstyle{CVal}}
\newcommand{\loc}{\nstyle{loc}}
\newcommand{\sto}{\nstyle{sto}}
\newcommand{\val}{\nstyle{val}}
\newcommand{\DEF}{\nstyle{DEF}}
\newcommand{\OUT}{\nstyle{OUT}}
\newcommand{\I}{\mathcal{I}}


\begin{document}


%%
%% The programming language
%%

\chapter{The programming language}

As target language, we use call-by-value PCF with unit, records and imperative concepts.


%%
%% Abstract syntax
%%

\section{Abstract syntax}

We assume that an infinite set of variables $x \in \Var$ and an infinite
set of names $m\in \Member$ is given, not necessarily disjoint.
Variables are used to abstract expressions; programs are considered equal modulo
renaming of bound variables. Names $m$ are used to name fields in records. These
names are always free and not subject to $\alpha$-conversion.
$\Bool = \{\true,\false\}$ is the set of boolean constants $b$,
$\Int$ is a set of integer constants $z$; we do not distinguish
between integer values and their program representation.

\begin{definition}[Types]
  The set $\Type$ of all {\em types} $\tau$ is defined by the context-free grammar
  \[\GRbeg
  \tau \GRis \tbool \GRmid \tint \GRmid \tunit
          \GRal \tref{\tau}
          \GRal \tarrow{\tau_1}{\tau_2}
          \GRal \trecord{m_1:\tau_1,\ldots,m_n:\tau_n},
  \GRend\]
  with distinct member names $m_1,\ldots,m_n$ for record types.
\end{definition}

We assume an infinite set $\Loc^\tau$ of {\em $\tau$-locations} for each type $\tau \in \Type$,
where the sets $\Loc^\tau$ are required to be disjoint. Given those sets,
$\Loc = \bigcup_{\tau\in\Type} \Loc^\tau$ is the set of all {\em locations} $l$.

\begin{definition}[Expressions]
  The set $\Op$ of {\em operators} $\op$ and the set $\Const$ of {\em constants} $c$
  are defined by the grammars
  \[\GRbeg
    \op \GRis + \GRmid - \GRmid * \GRmid = \GRmid < \GRmid > \GRmid \le \GRmid \ge \\
    c \GRis z \GRmid b \GRmid \unit \GRmid \cref \GRmid !
             \GRmid \assign \GRmid \fix,
  \GRend\]
  the set $\Exp$ of {\em expressions} $e$ is defined by
  \[\GRbeg
    e \GRis c \GRmid x \GRmid l \GRmid \abstr{x}{e} \GRmid \app{e_1}{e_2}
           \GRmid \ifte{e_0}{e_1}{e_2}
           \GRal \proj{e}{m} \GRmid \record{m_1=e_1,\ldots,m_n=e_n}
  \GRend\]
  and the set $\Val \subseteq \Exp$ of {\em values} $v$ is defined by
  \[\GRbeg
    v \GRis c \GRmid l \GRmid \abstr{x}{e} \GRmid \app{\op}{v} \GRmid \record{m_1=v_1,\ldots,m_n=v_n}
  \GRend\]
  with distinct member names $m_1,\ldots,m_n$ for records.
\end{definition}

$\free{e}$ denotes the set of free variables and $\locns{e}$ denotes the set of locations in the
expression $e$. We say that an expression $e$ is a {\em program} if both $\free{e}=\emptyset$
and $\locns{e}=\emptyset$ holds, that is if $e$ contains neither unbound variables nor
locations (i.e. memory addresses in terms of the underlying machine). We demand that all expressions
considered for the logic later are valid programs (i.e. programmers aren't permitted to access
abritrary memory locations).


%%
%% Static semantics
%%

\section{Static semantics}

The static semantics of the programming language are defined by the typing relation
$\Tj{\Gamma}{e}{\tau}$.

\begin{definition}[Typing relation]
  A {\em typing judgement} is a formula $\Tj{\Gamma}{e}{\tau}$ where
  \begin{itemize}
    \item $\Gamma:\Var \pto \Type$, $\dom{\Gamma}$ is finite
    \item $e\in\Exp$ and
    \item $\tau\in\Type$.
  \end{itemize}
\end{definition}

We write $\tj{e}{\tau}$ for $\Tj{[\,]}{e}{\tau}$. The set $\CExp^\tau = \{e\in\Exp\,|\,\tj{e}{\tau}\}$
includes the closed expressions of type $\tau$ and $\CExp = \bigcup_{\tau\in\Type} \CExp^\tau$ includes
all well-typed, closed expressions. Likewise $\CVal^\tau = \CExp^\tau \cap \Val$ includes
the closed values of type $\tau$ and $\CVal = \bigcup_{\tau\in\Type} \CVal^\tau$ includes all well-typed,
closed values.


%%
%% Operational semantics
%%

\section{Operational semantics}

\begin{definition}[Store] \label{definition:Store} \
  \begin{enumerate}
    \item A {\em store} is a (total) function $s:\Loc \to \CVal_\bot$ with
          \begin{itemize}
            \item $s(\Loc^\tau) \subseteq (\CVal^\tau)_\bot$ for all $\tau \in \Type$, and
            \item $s(\locns{s(\Loc)}) \subseteq \CVal$.
          \end{itemize}

    \item For each $L\in\powersetfin{\Loc}$ the set of {\em $L$-stores} is defined as
          \[\begin{array}{l}
            \Store_L = \{s\,|\,\text{$s$ is a store with $L\subseteq\dom{s}$}\}.
          \end{array}\]

    \item $\Store = \bigcup_{L\in\powersetfin{\Loc}} \Store_L$ denotes the set of all {\em stores}.
  \end{enumerate}
\end{definition}

We thereby require stores to be well-typed and closed by definition, i.e. we don't permit {\em dangling
references} within stores.

\begin{definition}[Configuration]
  Let $e\in\CExp$ and $s\in\Store$. The pair $(e,s)$ is a {\em configuration}
  if $\locns{e} \subseteq \dom{s}$.
\end{definition}

The stores $s$ are already guarantied to be closed due to definition~\ref{definition:Store} and
now expressions that appear as part of an evaluation are also required to be closed, i.e. do not
include {\em dangling references} to unallocated store cells.

\begin{definition}[Big step semantics]
  Let $e\in\Exp$, $v\in\Val$ and $s,s'\in\Store$. A {\em big step} is a formula
  $(e,s) \Downarrow (v,s')$. A big step is said to be valid if it was derived with the
  big step rules given {\bf somewhere else}.
\end{definition}

\begin{corollary}
  Let $e\in\Exp$, $v\in\Val$ and $s,s'\in\Store$. If $(e,s)$ is a configuration
  and $(e,s) \Downarrow (v,s')$, then
  \begin{enumerate}
    \item $(v,s')$ is also a configuration, and
    \item $\dom{s} \subseteq \dom{s'}$.
  \end{enumerate}
\end{corollary}

\begin{proof}
  Straight forward induction.
\end{proof}

The big step semantic is thereby well-defined with regard to configurations, i.e. it preserves
the configuration property. In the remainder
we thereby consider only pairs $(e,s)$ that are valid configurations. In addition we
assume that type safety holds for the semantics (see Pierce, {\bf TODO}).


%%
%% The assertion language
%%

\chapter{The assertion language}


%%
%% Syntax of the assertion language
%%

\section{Syntax of the assertion language}

Based on the types $\tau$ of the programming language, we define
\begin{itemize}
  \item the set $\DType$ of {\em data types} $\theta$ and
  \item the set $\LType$ of {\em logic types} $\pi$
\end{itemize}
using the context-free grammar given below:
\[\GRbeg
  \theta \GRis \tau \GRmid \ttarrow{\theta_1}{\theta_2} \\
  \pi \GRis \tassn \GRmid \theta \GRmid \ttarrow{\theta}{\pi}
\GRend\]
Given an appropriate set $F$ of predefined {\em function symbols} $f$, we
define the set $\Assn$ of {\em assertions} $p,q$ using the following grammar:
\[\GRbeg
  p,q \GRis x
      \GRal v
      \GRal f
      \GRal p \mapsto q
      \GRal p \leadsto q
      \GRal \app{p}{q}
      \GRal \abstr{x:\theta}{p}
      \GRal \neg p
      \GRal p \wedge q
      \GRal \exists x:\pi \in [p]. q
      \GRal p = q
      \GRal \triple{p}{e}{q}
\GRend\]


%%
%% Axiomatic semantics
%%

\section{Semantics of the logic}


%%
%% Reachability
%%

%\subsection{Reachability}
%
%
%\begin{definition}[Reachability]
%  Let $L\in\Locfin$ and $s\in\Store_L$. By induction we define sets $\reachn{i}{L}{s} \in \Loc$
%  \begin{itemize}
%    \item $\reachn{0}{L}{s} = L$
%    \item $\reachn{i+1}{L}{s} = \reachn{i}{L}{s} \cup \bigcup_{l\in\reachn{i}{L}{s}} \locns{s(l)}$.
%  \end{itemize}
%  The set $\reach{L}{s} = \bigcup_{i\in\N} \reachn{i}{L}{s}$ includes all locations in $s$ reachable
%  by $L$.
%\end{definition}
%
%Since a store $s\in\Store_L$ is always closed by definition, it is obvious that
%$\reach{L}{s} \subseteq \dom{s}$ holds.
%
%\begin{lemma}
%  For all $L,L'\in\Locfin$ with $L \subseteq L'$ and $s,s'\in\Store_{L'}$ we have:
%  \begin{enumerate}
%    \item $\Store_{L'} \subseteq \Store_L$
%    \item $\reach{L}{s} \subseteq \reach{L'}{s}$
%    \item If $\reach{L'}{s} = \reach{L'}{s'}$ and $s(l) = s'(l)$ for every $l\in\reach{L'}{s}$,
%          then $\reach{L}{s} = \reach{L}{s'}$.
%  \end{enumerate}
%\end{lemma}
%
%\begin{proof}
%  Trivial.
%\end{proof}
%
%\begin{definition}[$L$-similarity, $L$-equivalence, $L$-definability]
%  Given a set $L\in\Locfin$ we define:
%  \begin{enumerate}
%    \item Two stores $s,s'\in\Store$ with $L \subseteq \dom{s} \cup \dom{s'}$ are
%          {\em $L$-similar}, denoted $s \simeq_L s'$, if $s(l)=s'(l)$ for every
%          $l \in L$.
%
%    \item Two stores $s,s'\in\Store_L$ are {\em $L$-equivalent}, denoted $s \equiv_L s'$,
%          if
%          \begin{itemize}
%            \item $\reach{L}{s} = \reach{L}{s'}$ and
%            \item $s \simeq_{\reach{L}{s}} s'$.
%          \end{itemize}
%
%    \item A predicate $\phi:\Store_{L'} \to \Bool$ with $L' \subseteq L$ is said to be {\em $L$-definable} if
%          $\app{\phi}{s} = \app{\phi}{s'}$ holds for all
%          stores $s,s'\in\Store_L$ with $s \equiv_L s'$.
%  \end{enumerate}
%\end{definition}
%
%\begin{lemma}
%  For all sets $L,L'\in\Locfin$ with $L \subseteq L'$ we have:
%  \begin{enumerate}
%    \item $s \simeq_{L'} s'$ implies $s \simeq_L s'$ for all $s,s'\in\Store$.
%    \item $s \equiv_{L'} s'$ implies $s \equiv_L s'$ for all $s,s'\in\Store_{L'}$.
%  \end{enumerate}
%\end{lemma}
%
%\begin{proof}
%  Trivial.
%\end{proof}


%% TODO - statt \pi besser \theta fuer die logischen Typen verwenden
%%        und \pi fuer die Permutationen

%%
%% Semantic domains
%%

\subsection{Semantic domains}

\begin{definition}
  Let $(W,\sqsubseteq)$ be a partial order (of `worlds' $w$).
  A {\em $W$-set} is a set $D$ together with a family of subsets $(D_w)_{w\in W}$ such that
  $D = \bigcup_{w\in W} D_w$ and $w \ne v \Rightarrow D_w \cap D_v = \emptyset$
  for all $v,w \in W$.
\end{definition}

We define the {\em support} of an element $d\in D_w$ to be the world
\[\begin{array}{rcl}
  \supp{d} &=& w
\end{array}\]
The support is guarantied to be unique for every $d \in D$, since the subsets
$(D_w)_{w\in W}$ of $D$ are disjoint.

\begin{definition}
  A {\em $W$-sorted (relation) signature} is a family $\Sigma=(\Sigma^w_n)_{w\in W,n\in \N}$
  of sets $\Sigma^w_n$ such that for all $m,n\in \N$ and $v,w\in W$
  \begin{itemize}
    \item $m \ne n \Rightarrow \Sigma^v_m \cap \Sigma^w_n = \emptyset$
    \item $v \sqsubseteq w \Rightarrow \Sigma^v_n \supseteq \Sigma^w_n$
  \end{itemize}
\end{definition}

An element $r \in \Sigma_n$ is called a {\em relation symbol of arity $n$}. We use the abbreviations
\[\begin{array}{rcl}
  \Sigma_n =_{def} \bigcup_{w\in W} \Sigma^w_n, &
  \Sigma^w =_{def} \bigcup_{n\in \N} \Sigma^w_n, &
  \Sigma =_{def} \bigcup_{w\in W} \Sigma^w.
\end{array}\]

\begin{definition}
  Let $\Sigma$ be a $W$-sorted signature. A {\em $W$-$\Sigma$-set} is a pair
  $(D,\I)$, where $D$ is a $W$-set and $\I$ is a function which maps every
  $r \in \Sigma_n$ to a relation $\I(r) \subseteq D^n$ such that
  $r \in \Sigma^w_n \Rightarrow \delta^n D_w \subseteq \I(r)$ for all $w \in W$.
\end{definition}

We choose $(W,\sqsubseteq) = (\powersetfin{\Loc},\subseteq)$ as the partial order of `worlds'.

\begin{definition}[Ground relation]
  A {\em ground relation of arity $n$} is simply a triple $R = (R^\gamma)_{\gamma\in\{\loc,\sto,\val\}}$ with
  \begin{itemize}
    \item $R^\loc \subseteq \bigcup_{\tau\in\Type} \left(\Loc^\tau\right)^n$,
    \item $R^\sto \subseteq \Store^n$ and
    \item $R^\val = \bigcup_{\tau\in\Type} \left(\CVal^\tau\right)^n \cap R^\Exp$
  \end{itemize}
  where the relation $R^\Exp \subseteq \Exp^n$ is defined by induction as follows:
  \begin{enumerate}
    \item $\vec{l}\in R^\Exp$ iff $\vec{l}\in R^\loc$.
    \item $(c,\ldots,c)\in R^\Exp$ iff $c\in\Const$.
    \item $(x,\ldots,x)\in R^\Exp$ iff $x\in\Var$.
    \item $(\abstr{x}{e_1},\ldots,\abstr{x}{e_n})\in R^\Exp$ iff $\vec{e}\in R^\Exp$.
    \item $(\app{e_1}{e_1'},\ldots,\app{e_n}{e_n'})\in R^\Exp$ iff $\vec{e},\vec{e'}\in R^\Exp$.
    \item $(\ifte{e_1}{e_1'}{e_1''},\ldots,\ifte{e_n}{e_n'}{e_n''})\in R^\Exp$ iff
          $\vec{e},\vec{e'},\vec{e''}\in R^\Exp$.
  \end{enumerate}
\end{definition}

\begin{definition}[$L$-definability]
  Let $L \in W$. A ground relation $R$ of arity $n$ is called {\em $L$-definable}, if there
  exists a relation $R_L \in (\Store_L)^n$ such that
  \begin{enumerate}
    \item $\vec{s}\in R^\sto$ iff $(\vec{s}|L) \in R_L$ and
          $\vec{s}\,(\delta^n\,(\Loc \setminus L)) \subseteq \delta^n \CVal_\bot$
    \item $\vec{l}\in R^\loc$ and $\vec{s}\in R^\sto$ iff
          $\vec{s}\,(\vec{l})\in R^\val \cup \{\bot\}^n$
    \item for every $\tau\in\Type$, $\vec{s}\in\Store^n$, $\vec{l}\in(\Loc^\tau)^n$
          and $\vec{v}\in(\CVal^\tau)^n$: \\
          $\vec{s}\in R^\sto$, $\vec{l}\in R^\loc$ and $\vec{v}\in R^\val$ iff
          $\vec{s}\,[\vec{v}/\vec{l}]\in R^\sto$
  \end{enumerate}
\end{definition}

We let $\DEF^L_n$ denote the set of all $L$-definable ground relations
of arity $n$ and $\OUT^L_n = \bigcup_{L'\in W\wedge L \cap L' = \emptyset} \DEF^{L'}_n$ denote
the set of those which are definable outside of $L$.

\begin{lemma}[$L$-definability] \label{lemma:L_definability}
  Let $n \in \N$, $L \in W$ and $R \in \OUT^L_n$, then
  \begin{enumerate}
    \item $\delta^n L \subseteq R^\loc$
    \item $\delta^n\{v\in\CVal\,|\,\locns{v} = L\} \subseteq R^\val$
  \end{enumerate}
\end{lemma}

\begin{proof}
  $R \in \OUT^L_n$ implies that a $L' \in W$ exists such that $L' \cap L = \emptyset$
  and $R \in \DEF^{L'}_n$. Without loss of generality let $L \ne \emptyset$.
  \begin{enumerate}
    \item Let $l \in L$ and $\vec{s} \in R^\sto$. Then $\vec{s}\,(l,\ldots,l) \in \delta^n \CVal_\bot$ and
          hence there exists a $d \in \CVal_\bot$ such that $\vec{s}\,(l,\ldots,l) = (d,\ldots,d)$.
          \begin{itemize}
            \item $d = \bot \Rightarrow \vec{s}\,(l,\ldots,l) \in \{\bot\}^n
                     \Rightarrow (l,\ldots,l) \in R^\loc$
            \item $d \ne \bot \Rightarrow \vec{s}\,[(d,\ldots,d)/(l,\ldots,l)] = \vec{s} \in R^\sto
                     \Rightarrow (l,\ldots,l) \in R^\loc$
          \end{itemize}

    \item Immediate consequence of the above.
  \end{enumerate}
\end{proof}

Now let $\Sigma = (\Sigma^L_n)_{L\in W,n\in \N}$ where $\Sigma^L_n = \OUT^L_n$. It is trivial to see
that $\Sigma$ is indeed a $W$-sorted signature.

\begin{definition}[Semantic domains]
  For every type $\pi\in\LType$ we define a $W$-$\Sigma$-set $\semantic{\pi}=(D^\pi,\I^\pi)$
  with $D^\pi = \bigcup_{L\in W} D^\pi_L$ by
  \begin{itemize}
    \item $D^\tau_L = \{v\in\CVal^\tau\,|\,\locns{v} = L\}$ \\
          $\I^\tau(R) = \{\vec{v}\in(D^\tau)^n\,|\,\vec{v}\in R^\val\}$

    \item $D^\tassn_L = \{ \phi:\Store\to\Bool_\bot\,|\,
                              \forall R\in\Sigma^L_n.\,\phi\,(R^\sto) \subseteq \delta^n\Bool_\bot \\
                              \hspace*{5cm} \wedge \dom{\phi} = \Store_L
                         \}$ \\
          $\I^\tassn(R) = \{ \vec{\phi} \in (D^\tassn)^n\,|\,
                                  \vec{\phi}\,(R^\sto) \subseteq \delta^n \Bool_\bot \}$

    \item $D^{\ttarrow{\theta}{\pi}}_L = \{ \psi:D^\theta \to D^\pi\,|\,
                              \forall L'\in W.\, \psi\,(D^\theta_{L'}) \subseteq D^\pi_{L \cup L'} \\
                              \hspace*{4.2cm} \wedge \forall R\in\Sigma^L.\, \psi\,(\I^\theta(R))\subseteq\I^\pi(R)\}$ \\
          $\I^{\ttarrow{\theta}{\pi}}(R) = \{ \vec{\psi} \in (D^{\ttarrow{\theta}{\pi}})^n\,|\,
                                          \vec{\psi}\,(\I^\theta(R)) \subseteq \I^\pi(R) \}$
  \end{itemize}
\end{definition}

We follow the usual mathematical convention and use $\semantic{\pi}$ not only as a notation
for the $W$-$\Sigma$-set $(D^\pi,\I^\pi)$ but also for the underlying $W$-set $D^\pi$,
hence $\semantic{\pi}_L$ denotes the set $D^\pi_L$. Moreover, we use $R^\pi$ as an abbreviation
for $\I^\pi(R)$. However we still need to prove that the semantic domains $\semantic{\pi}$
are indeed $W$-$\Sigma$-sets.

\begin{lemma}
  Let $\pi\in\LType$ and $d \in \semantic{\pi}$.
  \begin{enumerate}
    \item $\forall L_1,L_2\in W.\, d \in \semantic{\pi}_{L_1} \cap \semantic{\pi}_{L_2} \Rightarrow L_1 = L_2$
    \item $\forall L \in W, n \in \N, R\in \OUT^L_n.\, d \in \semantic{\pi}_L \Rightarrow (d,\ldots,d) \in R^\pi$
  \end{enumerate}
\end{lemma}

\begin{proof} \
  \begin{enumerate}
    \item Straight forward induction.

    \item The proof for $v\in\semantic{\tau}_L$ follows immediately with lemma~\ref{lemma:L_definability},
          the proofs for $\phi\in\semantic{\tassn}_L$ and $\psi\in\semantic{\ttarrow{\theta}{\pi}}_L$ are
          trivial.
  \end{enumerate}
\end{proof}


%%
%% Total correctness
%%

\subsection{Total correctness}

\begin{definition}[Total correctness]
  Let $\tau\in\Type$, $e\in\CExp^\tau$, $\phi\in\semantic{\tassn}$, $\psi\in\semantic{\ttarrow{\tau}{\tassn}}$
  and $L\in W$. $e$ is {\em $L$-totally correct}, or {\em $L$-correct} for short, with respect to $\phi$ and $\psi$ if for
  all $s\in\Store_L$ with $\phi\,s=\true$: each computation for $(e,s)$ terminates with some
  $(v',s')$ such that $\psi\,v'\,s'=\true$. We write $L \models \triple{\phi}{e}{\psi}$ in this case.
\end{definition}

While the above definition reflects the usual understanding of {\em total correctness}, it is not really handy
in proving the soundness of the calculus later. This is because having to argue about all computations each
time is a tedious task. To get around this, we will show that it suffices to prove that if there exists a
terminating computation for $(e,s)$, then all possible computations terminate, and $\psi$ either holds for all
or none of the computations.

\begin{lemma}
  Let $\tau\in\Type$, $e \in \CExp^\tau$, $s\in\Store$ with $\locns{e} \subseteq \dom{s}$.
  If $(e,s)$ terminates with $(v',s')$ then all computations for $(e,s)$ terminate with some
  $(v'',s'')$ such that there is a $R \in \Sigma_{\dom{s}}$ with $(v',v'')\in R^\tau$ and $(s',s'') \in R^\sto$.
\end{lemma}

\begin{lemma}
  Let $\tau\in\Type$, $\psi\in\semantic{\ttarrow{\tau}{\tassn}}$,
  $e\in\CExp^\tau$, $v',v''\in\CVal^\tau$ and $s,s',s''\in\Store$.
  If $\supp{\psi} \subseteq \dom{s}$, $(e,s)\xrightarrow*(v',s')$
  and $(e,s)\xrightarrow*(v'',s'')$ then $\psi\,v'\,s'=\psi\,v''\,s''$.
\end{lemma}

\begin{lemma}
  Let $\tau\in\Type$, $e\in\CExp^\tau$, $v'\in\CVal^\tau$, $s,s'\in\Store$
  and $R \in \Sigma_{\dom{s}}$. If $(e,s)\xrightarrow*(v',s')$ then there
  exists a configuration $(v'',s'')$ such that
  \begin{enumerate}
    \item $(v',v'')\in R^\tau$
    \item $(s',s'')\in R^\sto$
    \item $(e,s) \xrightarrow* (v'',s'')$
  \end{enumerate}
\end{lemma}

\begin{lemma}
  If $(e,s)$ has a terminating computation and $\grph{s} \subseteq \grph{s_1}$ then
  there exist $v\in\CVal^\tau$, $s',s_1'\in\Store$ with
  \begin{enumerate}
    \item $(e,s) \xrightarrow* (v,s')$
    \item $(e,s_1) \xrightarrow* (v,s_1')$
    \item $\grph{s'} \subseteq \grph{s_1'}$
  \end{enumerate}
\end{lemma}

\begin{lemma}[Total correctness]
  Let $\tau\in\Type$, $e\in\CExp^\tau$, $\phi\in\semantic{\tassn}$, $\psi\in\semantic{\ttarrow{\tau}{\tassn}}$
  and $L,L'\in W$.
  \begin{enumerate}
    \item If for every $s \in \Store_L$ with $\phi\,s=\true$ a $v'\in\CVal^\tau$ and a
          $s'\in\Store_L$ exist so that $(e,s) \xrightarrow* (v',s')$ and $\psi\,v'\,s'=\true$ then
          \[\begin{array}{rcl}
            L\models\triple{\phi}{e}{\psi}.
          \end{array}\]

    \item $\locns{e}\cup\supp{\phi}\cup\supp{\psi} \subseteq L \cap L'$ implies
          \[\begin{array}{rcl}
            L\models\triple{\phi}{e}{\psi} &\text{ iff }& L'\models\triple{\phi}{e}{\psi}.
          \end{array}\]
  \end{enumerate}
\end{lemma}

\begin{proof} \
  \begin{enumerate}
    \item

    \item Without loss of generality let $L= \locns{e}\cup\supp{\phi}\cup\supp{\psi}$ and hence $L \subseteq L'$.
          \begin{itemize}
            \item[`$\Rightarrow$']
                  Let $s\in\Store_{L'}$ with $\phi\,s=\true$. Since $\Store_{L'} \subseteq \Store_L$ we have
                  a $v\in\CVal^\tau$ and a $s'\in\Store_L$ such that $(e,s)\Downarrow(v,s')$ and
                  $\psi\,v\,s'=\true$. $s'\in\Store_{L'}$ since $L'\subseteq\dom{s}\subseteq\dom{s'}$.

            \item[`$\Leftarrow$']
          \end{itemize}
  \end{enumerate}
\end{proof}

From these results it is easy to see that the exact $L$ doesn't matter for the total correctness, since
if $L\models\triple{\phi}{e}{\psi}$ holds for some $L \supseteq \supp{\phi}\cup\locns{e}\cup\supp{\psi}$
then $L'\models\triple{\phi}{e}{\psi}$ holds for all $L'\supseteq \supp{\phi}\cup\locns{e}\cup\supp{\psi}$.
Hence we write $\triple{\phi}{e}{\psi}$ to denote that $e$ is {\em totally correct} with respect to
$\phi$ and $\psi$.

%\begin{definition}[Total correctness]
%  Let $\tau\in\Type$, $e\in\CExp^\tau$, $\phi\in\semantic{\tassn}$, $\psi\in\semantic{\ttarrow{\tau}{\tassn}}$
%  and $L\in W$. $e$ is {\em $L$-total correct}, or {\em $L$-correct} for short, with regard to $\phi$ and $\psi$ if for
%  all $s\in\Store_L$ with $\app{\phi}{s}=\true$ there exists a $v\in\CVal^\tau$ and a $s'\in\Store_L$,
%  for which $(e,s) \Downarrow (v,s')$ and $\app{\app{\psi}{v}}{s'} = \true$ holds. We write
%  $L \models \triple{\phi}{e}{\psi}$ in this case.
%\end{definition}
%
%\begin{lemma}[Total correctness]
%  Let $\tau\in\Type$, $e\in\CExp^\tau$, $\phi\in\semantic{\tassn}$, $\psi\in\semantic{\ttarrow{\tau}{\tassn}}$
%  and $L,L'\in W$. $\locns{e}\cup\supp{\phi}\cup\supp{\psi} \subseteq L \cap L'$ implies
%  \[\begin{array}{rcl}
%    L\models\triple{\phi}{e}{\psi} &\text{ iff }& L'\models\triple{\phi}{e}{\psi}.
%  \end{array}\]
%\end{lemma}
%
%\begin{proof}
%  Without loss of generality let $L= \locns{e}\cup\supp{\phi}\cup\supp{\psi}$ and hence $L \subseteq L'$.
%  \begin{itemize}
%    \item[`$\Rightarrow$']
%          Let $s\in\Store_{L'}$ with $\phi\,s=\true$. Since $\Store_{L'} \subseteq \Store_L$ we have
%          a $v\in\CVal^\tau$ and a $s'\in\Store_L$ such that $(e,s)\Downarrow(v,s')$ and
%          $\psi\,v\,s'=\true$. $s'\in\Store_{L'}$ since $L'\subseteq\dom{s}\subseteq\dom{s'}$.
%
%    \item[`$\Leftarrow$']
%  \end{itemize}
%\end{proof}


\end{document}

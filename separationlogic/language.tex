\documentclass[12pt,a4paper]{article}

\usepackage{amsmath}
\usepackage{amssymb}
\usepackage{amstext}
\usepackage{array}
\usepackage[american]{babel}
\usepackage{color}
\usepackage{enumerate}
\usepackage[a4paper,%
            colorlinks=false,%
            final,%
            pdfkeywords={},%
            pdftitle={},%
            pdfauthor={Benedikt Meurer},%
            pdfsubject={},%
            pdfdisplaydoctitle=true]{hyperref}
\usepackage{ifthen}
\usepackage[latin1]{inputenc}
\usepackage{latexsym}
\usepackage[final]{listings}
\usepackage{makeidx}
\usepackage[standard,thmmarks]{ntheorem}
\usepackage{stmaryrd}
\usepackage{url}
\usepackage[arrow, matrix, curve]{xy}

%% LaTeX macros
%%
%% macros.tex - Useful LaTeX macros
%%
%% Copyright (c) 2006-2008 Benedikt Meurer <benedikt.meurer@googlemail.com>
%%


%%
%% Generic style macros
%%

\newcommand{\nstyle}[1]{\ensuremath{\mathit{#1}}}
\newcommand{\pstyle}[1]{\ensuremath{\mathsf{#1}}}


%%
%% Misc
%%

\newcommand{\coloncolon}{\mathrel{::}}
\newcommand{\pto}{\hookrightarrow}
\newcommand{\tto}{\rightarrow_t}


%%
%% Sets
%%

\newcommand{\N}{\ensuremath{\mathbb N}}
\newcommand{\Z}{\ensuremath{\mathbb Z}}

\newcommand{\Bool}{\nstyle{Bool}}
\newcommand{\Const}{\nstyle{Const}}
\newcommand{\Exp}{\nstyle{Exp}}
\newcommand{\Member}{\ensuremath{\mathcal{M}}}
\newcommand{\Var}{\ensuremath{\mathcal{X}}}
\newcommand{\Int}{\Z}
\newcommand{\Loc}{\nstyle{Loc}}
\newcommand{\Locfin}{\ensuremath{\Loc_{fin}}}
\newcommand{\Op}{\nstyle{Op}}
\newcommand{\Store}{\nstyle{Store}}
\newcommand{\Type}{\nstyle{Type}}
\newcommand{\Val}{\nstyle{Val}}

\newcommand{\AType}{\nstyle{AType}}
\newcommand{\DType}{\nstyle{DType}}
\newcommand{\LType}{\nstyle{LType}}

\newcommand{\Assn}{\nstyle{Assn}}
\newcommand{\Formula}{\nstyle{Formula}}
\newcommand{\Term}{\nstyle{Term}}

\newcommand{\Perm}{\nstyle{Perm}}


%%
%% Functions (TODO - DeclareMathOperator)
%%

\newcommand{\dom}[1]{\ensuremath{\nstyle{dom}\,(#1)}}
\newcommand{\free}[1]{\ensuremath{\nstyle{free}\,(#1)}}
\newcommand{\locns}[1]{\ensuremath{\nstyle{locns}\,(#1)}}
\newcommand{\grph}[1]{\ensuremath{\nstyle{graph}\,(#1)}}
\newcommand{\supp}[1]{\ensuremath{\nstyle{supp}\,(#1)}}
\newcommand{\reach}[2]{\ensuremath{\nstyle{reach}\,(#1,#2)}}
\newcommand{\reachn}[3]{\ensuremath{\nstyle{reach}_{#1}\,(#2,#3)}}
\newcommand{\semantic}[1]{\ensuremath{\llbracket#1\rrbracket}}

\newcommand{\powerset}[1]{\ensuremath{\wp\,(#1)}}
\newcommand{\powersetfin}[1]{\ensuremath{\wp_{fin}\,(#1)}}


%%
%% Names
%%

\newcommand{\op}{\nstyle{op}}


%%
%% Types
%%

\newcommand{\tassn}{\pstyle{assn}}
\newcommand{\tunit}{\pstyle{unit}}
\newcommand{\tbool}{\pstyle{bool}}
\newcommand{\tint}{\pstyle{int}}
\newcommand{\tarrow}[2]{#1\to#2}
\newcommand{\ttarrow}[2]{#1\tto#2}
\newcommand{\trecord}[1]{\left\{#1\right\}}
\newcommand{\tref}[1]{#1\,\pstyle{ref}}


%%
%% Constants
%%

\newcommand{\true}{\pstyle{true}}
\newcommand{\false}{\pstyle{false}}
\newcommand{\assign}{\pstyle{:=}}
\newcommand{\fix}{\pstyle{fix}}
\newcommand{\cref}{\pstyle{ref}}
\newcommand{\unit}{\pstyle{()}}


%%
%% Expressions
%%

\newcommand{\abstr}[2]{\lambda #1.#2}
\newcommand{\app}[2]{#1\,#2}
\newcommand{\ifte}[3]{\pstyle{if}\,#1\,\pstyle{then}\,#2\,\pstyle{else}\,#3}
\newcommand{\record}[1]{\left\{#1\right\}}
\newcommand{\proj}[2]{#1.#2}

\newcommand{\triple}[3]{\{#1\}\,#2\,\{#3\}}


%%
%% Grammars
%%

\newcommand{\GRbeg}{\begin{array}{rrlll}}
\newcommand{\GRend}{\end{array}}

\newcommand{\GRis}{& ::= &}
\newcommand{\GRal}{\\ & \mid &}
\newcommand{\GRnl}{\vspace{2mm}\\}
\newcommand{\GRmid}{\,\mid\,}
\newcommand{\GRtext}[1]{\hfill & \text{#1}}


%%
%% Rules
%%

\newcommand{\tj}[2]{#1\,\coloncolon\,#2}
\newcommand{\Tj}[3]{#1\,\triangleright\,#2\coloncolon#3}



% TODO
\newcommand{\aop}{\nstyle{aop}}
\newcommand{\rop}{\nstyle{rop}}
\newcommand{\CAssn}{\sstyle{CAssn}}
\newcommand{\CTerm}{\sstyle{CTerm}}
\newcommand{\CExp}{\sstyle{CExp}}
\newcommand{\CVal}{\sstyle{CVal}}
\newcommand{\Conf}{\sstyle{Conf}}
\newcommand{\Con}{\sstyle{Con}}
\newcommand{\Sto}{\sstyle{Sto}}
\newcommand{\scon}{\nstyle{con}}
\newcommand{\sexp}{\nstyle{exp}}
\newcommand{\ssto}{\nstyle{sto}}
\newcommand{\senv}{\nstyle{env}}
\newcommand{\z}{\nstyle{int}}
\newcommand{\DEF}{\sstyle{DEF}}
\newcommand{\id}{\nstyle{id}}
\newcommand{\I}{\mathcal{I}}
\newcommand{\Fix}[1]{\nstyle{Fix}\,(#1)}
\newcommand{\returns}[2]{\ensuremath{\mathbf{returns}\,{#1}.\,{#2}}}

\newcommand{\AUX}{\nstyle{AUX}}
\newcommand{\Neg}{\nstyle{Neg}}
\newcommand{\Or}{\nstyle{Or}}
\newcommand{\Cont}{\nstyle{Cont}}
\newcommand{\Disj}{\nstyle{Disj}}
\newcommand{\Exists}{\nstyle{Exists}}

\newcommand{\emp}{\ensuremath{\mathbf{emp}}}


\begin{document}


\section{Types}

\begin{definition}
  The sets $\Type$ of {\em expression types} $\tau$, $\TType$ of {\em term types} $\sigma$ and
  $\LType$ of {\em logical types} $\theta$ are defined by the following grammar:
  \[\GRbeg
  \tau \GRis \tbool \GRmid \tint \GRmid \tref{\tau}
  \GRmid \tarrow{\tau_1}{\tau_2} \GRmid \tuple{\tau_1,\ldots,\tau_n} \quad (n \ge 0)
  \\
  \sigma \GRis \tau \GRmid \ttarrow{\sigma_1}{\sigma_2} \GRmid \ttuple{\sigma_1,\ldots,\sigma_n} \quad (n \ge 2)
  \\
  \theta \GRis \sigma \GRmid \tassn[\sigma]
  \GRend\]
\end{definition}

We write $\tunit$ instead of $\tuple{}$ and $\tassn$ instead of $\tassn[\tunit]$.

We assume that disjoint infinite sets $\Var^\theta$ of {\em $\tau$-variables} $x^\theta$ and $\Loc^\tau$ of
{\em $\tau$-locations} $\ell^\tau$ are given. Then we have sets $\Var = \bigcup_{\theta\in\LType}\Var^\theta$ of
{\em variables} $x$ and $\Loc = \bigcup_{\tau\in\Type}\Loc^\tau$ of {\em locations} $\ell$, where variables are
used to abstract expressions and locations are used to identify memory cells.


\section{The programming language}

Expressions are considered equal modulo renaming of bound variables. $\Bool=\{\true,\false\}$ is
the set of boolean constants $b$, $\Int$ is the set of integer constants $z$. We do not distinguish
between integer values their program representation.

\begin{definition}[Expressions]
  The set $\Op$ of {\em operators} $\op$ and the set $\Const$ of {\em constants} $c$ are defined by
  the grammars
  \[\GRbeg
  \aop \GRis + \GRmid - \GRmid * \\
  \rop \GRis = \GRmid < \GRmid > \GRmid \le \GRmid \ge \\
  \op \GRis \aop \GRmid \rop \\
  c \GRis z \GRmid b \GRmid \cref \GRmid ! \GRmid \assign
% \GRmid \fix
  \GRmid \op
  \GRmid \proj{n}{i},
% \GRmid \fst \GRmid \snd,
  \GRend\]
  the set $\Exp$ of {\em expressions} $e$ is defined by
  \[\GRbeg
  e \GRis c \GRmid x^\tau \GRmid \ell \GRmid \abstr{x^\tau}{e} \GRmid \app{e_1}{e_2}
  \GRmid \ifte{e_0}{e_1}{e_2} \GRmid \tuple{e_1,\ldots,e_n}
  \GRend\]
  and the set $\Val \subseteq \Exp$ of {\em values} $v$ is defined by
  \[\GRbeg
  v \GRis c \GRmid \ell \GRmid \abstr{x^\tau}{e} \GRmid \app{\op}{v} \GRmid \app{\assign}{v} \GRmid \tuple{v_1,\ldots,v_n}
  \GRend\]
\end{definition}

The empty tuple $\tuple{}$ is sometimes referred to as the {\em unit constant}.

$\free{e}$ denotes the set of free variables and $\locns{e}$ denotes the set of locations in the expression $e$.
We say that $e$ is {\em closed} if $\free{e}=\emptyset$. $e$ is called a {\em program} if
$\free{e}\cup\locns{e}=\emptyset$, that is, if $e$ contains neither unbound variables nor locations (i.e.
memory addresses in terms of the underlying machine). We demand that all expressions considered for the
logic later are valid programs (i.e. programmers aren't permitted to access arbitrary memory locations).


\subsection{Static semantics}

\begin{definition}[Typing relation for expressions]
  The {\em typing relation for expressions}, written $\tj{e}{\tau}$, is the smallest relation between expressions
  and types satisfying all instances of the rules in figure~\ref{fig:Typing_rules_for_constants} and
  figure~\ref{fig:Typing_rules_for_expressions}. An expression $e$ is {\em well typed} if there is some
  $\tau\in\Type$ such that $\tj{e}{\tau}$.
\end{definition}

\begin{figure}[htb]
  \centering
  \begin{tabular}{rll}
    \RN{T-Int} & $\tj{z}{\tint}$ \\[1mm]
    \RN{T-Bool} & $\tj{b}{\tbool}$ \\[1mm]
    \RN{T-Ref} & $\tj{\cref}{\tarrow{\tau}{\tref{\tau}}}$ \\[1mm]
    \RN{T-Deref} & $\tj{!}{\tarrow{\tref{\tau}}{\tau}}$ \\[1mm]
    \RN{T-Assign} & $\tj{\assign}{\tarrow{\tref{\tau}}{\tarrow{\tau}{\tunit}}}$ \\[1mm]
    \RN{T-Aop} & $\tj{\aop}{\tarrow{\tint}{\tarrow{\tint}{\tint}}}$ \\[1mm]
    \RN{T-Rop} & $\tj{\rop}{\tarrow{\tint}{\tarrow{\tint}{\tbool}}}$ \\[1mm]
    \RN{T-Proj} & $\tj{\proj{n}{i}}{\tarrow{\tuple{\tau_1,\ldots,\tau_n}}{\tau_i}}$
  \end{tabular}
  \caption{Typing rules for constants}
  \label{fig:Typing_rules_for_constants}
\end{figure}

\begin{figure}[htb]
  \centering
  \begin{tabular}{rl}
    \RN{T-Var} & $\tj{x^\tau}{\tau}$ \\[1mm]
    \RN{T-Loc} & $\tj{\ell^\tau}{\tref{\tau}}$ \\[1mm]
    \RN{T-Abstr} & $\RULE{\tj{e}{\tau'}}{\tj{\abstr{x^\tau}{e}}{\tarrow{\tau}{\tau'}}}$ \\[3mm]
    \RN{T-App} & $\RULE{\tj{e_1}{\tarrow{\tau}{\tau'}} \quad \tj{e_2}{\tau}}{\tj{\app{e_1}{e_2}}{\tau'}}$ \\[3mm]
    \RN{T-Cond} & $\RULE{\tj{e_0}{\tbool} \quad \tj{e_1}{\tau} \quad \tj{e_2}{\tau}}{\tj{\ifte{e_0}{e_1}{e_2}}{\tau}}$ \\[3mm]
    \RN{T-Tuple} & $\RULE{\tj{e_1}{\tau_1} \quad \ldots \quad \tj{e_n}{\tau_n}}{\tj{\tuple{e_1,\ldots,e_n}}{\tuple{\tau_1,\ldots,\tau_n}}}$
  \end{tabular}
  \caption{Typing rules for expressions}
  \label{fig:Typing_rules_for_expressions}
\end{figure}


\subsection{Operational semantics}

Let $(W,\subseteq) = (\powersetfin{\Loc},\subseteq)$ be a partial order of ``possible worlds''.

\begin{definition}[Store]
  A {\em store} is a finite partial mapping $s:\Loc \pto \CVal$, such that
  $s(\Loc^\tau) \subseteq \CVal^\tau$ for all $\tau\in\Type$.
  $\Store$ denotes the set of all such stores.
\end{definition}

A {\em configuration} $k$ is a pair $(e,s)$.
$\Conf$ denotes the set of all configurations.

\begin{figure}[ht]
  \centering
  \begin{tabular}{rl}
    \RN{App-Left} & $\RULE{(e_1,s) \to (e_1',s')}{(\app{e_1}{e_2},s)\to(\app{e_1'}{e_2},s')}$ \\[3mm]
    \RN{App-Right} & $\RULE{(e,s) \to (e',s')}{(\app{v}{e},s)\to(\app{v}{e'},s')}$ \\[3mm]
    \RN{Op} & $(\app{\app{\op}{z_1}}{z_2},s) \to (\op^I(z_1,z_2),s)$ \\[1mm]
    \RN{Beta-V} & $(\app{(\abstr{x}{e})}{v},s) \to (e[v/x],s)$ \\[1mm]
    \RN{Proj} & $(\app{\proj{n}{i}}{\tuple{v_1,\ldots,v_n}},s) \to (v_i,s)$ \\[1mm]
    \RN{Deref} & $\RULE{s(\ell) = v}{(\app{!}{\ell},s) \to (v,s)}$ \\[3mm]
    \RN{Assign} & $\RULE{\ell\in \dom{s}}{(\app{\app{\assign}{\ell}}{v},s) \to (\unit,s[v/\ell])}$ \\[3mm]
    \RN{Ref} & $\RULE{\tj{v}{\tau} \quad \ell \in \Loc^\tau \setminus \dom{s}}{(\app{\cref}{v},s) \to (\ell,s[v/\ell])}$ \\[3mm]
    \RN{Cond-Eval} & $\RULE{(e_0,s) \to (e_0',s')}{(\ifte{e_0}{e_1}{e_2},s)\to(\ifte{e_0'}{e_1}{e_2},s')}$ \\[3mm]
    \RN{Cond-True} & $(\ifte{\true}{e_1}{e_2},s) \to (e_1,s)$ \\[1mm]
    \RN{Cond-False} & $(\ifte{\false}{e_1}{e_2},s) \to (e_2,s)$ \\[1mm]
    \RN{Tuple} & $\RULE{(e_i,s) \to (e_i',s')}{(\tuple{v_1,\ldots,v_{i-1},e_i,\ldots,e_n},s) \to (\tuple{v_1,\ldots,v_{i-1},e_i',\ldots,e_n},s')}$
  \end{tabular}
  \caption{Small step rules}
  \label{fig:Small_step_rules}
\end{figure}

\begin{definition}[Small step relation]
  The {\em small step relation} $\to$ is the smallest binary relation between configurations
  satisfying all instances of the rules in figure~\ref{fig:Small_step_rules}. We write $\to^*$ for
  the reflexive, transitive closure of $\to$.
\end{definition}

We write $k \not\to$ if there exists no $k'$ such that $k \to k'$.
Obviously the small step relation is well-defined. Let's say we have $(e,s)\to(e',s')$. Then
\begin{itemize}
\item if $\tj{e}{\tau}$, then $\tj{e'}{\tau}$, and
\item if $\locns{e} \subseteq \dom{s}$ then $\locns{e'} \subseteq \dom{s'}$, and
\item if $\free{e'} \subseteq \free{e}$.
\end{itemize}

A {\em finite computation} for $k$ is a finite sequence of configurations $k_1,\ldots,k_n$
such that $k = k_1$, $k_n \not\to$ and $k_i \to k_{i+1}$ for every $i=1,\ldots,n-1$. Such a
finite computation {\em terminates} if $k_n \in \Val\times\Store$. Otherwise
we say that the computation {\em got stuck}.
An {\em infinite computation} for $k$ is an infinite sequence of configurations $k_1,k_2,\ldots$
such that $k=k_1$ and for each $i=1,2,\ldots$ there is some $k_{i+1}$ such that $k_i \to k_{i+}$.
We say that such an infinite computation {\em diverges}.
A configuration $k$ is {\em safe} if there is no computation for $k$, which gets
stuck. Likewise $k$ {\em must terminate normally} if all computations for $k$
terminate.

Similar to O'Hearn ({\em ``A semantic basis for Local Reasoning''}) we state:

\begin{lemma}[Safety and Termination Monotonicity] \
  \begin{enumerate}
  \item If $(e,s)$ is safe and $s' = s \oplus s''$, then $(e,s')$ is also safe.
  \item If $(e,s)$ must terminate normally and $s' = s \oplus s''$, then $(e,s')$ must also terminate normally.
  \end{enumerate}
\end{lemma}

\begin{lemma}[Frame Property]
  Suppose $(e,s_0)$ is safe, and $(e,s_0 \oplus s_1) \to^* (v,s')$. Then there exists some $s_0'$ such that
  $(s,s_0) \to^* (v,s_0')$ and $s' = s_0' \oplus s_1$.
\end{lemma}


\section{Assertions}

Let $\mathcal{F}^\sigma$ be some predefined set of {\em $\sigma$-function symbols} $f^\sigma$ and
$\mathcal{F} = \bigcup_{\sigma\in\TType}\mathcal{F}^\sigma$ be the set of all {\em function symbols} $f$, such that
\begin{itemize}
\item $f_\aop \in \mathcal{F}^{\ttarrow{\tint}{\ttarrow{\tint}{\tint}}}$, and
\item $f_\rop \in \mathcal{F}^{\ttarrow{\tint}{\ttarrow{\tint}{\tbool}}}$,
\end{itemize}
and let $\mathcal{R}^\sigma$ be some predefined set of {\em $\sigma$-relation symbols} $R^\sigma$ and
$\mathcal{R} = \bigcup_{\sigma\in\TType}\mathcal{R}^\sigma$ be the set of all {\em relation symbols} $R$, such that
\begin{itemize}
\item $\btrue,\bfalse\in \mathcal{R}^\tunit$,
\item $=\ \in \mathcal{R}^\sigma$ for each $\sigma\in\TType$, and
\item $<,>,\le,\ge\ \in \mathcal{R}^{\ttuple{\tint,\tint}}$.
\end{itemize}

\begin{definition}[Assertions]
  The sets $\Term$ of {\em terms} $t$ and $\Assn$ of {\em assertions} $p,q,r$ are defined by the following grammar:
  \[\GRbeg
  t \GRis f \GRmid x \GRmid v \GRmid \app{t_1}{t_2} \GRmid \ttuple{t_1,\ldots,t_n}
  \\
  p \GRis \app{R}{t} \GRmid \app{x}{t} \GRmid \neg p \GRmid p \wedge q \GRmid \exists x.\,p
  \GRal \emp \GRmid t_1 \mapsto t_2 \GRmid p \sepcon q \GRmid p \sepimp q  \GRmid \triplex{p}{e}{x}{q}
  \GRend\]
\end{definition}

\begin{definition}[Hoare formulae]
  The set $\Formula \subseteq \Assn$ of {\em hoare formulae} $h$ is defined by the following grammar:
  \[\GRbeg
  h \GRis \app{R}{t} \GRmid \neg h \GRmid h_1 \wedge h_2 \GRmid \exists x.\,h
  \GRal h_1 \sepcon h_2 \GRmid h_1 \sepimp h_2 \GRmid \triplex{p}{e}{x}{q}
  \GRend\]
\end{definition}

\begin{definition}
  The {\em typing relation for terms}, written $\tj{t}{\sigma}$, is the smallest relation between terms
  and term types satisfying all instances of the rules in figure~\ref{fig:Typing_rules_for_constants},
  figure~\ref{fig:Typing_rules_for_expressions} and figure~\ref{fig:Typing_rules_for_terms}. A term $t$
  is {\em well typed} if there is some $\sigma\in\TType$ such that $\tj{t}{\sigma}$.
\end{definition}

\begin{figure}[htb]
  \centering
  \begin{tabular}{rl}
    \RN{T-Fun} & $\tj{f^\sigma}{\sigma}$ \\[1mm]
    \RN{T-Var} & $\tj{x^\sigma}{\sigma}$ \\[1mm]
    \RN{T-App} & $\RULE{\tj{t_1}{\ttarrow{\sigma}{\sigma'}} \quad \tj{t_2}{\sigma}}{\tj{\app{t_1}{t_2}}{\sigma'}}$ \\[3mm]
    \RN{T-Tuple} & $\RULE{\tj{t_1}{\sigma_1} \quad \ldots \quad \tj{t_n}{\sigma_n}}{\tj{\ttuple{t_1,\ldots,t_n}}{\ttuple{\sigma_1,\ldots,\sigma_n}}}$
  \end{tabular}
  \caption{Typing rules for terms}
  \label{fig:Typing_rules_for_terms}
\end{figure}

\begin{definition}
  The set of {\em well typed} assertions is the smallest set of assertions satisfying all instances of the
  rules in figure~\ref{fig:Typing_rules_for_assertions}. We write $\tj{p}{\tassn}$ to indicate that $p$ is
  well typed.
\end{definition}

\begin{figure}[htb]
  \centering
  \begin{tabular}{rl}
    \RN{T-Rel} & $\RULE{\tj{t}{\sigma}}{\tj{\app{R^\sigma}{t}}{\tassn}}$ \\[3mm]
    \RN{T-Var} & $\RULE{\tj{t}{\sigma}}{\tj{\app{x^{\tassn[\sigma]}}{t}}{\tassn}}$ \\[3mm]
    \RN{T-Neg} & $\RULE{\tj{p}{\tassn}}{\tj{\neg p}{\tassn}}$ \\[3mm]
    \RN{T-Conj} & $\RULE{\tj{p}{\tassn} \quad \tj{q}{\tassn}}{\tj{p \wedge q}{\tassn}}$ \\[3mm]
    \RN{T-Exists} & $\RULE{\tj{p}{\tassn}}{\tj{\exists x^\theta.\,p}{\tassn}}$ \\[3mm]
    \RN{T-Emp} & $\tj{\emp}{\tassn}$ \\[1mm]
    \RN{T-Cont} & $\RULE{\tj{t_1}{\tref{\tau}} \quad \tj{t_2}{\tau}}{\tj{t_1 \mapsto t_2}{\tassn}}$ \\[3mm]
    \RN{T-SepConj} & $\RULE{\tj{p}{\tassn} \quad \tj{q}{\tassn}}{\tj{p \sepcon q}{\tassn}}$ \\[3mm]
    \RN{T-SepImpl} & $\RULE{\tj{p}{\tassn} \quad \tj{q}{\tassn}}{\tj{p \sepimp q}{\tassn}}$ \\[3mm]
    \RN{T-Spec} & $\RULE{\tj{p}{\tassn} \quad \tj{e}{\tau} \quad \tj{q}{\tassn}}{\tj{\triplex{p}{e}{x^\tau}{q}}{\tassn}}$
  \end{tabular}
  \caption{Typing rules for assertions}
  \label{fig:Typing_rules_for_assertions}
\end{figure}


\subsection{Meaning of assertions}

For each logical type $\theta\in\LType$ we define a semantic domain $\semantic{\theta}$ by:
\[\begin{array}{rcl}
  \semantic{\tau} &=& \{v\in\CVal\,|\,\tj{v}{\tau}\} \\
  \semantic{\ttarrow{\sigma_1}{\sigma_2}} &=& (\semantic{\sigma_1} \to \semantic{\sigma_2}) \\
  \semantic{\ttuple{\tau_1,\ldots,\tau_n}} &=& \semantic{\tau_1} \times \ldots \times \semantic{\tau_n} \\
  \semantic{\tassn[\sigma]} &=& \mathcal{P}(\semantic{\sigma} \times \Store)
\end{array}\]

For each predefined function symbol $f^\sigma$, let $\semantic{f^\sigma}\in\semantic{\sigma}$ be the meaning of
$f$, with
\begin{itemize}
\item $\semantic{f_\aop} : \semantic{\tint} \to \semantic{\tint} \to \semantic{\tint},
  z_1 z_2 \mapsto \aop^I(z_1,z_2)$, and
\item $\semantic{f_\rop} : \semantic{\tint} \to \semantic{\tint} \to \semantic{\tbool},
  z_1 z_2 \mapsto \rop^I(z_1,z_2)$.
\end{itemize}
For each predefined relation symbol $R^\sigma$, let $\semantic{R^\sigma}\subseteq\semantic{\sigma}$ be the meaning
of $f$, with
\begin{itemize}
\item $\semantic{\btrue^\tunit} = \{\unit\}$ and $\semantic{\bfalse^\tunit} = \emptyset$,
\item $\semantic{=^\sigma} = \{(d,d)\,|\,d\in\semantic{\sigma}\}$ for each $\sigma\in\TType$, and
\item $\semantic{\op^{\ttuple{\tint,\tint}}} = \{(z_1,z_2)\in\semantic{\ttuple{\tint,\tint}}\,|\,\op^I(z_1,z_2)=\true\}$
  for each $\op\in\{<,>,\le,\ge\}$.
\end{itemize}

An {\em environment} is a partial finite mapping $\eta:\Var \pto \bigcup_{\theta\in\LType}\semantic{\theta}$ such
that $\eta(\Var^\theta) \subseteq \semantic{\theta}$ for each $\theta\in\LType$.

\begin{definition}[Meaning of terms]
  Let $t$ be a term of type $\sigma$ and $\eta$ a suitable environment with $\free{t} \subseteq \dom{\eta}$.
  The {\em meaning of $t$} wrt. $\eta$, written $\semantic{\tj{t}{\sigma}}\,\eta$, is defined by:
  \[\begin{array}{rcl}
    \semantic{\tj{f}{\sigma}}\,\eta
    &=& \semantic{f^\sigma} \\
    \semantic{\tj{x}{\sigma}}\,\eta
    &=& \eta(x) \\
    \semantic{\tj{v}{\tau}}\,\eta
    &=& v\,\eta \\
    \semantic{\tj{\app{t_1}{t_2}}{\sigma}}\,\eta
    &=& \app{(\semantic{\tj{t_1}{\ttarrow{\sigma'}{\sigma}}}\,\eta)}{(\semantic{\tj{t_2}{\sigma'}}\,\eta)} \\
    \semantic{\tj{\ttuple{t_1,\ldots,t_n}}{\ttuple{\sigma_1,\ldots,\sigma_n}}}\,\eta
    &=& (\semantic{\tj{t_1}{\sigma_1}}\,\eta,\ldots,\semantic{\tj{t_n}{\sigma_n}}\,\eta) 
  \end{array}\]
\end{definition}

\begin{lemma}
  $\free{t} \subseteq \dom{\eta} \Rightarrow (\semantic{\tj{t}{\sigma}}\,\eta)\in\semantic{\sigma}$
\end{lemma}

We write $\semantic{t}$ instead of $\semantic{\tj{t}{\sigma}}$, since for each well typed term $t$ there is
exactly one $\sigma$ such that $\tj{t}{\sigma}$.

\begin{definition}[Meaning of assertions]
  The {\em satisfaction relation}, written $\eta,s \models p$, is the smallest relation between 
  environments, stores and well typed assertions, satisfying $\free{p} \subseteq \dom{\eta}$ and all of the
  following properties:
  \[\begin{array}{lcl}
    \eta,s \models \app{R^\sigma}{t}
    &\text{iff}& (\semantic{\tj{t}{\sigma}}\,\eta) \in \semantic{R^\sigma} \\
    \eta,s \models \app{x^{\tassn[\sigma]}}{t}
    &\text{iff}& (\semantic{\tj{t}{\sigma}}\,\eta,s) \in \eta(x) \\
    \eta,s \models \neg p
    &\text{iff}& \eta,s \not\models p \\
    \eta,s \models p \wedge q
    &\text{iff}& \eta,s \models p \text{ and } \eta,s \models q \\
    \eta,s \models \exists x^\theta.\,p
    &\text{iff}& \exists d \in \semantic{\theta}.\,\eta[d/x],s \models p \\
    \eta,s \models \emp
    &\text{iff}& \grph{s} = \emptyset \\
    \eta,s \models t_1 \mapsto t_2
    &\text{iff}& \grph{s} = \{(\semantic{t_1}\,\eta,\semantic{t_2}\,\eta)\} \\
    \eta,s \models p \sepcon q
    &\text{iff}& \exists s_1,s_2.\,s=s_1 \oplus s_2 \text{ and } \eta,s_1 \models p \text{ and } \eta,s_2 \models q \\
    \eta,s \models p \sepimp q
    &\text{iff}& \forall s_1,s_2.\,(s_2 = s \oplus s_1 \text{ and } \eta,s_1 \models p) \text{ implies } \eta,s_2 \models q \\
    \eta,s \models \triplex{p}{e}{x}{q}
    &\text{iff}& \forall s'.\,\eta,s' \models p \text{ implies that all computations for } (e\,\eta,s') \\
    && \text{terminate with some $(v,s'')$ such that } \eta[v/x],s'' \models q
  \end{array}\]
\end{definition}

We write $R^\tunit$ instead of $\app{R^\tunit}{\unit}$ and likewise $x^{\tassn}$ instead of $\app{x^{\tassn}}{\unit}$.
Furtheron we abbreviate:
\[\begin{array}{rcl}
  \forall x.\,p &=_{def}& \neg \exists x.\,\neg p \\
  p \vee q &=_{def}& \neg (\neg p \wedge \neg q) \\
  p \Rightarrow q &=_{def}& \neg p \vee q \\
  p \Leftrightarrow q &=_{def}& (p \Rightarrow q) \wedge (q \Rightarrow p)
\end{array}\]
It is useful to introduce several more complex forms as abbreviations:
\[\begin{array}{rcl}
  t \mapsto - &=_{def}& \exists x.\,t \mapsto x \quad \text{where $x\not\in\free{t}$} \\
  t_1 \hookrightarrow t_2 &=_{def}& t_1 \mapsto t_2 \sepcon \btrue \\
  \{p\} &=_{def}& \triplex{\btrue}{\unit}{x}{p} \quad \text{where $x\not\in\free{p}$}
\end{array}\]

Meaning of abbreviations:
\[\begin{array}{lcl}
  \eta,s \models \btrue
  && \text{always} \\
  \eta,s \models \bfalse
  && \text{never} \\
  \eta,s \models p \vee q
  &\text{iff}& \eta,s \models p \text{ or } \eta,s \models q \\
  \eta,s \models p \Rightarrow q
  &\text{iff}& \eta,s \models p \text{ implies } \eta,s \models q \\
  \eta,s \models p \Leftrightarrow q
  &\text{iff}& \eta,s \models p \text{ iff } \eta,s \models q \\
  \eta,s \models \forall x^\theta.\,p
  &\text{iff}& \forall d\in \semantic{\theta}.\,\eta[d/x],s \models p \\
  \eta,s \models t \mapsto -
  &\text{iff}& \dom{s} = \{\semantic{t}\,\eta\} \\
  \eta,s \models t_1 \hookrightarrow t_2
  &\text{iff}& (\semantic{t_1}\,\eta,\semantic{t_2}\,\eta)\in\grph{s} \\
  \eta,s \models \{p\}
  &\text{iff}& \forall s'.\,\eta,s' \models p
\end{array}\]

\subsubsection{Pure assertions}

\begin{definition}
  An assertion is said to be {\em pure} if, for any given environment, it is independent of the store.
  More precisely, an assertion $p$ is {\em pure} iff, for all environments $\eta$ and all stores $s, s'$,
  \[\begin{array}{rcl}
    \eta,s \models p &\text{iff}& \eta,s' \models p.
  \end{array}\]
\end{definition}

Obviously all hoare formulae $h$ are pure.


\subsection{The calculus}

\subsubsection{Rules for the programming language}

{\small
\begin{tabular}{rl}
  \RN{Val} & $\triple{\emp}{x_{:y}}{y = x \wedge \emp}$ \\[1mm]
  \RN{App} & $\triplex{p}{e_1}{x}{p'} \wedge (\forall x.\,\triplex{p'}{e_2}{y}{q'}) \wedge (\forall x,y.\,\triplex{q'}{\app{x}{y}}{z}{q})$ \\
  & $\Rightarrow \triplex{p}{\app{e_1}{e_2}}{z}{q}$ \\
  & {\footnotesize where $x \not\in \free{e_2}$, $x,y \not\in\free{q}$} \\[1mm]
  \RN{Op} & $\triplex{\emp}{\app{\app{\op}{x}}{y}}{z}{z = \app{f_{\op}}{\tuple{x,y}} \wedge \emp}$ \\[1mm]
  \RN{Beta-V} & $\forall x.\,\triplex{p}{e}{y}{q} \Rightarrow \triplex{p}{\app{(\abstr{x}{e})}{x}}{y}{q}$ \\[1mm]
%  \RN{Unfold} & $\triplex{p}{e[\app{\fix}{(\abstr{x}{e})}/x]}{y}{q} \Rightarrow \triplex{p}{\app{\fix}{(\abstr{x}{e})}}{y}{q}$ \\[1mm]
  \RN{Fst} & $\forall x,y.\,\triplex{\emp}{\app{\fst}{\pair{x}{y}}}{z}{z = x \wedge \emp}$ \\[1mm]
  \RN{Snd} & $\forall x,y.\,\triplex{\emp}{\app{\snd}{\pair{x}{y}}}{z}{z = y \wedge \emp}$ \\[1mm]
  \RN{Deref} & $\forall x,y.\,\triplex{x \mapsto y}{!x}{z}{z = y \wedge x \mapsto y}$ \\[1mm]
  \RN{Assign} & $\forall x,y.\,\triplex{x \mapsto -}{\app{\app{\assign}{x}}{y}}{u}{u = \unit \wedge x \mapsto y}$ \\[1mm]
  \RN{Ref} & $\forall x.\,\triplex{\emp}{\cref\,x}{y}{y \mapsto x}$ \\[1mm]
  \RN{Cond-True} & $\triplex{p}{e_0}{x}{x = \true \wedge r} \wedge \triplex{r}{e_1}{y}{q}$ \\
  & $\Rightarrow \triplex{p}{\ifte{e_0}{e_1}{e_2}}{y}{q}$ \\
  & {\footnotesize where $x \not\in \free{r}$} \\[1mm]
  \RN{Cond-False} & $\triplex{p}{e_0}{x}{x = \false \wedge r} \wedge \triplex{r}{e_2}{y}{q}$ \\
  & $\Rightarrow \triplex{p}{\ifte{e_0}{e_1}{e_2}}{y}{q}$ \\
  & {\footnotesize where $x \not\in \free{r}$} \\[1mm]
  \RN{Pair} & $\triplex{p}{e_1}{x}{p'} \wedge (\forall x.\,\triplex{p'}{e_2}{y}{q'}) \wedge (\forall x,y.\,\triplex{q'}{\pair{x}{y}}{z}{q})$ \\
  & $\Rightarrow \triplex{p}{\pair{e_1}{e_2}}{z}{q}$ \\
  & \footnotesize{where $x \not\in \free{e_2}$, $x,y \not\in \free{q}$}
\end{tabular}}

\subsubsection{Rules for separation}

{\small
\begin{tabular}{rl}
  \RN{Sep-1} & $p_1 \sepcon p_2 \Leftrightarrow p_2 \sepcon p_1$ \\[1mm]
  \RN{Sep-2} & $(p_1 \sepcon p_2) \sepcon p_3 \Leftrightarrow p_1 \sepcon (p_2 \sepcon p_3)$ \\[1mm]
  \RN{Sep-3} & $p \sepcon \emp \Leftrightarrow p$ \\[1mm]
  \RN{Sep-4} & $(p_1 \vee p_2) \sepcon q \Leftrightarrow (p_1 \vee q) \sepcon (p_2 \vee q)$ \\[1mm]
  \RN{Sep-5} & $(p_1 \wedge p_2) \sepcon q \Rightarrow (p_1 \wedge q) \sepcon (p_2 \wedge q)$ \\[1mm]
  \RN{Sep-6} & $(\exists x.\,p) \sepcon q \Leftrightarrow \exists x.\,p \sepcon q$ $\quad$ where $x\not\in\free{q}$ \\[1mm]
  \RN{Sep-7} & $(\exists X.\,p) \sepcon q \Leftrightarrow \exists X.\,p \sepcon q$ $\quad$ where $X\not\in\free{q}$ \\[1mm]
  \RN{Sep-8} & $(\forall x.\,p) \sepcon q \Rightarrow \forall x.\,p \sepcon q$ $\quad$ where $x\not\in\free{q}$ \\[1mm]
  \RN{Sep-9} & $(\forall X.\,p) \sepcon q \Rightarrow \forall X.\,p \sepcon q$ $\quad$ where $X\not\in\free{q}$ \\[1mm]
  \RN{Sep-10} & $\{p_1 \Rightarrow p_2\} \wedge \{q_1 \Rightarrow q_2\} \Rightarrow \{p_1 \sepcon q_1 \Rightarrow p_2 \sepcon q_2\}$ \\[1mm]
  \RN{Sep-11} & $\{p_1 \sepcon p_2 \Rightarrow p_3\} \Leftrightarrow \{p_1 \Rightarrow (p_2 \sepimp p_3)\}$ 
\end{tabular}}


\subsubsection{Rules for total correctness}

{\small
\begin{tabular}{rl}
  \RN{Conseq} & $(\{p \Rightarrow p'\} \wedge \triplex{p'}{e}{x}{q'} \wedge \forall x.\,\{q' \Rightarrow q\}) \Rightarrow \triplex{p}{e}{x}{q}$ \\[1mm]
  \RN{Frame} & $\triplex{p}{e}{x}{q} \Rightarrow \triplex{p \sepcon r}{e}{x}{q \sepcon r}$ \\
  & {\footnotesize where $x \not\in \free{r}$}
\end{tabular}}


\subsubsection{Structural rules}

{\small
\begin{tabular}{rl}
  \RN{Refl-Term} & $\forall x.\,x = x$ \\[1mm]
  \RN{Subst-Term} & $t_1 = t_2 \Rightarrow (p[t_1/x] \Leftrightarrow p[t_2/x])$ \\
  & {\footnotesize where $(p[t_1/x]), (p[t_2/x]) \in \Assn$} \\[1mm]
  \RN{Spec-Term} & $(\forall x.\,p) \Rightarrow p[t/x]$ \\
  & {\footnotesize where $(p[t/x])\in\Assn$} \\[1mm]
\end{tabular}}

\end{document}

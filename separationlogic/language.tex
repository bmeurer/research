\documentclass[12pt,a4paper]{article}

\usepackage{amsmath}
\usepackage{amssymb}
\usepackage{amstext}
\usepackage{array}
\usepackage[american]{babel}
\usepackage{color}
\usepackage{enumerate}
\usepackage[a4paper,%
            colorlinks=false,%
            final,%
            pdfkeywords={},%
            pdftitle={},%
            pdfauthor={Benedikt Meurer},%
            pdfsubject={},%
            pdfdisplaydoctitle=true]{hyperref}
\usepackage[latin1]{inputenc}
\usepackage{latexsym}
\usepackage[final]{listings}
\usepackage{makeidx}
\usepackage[standard,thmmarks]{ntheorem}
\usepackage{stmaryrd}
\usepackage{url}
\usepackage[arrow, matrix, curve]{xy}

%% LaTeX macros
%%
%% macros.tex - Useful LaTeX macros
%%
%% Copyright (c) 2006-2011 Benedikt Meurer <benedikt.meurer@googlemail.com>
%% 


%%
%% Styles
%%

\newcommand{\nstyle}[1]{\ensuremath{\mathsf{#1}}}
\newcommand{\sstyle}[1]{\ensuremath{\mathit{#1}}}


%%
%% Misc
%%

\newcommand{\abort}{\ensuremath{\mathbf{abort}}}
\newcommand{\pto}{\rightharpoonup}
\newcommand{\step}{\ensuremath{\rightsquigarrow}}

\newcommand{\sem}[1]{\ensuremath{[\![#1]\!]}}


%%
%% Names
%%

\newcommand{\arity}{\ensuremath{\mathit{arity}}}
\newcommand{\cl}{\ensuremath{\mathit{cl}}}
\newcommand{\fr}{\ensuremath{\mathit{fr}}}
\newcommand{\free}{\ensuremath{\mathit{free}}}
\newcommand{\graph}{\ensuremath{\mathit{graph}}}
\newcommand{\id}{\ensuremath{\mathit{id}}}


%%
%% Sets
%%

\newcommand{\I}{\ensuremath{\mathcal I}}
\newcommand{\N}{\ensuremath{\mathbb N}}
\renewcommand{\O}{\ensuremath{\mathcal O}}
\newcommand{\Z}{\ensuremath{\mathbb Z}}
\newcommand{\Cl}{\sstyle{Cl}}
\newcommand{\Env}{\sstyle{Env}}
\newcommand{\Exp}{\sstyle{Exp}}
\newcommand{\Frame}{\sstyle{Frame}}
\newcommand{\Id}{\sstyle{Id}}
\newcommand{\Node}{\sstyle{Node}}
\newcommand{\Val}{\sstyle{Val}}


%%
%% Expressions
%%

\newcommand{\app}[2]{{#1}\,{#2}}
\newcommand{\abstr}[2]{\lambda{#1}.\,{#2}}
\newcommand{\ifte}[3]{\mathbf{if}\,{#1}\,\mathbf{then}\,{#2}\,\mathbf{else}\,{#3}}


%%
%% Values
%%

\newcommand{\clov}[2]{\langle{#1},{#2}\rangle}
\newcommand{\false}{\mathbf{false}}
\newcommand{\true}{\mathbf{true}}


%%
%% Grammars
%%

\newenvironment{grammar}{\begin{array}{rrlll}}{\end{array}}

\newcommand{\is}{& ::= &}
\newcommand{\al}{\\ & \mid &}
\newcommand{\nl}{\vspace{2mm}\\}


%%
%% Other environments
%%

\newenvironment{case}{\left\{\!\!\!\begin{array}{ll}}{\end{array}\right.}


%%% Local Variables: 
%%% mode: latex
%%% TeX-master: "compiler"
%%% End: 


% TODO
\newcommand{\CAssn}{\sstyle{CAssn}}
\newcommand{\CTerm}{\sstyle{CTerm}}
\newcommand{\CExp}{\sstyle{CExp}}
\newcommand{\CVal}{\sstyle{CVal}}
\newcommand{\Conf}{\sstyle{Conf}}
\newcommand{\Con}{\sstyle{Con}}
\newcommand{\Sto}{\sstyle{Sto}}
\newcommand{\BType}{\sstyle{BType}}
\newcommand{\scon}{\nstyle{con}}
\newcommand{\sexp}{\nstyle{exp}}
\newcommand{\ssto}{\nstyle{sto}}
\newcommand{\senv}{\nstyle{env}}
\newcommand{\z}{\nstyle{int}}
\newcommand{\DEF}{\sstyle{DEF}}
\newcommand{\id}{\nstyle{id}}
\newcommand{\I}{\mathcal{I}}
\newcommand{\Fix}[1]{\nstyle{Fix}\,(#1)}
\newcommand{\returns}[2]{\ensuremath{\mathbf{returns}\,{#1}.\,{#2}}}

\newcommand{\AUX}{\nstyle{AUX}}
\newcommand{\Neg}{\nstyle{Neg}}
\newcommand{\Or}{\nstyle{Or}}
\newcommand{\Cont}{\nstyle{Cont}}
\newcommand{\Disj}{\nstyle{Disj}}
\newcommand{\Exists}{\nstyle{Exists}}


\begin{document}

\section{The programming language}

We assume that infinite sets of variables $x \in \Var$ and locations $\ell \in \Loc$ are given,
where variables are used to abstract expressions and locations are used to address memory cells.
Expressions are considered equal modulo renaming of bound variables. $\Bool=\{\true,\false\}$ is
the set of boolean constants $b$, $\Int$ is the set of integer constants $z$. We do not distinguish
between integer values their program representation.

\begin{definition}[Expressions]
  The set $\Op$ of {\em operators} $\op$ and the set $\Const$ of {\em constants} $c$ are defined by
  the grammars
  \[\GRbeg
  \op \GRis + \GRmid - \GRmid * \GRmid = \GRmid < \GRmid > \GRmid \le \GRmid \ge \\
  c \GRis z \GRmid b \GRmid \unit \GRmid \cref \GRmid ! \GRmid \assign \GRmid \fix \GRmid \op \GRmid \fst \GRmid \snd,
  \GRend\]
  the set $\Exp$ of {\em expressions} $e$ is defined by
  \[\GRbeg
  e \GRis c \GRmid x \GRmid \ell \GRmid \abstr{x}{e} \GRmid \app{e_1}{e_2}
  \GRmid \ifte{e_0}{e_1}{e_2} \GRmid (e_1,e_2)
  \GRend\]
  and the set $\Val \subseteq \Exp$ of {\em values} $v$ is defined by
  \[\GRbeg
  v \GRis c \GRmid \ell \GRmid \abstr{x}{e} \GRmid \app{\op}{v} \GRmid \app{\assign}{v} \GRmid (v_1,v_2).
  \GRend\]
\end{definition}

$\free{e}$ denotes the set of free variables and $\locns{e}$ denotes the set of locations in the expression $e$.
We say that $e$ is {\em closed} if $\free{e}=\emptyset$. $e$ is called a {\em program} if
$\free{e}\cup\locns{e}=\emptyset$, that is, if $e$ contains neither unbound variables nor locations (i.e.
memory addresses in terms of the underlying machine). We demand that all expressions considered for the
logic later are valid programs (i.e. programmers aren't permitted to access arbitrary memory locations).


\subsection{Operational semantics}

Let $(W,\subseteq) = (\powersetfin{\Loc},\subseteq)$ be a partial order of ``possible worlds''.

\begin{definition}[Store] \
  \begin{enumerate}
  \item A {\em store} is a finite partial mapping $s:\Loc \pto \CVal$. $\Store$ denotes the set of all
    such stores.
  \item For every $L \in W$ a subset $\Store_L = \{s\in\Store\,|\,L\subseteq\dom{s}\}$ is defined.
  \end{enumerate}
\end{definition}

\begin{definition}[Configuration]
  Let $e\in\Exp$ and $s\in\Store$. The pair $(e,s)$ is a {\em configuration} if
  $\locns{e} \subseteq \dom{s}$. $\Conf$ denotes the set of all configurations.
\end{definition}

\begin{figure}[ht]
  \centering
  \begin{tabular}{rl}
    \RN{App-Left} & $\RULE{(e_1,s) \to (e_1',s')}{(\app{e_1}{e_2},s)\to(\app{e_1'}{e_2},s')}$ \\[3mm]
    \RN{App-Right} & $\RULE{(e,s) \to (e',s')}{(\app{v}{e},s)\to(\app{v}{e'},s')}$ \\[3mm]
    \RN{Op} & $(\app{\app{\op}{z_1}}{z_2},s) \to (\op^I(z_1,z_2),s)$ \\[1mm]
    \RN{Beta-V} & $(\app{(\abstr{x}{e})}{v},s) \to (e[v/x],s)$ \\[1mm]
    \RN{Unfold} & $(\app{\fix}{(\abstr{x}{e})},s) \to (e[\app{\fix}{(\abstr{x}{e})}/x],s)$ \\[1mm]
    \RN{Fst} & $(\app{\fst}{(v_1,v_2)},s) \to (v_1,s)$ \\[1mm]
    \RN{Snd} & $(\app{\snd}{(v_1,v_2)},s) \to (v_2,s)$ \\[1mm]
    \RN{Deref} & $(\app{!}{\ell},s) \to (s(\ell),s)$ \\[1mm]
    \RN{Assign} & $(\app{\app{\assign}{\ell}}{v},s) \to (\unit,s[v/\ell])$ \\[1mm]
    \RN{Ref} & $\RULE{\ell \not\in \dom{s}}{(\app{\cref}{v},s) \to (\ell,s[v/\ell])}$ \\[3mm]
    \RN{Cond-Eval} & $\RULE{(e_0,s) \to (e_0',s')}{(\ifte{e_0}{e_1}{e_2},s)\to(\ifte{e_0'}{e_1}{e_2},s')}$ \\[3mm]
    \RN{Cond-True} & $(\ifte{\true}{e_1}{e_2},s) \to (e_1,s)$ \\[1mm]
    \RN{Cond-False} & $(\ifte{\false}{e_1}{e_2},s) \to (e_2,s)$ \\[1mm]
    \RN{Pair-Left} & $\RULE{(e_1,s)\to(e_1',s')}{((e_1,e_2),s) \to ((e_1',e_2),s')}$ \\[3mm]
    \RN{Pair-Right} & $\RULE{(e,s)\to(e',s')}{((v,e),s) \to ((v,e'),s')}$
  \end{tabular}
  \caption{Small step rules}
  \label{fig:Small_step_rules}
\end{figure}

\begin{definition}[Small step relation]
  The {\em small step relation}, written $k \to k'$, is the smallest binary relation between configurations
  satisfying all instances of the rules in figure~\ref{fig:Small_step_rules}.
\end{definition}

Obviously the small step relation is well-defined. Let's say we have $(e,s)\to(e',s')$. Then
\begin{itemize}
\item if $(e,s)\in\Conf$ then $(e',s')\in\Conf$, and
\item if $\free{e'} \subseteq \free{e}$.
\end{itemize}


\subsection{Static semantics}

\begin{definition}[Types for the programming language]
  The set $\Type$ of {\em types} $\tau$ is defined by the grammar
  \[\GRbeg
  \tau \GRis \tbool \GRmid \tbool \GRmid \tunit \GRmid \tref{\tau}
  \GRal \tarrow{\tau_1}{\tau_2} \GRmid \tau_1 \times \tau_2.
  \GRend\]
\end{definition}

\begin{definition}[Typing relation]
  A {\em type environment} is a finite partial mapping $\Gamma:\Var \pto \Type$. $\TEnv$ denotes the
  set of all such type environments. The {\em typing relation}, written $\Tj{\Gamma}{e}{\tau}$, is
  the smallest relation between type environments, expressions and types satisfying all instances
  of the rules in figure~\ref{fig:Typing_rules_for_expressions}. An expression $e$ is {\em well typed} wrt.
  $\Gamma$ if there is some $\tau\in\Type$ such that $\Tj{\Gamma}{e}{\tau}$.
\end{definition}

\begin{figure}[ht]
  \centering
  \begin{tabular}{rl}
    \RN{Const} & $\RULE{\tj{c}{\tau}}{\Tj{\Gamma}{c}{\tau}}$ \\[3mm]
    \RN{Var} & $\RULE{\Gamma(x) = \tau}{\Tj{\Gamma}{x}{\tau}}$ \\[3mm]
    \RN{Loc} & {\bf TODO} \\[1mm]
    \RN{Abstr} & $\RULE{\Tj{\Gamma[\tau/x]}{e}{\tau'}}{\Tj{\Gamma}{\abstr{x}{e}}{\tarrow{\tau}{\tau'}}}$ \\[3mm]
    \RN{App} & $\RULE{\Tj{\Gamma}{e_1}{\tarrow{\tau}{\tau'}} \quad \Tj{\Gamma}{e_2}{\tau}}{\Tj{\Gamma}{\app{e_1}{e_2}}{\tau'}}$ \\[3mm]
    \RN{Cond} & $\RULE{\Tj{\Gamma}{e_0}{\tbool} \quad \Tj{\Gamma}{e_1}{\tau} \quad \Tj{\Gamma}{e_2}{\tau}}{\Tj{\Gamma}{\ifte{e_0}{e_1}{e_2}}{\tau}}$ \\[3mm]
    \RN{Pair} & $\RULE{\Tj{\Gamma}{e_1}{\tau_1} \quad \Tj{\Gamma}{e_2}{\tau_2}}{\Tj{\Gamma}{(e_1,e_2)}{\tau_1 \times \tau_2}}$
  \end{tabular}
  \caption{Typing rules for expressions}
  \label{fig:Typing_rules_for_expressions}
\end{figure}

\end{document}

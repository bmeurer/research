\documentclass[12pt,a2paper,draft]{article}

\usepackage{amssymb}
\usepackage{amsthm}
\usepackage[blocks]{authblk}
\usepackage[english]{babel}
\usepackage{color}
\usepackage[colorlinks=false,%
            pdfkeywords={type safety, big-step, operational, semantics},%
            pdftitle={A new approach to prove type safety using big-step operational semantics},%
            pdfauthor={Benedikt Meurer and Kurt Sieber},%
            pdfsubject={},%
            pdfdisplaydoctitle=true]{hyperref}
\usepackage{mathpartir}
\usepackage{varwidth}

\newcommand{\abstr}[2]{\ensuremath{\lambda{#1}.\,{#2}}}
\newcommand{\app}[2]{\ensuremath{{#1}\,{#2}}}
\renewcommand{\int}{\ensuremath{\textbf{int}}}
\newcommand{\rec}[2]{\ensuremath{\textbf{rec}\,{#1}.\,{#2}}}
\newcommand{\tj}[2]{\mbox{\ensuremath{{#1}:{#2}}}}
\newcommand{\Tj}[3]{\mbox{\ensuremath{{#1}\vdash\tj{#2}{#3}}}}

\newtheorem{lemma}{Lemma}
\newtheorem{theorem}{Theorem}

\begin{document}

\title{%
  A new approach to prove type safety\\using big-step operational semantics
}
\author{Benedikt Meurer}
\author{Kurt Sieber}
\affil{%
  Compilerbau und Softwareanalyse\\
  Universit\"at Siegen\\
  D-57072 Siegen, Germany\\
  {\tt \{meurer,sieber\}@informatik.uni-siegen.de}
}
% \author{%
%   Benedikt Meurer\\
%   \small Compiler Construction and Software Analysis Group\\
%   \small University of Siegen\\
%   \small D-57072 Siegen, Germany\\
%   \small \url{meurer@informatik.uni-siegen.de}
%   \and
%   Kurt Sieber\\
%   \small Compiler Construction and Software Analysis Group\\
%   \small University of Siegen\\
%   \small D-57072 Siegen, Germany\\
%   \small \url{sieber@informatik.uni-siegen.de}
% }
\date{}
\maketitle
\begin{abstract}
  TODO
\end{abstract}


%% Introduction
\section{Introduction}


%% The language and its big-step semantics
\section{The language and its big-step semantics}

This section describes the programming language considered in this paper,
the simply typed $\lambda$-calculus extended with constants and recursion, which
presents the simplest possible functional language that exhibits run-time
errors (closed expressions that ``go wrong''). Figure~\ref{figure:Basic_syntax}
shows the basic syntax of the programming language.

\begin{figure}[htb]
  \centering
  \begin{tabular}{llcl}
    Variables   & \multicolumn{3}{l}{$x,y,z,\ldots$} \\
    Constants   & $c$ & $::=$ & $0 \mid 1 \mid 2 \mid \ldots$ \\
    Expressions & $e$ & $::=$ & $c \mid x \mid \abstr{x}{e} \mid \app{e_1}{e_2} \mid \rec{x}{e}$ \\
    Values      & $v$ & $::=$ & $c \mid \abstr{x}{e}$
  \end{tabular}
  \caption{Basic syntax}
  \label{figure:Basic_syntax}
\end{figure}

We write $e'[e/x]$ for the capture-avoiding substitution of $e$ for all free occurrences
of $x$ in $e'$. The standard call-by-value semantics in big-step style for this language is
defined inductively by the inference rules given in figure~\ref{figure:Call_by_value_inference_rules}.
To be exact, the \emph{evaluation relation} $e \Downarrow v$ is the smallest fixpoint of
the given inference rules. That is, $e \Downarrow v$ holds if and only if it is the conclusion
of a finite derivation tree built from the inference rules.

\begin{figure}[htb]
  \centering
  \begin{mathpar}
    \inferrule[(Val)]{}{v \Downarrow v}
    \and
    \inferrule[(App)]{%
      e_1 \Downarrow \abstr{x}{e} \\
      e_2 \Downarrow v_2 \\
      e[v_2/x] \Downarrow v
    }{%
      \app{e_1}{e_2} \Downarrow v
    }
    \and
    \inferrule[(Rec)]{%
      e[\rec{x}{e}/x] \Downarrow v
    }{%
      \rec{x}{e} \Downarrow v
    }
  \end{mathpar}
  \caption{Call-by-value inference rules}
  \label{figure:Call_by_value_inference_rules}
\end{figure}

\begin{figure}[htb]
  \centering
  \begin{mathpar}
    \inferrule[(App-1)]{}{\app{e_1}{e_2} \Rightarrow e_1}
    \and
    \inferrule[(App-2)]{%
      e_1 \Downarrow \abstr{x}{e}
    }{%
      \app{e_1}{e_2} \Rightarrow e_2
    }
    \and
    \inferrule[(App-3)]{%
      e_1 \Downarrow \abstr{x}{e} \\
      e_2 \Downarrow v_2
    }{%
      \app{e_1}{e_2} \Rightarrow e[v_2/x]
    }
    \and
    \inferrule[(Rec-1)]{}{\rec{x}{e} \Rightarrow e[\rec{x}{e}/x]}
  \end{mathpar}
  \caption{Call-by-value prerequisite rules}
  \label{figure:Call_by_value_prerequisite_rules}
\end{figure}


%% Type safety
\section{Type safety}

\begin{figure}[htb]
  \centering
  \begin{mathpar}
    \inferrule[(Const)]{}{\Tj{\Gamma}{c}{\int}}
    \and
    \inferrule[(Var)]{%
      \Gamma(x) = \tau
    }{%
      \Tj{\Gamma}{x}{\tau}
    }
    \and
    \inferrule[(Abstr)]{%
      \Tj{\Gamma[\tau/x]}{e}{\tau'}
    }{%
      \Tj{\Gamma}{\abstr{x}{e}}{\tau \to \tau'}
    }
    \\
    \inferrule[(Rec)]{%
      \Tj{\Gamma[\tau/x]}{e}{\tau}
    }{%
      \Tj{\Gamma}{\rec{x}{e}}{\tau}
    }
    \and
    \inferrule[(App)]{%
      \Tj{\Gamma}{e_1}{\tau' \to \tau} \\
      \Tj{\Gamma}{e_2}{\tau'}
    }{%
      \Tj{\Gamma}{\app{e_1}{e_2}}{\tau}
    }
  \end{mathpar}
  \caption{Typing rules}
  \label{figure:Typing_rules}
\end{figure}

\begin{lemma}[Preservation of types under substitution]
  \label{lemma:Preservation_of_types_under_substitution}
  If $\Tj{\Gamma[\tau/x]}{e'}{\tau'}$ and $\Tj{\Gamma}{e}{\tau}$,
  then $\Tj{\Gamma}{e'[e/x]}{\tau'}$.
\end{lemma}

\begin{theorem}[Preservation]
  \label{theorem:Preservation}
  Let $\Tj{\Gamma}{e}{\tau}$.
  \begin{enumerate}
  \item If $e \Downarrow v$, then $\Tj{\Gamma}{v}{\tau}$.
  \item If $e \Rightarrow e'$, then there is some $\tau'$ such that $\Tj{\Gamma}{e'}{\tau'}$
  \end{enumerate}
\end{theorem}

\begin{proof}
  The proof is by simultaneous induction on the derivations of $e \Downarrow v$ and $e \Rightarrow e'$.
  The cases for $e \Downarrow v$ are standard and therefore omitted.
  $e \Rightarrow e'$ implies that $e$ is either an application $\app{e_1}{e_2}$ or
  a recursion $\rec{x}{e_1}$.

  If $e = \app{e_1}{e_2}$, then we have $\Tj{\Gamma}{\app{e_1}{e_2}}{\tau}$, which
  must have been derived from $\Tj{\Gamma}{e_1}{\tau_2 \to \tau}$ and $\Tj{\Gamma}{e_2}{\tau_2}$
  using \textsc{(App)}. We consider only the interesting case of \textsc{(App-3)}. Then we have
  $e_1 \Downarrow \abstr{x}{e_1'}$ and $e_2 \Downarrow v_2$. By
  induction hypothesis we get $\Tj{\Gamma}{\abstr{x}{e_1'}}{\tau_2 \to \tau}$, derived from
  $\Tj{\Gamma[\tau_2/x]}{e_1'}{\tau}$ using \textsc{(Abstr)}, and $\Tj{\Gamma}{v_2}{\tau_2}$.
  We conclude using Lemma~\ref{lemma:Preservation_of_types_under_substitution}.

  If $e = \rec{x}{e_1}$, then we have $\Tj{\Gamma}{\rec{x}{e_1}}{\tau}$, which must have
  been derived from $\Tj{\Gamma[\tau/x]}{e_1}{\tau}$ using \textsc{(Rec)}.
  We conclude using Lemma~\ref{lemma:Preservation_of_types_under_substitution}.
\end{proof}

\begin{theorem}[Progress]
  \label{theorem:Progress}
  If $\tj{e}{\tau}$, then there is either a $v$ such that $e \Downarrow v$, or an $e'$ such that
  $e \Rightarrow e'$.
\end{theorem}

\begin{proof}
  For values $e = v$ we have $v \Downarrow v$ using \textsc{(Val)}, for applications
  $e = \app{e_1}{e_2}$ we have $\app{e_1}{e_2} \Rightarrow e_1$ using \textsc{(App-1)},
  and for recursions $e = \rec{x}{e'}$ we have $\rec{x}{e'} \Rightarrow e'[\rec{x}{e'}/x]$
  using \textsc{(Rec-1)}.
\end{proof}

\begin{theorem}[Type safety]
  If $\tj{e}{\tau}$, then there is no $e'$ with
  $e' \not\Downarrow$ and $e' \not\Rightarrow$,
  such that $e \stackrel{*}{\Rightarrow} e'$.
\end{theorem}

\begin{proof}
  Consider $e \stackrel{*}{\Rightarrow} e'$. According to theorem~\ref{theorem:Preservation} there
  exists some $\tau'$ such that $\tj{e'}{\tau'}$, and according to theorem~\ref{theorem:Progress}
  there is either a $v$ such that $e' \Downarrow v$ or an $e''$ such that $e' \Rightarrow e''$.
\end{proof}


%% Related work
\section{Related work}


%% Conclusion
\section{Conclusions}


%% References
\bibliographystyle{abbrv}
\bibliography{citations}


\end{document}
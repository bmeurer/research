\documentclass[12pt,a2paper,draft]{article}

\usepackage{amssymb}
\usepackage{amsthm}
\usepackage[blocks]{authblk}
\usepackage[english]{babel}
\usepackage{color}
\usepackage[colorlinks=false,%
            pdfkeywords={type safety, big-step, operational, semantics},%
            pdftitle={A new approach to prove type safety using big-step operational semantics},%
            pdfauthor={Benedikt Meurer and Kurt Sieber},%
            pdfsubject={},%
            pdfdisplaydoctitle=true]{hyperref}
\usepackage{mathpartir}
\usepackage{varwidth}

\newcommand{\abstr}[2]{\ensuremath{\lambda{#1}.\,{#2}}}
\newcommand{\app}[2]{\ensuremath{{#1}\,{#2}}}
\newcommand{\rec}[2]{\ensuremath{{\normalfont\textsf{rec}}\,{#1}.\,{#2}}}
\newcommand{\unit}{\ensuremath{\normalfont\textsf{unit}}}
\newcommand{\Unit}{\ensuremath{\normalfont\textsf{Unit}}}

\newcommand{\Tj}[3]{\mbox{\ensuremath{{#1}\vdash{#2}:{#3}}}}
\newcommand{\tj}[2]{\Tj{\emptyset}{#1}{#2}}
\newcommand{\Pj}[3]{\mbox{\ensuremath{{#1}\vdash{#2}\rightarrow{#3}}}}
\newcommand{\tree}[1]{\mathcal{D}(#1)}
\newcommand{\rn}[1]{\mbox{\ensuremath{\textsc{(#1)}}}}

\newtheorem{lemma}{Lemma}
\newtheorem{theorem}{Theorem}
\newtheorem{definition}{Definition}

\begin{document}

\title{%
  A new approach to prove type safety\\using big-step operational semantics
}
\author{Benedikt Meurer}
\author{Kurt Sieber}
\affil{%
  Compilerbau und Softwareanalyse\\
  Universit\"at Siegen\\
  D-57072 Siegen, Germany\\
  {\tt \{meurer,sieber\}@informatik.uni-siegen.de}
}
\date{}
\maketitle
\begin{abstract}
  TODO
\end{abstract}


%% Introduction
\section{Introduction}

Two widely-used styles of \emph{operational semantics} are available today: \emph{big step semantics} \cite{Kahn87},
also known as \emph{natural semantics}, relates program expressions to the final results of their
evaluation; \emph{small step semantics} \cite{Plotkin81,Plotkin04}, also known as
\emph{structured operational semantics}, repeatedly applies a one-step transition relation to
form reduction sequences.


%% The language and its big-step semantics
\section{The language and its big-step semantics}

This section describes the programming language considered in this paper,
the simply typed $\lambda$-calculus with $\unit$ and general recursion, which
presents the simplest possible functional language that exhibits run-time
errors (closed expressions that ``go wrong'') and divergence. Figure~\ref{fig:Basic_syntax}
shows the basic syntax of the programming language.

\begin{figure}[htb]
  \centering
  \begin{tabular}{llcl}
    Variables   & \multicolumn{3}{l}{$x,y,z,\ldots$} \\
    Expressions & $e$ & $::=$ & $\unit \mid x \mid \abstr{x}{e} \mid \app{e_1}{e_2} \mid \rec{x}{e}$ \\
    Values      & $v$ & $::=$ & $\unit \mid \abstr{x}{e}$
  \end{tabular}
  \caption{Basic syntax}
  \label{fig:Basic_syntax}
\end{figure}

We write $e'[x \mapsto e]$ for the capture-avoiding substitution of $e$ for all free occurrences
of $x$ in $e'$. The standard call-by-value semantics in big-step style for this language is
defined inductively by the inference rules given in figure~\ref{fig:Call_by_value_inference_rules}.
To be exact, the \emph{evaluation relation} $e \Rightarrow v$ is the smallest fixpoint of
the given inference rules. That is, $e \Rightarrow v$ holds if and only if it is the conclusion
of a finite derivation tree built from the inference rules.

\begin{figure}[htb]
  \centering
  \begin{mathpar}
    \inferrule[(Val)]{}{v \Rightarrow v}
    \and
    \inferrule[(App)]{%
      e_1 \Rightarrow \abstr{x}{e} \\
      e_2 \Rightarrow v \\
      e[x \mapsto v] \Rightarrow v'
    }{%
      \app{e_1}{e_2} \Rightarrow v'
    }
    \and
    \inferrule[(Rec)]{%
      e[x \mapsto \rec{x}{e}] \Rightarrow v
    }{%
      \rec{x}{e} \Rightarrow v
    }
  \end{mathpar}
  \caption{Call-by-value inference rules}
  \label{fig:Call_by_value_inference_rules}
\end{figure}

\begin{figure}[htb]
  \centering
  \begin{mathpar}
    \inferrule[(App-1)]{}{\app{e_1}{e_2} \leadsto e_1}
    \and
    \inferrule[(App-2)]{%
      e_1 \Rightarrow \abstr{x}{e}
    }{%
      \app{e_1}{e_2} \leadsto e_2
    }
    \and
    \inferrule[(App-3)]{%
      e_1 \Rightarrow \abstr{x}{e} \\
      e_2 \Rightarrow v
    }{%
      \app{e_1}{e_2} \leadsto e[x \mapsto v]
    }
    \and
    \inferrule[(Rec-1)]{}{\rec{x}{e} \leadsto e[x \mapsto \rec{x}{e}]}
  \end{mathpar}
  \caption{Call-by-value prerequisite rules}
  \label{fig:Call_by_value_prerequisite_rules}
\end{figure}

\begin{lemma} \label{lemma:Finite_prerequisites}
  If $e \leadsto e'$ and $e \not\stackrel{\infty}{\leadsto}$, then
  $e' \not\stackrel{\infty}{\leadsto}$.
\end{lemma}


%% Type safety
\section{Type safety}

\begin{figure}[htb]
  \centering
  \begin{mathpar}
    \inferrule[(T-Unit)]{}{\Tj{\Gamma}{\unit}{\Unit}}
    \and
    \inferrule[(T-Var)]{%
      \Gamma(x) = \tau
    }{%
      \Tj{\Gamma}{x}{\tau}
    }
    \and
    \inferrule[(T-Abstr)]{%
      \Tj{\Gamma + \{x : \tau \}}{e}{\tau'}
    }{%
      \Tj{\Gamma}{\abstr{x}{e}}{\tau \to \tau'}
    }
    \\
    \inferrule[(T-Rec)]{%
      \Tj{\Gamma + \{x:\tau\}}{e}{\tau}
    }{%
      \Tj{\Gamma}{\rec{x}{e}}{\tau}
    }
    \and
    \inferrule[(T-App)]{%
      \Tj{\Gamma}{e_1}{\tau' \to \tau} \\
      \Tj{\Gamma}{e_2}{\tau'}
    }{%
      \Tj{\Gamma}{\app{e_1}{e_2}}{\tau}
    }
  \end{mathpar}
  \caption{Typing rules}
  \label{figure:Typing_rules}
\end{figure}

\begin{lemma}[Preservation of types under substitution] \label{lemma:Preservation_of_types_under_substitution}
  If $\Tj{\Gamma + \{x:\tau\}}{e'}{\tau'}$ and $\Tj{\Gamma}{e}{\tau}$,
  then $\Tj{\Gamma}{e'[x \mapsto e]}{\tau'}$.
\end{lemma}

\begin{lemma}[Canonical Forms] \label{lemma:Canonical_Forms} \
  \begin{enumerate}
  \item If $\tj{v}{\Unit}$, then $v = \unit$.
  \item If $\tj{v}{\tau' \to \tau}$, then $v = \abstr{x}{e}$.
  \end{enumerate}
\end{lemma}

\begin{theorem}[Termination]
  If $\tj{e}{\tau}$ and $e \not\stackrel{\infty}{\leadsto}$, then there is some
  $v$ such that $e \Rightarrow v$ and $\tj{v}{\tau}$.
\end{theorem}

\begin{proof}
  Straightforward induction on typing derivations. The \rn{T-Unit} and \rn{T-Abstr}
  cases are immediate, since $e$ is already a value. The \rn{T-Var} case cannot occur
  because $e$ is closed.

  In case of \rn{T-App} we have $\tj{\app{e_1}{e_2}}{\tau}$, $\tj{e_1}{\tau' \to \tau}$
  and $\tj{e_2}{\tau'}$. By \rn{App-1}, lemma~\ref{lemma:Finite_prerequisites} and
  induction hypothesis, there is some $v_1$ such that $e_1 \Rightarrow v_1$ and
  $\tj{v_1}{\tau' \to \tau}$, which implies $v_1 = \abstr{x}{e}$ according to
  lemma~\ref{lemma:Canonical_Forms}. Similarly we get $e_2 \Rightarrow v$ and
  $\tj{v}{\tau'}$ using \rn{App-2}. By lemma~\ref{lemma:Preservation_of_types_under_substitution}
  we also have $\tj{e[x \mapsto v]}{\tau}$, yielding $e[x \mapsto v] \Rightarrow v'$
  and $\tj{v'}{\tau}$ using \rn{App-3}, lemma~\ref{lemma:Finite_prerequisites}
  and the induction hypothesis. We conclude $\app{e_1}{e_2} \Rightarrow v'$ using
  inference rule \rn{App}.

  In case of \rn{T-Rec} we have $\tj{\rec{x}{e}}{\tau}$ and $\Tj{\{x:\tau\}}{e}{\tau}$,
  which yields $\tj{e[x \mapsto \rec{x}{e}]}{\tau}$ using
  lemma~\ref{lemma:Preservation_of_types_under_substitution}. By \rn{Rec-1},
  lemma~\ref{lemma:Finite_prerequisites} and induction hypothesis, there is some
  $v$ such that $e[x \mapsto \rec{x}{e}] \Rightarrow v$ and $\tj{v}{\tau}$, so
  there is also a big-step $\rec{x}{e} \Rightarrow v$ using \rn{Rec}.
\end{proof}

\begin{theorem}[Type safety]
  If $\tj{e}{\tau}$, then $e \Rightarrow v$ or $e \stackrel{\infty}{\leadsto}$.
\end{theorem}


%% Related work
\section{Related work}

So far, three different big-step semantics have been proposed for the $\lambda$-calculus to proof
type-safety of a programming language:

The standard approach \cite{Tofte87} for proving type safety using big-step semantics is to use
additional inductive rules to define a predicate $e \Rightarrow \textsf{err}$ characterizing expressions that
``go wrong'', and prove $\neg(e \Rightarrow \textsf{err})$ for all closed,
well-typed expressions $e$. This formulation is both easy to understand and easy to prove, but
it has two important drawbacks compared to our approach:
(1) it requires extra rules to be provided to define $e \Rightarrow \textsf{err}$,
which increases the size of the semantics, and (2) there is a risk that the set of rules for
$e \Rightarrow \textsf{err}$ is incomplete and misses some expressions that ``go wrong'',
in which case the type safety theorem no longer guarantees that closed, well-typed expressions
either evaluate to a value or diverge.

The second approach \cite{CousotCousot07,LeroyGrall09} complements the normal inductive big-step
inference rules for finite evaluations $e \Rightarrow v$ with coinductive big-step rules describing
diverging evaluations $e \stackrel{\infty}{\Rightarrow}$. Type safety is proved by showing
$e \stackrel{\infty}{\Rightarrow}$ for all closed, well-typed expressions $e$ that do not terminate
with a value. As with the standard approach the size of the semantics is increased, but the risk of
putting too few or too many divergence rules is low, hence this approach seems rather robust with
respect to mistakes in the specification of the semantics. Compared to our approach, it is less
intuitive and requires knowledge of coinductive techniques.

The third interprets coinductively the normal big-step inference rules, leading to a \emph{coevaluation
relation} $e \stackrel{\mathsf{co}}{\Rightarrow} v$, thus enabling them to describe both terminating
and non-terminating evaluations \cite{LeroyGrall09}.


%% Conclusion
\section{Conclusions}


%% References
\bibliographystyle{abbrv}
\bibliography{citations}


\end{document}
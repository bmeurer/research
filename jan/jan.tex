\documentclass[12pt,a4paper]{article}

\usepackage{amsmath}
\usepackage{amssymb}
\usepackage{amstext}
\usepackage{array}
\usepackage[american]{babel}
\usepackage{color}
\usepackage{enumerate}
\usepackage[a4paper,%
            colorlinks=false,%
            final,%
            pdfkeywords={},%
            pdftitle={},%
            pdfauthor={Benedikt Meurer},%
            pdfsubject={},%
            pdfdisplaydoctitle=true]{hyperref}
\usepackage{ifthen}
\usepackage[latin1]{inputenc}
\usepackage{latexsym}
\usepackage[final]{listings}
\usepackage{makeidx}
\usepackage{ngerman}
\usepackage[standard,thmmarks]{ntheorem}
\usepackage{stmaryrd}

%% LaTeX macros
%%
%% macros.tex - Useful LaTeX macros
%%
%% Copyright (c) 2006-2011 Benedikt Meurer <benedikt.meurer@googlemail.com>
%% 


%%
%% Styles
%%

\newcommand{\nstyle}[1]{\ensuremath{\mathsf{#1}}}
\newcommand{\sstyle}[1]{\ensuremath{\mathit{#1}}}


%%
%% Misc
%%

\newcommand{\abort}{\ensuremath{\mathbf{abort}}}
\newcommand{\pto}{\rightharpoonup}
\newcommand{\step}{\ensuremath{\rightsquigarrow}}

\newcommand{\sem}[1]{\ensuremath{[\![#1]\!]}}


%%
%% Names
%%

\newcommand{\arity}{\ensuremath{\mathit{arity}}}
\newcommand{\cl}{\ensuremath{\mathit{cl}}}
\newcommand{\fr}{\ensuremath{\mathit{fr}}}
\newcommand{\free}{\ensuremath{\mathit{free}}}
\newcommand{\graph}{\ensuremath{\mathit{graph}}}
\newcommand{\id}{\ensuremath{\mathit{id}}}


%%
%% Sets
%%

\newcommand{\I}{\ensuremath{\mathcal I}}
\newcommand{\N}{\ensuremath{\mathbb N}}
\renewcommand{\O}{\ensuremath{\mathcal O}}
\newcommand{\Z}{\ensuremath{\mathbb Z}}
\newcommand{\Cl}{\sstyle{Cl}}
\newcommand{\Env}{\sstyle{Env}}
\newcommand{\Exp}{\sstyle{Exp}}
\newcommand{\Frame}{\sstyle{Frame}}
\newcommand{\Id}{\sstyle{Id}}
\newcommand{\Node}{\sstyle{Node}}
\newcommand{\Val}{\sstyle{Val}}


%%
%% Expressions
%%

\newcommand{\app}[2]{{#1}\,{#2}}
\newcommand{\abstr}[2]{\lambda{#1}.\,{#2}}
\newcommand{\ifte}[3]{\mathbf{if}\,{#1}\,\mathbf{then}\,{#2}\,\mathbf{else}\,{#3}}


%%
%% Values
%%

\newcommand{\clov}[2]{\langle{#1},{#2}\rangle}
\newcommand{\false}{\mathbf{false}}
\newcommand{\true}{\mathbf{true}}


%%
%% Grammars
%%

\newenvironment{grammar}{\begin{array}{rrlll}}{\end{array}}

\newcommand{\is}{& ::= &}
\newcommand{\al}{\\ & \mid &}
\newcommand{\nl}{\vspace{2mm}\\}


%%
%% Other environments
%%

\newenvironment{case}{\left\{\!\!\!\begin{array}{ll}}{\end{array}\right.}


%%% Local Variables: 
%%% mode: latex
%%% TeX-master: "compiler"
%%% End: 


\begin{document}


\section{Kalk"ul}

\begin{definition}[Kalk"ul]
  Ein \emph{Kalk"ul} $\mathcal{K} = (\mathcal{U}, \mathcal{R})$ besteht aus
  \begin{enumerate}
  \item einer Formelmenge $\mathcal{U}$, dem \emph{Universum}, und
  \item einer entscheidbaren Relation $\mathcal{R} \subseteq \mathcal{U}^+$, der \emph{Ableitungsrelation}.
  \end{enumerate}
\end{definition}
Zu jedem Kalk"ul $\mathcal{K}$ l"asst sich eine Funktion
$\mathcal{F}:\powerset{\mathcal{U}} \to \powerset{\mathcal{U}}$ definieren durch
\[\begin{array}{rcl}
\mathcal{F}(X) &=& \{u \in \mathcal{U} \mid \exists n\in\N,u_1,\ldots,u_n\in X.\,(u_1 \ldots u_n\,u) \in\mathcal{R}\}.
\end{array}\]
$\mathcal{F}$ ist stetig auf dem vollst"andigen Verband $\powerset{\mathcal{U}}$, d.h. es
existiert
\begin{itemize}
\item ein kleinster Fixpunkt $\mu\mathcal{F} = \bigcup_{n\in\N}\mathcal{F}^n(\emptyset)$, und
\item ein gr"o"ster Fixpunkt $\nu\mathcal{F} = \bigcap_{n\in\N}\mathcal{F}^n(\mathcal{U})$.
\end{itemize}
Ein \emph{$\mathcal{K}$-Ableitungsbaum} f"ur eine Formel $u \in \mathcal{U}$ ist (endl. oder unendl.)
$\mathcal{U}$-markierter Baum $t$, so dass:
\begin{enumerate}
\item $t(\varepsilon) = u$, und
\item $(t(k.1),\ldots,t(k.n),t(k)) \in \mathcal{R}$ f"ur jeden Knoten $k$ mit den Kindern $k.1,\ldots,k.n$.
\end{enumerate}

\begin{theorem} \
  \begin{enumerate}
  \item $u$ besitzt einen endlichen Ableitungsbaum gdw. $u \in \mu\mathcal{F}$.
  \item $u$ besitzt einen beliebigen Ableitungsbaum gdw. $u \in \nu\mathcal{F}$.
  \end{enumerate}
\end{theorem}

\begin{proof} \
  \begin{enumerate}
  \item $u \in \mathcal{F}^n(\emptyset)$ gdw. $u$ besitzt Ableitugnsbaum der H"ohe $\le n$.
  \item 
    \begin{itemize}
    \item[``$\Rightarrow$'']
      Sei $X$ die Menge der Markierungen, dann ist $u \in X$ und f"ur jedes $w \in X$ ex.
      $w_1,\ldots,w_n \in X$ mit $(w_1 \ldots w_n\,w) \in \mathcal{R}$, also
      $w \in \mathcal{F}(X)$. D.h. $u \in X$ und $X$ ist $\mathcal{F}$-konsistent, also
      $u \in X \subseteq \nu \mathcal{F}$.
    \item[``$\Leftarrow$'']
      Der Ableitungsbaum $t$ ist eine partielle Funktion $t: \N^* \pto \nu\mathcal{F}$ induktiv definiert
      durch:
      \begin{itemize}
      \item $(\varepsilon,u) \in t$
      \item Wenn $(k,w) \in t$, dann ex. $n\in\N,w_1,\ldots,w_n \in \nu\mathcal{F}$ --
        denn $w \in \nu\mathcal{F} = \mathcal{F}(\nu\mathcal{F})$ -- mit
        $(w_1 \ldots w_n\,w) \in \mathcal{R}$ \\
        ind. def. $\leadsto$ $(k.i,w_i) \in t$ f"ur $i=1,\ldots,n$.
      \end{itemize}
    \end{itemize}
  \end{enumerate}
\end{proof}


\section{Rekursion}

\begin{definition}[Rekursionsschema]
  Seien $A,B$ disjunkte Mengen. Ein \emph{$(A,B)$-Rekursionsschema} ist eine Relation
  $\sigma \subseteq A \times B^* \times (A \uplus B)$. $\Sigma^A_B$ bezeichnet die Menge
  aller $(A,B)$-Rekursionsschemata.
\end{definition}

\begin{definition}[Rekursion]
  Sei $\sigma \in \Sigma^A_B$. Die durch $\sigma$ definierte \emph{Rekursion}
  $\Rightarrow_\sigma\ \subseteq A \times B$ ist induktiv definiert durch:
  \begin{enumerate}
  \item Wenn $(a,\varepsilon, b) \in \sigma$, dann $a \Rightarrow_\sigma b$.
  \item Wenn $n \ge 1$, $a_i \Rightarrow_\sigma b_i$ und $(a,b_1 \ldots b_{i-1},a_i) \in \sigma$ f"ur $i=1,\ldots,n$,
    und $(a,b_1 \ldots b_n,b)\in\sigma$, dann $a \Rightarrow_\sigma b$.
  \end{enumerate}
\end{definition}
Diese induktive Definition von $\Rightarrow_\sigma$ l"asst sich auch k"urzer notieren:
\begin{quote}
  F"ur alle $n \in \N$, $a, a_1,\ldots,a_n \in A$ und $b,b_1,\ldots,b_n\in B$ gilt: 
  Wenn $a_i \Rightarrow_\sigma b_i$ und $(a,b_1 \ldots b_{i-1},a_i)\in\sigma$ f"ur $i = 1,\ldots,n$ und
  $(a,b_1 \ldots b_n,b)\in\sigma$, dann $a \Rightarrow_\sigma b$.
\end{quote}
Wie man leicht sieht, ist $\Rightarrow_\sigma$ eine Funktion wenn $\sigma$ eine Funktion ist.

\begin{definition}[Co-Rekursion]
  Sei $\sigma \in \Sigma^A_B$. Die durch $\sigma$ definierte \emph{Co-Rekursion}
  $\stackrel{\infty}{\Rightarrow}_\sigma\ \subseteq A$ ist co-induktiv definiert durch:
  \begin{quote}
    Wenn $n \in \N$, $a, a_1,\ldots,a_n,a_{n+1} \in A$, $b_1,\ldots,b_n\in B$,
    $a_i \Rightarrow_\sigma b_i$ und $(a,b_1 \ldots b_{i-1},a_i)\in\sigma$ f"ur $i=1,\ldots,n$,
    und $a_{n+1} \stackrel{\infty}{\Rightarrow}_\sigma$ und $(a,b_1 \ldots b_n,a_{n+1})\in\sigma$,
    dann $a \stackrel{\infty}{\Rightarrow}_\sigma$.
  \end{quote}
\end{definition}
Statt $\Rightarrow_\sigma$ und $\stackrel{\infty}{\Rightarrow}_\sigma$ schreiben wir kurz $\Rightarrow$
und $\stackrel{\infty}{\Rightarrow}$ wenn $\sigma$ aus dem Kontext ersichtlich ist.


\section{Iteration}

\begin{definition}[Iteration]
  Sei $\sigma: A \Rightarrow B$ ein Rekursionsschema. Die durch $\sigma$ definierte \emph{Iteration}
  $\to_\sigma\ \in (A \uplus B)^* \times (A \uplus B)^*$ ist induktiv definiert durch:
  \begin{enumerate}
  \item Wenn $(a,b_1 \ldots b_n,b) \in \sigma$,
    dann $w\,a\,b_1 \ldots b_n \to_\sigma w\,b$.
  \item Wenn $(a,b_1 \ldots b_n,a') \in \sigma$,
    dann $w\,a\,b_1 \ldots b_n \to_\sigma w\,a\,b_1 \ldots b_n\,a'$.
  \end{enumerate}
\end{definition}
Wir schreiben $\to$ statt $\to_\sigma$ wenn $\sigma$ aus dem Kontext ersichtlich ist.
\\[5mm]
\begin{tabular}{ll}
  $\Rule{w \to v \quad v \stackrel{\infty}{\to}}{w \stackrel{\infty}{\to}}$ 
\end{tabular} \\[5mm]
$\stackrel{*}{\to}$ ist die reflexive, transitive H"ulle von $\to$. $w \stackrel{\infty}{\to}$
beschreibt unendliche Folgen von Reduktionen.

\begin{lemma} \label{lemma:Redex_Star_und_Infty}
  Wenn $v \stackrel{*}{\rightarrow} w$ und $w \stackrel{\infty}{\rightarrow}$,
  dann $v \stackrel{\infty}{\rightarrow}$.
\end{lemma}

\begin{theorem}[Vollst"andigkeit] \label{theorem:Vollstaendigkeit}
  F"ur alle $n \in \N$, $a,a_1,\ldots,a_n \in A$, $b,b_1,\ldots,b_n \in B$ und $w \in (A \uplus B)^*$ gilt:
  \begin{enumerate}
  \item Wenn $a \Rightarrow b$, dann gilt $w\,a \stackrel{*}{\to} w\,b$.
  \item Wenn $a_i \Rightarrow b_i$ und $(a,b_1 \ldots b_{i-1},a_i) \in \sigma$ f"ur $i=1,\ldots,n$, dann gilt
    $w\,a \stackrel{*}{\to} w\,a\,b_1 \ldots b_n$.
  \end{enumerate}
\end{theorem}

\begin{proof}
  Per Induktion "uber die Herleitung von $a \Rightarrow b$ und $n$.
  \begin{enumerate}
  \item Es existieren $n \in \N$, $a_1,\ldots,a_n \in A$ und $b_1,\ldots,b_n \in B$ mit
    $a_i \Rightarrow b_i$ und $(a,b_1 \ldots b_{i-1},a_i) \in \sigma$ f"ur $i = 1,\ldots,n$,
    und es gilt $(a,b_1 \ldots b_n,b) \in \sigma$. Also existiert eine Berechnung der Form
    $w\,a \stackrel{*}{\to} w\,a\,b_1 \ldots b_n$, und wegen $(a,b_1 \ldots b_n,b) \in \sigma$
    folgt $w\,a \stackrel{*}{\to} w\,a\,b_1 \ldots b_n \to w\,a\,b$.

  \item F"ur $n = 0$ folgt unmittelbar $w\,a \stackrel{*}{\to} w\,a$.

    F"ur $n > 0$ gilt $w\,a \stackrel{*}{\to} w\,a\,b_1 \ldots b_{n-1}$ nach Induktionsvoraussetzung und wegen
    $(a,b_1 \ldots b_{n-1},a_n)\in\sigma$ folgt $w\,a\,b_1 \ldots b_{n-1} \to w\,a\,b_1 \ldots b_{n-1}\,a_n$.
    Nach Induktionsvoraussetzung folgt
    $w\,a\,b_1 \ldots b_{n-1}\,a_n \stackrel{*}{\to} w\,a\,b_1 \ldots b_n$ wegen $a_n \Rightarrow b_n$.
    Also gilt insgesamt $w\,a \stackrel{*}{\to} w\,a\,b_1 \ldots b_n$.
  \end{enumerate}
\end{proof}

\begin{theorem}[Korrektheit] \label{theorem:Korrektheit}
  F"ur alle $n,l \in N$, $\hat{a} \in A$, $b, b_1,\ldots,b_n \in B$ und
  $w \in (A \uplus B)^*$ gilt:
  \begin{enumerate}
  \item Wenn $\hat{a} \stackrel{l}{\to} w\,b$, dann existieren $k < l$ und $a \in A$ mit
    $a \Rightarrow b$ und $\hat{a} \stackrel{k}{\to} w\,a$.
  \item Wenn $\hat{a} \stackrel{l}{\to} w\,a\,b_1 \ldots b_n$, dann existieren $k \le l$
    und $a_1,\ldots,a_n \in A$ mit
    $a_i \Rightarrow b_i$ und $(a,b_1 \ldots b_{i-1},a_i) \in \sigma$ f"ur $i=1,\ldots,n$, und
    $\hat{a} \stackrel{k}{\to} w\,a$.
  \end{enumerate}
\end{theorem}

\begin{proof}
  Per Induktion "uber $l$.
  \begin{enumerate}
  \item Der letzte Schritt kann nur mit der 1. Regel folgen, es existieren also $n\in\N$, $a \in A$
    und $b_1,\ldots,b_n \in B$, so dass gilt $(a,b_1 \ldots b_n,b) \in \sigma$ und
    $\hat{a} \stackrel{l-1}{\to} w\,a\,b_1 \ldots b_n \to w\,b$. Nach Induktionsvoraussetzung existieren
    nun $k < l-1$ und $a_1,\ldots,a_n \in A$ mit $a_i \Rightarrow b_i$ und $(a,b_1 \ldots b_{i-1},a_i)\in\sigma$
    f"ur $i=1,\ldots,n$ und es gilt $\hat{a} \stackrel{k}{\to} w\,a$. Also existiert insgesamt eine
    Herleitung f"ur $a \Rightarrow b$.
  \item Trivial im Fall von $n = 0$.

    F"ur $n > 0$ muss der letzte Schritt mit der 1. Regel hergeleitet worden sein, d.h. es existieren
    $m \in \N$, $a_n \in A$, $b_1',\ldots,b_m' \in B$ mit $(a_n,b_1' \ldots b_m',b_n) \in \sigma$ und
    $\hat{a} \stackrel{l-1}{\to} w\,a\,b_1 \ldots b_{n-1}\,a_n\,b_1'\ldots b_m' \to w\,a\,b_1 \ldots b_n$.

    Nach Induktionsvoraussetzung existeren $k' \le l-1$ und $a_1',\ldots,a_m' \in A$ mit
    $a_i' \Rightarrow b_i'$ und $(a_n,b_1' \ldots b_{i-1}',a_i') \in \sigma$ f"ur $i=1,\ldots,m$
    und es gilt $\hat{a} \stackrel{k'}{\to} w\,a\,b_1 \ldots b_{n-1}\,a_n$. Also folgt sofort
    $a_n \Rightarrow b_n$.

    Der letzte Schritt kann nur mit der 2. Regel hergeleitet worden sein, d.h. es gilt
    $(a,b_1 \ldots b_{n-1},a_n) \in \sigma$ und $\hat{a} \stackrel{k'-1}{\to} w\,a\,b_1 \ldots b_{n-1}$.
    Nach Induktionsvoraussetzung existeren $k \le k' < l$ und $a_1,\ldots,a_{n-1} \in A$ mit
    $a_i \Rightarrow b_i$ und $(a,b_1 \ldots b_{i-1},a_i) \in \sigma$ f"ur alle $i=1,\ldots,n-1$, und
    es gilt $\hat{a} \stackrel{k}{\to} w\,a$.
  \end{enumerate}
\end{proof}

\begin{theorem}
  Wenn $a \stackrel{\infty}{\Rightarrow}$, dann $w\,a \stackrel{\infty}{\rightarrow}$.
\end{theorem}

\begin{proof}
  Seien $\stackrel{\infty}{\Rightarrow}\ = \nu\mathcal{F}$
  und $\stackrel{\infty}{\rightarrow}\ = \nu\mathcal{G}$. Z.z.:
  $\underbrace{\bigcup_w w\,\nu\mathcal{F}}_X \subseteq \nu\mathcal{G}$ \\
  Dazu gen"ugt es zu zeigen, dass $X \subseteq \mathcal{G}(X)$, d.h.
  $\bigcup_w w\,\nu\mathcal{F} \subseteq \mathcal{G}(\bigcup_w w\,\nu\mathcal{F})$.
  Sei $wa \in \bigcup_w w\,\nu\mathcal{F}$, dann ist $a \in \nu\mathcal{F}$, d.h. $a \stackrel{\infty}{\Rightarrow}$.

  Nach Voraussetzung existieren also $n \in \N$, $a_1,\ldots,a_{n+1} \in A$ und
  $b_1,\ldots,b_n \in B$, so dass $a_i \Rightarrow b_i$ und $(a,b_1 \ldots b_{i-1},a_i) \in \sigma$
  f"ur $i=1,\ldots,n$, und $a_{n+1} \stackrel{\infty}{\Rightarrow}$ (d.h. $a_{n+1} \in \nu\mathcal{F}$).
  Also existiert nach Theorem~\ref{theorem:Vollstaendigkeit} eine Folge der Form
  \[
  w\,a \stackrel{*}{\rightarrow} w\,a\,b_1 \ldots b_n \rightarrow w\,a\,b_1 \ldots b_n\,a_{n+1}
  \]
  und weiterhin gilt $w\,a\,b_1 \ldots b_n\,a_{n+1} \in \bigcup_w w\,\nu\mathcal{F}$,
  da $a_{n+1} \in \nu\mathcal{F}$. Also gilt $(w\,a\,b_1 \ldots b_n) \in \mathcal{G}(X)$,
  was induktiv $(w\,a) \in \mathcal{G}(X)$ bedingt. Also ist $X$ $\mathcal{G}$-konsistent,
  d.h. $X$ ist Teilmenge des gr"o"sten Fixpunktes, was wiederum bedeutet, dass
  $w\,a \stackrel{\infty}{\rightarrow}$ gilt.
\end{proof}

\begin{theorem}["Aquivalenz]
  Sei $\sigma \in \Sigma^A_B$. Dann gilt f"ur alle $a \in A, b \in B$:
  \begin{enumerate}
  \item $a \Rightarrow b$ gdw. $a \stackrel{*}{\to} b$.
  \item $a \stackrel{\infty}{\Rightarrow}$ gdw. $a \stackrel{\infty}{\to}$.
  \end{enumerate}
\end{theorem}

\begin{proof} \
  \begin{enumerate}
  \item Folgt unmittelbar aus Theorem~\ref{theorem:Korrektheit} und Theorem~\ref{theorem:Vollstaendigkeit}.
  \end{enumerate}
\end{proof}


\section{Implementationen}

\begin{definition}[Transitionssystem]
  Ein \emph{Transitionssystem} ist ein Quadrupel $T = (S, S_0, S_F, \step)$ mit $S_0 \subseteq S$,
  $S_F \subseteq S$ und $\step\ \subseteq S \times S$.
\end{definition}

\begin{definition}[Implementation]
  Eine \emph{Implementation} von $\sigma\in\Sigma^A_B$ ist ein Paar
  $(T,\sim)$, mit $T = (S,S_0,S_F,\step)$ und $\sim\ \subseteq S \times (A \uplus B)^*$.
\end{definition}

\begin{definition}[Korrektheit von Implementationen]
  Eine Implementation $((S,S_0,S_F,\step),\sim)$ von $\sigma \in \Sigma^A_B$ hei"st \emph{korrekt},
  wenn gilt:
  \begin{enumerate}
  \item Wenn $s \in S_0$, dann existiert ein $a \in A$ mit $s \sim a$.
  \item Wenn $s \in S_F$, dann existiert ein $b \in B$ mit $s \sim b$.
  \item Wenn $s \step s'$ und $s \sim w$, dann existiert ein $w' \in (A \uplus B)^*$
    mit $w \to_\sigma w'$ und $s' \sim w'$.
  \end{enumerate}
\end{definition}

\begin{definition}[Vollst"andigkeit von Implementationen]
  Eine Implementation $((S,S_0,S_F,\step),\sim)$ von $\sigma \in \Sigma^A_B$ hei"st \emph{vollst"andig},
  wenn gilt:
  \begin{enumerate}
  \item Wenn $a \in A$, dann existiert ein $s \in S_0$ mit $s \sim a$.
  \item Wenn $b \in B$, dann existiert ein $s \in S_F$ mit $s \sim b$.
  \item Wenn $w \to_\sigma w'$ und $s \sim w$, dann existiert ein $s' \in S$
    mit $s \step s'$ und $s' \sim w'$.
  \end{enumerate}
\end{definition}

\begin{definition}[Modell]
  Ein \emph{Modell} von $\sigma \in \Sigma^A_B$ ist eine
  korrekte und vollst"andige Implementation $(T,\sim)$.
\end{definition}

\end{document}
\documentclass[12pt,a4paper]{article}

\usepackage{amsmath}
\usepackage{amssymb}
\usepackage{amstext}
\usepackage{array}
\usepackage[american]{babel}
\usepackage{color}
\usepackage{enumerate}
\usepackage[a4paper,%
            colorlinks=false,%
            final,%
            pdfkeywords={},%
            pdftitle={},%
            pdfauthor={Benedikt Meurer},%
            pdfsubject={},%
            pdfdisplaydoctitle=true]{hyperref}
\usepackage[latin1]{inputenc}
\usepackage{latexsym}
\usepackage[final]{listings}
\usepackage{makeidx}
\usepackage[standard,thmmarks]{ntheorem}
\usepackage{stmaryrd}
\usepackage{url}
\usepackage[arrow, matrix, curve]{xy}

%% LaTeX macros
%%
%% macros.tex - Useful LaTeX macros
%%
%% Copyright (c) 2006-2008 Benedikt Meurer <benedikt.meurer@googlemail.com>
%%


%%
%% Generic style macros
%%

\newcommand{\nstyle}[1]{\ensuremath{\mathit{#1}}}
\newcommand{\pstyle}[1]{\ensuremath{\mathsf{#1}}}


%%
%% Misc
%%

\newcommand{\coloncolon}{\mathrel{::}}
\newcommand{\pto}{\hookrightarrow}
\newcommand{\tto}{\rightarrow_t}


%%
%% Sets
%%

\newcommand{\N}{\ensuremath{\mathbb N}}
\newcommand{\Z}{\ensuremath{\mathbb Z}}

\newcommand{\Bool}{\nstyle{Bool}}
\newcommand{\Const}{\nstyle{Const}}
\newcommand{\Exp}{\nstyle{Exp}}
\newcommand{\Member}{\ensuremath{\mathcal{M}}}
\newcommand{\Var}{\ensuremath{\mathcal{X}}}
\newcommand{\Int}{\Z}
\newcommand{\Loc}{\nstyle{Loc}}
\newcommand{\Locfin}{\ensuremath{\Loc_{fin}}}
\newcommand{\Op}{\nstyle{Op}}
\newcommand{\Store}{\nstyle{Store}}
\newcommand{\Type}{\nstyle{Type}}
\newcommand{\Val}{\nstyle{Val}}

\newcommand{\AType}{\nstyle{AType}}
\newcommand{\DType}{\nstyle{DType}}
\newcommand{\LType}{\nstyle{LType}}

\newcommand{\Assn}{\nstyle{Assn}}
\newcommand{\Formula}{\nstyle{Formula}}
\newcommand{\Term}{\nstyle{Term}}

\newcommand{\Perm}{\nstyle{Perm}}


%%
%% Functions (TODO - DeclareMathOperator)
%%

\newcommand{\dom}[1]{\ensuremath{\nstyle{dom}\,(#1)}}
\newcommand{\free}[1]{\ensuremath{\nstyle{free}\,(#1)}}
\newcommand{\locns}[1]{\ensuremath{\nstyle{locns}\,(#1)}}
\newcommand{\grph}[1]{\ensuremath{\nstyle{graph}\,(#1)}}
\newcommand{\supp}[1]{\ensuremath{\nstyle{supp}\,(#1)}}
\newcommand{\reach}[2]{\ensuremath{\nstyle{reach}\,(#1,#2)}}
\newcommand{\reachn}[3]{\ensuremath{\nstyle{reach}_{#1}\,(#2,#3)}}
\newcommand{\semantic}[1]{\ensuremath{\llbracket#1\rrbracket}}

\newcommand{\powerset}[1]{\ensuremath{\wp\,(#1)}}
\newcommand{\powersetfin}[1]{\ensuremath{\wp_{fin}\,(#1)}}


%%
%% Names
%%

\newcommand{\op}{\nstyle{op}}


%%
%% Types
%%

\newcommand{\tassn}{\pstyle{assn}}
\newcommand{\tunit}{\pstyle{unit}}
\newcommand{\tbool}{\pstyle{bool}}
\newcommand{\tint}{\pstyle{int}}
\newcommand{\tarrow}[2]{#1\to#2}
\newcommand{\ttarrow}[2]{#1\tto#2}
\newcommand{\trecord}[1]{\left\{#1\right\}}
\newcommand{\tref}[1]{#1\,\pstyle{ref}}


%%
%% Constants
%%

\newcommand{\true}{\pstyle{true}}
\newcommand{\false}{\pstyle{false}}
\newcommand{\assign}{\pstyle{:=}}
\newcommand{\fix}{\pstyle{fix}}
\newcommand{\cref}{\pstyle{ref}}
\newcommand{\unit}{\pstyle{()}}


%%
%% Expressions
%%

\newcommand{\abstr}[2]{\lambda #1.#2}
\newcommand{\app}[2]{#1\,#2}
\newcommand{\ifte}[3]{\pstyle{if}\,#1\,\pstyle{then}\,#2\,\pstyle{else}\,#3}
\newcommand{\record}[1]{\left\{#1\right\}}
\newcommand{\proj}[2]{#1.#2}

\newcommand{\triple}[3]{\{#1\}\,#2\,\{#3\}}


%%
%% Grammars
%%

\newcommand{\GRbeg}{\begin{array}{rrlll}}
\newcommand{\GRend}{\end{array}}

\newcommand{\GRis}{& ::= &}
\newcommand{\GRal}{\\ & \mid &}
\newcommand{\GRnl}{\vspace{2mm}\\}
\newcommand{\GRmid}{\,\mid\,}
\newcommand{\GRtext}[1]{\hfill & \text{#1}}


%%
%% Rules
%%

\newcommand{\tj}[2]{#1\,\coloncolon\,#2}
\newcommand{\Tj}[3]{#1\,\triangleright\,#2\coloncolon#3}



\newcommand{\CExp}{\sstyle{CExp}}
\newcommand{\CVal}{\sstyle{CVal}}
\newcommand{\Conf}{\sstyle{Conf}}
\newcommand{\scon}{\nstyle{con}}
\newcommand{\sexp}{\nstyle{exp}}
\newcommand{\ssto}{\nstyle{sto}}

\begin{document}

%%
%% Abstract syntax
%%

\section{Abstract syntax}

The target language is eager PCF with the usual primitive types and imperative concepts.

We assume that infinite sets of variables $x \in \Var$ and locations $l \in \Loc$ are given;
variables are used to abstract expressions, locations are used to address memory cells.
Expressions are considered equal modulo renaming of bound variables.
$\Bool=\{\true,\false\}$ is the set of boolean constants $b$, $\Int$ is the set of integer
constants $z$; we do not distinguish between integer values and their program representation.

\begin{definition}[Expressions]
  The set $\Op$ of {\em operators} $\op$ and the set $\Const$ of {\em constants} $c$
  are defined by the grammars
  \[\GRbeg
    \op \GRis + \GRmid - \GRmid * \GRmid = \GRmid < \GRmid > \GRmid \le \GRmid \ge \\
    c \GRis z \GRmid b \GRmid \unit \GRmid \cref \GRmid !
             \GRmid \assign \GRmid \fix \GRmid \op \GRmid \proj{n}{i},
  \GRend\]
  the set $\Exp$ of {\em expressions} $e$ is defined by
  \[\GRbeg
    e \GRis c \GRmid x \GRmid l \GRmid \abstr{x}{e} \GRmid \app{e_1}{e_2}
      \GRmid \ifte{e_0}{e_1}{e_2} \GRmid (e_1,\ldots,e_n)
  \GRend\]
  and the set $\Val \subseteq \Exp$ of {\em values} $v$ is defined by
  \[\GRbeg
    v \GRis c \GRmid x \GRmid l \GRmid \abstr{x}{e} \GRmid \app{\op}{v} \GRmid \app{\assign}{v}
      \GRmid (v_1,\ldots,v_n)
  \GRend\]
\end{definition}

$\free{e}$ denotes the set of free variables and $\locns{e}$ denotes the set of locations in the
expression $e$. We say that an expression $e$ is a {\em program} if $\free{e}\cup\locns{e}=\emptyset$,
that is if $e$ contains neither unbound variables nor
locations (i.e. memory addresses in terms of the underlying machine). We demand that all expressions
considered for the logic later are valid programs (i.e. programmers aren't permitted to access
arbitrary memory locations).

\begin{definition}[Types]
  The set $\Type$ of all {\em types} $\tau$ is defined by:
  \[\GRbeg
  \tau \GRis \tbool \GRmid \tint \GRmid \tunit
  \GRal \tref{\tau_1} \GRmid \tarrow{\tau_1}{\tau_2}
  \GRal \tau_1 \times \ldots \times \tau_n
  \GRend\]
\end{definition}

We don't want to fiddle with type environments and store typings, hence we assume that
the elements of $\Var$ and $\Loc$ are already tagged with types. Therefore for every
type $\tau\in\Type$, let $\Var^\tau$ denote the set of variables of type $\tau$ and $\Loc^\tau$ the
set of locations of type $\tau$. Then $\Var=\bigcup_{\tau\in\Type}\Var^\tau$ and
$\Loc=\bigcup_{\tau\in\Type}\Loc^\tau$, where the $\Var^\tau$ and $\Loc^\tau$ are
required to be disjoint. We write $x^\tau$ to identify elements of $\Var^\tau$ and
$l^\tau$ for elements of $\Loc^\tau$.


%%
%% Static semantics
%%

\section{Static semantics}

$\CExp$ includes all well-typed, closed expressions. $\CVal = \CExp \cap \Val$.


%%
%% Operational semantics
%%

\section{Operational semantics}

Let $(W,\subseteq) = (\powersetfin{\Loc},\subseteq)$ be the partial order of possible `worlds'.

\begin{definition}[Store] \label{definition:Store} \
  A {\em store} is a partial function $s:\Loc \pto \CVal$
  with a finite domain such that $\locns{s(\Loc)} \subseteq \dom{s}$ and
  $s(\Loc^\tau) \subseteq \CVal^\tau$ for every $\tau\in\Type$.
  $\Store$ denotes the set of all such stores.
\end{definition}

\begin{definition}[Configuration]
  Let $e\in\Exp$ and $s\in\Store$. The pair $(e,s)$ is a {\em configuration}
  if $\locns{e} \subseteq \dom{s}$. We let $\Conf$ denote the set of all configurations.
\end{definition}

For every $L\in W$ we abbreviate
\[\begin{array}{rcl}
  \Exp(L) &\defeqn& \{e\in\Exp\,|\,L = \locns{e}\} \\
  \Store(L) &\defeqn& \{s\in\Store\,|\,L \subseteq \dom{s}\} \\
  \Conf(L) &\defeqn& \Exp(L) \times \Store(L)
\end{array}\]

\begin{tabular}{ll}
  \RN{Op} & $(\app{\app{\op}{n_1}}{n_2},s) \to (\op^I(n_1,n_2),s)$ \\[3mm]
  \RN{Beta-V} & $(\app{(\abstr{x}{e})}{v},s) \to (e[v/x],s)$ \\[3mm]
  \RN{App-Left} & $\RULE{(e_1,s) \to (e_1',s')}{(\app{e_1}{e_2},s) \to (\app{e_1'}{e_2},s')}$ \\[5mm]
  \RN{App-Right} & $\RULE{(e,s) \to (e',s')}{(\app{v}{e},s) \to (\app{v}{e'},s')}$ \\[5mm]
  \RN{Unfold} & $(\app{\fix}{(\abstr{x}{e})},s) \to (e[\app{\fix}{(\abstr{x}{e})}/x],s)$ \\[3mm]
  \RN{Cond-Eval} & $\RULE{(e_0,s) \to (e_0',s')}{(\ifte{e_0}{e_1}{e_2},s) \to (\ifte{e_0'}{e_1}{e_2},s')}$ \\[5mm]
  \RN{Cond-True} & $(\ifte{\true}{e_1}{e_2},s) \to (e_1,s)$ \\[3mm]
  \RN{Cond-False} & $(\ifte{\false}{e_1}{e_2},s) \to (e_2,s)$ \\[3mm]
  \RN{Tuple} & $\RULE{(e_i,s)\to(e_i',s')}{((v_1,\ldots,e_i,\ldots,e_n),s)\to((v_1,\ldots,e_i',\ldots,e_n),s')}$ \\[5mm]
  \RN{Proj} & $(\app{\proj{n}{i}}{(v_1,\ldots,v_n)},s)\to(v_i,s)$ \\[3mm]
  \RN{Ref} & $\RULE{l\not\in\dom{s}}{(\app{\cref}{v},s)\to(l,s[v/l])}$ \\[5mm]
  \RN{Assign} & $\RULE{l\in\dom{s}}{(\app{\app{\assign}{l}}{v},s)\to(\unit,s[v/l])}$ \\[5mm]
  \RN{Deref} & $\RULE{s(l)=v}{(\app{!}{l},s)\to(v,s)}$ \\[3mm]
\end{tabular}


%%
%% Operational properties
%%

\section{Operational properties}

Given finite worlds $L_1,L_2 \in W$ of locations, we write $R: L_1 \rightleftharpoons L_2$ to
indicate that $R$ is (the graph of) a type-preserving {\em partial bijection} from $L_1$ to $L_2$.
In other words, $R \subseteq L_1 \times L_2$ satifies the following properties:
\begin{enumerate}
\item $(l_1,l_2)\in R \Rightarrow \exists \tau\in\Type:\,l_1,l_2\in\Loc^\tau$
\item $(l_1,l_2) \in R \wedge (l_1',l_2')\in R \Rightarrow (l_1=l_1' \Leftrightarrow l_2=l_2')$
\end{enumerate}
Note that $R \oplus R'$ is a partial bijection $L_1 \oplus L_1' \rightleftharpoons L_2 \oplus L_2'$
if $R: L_1 \rightleftharpoons L_2$ and $R': L_1' \rightleftharpoons L_2'$ are disjoint partial
bijections. The {\em identity} partial bijection, $\Id_L: L \rightleftharpoons L$, is given by
$\Id_L \defeqn \delta L$.

The domain and codomain of definition of a partial bijection $R: L_1 \leftrightharpoons L_2$ will
be denoted
\[\begin{array}{rcl}
  \dom{R} &\defeqn& \{ l_1 \in L_1\,|\, \exists l_2\in L_2:\,l_1\ R\ l_2\} \\
  \cod{R} &\defeqn& \{ l_2 \in L_2\,|\, \exists l_1\in L_1:\,l_1\ R\ l_2\} .
\end{array}\]
Therefore $R$ is a bijection if $\dom{R} = L_1$ and $\cod{R} = L_2$, in which case we
write $R: L_1 \leftrightarrow L_2$.

For every partial bijection $R: L_1 \rightleftharpoons L_2$ we define relations
$R_{\sexp} \subseteq \Exp(L_1) \times \Exp(L_2)$, $R_{\ssto} \subseteq \Store(L_1) \times \Store(L_2)$
and $R_{\scon} \subseteq \Conf(L_1) \times \Conf(L_2)$ by:
\[\begin{array}{rcl}
  R_{\sexp} &\defeqn& \text{by induction on the structure of expressions} \\
  R_{\ssto} &\defeqn& \{ \vec{s} \in \Store(L_1) \times \Store(L_2)\,|\,\vec{s}(R) \subseteq R_{\sexp}\} \\
  R_{\scon} &\defeqn& \{((e_1,s_1),(e_2,s_2))\in\Conf(L_1)\times\Conf(L_2)\,|\,\vec{e}\in R_{\sexp}\wedge\vec{s}\in R_{\ssto}\}
\end{array}\]
We write $R$ instead of $R_\gamma$ if $\gamma\in\{\sexp,\ssto,\scon\}$ is clear from the context.

\begin{theorem}
  Let $L_i\in W$, $k_i\in\Conf(L_i)$ for $i=1,\ldots,4$ and  $R: L_1 \rightleftharpoons L_3$. If
  $k_1\ R\ k_3$, $k_1 \to k_2$ and $k_3 \to k_4$, then there exists a
  $R': (L_2 \setminus L_1) \rightleftharpoons (L_4\setminus L_3)$ such that $k_2\ (R \oplus R')\ k_4$.
\end{theorem}

The theorem is probably easier to understand if we visualize the implication using a commutative
diagram as shown below:
\[
\xymatrix{
  k_1 \ar[r]^{\to} \ar@{<->}[d]_{R} & k_2 \ar@{<.>}[d]^{R\oplus R'} \\
  k_3 \ar[r]^{\to} & k_4
}
\]

\begin{proof}
  {\bf TODO}
\end{proof}

{\bf TODO:} obviously $\oplus$ is associative (for the corollary)

{\bf TODO:} check $s \sqsubseteq s' \Leftrightarrow s\ \Id_{\dom{s}}\ s'$.
That should help to get from $L$-correctness to total correctness (independent of the $L$). It may be a trivial
consequence of the theorem above.

\end{document}

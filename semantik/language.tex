\documentclass[12pt,a4paper]{article}

\usepackage{amsmath}
\usepackage{amssymb}
\usepackage{amstext}
\usepackage{array}
\usepackage[american]{babel}
\usepackage{color}
\usepackage{enumerate}
\usepackage[a4paper,%
            colorlinks=false,%
            final,%
            pdfkeywords={},%
            pdftitle={},%
            pdfauthor={Benedikt Meurer},%
            pdfsubject={},%
            pdfdisplaydoctitle=true]{hyperref}
\usepackage[latin1]{inputenc}
\usepackage{latexsym}
\usepackage[final]{listings}
\usepackage{makeidx}
\usepackage[standard,thmmarks]{ntheorem}
\usepackage{stmaryrd}
\usepackage{url}
\usepackage[arrow, matrix, curve]{xy}

%% LaTeX macros
%%
%% macros.tex - Useful LaTeX macros
%%
%% Copyright (c) 2006-2011 Benedikt Meurer <benedikt.meurer@googlemail.com>
%% 


%%
%% Styles
%%

\newcommand{\nstyle}[1]{\ensuremath{\mathsf{#1}}}
\newcommand{\sstyle}[1]{\ensuremath{\mathit{#1}}}


%%
%% Misc
%%

\newcommand{\abort}{\ensuremath{\mathbf{abort}}}
\newcommand{\pto}{\rightharpoonup}
\newcommand{\step}{\ensuremath{\rightsquigarrow}}

\newcommand{\sem}[1]{\ensuremath{[\![#1]\!]}}


%%
%% Names
%%

\newcommand{\arity}{\ensuremath{\mathit{arity}}}
\newcommand{\cl}{\ensuremath{\mathit{cl}}}
\newcommand{\fr}{\ensuremath{\mathit{fr}}}
\newcommand{\free}{\ensuremath{\mathit{free}}}
\newcommand{\graph}{\ensuremath{\mathit{graph}}}
\newcommand{\id}{\ensuremath{\mathit{id}}}


%%
%% Sets
%%

\newcommand{\I}{\ensuremath{\mathcal I}}
\newcommand{\N}{\ensuremath{\mathbb N}}
\renewcommand{\O}{\ensuremath{\mathcal O}}
\newcommand{\Z}{\ensuremath{\mathbb Z}}
\newcommand{\Cl}{\sstyle{Cl}}
\newcommand{\Env}{\sstyle{Env}}
\newcommand{\Exp}{\sstyle{Exp}}
\newcommand{\Frame}{\sstyle{Frame}}
\newcommand{\Id}{\sstyle{Id}}
\newcommand{\Node}{\sstyle{Node}}
\newcommand{\Val}{\sstyle{Val}}


%%
%% Expressions
%%

\newcommand{\app}[2]{{#1}\,{#2}}
\newcommand{\abstr}[2]{\lambda{#1}.\,{#2}}
\newcommand{\ifte}[3]{\mathbf{if}\,{#1}\,\mathbf{then}\,{#2}\,\mathbf{else}\,{#3}}


%%
%% Values
%%

\newcommand{\clov}[2]{\langle{#1},{#2}\rangle}
\newcommand{\false}{\mathbf{false}}
\newcommand{\true}{\mathbf{true}}


%%
%% Grammars
%%

\newenvironment{grammar}{\begin{array}{rrlll}}{\end{array}}

\newcommand{\is}{& ::= &}
\newcommand{\al}{\\ & \mid &}
\newcommand{\nl}{\vspace{2mm}\\}


%%
%% Other environments
%%

\newenvironment{case}{\left\{\!\!\!\begin{array}{ll}}{\end{array}\right.}


%%% Local Variables: 
%%% mode: latex
%%% TeX-master: "compiler"
%%% End: 


\newcommand{\CExp}{\sstyle{CExp}}
\newcommand{\CVal}{\sstyle{CVal}}
\newcommand{\Conf}{\sstyle{Conf}}
\newcommand{\scon}{\nstyle{con}}
\newcommand{\sexp}{\nstyle{exp}}
\newcommand{\ssto}{\nstyle{sto}}

\begin{document}

%%
%% Abstract syntax
%%

\section{Abstract syntax}

The target language is eager PCF with the usual primitive types and imperative concepts.

We assume that infinite sets of variables $x \in \Var$ and locations $l \in \Loc$ are given;
variables are used to abstract expressions, locations are used to address memory cells.
Expressions are considered equal modulo renaming of bound variables.
$\Bool=\{\true,\false\}$ is the set of boolean constants $b$, $\Int$ is the set of integer
constants $z$; we do not distinguish between integer values and their program representation.

\begin{definition}[Expressions]
  The set $\Op$ of {\em operators} $\op$ and the set $\Const$ of {\em constants} $c$
  are defined by the grammars
  \[\GRbeg
    \op \GRis + \GRmid - \GRmid * \GRmid = \GRmid < \GRmid > \GRmid \le \GRmid \ge \\
    c \GRis z \GRmid b \GRmid \unit \GRmid \cref \GRmid !
             \GRmid \assign \GRmid \fix \GRmid \op \GRmid \proj{n}{i},
  \GRend\]
  the set $\Exp$ of {\em expressions} $e$ is defined by
  \[\GRbeg
    e \GRis c \GRmid x \GRmid l \GRmid \abstr{x}{e} \GRmid \app{e_1}{e_2}
      \GRmid \ifte{e_0}{e_1}{e_2} \GRmid (e_1,\ldots,e_n)
  \GRend\]
  and the set $\Val \subseteq \Exp$ of {\em values} $v$ is defined by
  \[\GRbeg
    v \GRis c \GRmid x \GRmid l \GRmid \abstr{x}{e} \GRmid \app{\op}{v} \GRmid \app{\assign}{v}
      \GRmid (v_1,\ldots,v_n)
  \GRend\]
\end{definition}

$\free{e}$ denotes the set of free variables and $\locns{e}$ denotes the set of locations in the
expression $e$. We say that an expression $e$ is a {\em program} if $\free{e}\cup\locns{e}=\emptyset$,
that is if $e$ contains neither unbound variables nor locations (i.e. memory addresses in terms
of the underlying machine).


%%
%% Operational semantics
%%

\section{Operational semantics}

Let $(W,\subseteq) = (\powersetfin{\Loc},\subseteq)$ be the partial order of possible `worlds'.

\begin{definition}[Store] \label{definition:Store} \
  A {\em store} is a partial function $s:\Loc \pto \Val$
  with a finite domain such that $\locns{s(\Loc)} \subseteq \dom{s}$.
  $\Store$ denotes the set of all such stores.
\end{definition}

\begin{definition}[Configuration]
  Let $e\in\Exp$ and $s\in\Store$. The pair $(e,s)$ is a {\em configuration}
  if $\locns{e} \subseteq \dom{s}$. We let $\Conf$ denote the set of all configurations.
\end{definition}

For every $L\in W$ we abbreviate
\[\begin{array}{rcl}
  \Exp(L) &\defeqn& \{e\in\Exp\,|\,\locns{e} \subseteq L\} \\
  \Store(L) &\defeqn& \{s\in\Store\,|\,L \subseteq \dom{s}\} \\
  \Conf(L) &\defeqn& \Exp(L) \times \Store(L)
\end{array}\]

\begin{tabular}{ll}
  \RN{E-Op} & $(\app{\app{\op}{z_1}}{z_2},s) \to (\op^I(z_1,z_2),s)$ \\[3mm]
  \RN{E-BetaV} & $(\app{(\abstr{x}{e})}{v},s) \to (e[v/x],s)$ \\[3mm]
  \RN{E-AppLeft} & $\RULE{(e_1,s) \to (e_1',s')}{(\app{e_1}{e_2},s) \to (\app{e_1'}{e_2},s')}$ \\[5mm]
  \RN{E-AppRight} & $\RULE{(e,s) \to (e',s')}{(\app{v}{e},s) \to (\app{v}{e'},s')}$ \\[5mm]
  \RN{E-Unfold} & $(\app{\fix}{(\abstr{x}{e})},s) \to (e[\app{\fix}{(\abstr{x}{e})}/x],s)$ \\[3mm]
  \RN{E-CondEval} & $\RULE{(e_0,s) \to (e_0',s')}{(\ifte{e_0}{e_1}{e_2},s) \to (\ifte{e_0'}{e_1}{e_2},s')}$ \\[5mm]
  \RN{E-CondTrue} & $(\ifte{\true}{e_1}{e_2},s) \to (e_1,s)$ \\[3mm]
  \RN{E-CondFalse} & $(\ifte{\false}{e_1}{e_2},s) \to (e_2,s)$ \\[3mm]
  \RN{E-Tuple} & $\RULE{(e_i,s)\to(e_i',s')}{((v_1,\ldots,e_i,\ldots,e_n),s)\to((v_1,\ldots,e_i',\ldots,e_n),s')}$ \\[5mm]
  \RN{E-Proj} & $(\app{\proj{n}{i}}{(v_1,\ldots,v_n)},s)\to(v_i,s)$ \\[3mm]
  \RN{E-Ref} & $\RULE{l\not\in\dom{s}}{(\app{\cref}{v},s)\to(l,s[v/l])}$ \\[5mm]
  \RN{E-Assign} & $(\app{\app{\assign}{l}}{v},s)\to(\unit,s[v/l])$ \\[3mm]
  \RN{E-Deref} & $(\app{!}{l},s)\to(s(l),s)$ \\[3mm]
\end{tabular}


%%
%% Operational properties
%%

\section{Operational properties}

Given finite worlds $L_1,L_2 \in W$ of locations, we write $R: L_1 \rightleftharpoons L_2$ to
indicate that $R$ is (the graph of) a {\em partial bijection} from $L_1$ to $L_2$.
In other words, $R \subseteq L_1 \times L_2$ satifies the following property:
\[
(l_1,l_2) \in R \wedge (l_1',l_2')\in R \Rightarrow (l_1=l_1' \Leftrightarrow l_2=l_2')
\]
Note that $R \oplus R'$ is a partial bijection $L_1 \oplus L_1' \rightleftharpoons L_2 \oplus L_2'$
if $R: L_1 \rightleftharpoons L_2$ and $R': L_1' \rightleftharpoons L_2'$ are disjoint partial
bijections. The {\em identity} partial bijection, $\Id_L: L \rightleftharpoons L$, is given by
$\Id_L \defeqn \delta L$.

The domain and codomain of definition of a partial bijection $R: L_1 \leftrightharpoons L_2$ will
be denoted
\[\begin{array}{rcl}
  \dom{R} &\defeqn& \{ l_1 \in L_1\,|\, \exists l_2\in L_2:\,l_1\ R\ l_2\} \\
  \cod{R} &\defeqn& \{ l_2 \in L_2\,|\, \exists l_1\in L_1:\,l_1\ R\ l_2\} .
\end{array}\]
Therefore $R$ is a bijection if $\dom{R} = L_1$ and $\cod{R} = L_2$, in which case we
write $R: L_1 \leftrightarrow L_2$.

For every partial bijection $R: L_1 \rightleftharpoons L_2$ we define relations
$R_{\sexp} \subseteq \Exp(L_1) \times \Exp(L_2)$, $R_{\ssto} \subseteq \Store(L_1) \times \Store(L_2)$
and $R_{\scon} \subseteq \Conf(L_1) \times \Conf(L_2)$ by:
\[\begin{array}{rcl}
  R_{\sexp} &\defeqn& \text{by induction on the structure of expressions} \\
  R_{\ssto} &\defeqn& \{ \vec{s} \in \Store(L_1) \times \Store(L_2)\,|\,\vec{s}(R) \subseteq R_{\sexp}\} \\
  R_{\scon} &\defeqn& \{((e_1,s_1),(e_2,s_2))\in\Conf(L_1)\times\Conf(L_2)\,|\,\vec{e}\in R_{\sexp}\wedge\vec{s}\in R_{\ssto}\}
\end{array}\]
We write $R$ instead of $R_\gamma$ if $\gamma\in\{\sexp,\ssto,\scon\}$ is clear from the context.

\begin{theorem}
  Let $L_i\in W$, $k_i\in\Conf(L_i)$ for $i=1,\ldots,4$ and  $R: L_1 \rightleftharpoons L_3$. If
  $k_1\ R\ k_3$, $k_1 \to k_2$ and $k_3 \to k_4$, then there exists a
  $R': (L_2 \setminus L_1) \rightleftharpoons (L_4\setminus L_3)$ such that $k_2\ (R \oplus R')\ k_4$.
\end{theorem}

The theorem is probably easier to understand if we visualize the implication using a commutative
diagram as shown below:
\[
\xymatrix{
  k_1 \ar[r]^{\to} \ar@{<->}[d]_{R} & k_2 \ar@{<.>}[d]^{R\oplus R'} \\
  k_3 \ar[r]^{\to} & k_4
}
\]

\begin{proof}
  By induction on the size of the derivation of $k_1 \to k_3$:
  \begin{enumerate}
  \item In case of \RN{E-Ref} we have $k_1 = (\app{\cref}{v_1},s_1)$, $k_2 = (l_2,s_2)$ with
    $l_2 \not\in \dom{s_1}$ and $s_2 = s_1[v_1/l_2]$, and $k_3 = (\app{\cref}{v_3},s_3)$, $k_4 = (l_4,s_4)$
    with $l_4 \not\in \dom{s_3}$ and $s_4 = s_3[v_3/l_4]$. Choose $R': \{l_2\} \leftrightarrow \{l_4\}$ because
    \begin{enumerate}
    \item $l_2 \not\in \dom{s_1} \supseteq L_1$ and $l_4 \not\in \dom{s_3} \supseteq L_3$,
    \item and by assumption we have $(s_2,s_4)(R) \subseteq R_{\sexp} \subseteq (R \oplus R')_{\sexp}$
      and it is trivial to see that $(s_2,s_4)(R') \subseteq R_{\sexp} \subseteq (R \oplus R')_{\sexp}$.
    \end{enumerate}
    Hence we conclude $k_2\ (R \oplus R')\ k_4$.

  \item \RN{E-Op}, \RN{E-BetaV}, \RN{E-Unfold}, \RN{E-CondTrue}, \RN{E-CondFalse}, \RN{E-Proj}, \RN{E-Assign}
    and \RN{Deref} follow easily with $R':\emptyset \leftrightarrow \emptyset$.

  \item \RN{E-AppLeft}, \RN{E-AppRight}, \RN{E-CondEval} and \RN{E-Tuple} are straight-forward.
  \end{enumerate}
\end{proof}

\begin{theorem}
  Let $L_i\in W$, $k_i\in\Conf(L_i)$ for $i=1,\ldots,3$ and $R:L_1 \rightleftharpoons L_2$. If $k_1\ R\ k_3$
  and $k_1 \to k_2$, then there exists a $k_4 \in \Conf$ such that $k_3\to k_4$.
\end{theorem}

Again visualized:
\[
\xymatrix{
  k_1 \ar[r]^{\to} \ar@{<->}[d]_{R} & k_2 \\
  k_3 \ar@{.>}[r]^{\to} & k_4
}
\]

\begin{proof}
  Straight forward induction.
\end{proof}

\begin{corollary}
  Let $L_i\in W$, $k_i\in\Conf(L_i)$ for $i=1,\ldots,3$ with $k_1 \to k_2$ and $R: L_1 \rightleftharpoons L_3$ with
  $k_1\ R\ k_3$. Then there exist $L_4\in W$, $k_4 \in \Conf(L_4)$ and
  $R': (L_2 \setminus L_1) \rightleftharpoons (L_4 \setminus L_3)$ such that $k_3 \to k_4$
  and $k_2\ (R \oplus R') k_4$.
\end{corollary}

\begin{corollary}
  If $(e,s)\in\Conf$ has atleast one terminating computation $(e,s) \xrightarrow* (v_1,s_1)$, then all computations
  of $(e,s)$ terminate with some $(v_2,s_2)$ such that a partial bijection $R:\dom{s_1}\rightleftharpoons\dom{s_2}$
  exists with $\Id_{\dom{s}} \subseteq R$ and $(v_1,s_1)\ R\ (v_2,s_2)$.
\end{corollary}

\begin{proof}
  Trivial consequence of the first and second theorem.
\end{proof}

{\bf TODO:} check $s \sqsubseteq s' \Leftrightarrow s\ \Id_{\dom{s}}\ s'$.
That should help to get from $L$-correctness to total correctness (independent of the $L$). It may be a trivial
consequence of the theorem above.


%%
%% Static semantics
%%

\section{Static semantics}

$\CExp$ includes all well-typed, closed expressions. $\CVal = \CExp \cap \Val$.

\begin{definition}[Types]
  The set $\Type$ of all {\em types} $\tau$ is defined by:
  \[\GRbeg
  \tau \GRis \tbool \GRmid \tint \GRmid \tunit
  \GRal \tref{\tau_1} \GRmid \tarrow{\tau_1}{\tau_2}
  \GRal \tau_1 \times \ldots \times \tau_n
  \GRend\]
\end{definition}

% We don't want to fiddle with type environments and store typings, hence we assume that
% the elements of $\Var$ and $\Loc$ are already tagged with types. Therefore for every
% type $\tau\in\Type$, let $\Var^\tau$ denote the set of variables of type $\tau$ and $\Loc^\tau$ the
% set of locations of type $\tau$. Then $\Var=\bigcup_{\tau\in\Type}\Var^\tau$ and
% $\Loc=\bigcup_{\tau\in\Type}\Loc^\tau$, where the $\Var^\tau$ and $\Loc^\tau$ are
% required to be disjoint. We write $x^\tau$ to identify elements of $\Var^\tau$ and
% $l^\tau$ for elements of $\Loc^\tau$.

\begin{tabular}{ll}
  \RN{T-Unit} & $\tj{\unit}{\tunit}$ \\[3mm]
  \RN{T-Bool} & $\tj{b}{\tbool}$ \\[3mm]
  \RN{T-Int} & $\tj{z}{\tint}$ \\[3mm]
  \RN{T-Aop} & $\tj{\op}{\tarrow{\tint}{\tarrow{\tint}{\tint}}}$ \\[3mm]
  \RN{T-Rop} & $\tj{\op}{\tarrow{\tint}{\tarrow{\tint}{\tbool}}}$ \\[3mm]
  \RN{T-Fix} & $\tj{\fix}{\tarrow{(\tarrow{\tau}{\tau})}{\tau}}$ \\[3mm]
  \RN{T-Ref} & $\tj{\cref}{\tarrow{\tau}{\tref{\tau}}}$ \\[3mm]
  \RN{T-Deref} & $\tj{!}{\tarrow{\tref{\tau}}{\tau}}$ \\[3mm]
  \RN{T-Assign} & $\tj{\assign}{\tarrow{\tref{\tau}}{\tarrow{\tau}{\tunit}}}$ \\[3mm]
\end{tabular}

\begin{tabular}{ll}
  \RN{T-Const} & $\RULE{\tj{c}{\tau}}{\Tj{\Gamma}{c}{\tau}}$ \\[5mm]
  \RN{T-Var} & $\RULE{\Gamma(x) = \tau}{\Tj{\Gamma}{x}{\tau}}$ \\[5mm]
  \RN{T-Abstr} & $\RULE{\Tj{\Gamma[\tau/x]}{e}{\tau'}}{\Tj{\Gamma}{\abstr{x}{e}}{\tarrow{\tau}{\tau'}}}$ \\[5mm]
  \RN{T-Tuple} & $\RULE{\Tj{\Gamma}{e_1}{\tau_1} \quad \ldots \quad \Tj{\Gamma}{e_n}{\tau_n}}{\Tj{\Gamma}{(e_1,\ldots,e_n)}{\tau_1 \times \ldots \times \tau_n}}$ \\[5mm]
  \RN{T-App} & $\RULE{\Tj{\Gamma}{e_1}{\tarrow{\tau}{\tau'}} \quad \Tj{\Gamma}{e_2}{\tau}}{\Tj{\Gamma}{\app{e_1}{e_2}}{\tau'}}$ \\[5mm]
  \RN{T-Cond} & $\RULE{\Tj{\Gamma}{e_0}{\tbool} \quad \Tj{\Gamma}{e_1}{\tau} \quad \Tj{\Gamma}{e_2}{\tau}}{\Tj{\Gamma}{\ifte{e_0}{e_1}{e_2}}{\tau}}$ \\[5mm]
\end{tabular}


%%
%% The assertion language
%%

\section{The assertion language}

Static types $\sigma \in \SType$, logical types $\theta \in \LType$
\[\GRbeg
\sigma \GRis \tau \GRmid \ttarrow{\sigma_1}{\sigma_2} \GRnl
\theta \GRis \sigma \GRmid \tassn{\underline{\sigma}}
\GRend\]

\[
\DEF_L = \{R:L_1 \rightleftharpoons L_2\,|\,L_1,L_2\in W \wedge \Id_L \subseteq R\}
\]

\begin{definition}[Semantic domains]
  For every logical type $\theta \in \LType$ we define semantic domains
  $\semantic{\theta} = \bigcup_{L\in W} \semantic{\theta}_L$ by:
  \begin{itemize}
  \item $\semantic{\tau}_L
    = \{v \in \Val\,|\,\tj{v}{\tau} \wedge \locns{v} = L\}$
  \item $\semantic{\ttarrow{\sigma_1}{\sigma_2}}_L
    = \{\psi:\semantic{\sigma_1}\to\semantic{\sigma_2}\,|\,
    \forall L'\in W.\,\psi(\semantic{\sigma_1}_{L'}) \subseteq \semantic{\sigma_2}_{L \cup L'}\}$
  \item $\semantic{\tassn{\underline{\sigma}}}_L
    = \{\phi:\Store_L \to \powerset{\semantic{\underline{\sigma}}}\,|\,
    \forall R\in \DEF_L, (s_1,s_2)\in R.\, \phi(s_1) = \phi(s_2)\}$
  \end{itemize}
\end{definition}


\end{document}

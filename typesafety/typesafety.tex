\documentclass[a4paper,final,preprint,sort&compress]{elsarticle}

\usepackage[english]{babel}
\usepackage{bmeurer}
\usepackage{color}
\usepackage{enumitem}
\usepackage[colorlinks=false,%
            pdfkeywords={},%
            pdftitle={A Step-indexed Semantic Model of Types for the Call-by-Name Lambda Calculus},%
            pdfauthor={Benedikt Meurer},%
            pdfsubject={},%
            pdfdisplaydoctitle=true]{hyperref}
\usepackage{mathpartir}

\newdefinition{definition}{Definition}
\newtheorem{theorem}[definition]{Theorem}
\newtheorem{lemma}[definition]{Lemma}
\newproof{proof}{Proof}

\newcommand{\M}{\ensuremath{\mathcal{M}}}
\newcommand{\R}{\ensuremath{\mathcal{R}}}
\newcommand{\T}{\ensuremath{\mathcal{T}}}

\newcommand{\Rules}{\ensuremath{\mathit{Rules}}}
\newcommand{\States}{\ensuremath{\mathit{States}}}

\def \irulesinglefraction#1#2{\hbox{$\begin{array}{@{}c@{}}
    #1 \\[-1.2ex]
    \hrulefill \\
    #2
  \end{array}$}}
\def \iruledoublefraction#1#2{\hbox{$\begin{array}{@{}c@{}}
    #1 \\[-1.2ex]
    \hrulefill \\[-2.35ex]
    \hrulefill \\
    #2
  \end{array}$}}
\newcommand{\irulesingle}[3][]{\inferrule*[right={#1},myfraction=\irulesinglefraction]{#2}{#3}}
\newcommand{\iruledouble}[3][]{\inferrule*[right={#1},myfraction=\iruledoublefraction]{#2}{#3}}

\begin{document}

\begin{frontmatter}

\title{Type-safe recursion systems}

\author[fnt]{Jan Thomas K\"olzer}
\ead{jan.koelzer@student.uni-siegen.de}
\author[cus]{Benedikt Meurer\corref{cor1}}
\ead{meurer@informatik.uni-siegen.de}
\author[cus]{Kurt Sieber}
\ead{sieber@informatik.uni-siegen.de}
\cortext[cor1]{Corresponding author}
\address[fnt]{Naturwissenschaftlich-Technische Fakult\"at, Universit\"at Siegen, D-57068 Siegen, Germany}
\address[cus]{Compilerbau und Softwareanalyse, Universit\"at Siegen, D-57068 Siegen, Germany}

\begin{abstract}
  TODO
\end{abstract}

\begin{keyword}
  TODO
\end{keyword}

\end{frontmatter}


\section{Introduction}
\label{sec:Introduction}


\TODO{Introduction}


\section{Recursion systems}
\label{sec:Recursion_systems}


A \emph{semantic model} $\M$ is a triple $(A,B,\Rightarrow)$, where $A$ and $B$ are disjoint sets
and \mbox{$\Rightarrow~\subseteq A \times B_\bot^\top$} with $B_\bot^\top = B \uplus \{\bot,\top\}$. The $a \in A$
are called \emph{arguments} and the $b \in B$ are called \emph{results}.

\begin{definition}
  A semantic model $\M = (A,B,\Rightarrow)$ is called \emph{deterministic}, if $\Rightarrow$ is a function.
%  for every $a \in A$ there is at most one $r \in B_\bot^\top$ such that $a \Rightarrow r$.
\end{definition}

A \emph{recursion rule on $A$ and $B$} is a relation $R \subseteq A \times B^* \times (A \uplus B)$, the set
of all such rules is denoted by $\Rules(A,B)$. Each set of recursion rules \mbox{$\R \subseteq \Rules(A,B)$} forms a
\emph{recursion system}.

Every \emph{recursion system} $\R \subseteq \Rules(A,B)$ gives raise to a semantic model
\mbox{$\M(\R) = (A,B,\Rightarrow_\R)$}, where $\Rightarrow_\R$ is the disjoint union of
$\Downarrow_\R$ and $\Uparrow_\R$. 
\mbox{$\Downarrow_\R~\subseteq A \times B^\top$} is inductively defined by the inference
rules in Figure~\ref{fig:Inductive_inference_rules},
\begin{figure}[htb]
  \centering
  \begin{mathpar}
    \irulesingle[(Res)]{
      R \in \R \\
      R \vdash (a,b_1 \ldots b_n) \\
      (a,b_1 \ldots b_n,b) \in R
    }{
      a \Downarrow_\R b
    }
    \and
    \irulesingle[(Err-1)]{
      \forall R \in \R, x \in A \uplus B: (a,\varepsilon,x) \not\in R
    }{
      a \Downarrow_\R \top
    }
    \and
    \irulesingle[(Err-2)]{
      R \in \R \\
      R \vdash (a,b_1 \ldots b_{n+1}) \\
      \forall x \in A \uplus B: (a,b_1 \ldots b_{n+1},x) \not\in R
    }{
      a \Downarrow_\R \top
    }
    \and
    \irulesingle[(Err-3)]{
      R \in \R \\
      R \vdash (a,b_1 \ldots b_n) \\
      (a,b_1 \ldots b_n,a') \in R \\
      a' \Downarrow_\R \top
    }{
      a \Downarrow_\R \top
    }
  \end{mathpar}
  \caption{Inductive inference rules}
  \label{fig:Inductive_inference_rules}
\end{figure}
where we say that $b_1,\ldots,b_n$ are \emph{temporary results for $a$ with respect to $R$}, written
$R \vdash (a,b_1 \ldots b_n)$, if there are $a_1,\ldots,a_n$ such that for every
$i = 1,\ldots,n$ we have $a_i \Downarrow_\R b_i$ and $(a,b_1 \ldots b_{i-1},a_i) \in R$,
and \mbox{$\Uparrow_\R~\subseteq A \times \{\bot\}$} is coinductively defined by the
inference rule in Figure~\ref{fig:Coinductive_inference_rules}.
\begin{figure}[htb]
  \centering
  \begin{mathpar}
    \iruledouble[(Div)]{
      R \in \R \\
      R \vdash (a,b_1 \ldots b_n) \\
      (a,b_1 \ldots b_n,a') \in R \\
      a' \Uparrow_\R \bot
    }{
      a \Uparrow_\R \bot
    }
  \end{mathpar}
  \caption{Coinductive inference rules}
  \label{fig:Coinductive_inference_rules}
\end{figure}


\section{Safety}
\label{sec:Safety}


A \emph{type system} for $A$ consists of a set $\Pi$ of types $\pi$ and sets $A^\pi \subseteq A$
of well-typed arguments of type $\pi$ for all $\pi \in \Pi$.

\begin{definition} \label{def:Safety}
  A semantic model $\mathcal{M} = (A,B,\Rightarrow)$ is called \emph{safe} with respect to
  some type system $\Pi$, if for every type $\pi \in \Pi$ there is no $a \in A^\pi$ such that
  $a \Rightarrow \top$.
\end{definition}

A recursion system $\R$ is called safe with respect to $\Pi$ if $\M(\R)$ is safe with
respect to $\Pi$.

% \begin{lemma}
%   A transition system $\mathcal{T} = (\Sigma, \phi, \psi, \vdash)$ is safe with respect to
%   some type system $\Pi$ if, whenever $\pi \in \Pi$, $a \in A^\pi$ and $\sigma \in \Sigma$
%   with $\phi(a) \vdash^* \sigma$, then either $\sigma \in \dom(\psi)$ or there is some
%   $\sigma' \in \Sigma$ such that $\sigma \vdash \sigma'$.
% \end{lemma}

% \begin{proof}
%   Immediate consequence of definition \ref{def:Semantic_model_of_transition_system}
%   and \ref{def:Safety}.
% \end{proof}

\begin{theorem}[Safety]
  A recursion system $\R$ is safe with respect to $\Pi$, if the following two properties 
  hold for every $R \in \R$, $a \in A^\pi$ and $b_1,\ldots,b_n \in B$ with $n \ge 0$:
  \begin{description}[labelindent=\parindent,style=nextline]
  \item[Local Preservation]

    If \mbox{$R \vdash (a,b_1 \ldots b_n)$} and \mbox{$(a,b_1 \ldots b_n,a') \in R$} for
    some $a' \in A$, then there is some $\pi' \in \Pi$ such that $a' \in A^{\pi'}$.

  \item[Local Progress]

    If \mbox{$R \vdash (a,b_1 \ldots b_n)$}, then \mbox{$(a,b_1 \ldots b_n,x) \in R$} for
    some \mbox{$x \in A \uplus B$}

  \end{description}
\end{theorem}

\begin{proof}
  Assume that $a \Downarrow_\R \top$ holds. By induction on the derivation of $a \Downarrow_\R \top$ and
  case analysis on the inference rule:
  \begin{description}[font=\sc,labelindent=\parindent,style=nextline]
  \item[(Err-1)]

    Then there is no \mbox{$x \in A \uplus B$} such that \mbox{$(a,\varepsilon,x) \in R$},
    contrary to \textbf{Local Progress}.

  \item[(Err-2)]

    Then there is no \mbox{$x \in A \uplus B$} such that \mbox{$(a,b_1 \ldots b_{n+1},x) \in R$},
    contrary to \textbf{Local Progress}.

  \item[(Err-3)]

    By \textbf{Local Preservation} there is some $\pi'$ such that $a' \in A^{\pi'}$, and the result
    follows by induction hypothesis.
    
  \end{description}
\end{proof}


\section{Relation with transition systems}
\label{sec:Relation_with_transition_systems}


A \emph{transition system} $\T$ is a tuple $(\Sigma, \phi, \psi, \leadsto)$ where
\begin{itemize}
\item $\Sigma$ is a non-empty set of \emph{states} $\sigma$,
\item $\phi: A \pto \Sigma$ is the \emph{input function},
\item $\psi: \Sigma \pto B$ is the \emph{output function}, and
\item $\leadsto~\subseteq \Sigma \times \Sigma$ is the \emph{transition relation}.
\end{itemize}
The elements of $A_0 = \dom(\phi)$ are called \emph{initial arguments}.
We write \mbox{$\sigma_0 \leadsto^n \sigma_n$} if there exists a sequence of $n$ steps
such that \mbox{$\sigma_0 \leadsto \sigma_1 \leadsto \ldots \leadsto \sigma_n$}. We
write \mbox{$\sigma \leadsto^* \sigma'$} if \mbox{$\sigma \leadsto^n \sigma'$} for some
\mbox{$n \ge 0$}.

Let $\T = (\Sigma,\phi,\psi,\leadsto)$ be a transition system. The semantic model
\mbox{$\M(\T) = (A,B,\Rightarrow_\T)$} of $\T$ is defined by:
\[\begin{array}{ll}
  a \Rightarrow_\T b
  & \mbox{iff $\phi(a) = \sigma \leadsto^* \sigma'$ and $\psi(\sigma') = b$} \\
  a \Rightarrow_\T \top
  & \mbox{iff $\phi(a) = \sigma \leadsto^* \sigma' \not\leadsto$ and $\sigma' \not\in\dom(\psi)$} \\
  a \Rightarrow_\T \bot
  & \mbox{iff there is an infinite sequence $\phi(a) = \sigma_0 \leadsto \sigma_1 \leadsto \ldots$} \\
\end{array}\]

A transition system $\T$ is called safe with respect to $\Pi$ if $\M(\T)$ is safe with respect to $\Pi$.

$\Pi$ is a type system for $\T$ if it is a type system for $A$ and there are sets $\Sigma^\pi$ for
every $\pi \in \Pi$ such that $\sigma \in \Sigma^\pi$ whenever $a \in A^\pi$ and $\phi(a) = \sigma$.
Now we can formalize the usual safety theorem, which is proved using preservation and progress
theorems (Wright and Felleisen \cite{WrightFelleisen94}).

\begin{theorem}
  A transition system $\T = (\Sigma,\phi,\psi,\leadsto)$ is safe with respect to $\Pi$
  if the following two properties hold:
  \begin{description}[labelindent=\parindent,style=nextline]
  \item[Preservation] 

    If $\sigma \in \Sigma^\pi$ and $\sigma \leadsto \sigma'$, then $\sigma' \in \Sigma^\pi$.

  \item[Progress] 

    If \mbox{$\sigma \in \Sigma^\pi$}, then \mbox{$\sigma \in \dom(\psi)$} or there is some
    \mbox{$\sigma' \in \Sigma$} such that \mbox{$\sigma \leadsto \sigma'$}.
    
  \end{description}
\end{theorem}

\begin{proof}
  Assume that there is some $a \in A^\pi$ such that \mbox{$\phi(a) = \sigma \leadsto^* \sigma'$}.
  By \textbf{Preservation} we have $\sigma' \in \Sigma^\pi$ and by \textbf{Progress} we have either
  $\sigma' \in \dom(\psi)$ or there is some $\sigma''$ such that $\sigma' \leadsto \sigma''$,
  hence $a \not\Rightarrow_\T \top$.
\end{proof}

Every \emph{recursion system} $\R \subseteq \Rules(A,B)$ gives rise to a \emph{naive transition system}
$\T(\R) = (\States(\R),\phi_\R,\psi_\R,\leadsto_\R)$, where
\begin{itemize}
\item $\States(\R) = (A \times \R \cup B)^*(A \uplus B)$,
\item $\phi_\R: A \to \States(\R), a \mapsto a$,
\item $\psi_\R: \States(\R) \pto B, b \mapsto b$, and
\item $\leadsto_\R~\subseteq \States(\R) \times \States(\R)$ is the smallest relation closed under the following
  inference rules in Figure~\ref{fig:Naive_transition_rules}.
\end{itemize}

\begin{figure}[htb]
  \centering
  \begin{mathpar}
    \irulesingle[(AA)]{%
      R \in \R \\
      (a,\varepsilon,a') \in R
    }{%
      wa \leadsto_\R w(a,R) a'
    }%
    \and
    \irulesingle[(AB)]{%
      R \in \R \\
      (a,\varepsilon,b) \in R
    }{%
      wa \leadsto_\R wb
    }%
    \and
    \irulesingle[(BA)]{%
      R \in \R \\
      (a,b_1 \ldots b_n,a') \in R
    }{%
      w(a,R)b_1 \ldots b_n \leadsto_\R w(a,R) b_1 \ldots b_n a'
    }%
    \and
    \irulesingle[(BB)]{%
      R \in \R \\
      (a,b_1 \ldots b_n,b) \in R
    }{%
      w(a,R)b_1 \ldots b_n \leadsto_\R wb
    }%
  \end{mathpar}
  \caption{Naive transition rules}
  \label{fig:Naive_transition_rules}
\end{figure}

\begin{theorem}
  Let $\R \subseteq \Rules(A,B)$. Then \mbox{$\M\left(\R\right) = \M\left(\T\left(\R\right)\right)$}.
\end{theorem}

\TODO{Proof}


\section{Related work}
\label{sec:Related_work}


\paragraph{Big-step semantics}

The traditional approach to prove type safety using big-step operational semantics works by
providing additional inductive inference rules to define a predicate $a \Rightarrow \mathit{wrong}$
\cite{Tofte87}. One then proves that \mbox{$\neg(a \Rightarrow \mathit{wrong})$} whenever $a$ is
well-typed. Our approach can be seen as a generalization of the traditional approach: One no longer
needs to provide extra rules to define \mbox{$a \Rightarrow \mathit{wrong}$}, as these are derived from
the definition of the recursion system, and so there is no longer a risk that the rules for
\mbox{$a \Rightarrow \mathit{wrong}$} are incomplete and miss some cases of ``going wrong''.

Cousot and Cousot \cite{CousotCousot92}, and later Leroy \ETAL \cite{LeroyGrall09},
proposed two approaches to enable safety proofs using big-step semantics based on
coinductive definitions and proofs: the first complements the normal inductive big-step
evaluation rules for finite evaluations with coinductive big-step rules describing diverging
evaluations; the second -- called coevaluation -- simply interprets coinductively the normal
big-step evaluation rules, thus enabling them to describe both terminating and non-terminating
evaluations.
Both approaches do not work for non-determinstic big-step semantics; the
coevaluation approach is more elegant and simpler, but fails for interesting type systems.

Following up on \cite{CousotCousot92}, Cousot and Cousot recently introduced bi-inductive
semantics \cite{CousotCousot07} and applied it to the call-by-value $\lambda$-calculus.
Bi-inductive semantics are defined in terms of smallest fixed points with respect to a
non-standard ordering. This approach captures both terminating and diverging evaluations
using a common set of inference rules. They start from a big-step trace semantics and
systematically derive the other semantics (big-step and small-step) by abstracton.
\TODO{How does their approach relate to our approach? Their approach
should work with non-determinism, right?}


\paragraph{Small-step semantics}

Until recently, the most common way to prove type safety was by a syntactic proof
technique, based on small-step semantics, which was adopted from combinatory
logic by Wright and Felleisen \cite{WrightFelleisen94}. One shows that each step of
computation preserves typability (\emph{preservation} or \emph{subject reduction})
and that typable states are safe (\emph{progress}). There are
languages where these properties do not hold, but which can nevertheless be
considered type-safe. For example, if one formalizes the operational semantics
of Java in small-step style \cite{FlattKrishnamurthiFelleisen98,IgarashiPierceWadler01},
type preservation in its standard form fails. This is
not a defect in the language, but rather an artifact of the formalization, which
disappears, for example, in big-step operational semantics \cite{Pierce02}.

Later Appel \ETAL introduced step-indexed semantic models in the context of
foundational proof-carrying code \cite{AppelFelty00}, and applied their technique
to prove type safety of a pure $\lambda$-calculus with recursive types
\cite{AppelMcAllester01}. Ahmed \ETAL successfully extended this technique to
general references and impredicative polymorphmism \cite{AhmedAppelVirga02,Ahmed04},
and Hritcu \ETAL further extended it with object types and subtyping
\cite{Hritcu07,HritcuSchwinghammer09}.

Both the subject-reduction approach as well as the later semantic approaches
can be extended to handle fairly complex type systems \cite{Pierce02}, and also play well with
non-determinstic operational semantics, which is due to the fact that they are
based on small-step operational semantics. This is great for low-level languages,
where small-step operational semantics is the natural choice to describe the
system, but does not always provide the best option for higher-level languages.

\TODO{Other related work? Any recent big-step stuff?}


\section{Conclusion}
\label{sec:Conclusion}


\TODO{Conclusion}


\bibliographystyle{elsarticle-num}
\bibliography{citations}

\end{document}
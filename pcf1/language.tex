\documentclass[12pt,a4paper]{report}

\usepackage{amsmath}
\usepackage{amssymb}
\usepackage{amstext}
\usepackage{array}
\usepackage[american]{babel}
\usepackage{color}
\usepackage{enumerate}
%\usepackage[T1]{fontenc}
%\usepackage{german}
\usepackage[a4paper,%
            colorlinks=false,%
            final,%
            pdfkeywords={},%
            pdftitle={},%
            pdfauthor={Benedikt Meurer},%
            pdfsubject={},%
            pdfdisplaydoctitle=true]{hyperref}
\usepackage[latin1]{inputenc}
\usepackage{latexsym}
\usepackage[final]{listings}
\usepackage{makeidx}
%\usepackage{ngerman}
\usepackage[standard,thmmarks]{ntheorem}
%\usepackage{scrpage2}
\usepackage{stmaryrd}
\usepackage{url}

%% LaTeX macros
%%
%% macros.tex - Useful LaTeX macros
%%
%% Copyright (c) 2006-2011 Benedikt Meurer <benedikt.meurer@googlemail.com>
%% 


%%
%% Styles
%%

\newcommand{\nstyle}[1]{\ensuremath{\mathsf{#1}}}
\newcommand{\sstyle}[1]{\ensuremath{\mathit{#1}}}


%%
%% Misc
%%

\newcommand{\abort}{\ensuremath{\mathbf{abort}}}
\newcommand{\pto}{\rightharpoonup}
\newcommand{\step}{\ensuremath{\rightsquigarrow}}

\newcommand{\sem}[1]{\ensuremath{[\![#1]\!]}}


%%
%% Names
%%

\newcommand{\arity}{\ensuremath{\mathit{arity}}}
\newcommand{\cl}{\ensuremath{\mathit{cl}}}
\newcommand{\fr}{\ensuremath{\mathit{fr}}}
\newcommand{\free}{\ensuremath{\mathit{free}}}
\newcommand{\graph}{\ensuremath{\mathit{graph}}}
\newcommand{\id}{\ensuremath{\mathit{id}}}


%%
%% Sets
%%

\newcommand{\I}{\ensuremath{\mathcal I}}
\newcommand{\N}{\ensuremath{\mathbb N}}
\renewcommand{\O}{\ensuremath{\mathcal O}}
\newcommand{\Z}{\ensuremath{\mathbb Z}}
\newcommand{\Cl}{\sstyle{Cl}}
\newcommand{\Env}{\sstyle{Env}}
\newcommand{\Exp}{\sstyle{Exp}}
\newcommand{\Frame}{\sstyle{Frame}}
\newcommand{\Id}{\sstyle{Id}}
\newcommand{\Node}{\sstyle{Node}}
\newcommand{\Val}{\sstyle{Val}}


%%
%% Expressions
%%

\newcommand{\app}[2]{{#1}\,{#2}}
\newcommand{\abstr}[2]{\lambda{#1}.\,{#2}}
\newcommand{\ifte}[3]{\mathbf{if}\,{#1}\,\mathbf{then}\,{#2}\,\mathbf{else}\,{#3}}


%%
%% Values
%%

\newcommand{\clov}[2]{\langle{#1},{#2}\rangle}
\newcommand{\false}{\mathbf{false}}
\newcommand{\true}{\mathbf{true}}


%%
%% Grammars
%%

\newenvironment{grammar}{\begin{array}{rrlll}}{\end{array}}

\newcommand{\is}{& ::= &}
\newcommand{\al}{\\ & \mid &}
\newcommand{\nl}{\vspace{2mm}\\}


%%
%% Other environments
%%

\newenvironment{case}{\left\{\!\!\!\begin{array}{ll}}{\end{array}\right.}


%%% Local Variables: 
%%% mode: latex
%%% TeX-master: "compiler"
%%% End: 


% TODO
\newcommand{\CExp}{\nstyle{CExp}}
\newcommand{\CVal}{\nstyle{CVal}}
\newcommand{\loc}{\nstyle{loc}}
\newcommand{\sto}{\nstyle{sto}}
\newcommand{\val}{\nstyle{val}}
\newcommand{\z}{\nstyle{int}}
\newcommand{\DEF}{\nstyle{DEF}}
\newcommand{\OUT}{\nstyle{OUT}}
\newcommand{\I}{\mathcal{I}}


\begin{document}


%%
%% The programming language
%%

\chapter{The programming language}

As target language, we use eager PCF with the usual primitive types and imperative concepts.


%%
%% Abstract syntax
%%

\section{Abstract syntax}

We assume that an infinite set of variables $x \in \Var$ is given; variables are used to
abstract expressions. Expressions are considered equal modulo renaming of bound variables.
$\Bool=\{\true,\false\}$ is the set of boolean constants $b$, $\Int$ is the set of integer
constants $z$; we do not distinguish between integer values and their program representation.

\begin{definition}[Types]
  The set $\Type$ of all {\em types} $\tau$ is defined by the context-free grammar below:
  \[\GRbeg
    \tau  \GRis \tbool \GRmid \tint \GRmid \tunit
          \GRal \tref{\tint}
          \GRal \tarrow{\tau_1}{\tau_2}
  \GRend\]
\end{definition}

We assume an infinite set $\Loc$ of locations $l$.

\begin{definition}[Expressions]
  The set $\Op$ of {\em operators} $\op$ and the set $\Const$ of {\em constants} $c$
  are defined by the grammars
  \[\GRbeg
    \op \GRis + \GRmid - \GRmid * \GRmid = \GRmid < \GRmid > \GRmid \le \GRmid \ge \\
    c \GRis z \GRmid b \GRmid \unit \GRmid \cref \GRmid !
             \GRmid \assign \GRmid \fix,
  \GRend\]
  the set $\Exp$ of {\em expressions} $e$ is defined by
  \[\GRbeg
    e \GRis c \GRmid x \GRmid l \GRmid \abstr{x}{e} \GRmid \app{e_1}{e_2}
           \GRmid \ifte{e_0}{e_1}{e_2}
  \GRend\]
  and the set $\Val \subseteq \Exp$ of {\em values} $v$ is defined by
  \[\GRbeg
    v \GRis c \GRmid l \GRmid \abstr{x}{e} \GRmid \app{\op}{v} \GRmid \app{\assign}{v}
  \GRend\]
\end{definition}

$\free{e}$ denotes the set of free variables and $\locns{e}$ denotes the set of locations in the
expression $e$. We say that an expression $e$ is a {\em program} if both $\free{e}=\emptyset$
and $\locns{e}=\emptyset$ holds, that is if $e$ contains neither unbound variables nor
locations (i.e. memory addresses in terms of the underlying machine). We demand that all expressions
considered for the logic later are valid programs (i.e. programmers aren't permitted to access
abritrary memory locations).


%%
%% Static semantics
%%

\section{Static semantics}

The static semantics of the programming language are defined by the typing relation
$\Tj{\Gamma}{e}{\tau}$.

\begin{definition}[Typing relation]
  A {\em typing judgement} is a formula $\Tj{\Gamma}{e}{\tau}$ where
  \begin{itemize}
    \item $\Gamma:\Var \pto \Type$, $\dom{\Gamma}$ is finite
    \item $e\in\Exp$ and
    \item $\tau\in\Type$.
  \end{itemize}
\end{definition}

We write $\tj{e}{\tau}$ for $\Tj{[\,]}{e}{\tau}$. The set $\CExp^\tau = \{e\in\Exp\,|\,\tj{e}{\tau}\}$
includes the closed expressions of type $\tau$ and $\CExp = \bigcup_{\tau\in\Type} \CExp^\tau$ includes
all well-typed, closed expressions. Likewise $\CVal^\tau = \CExp^\tau \cap \Val$ includes
the closed values of type $\tau$ and $\CVal = \bigcup_{\tau\in\Type} \CVal^\tau$ includes all well-typed,
closed values.


%%
%% Operational semantics
%%

\section{Operational semantics}

\begin{definition}[Store] \label{definition:Store} \
  \begin{enumerate}
    \item A {\em store} is a (total) function $s:\Loc \to \Int_\bot$.

    \item For each $L\in\powersetfin{\Loc}$ the set of {\em $L$-stores} is defined as
          \[\begin{array}{l}
            \Store_L = \{s\,|\,\text{$s$ is a store with $L\subseteq\dom{s}$}\}.
          \end{array}\]

    \item $\Store = \bigcup_{L\in\powersetfin{\Loc}} \Store_L$ denotes the set of all {\em stores}.
  \end{enumerate}
\end{definition}

\begin{definition}[Configuration]
  Let $e\in\CExp$ and $s\in\Store$. The pair $(e,s)$ is a {\em configuration}
  if $\locns{e} \subseteq \dom{s}$.
\end{definition}

Expressions that appear as part of an evaluation are required to be closed, i.e. do not
include {\em dangling references} to unallocated store cells.

\begin{definition}[Small step semantics]
  Let $e,e'\in\Exp$ and $s,s'\in\Store$. A {\em small step} is a formula
  $(e,s) \to (e',s')$. A small step is said to be valid if it was derived with
  the small rules given {\bf somewhere else}.
\end{definition}

\begin{definition}[Big step semantics]
  Let $e\in\Exp$, $v\in\Val$ and $s,s'\in\Store$. A {\em big step} is a formula
  $(e,s) \Downarrow (v,s')$. A big step is said to be valid if it was derived with the
  big step rules given {\bf somewhere else}.
\end{definition}

\begin{corollary}
  Let $e\in\Exp$, $v\in\Val$ and $s,s'\in\Store$. If $(e,s)$ is a configuration
  and $(e,s) \Downarrow (v,s')$, then
  \begin{enumerate}
    \item $(v,s')$ is also a configuration, and
    \item $\dom{s} \subseteq \dom{s'}$.
  \end{enumerate}
\end{corollary}

\begin{proof}
  Straight forward induction.
\end{proof}

The big step semantic is thereby well-defined with regard to configurations, i.e. it preserves
the configuration property. In the remainder
we thereby consider only pairs $(e,s)$ that are valid configurations. In addition we
assume that type safety holds for the semantics (see Pierce, {\bf TODO}).


%%
%% The assertion language
%%

\chapter{The assertion language}


%%
%% Syntax of the assertion language
%%

\section{Syntax of the assertion language}


%%
%% Semantics of the logic
%%

\section{Semantics of the logic}


%%
%% Semantic domains
%%

\subsection{Semantic domains}

\begin{definition}
  Let $(W,\sqsubseteq)$ be a partial order (of `worlds' $w$).
  A {\em $W$-set} is a set $D$ together with a family of subsets $(D_w)_{w\in W}$ such that
  $D = \bigcup_{w\in W} D_w$ and $w \ne v \Rightarrow D_w \cap D_v = \emptyset$
  for all $v,w \in W$.
\end{definition}

We define the {\em support} of an element $d\in D_w$ to be the world
\[\begin{array}{rcl}
  \supp{d} &=& w
\end{array}\]
The support is guarantied to be unique for every $d \in D$, since the subsets
$(D_w)_{w\in W}$ of $D$ are disjoint.

\begin{definition}
  A {\em $W$-sorted (relation) signature} is a family $\Sigma=(\Sigma^w_n)_{w\in W,n\in \N}$
  of sets $\Sigma^w_n$ such that for all $m,n\in \N$ and $v,w\in W$
  \begin{itemize}
    \item $m \ne n \Rightarrow \Sigma^v_m \cap \Sigma^w_n = \emptyset$
    \item $v \sqsubseteq w \Rightarrow \Sigma^v_n \supseteq \Sigma^w_n$
  \end{itemize}
\end{definition}

An element $r \in \Sigma_n$ is called a {\em relation symbol of arity $n$}. We use the abbreviations
\[\begin{array}{rcl}
  \Sigma_n =_{def} \bigcup_{w\in W} \Sigma^w_n, &
  \Sigma^w =_{def} \bigcup_{n\in \N} \Sigma^w_n, &
  \Sigma =_{def} \bigcup_{w\in W} \Sigma^w.
\end{array}\]

\begin{definition}
  Let $\Sigma$ be a $W$-sorted signature. A {\em $W$-$\Sigma$-set} is a pair
  $(D,\I)$, where $D$ is a $W$-set and $\I$ is a function which maps every
  $r \in \Sigma_n$ to a relation $\I(r) \subseteq D^n$ such that
  $r \in \Sigma^w_n \Rightarrow \delta^n D_w \subseteq \I(r)$ for all $w \in W$.
\end{definition}

We choose $(W,\sqsubseteq) = (\powersetfin{\Loc},\subseteq)$ as the partial order of `worlds'.

\begin{definition}[Ground relation]
  A {\em ground relation of arity $n$} is simply a triple $R = (R^\gamma)_{\gamma\in\{\loc,\sto,\z\}}$ with
  \begin{itemize}
    \item $R^\loc \subseteq \Loc^n$,
    \item $R^\sto \subseteq \Store^n$ and
    \item $R^\z \subseteq \Int^n$.
  \end{itemize}
\end{definition}

\begin{definition}[$L$-definability]
  Let $L \in W$. A ground relation $R$ of arity $n$ is called {\em $L$-definable}, if there
  exists a relation $R_L \subseteq (L \to \Int)^n$ such that
  \begin{enumerate}
    \item $R^\sto = \{\vec{s}\in\Store^n\,|\,(\vec{s}|L)\in R_L
                      \wedge 
                      \vec{s}(\delta^n(\Loc \setminus L)) \subseteq \delta^n \Int\}$
    \item $R^\z = \delta^n \Int$
    \item $R^\loc = \{\vec{l}\in\Loc^n\,|\, \}$

    \item $\vec{l}\in R^\loc$ and $\vec{s}\in R^\sto$ implies
          $\vec{s}\,(\vec{l})\in R^\val \cup \{\bot\}^n$
    \item for every $\tau\in\Type$, $\vec{s}\in\Store^n$, $\vec{l}\in(\Loc^\tau)^n$
          and $\vec{v}\in(\CVal^\tau)^n$: \\
          $\vec{s}\in R^\sto$, $\vec{l}\in R^\loc$ and $\vec{v}\in R^\val$ implies
          $\vec{s}[\vec{v}/\vec{l}]\in R^\sto$
  \end{enumerate}
\end{definition}

We let $\DEF^L_n$ denote the set of all $L$-definable ground relations
of arity $n$ and $\OUT^L_n = \bigcup_{L'\in W\wedge L \cap L' = \emptyset} \DEF^{L'}_n$ denote
the set of those which are definable outside of $L$.

Let $\Sigma = (\Sigma^L_n)_{L\in W,n\in \N}$ where $\Sigma^L_n = \OUT^L_n$. It is trivial to see
that $\Sigma$ is indeed a $W$-sorted signature.

\begin{definition}[Semantic domains]
  For every type $\pi\in\LType$ we define a $W$-$\Sigma$-set $\semantic{\pi}=(D^\pi,\I^\pi)$
  with $D^\pi = \bigcup_{L\in W} W^\pi_L$ by
  \begin{itemize}
    \item $D^\tau_L = \{v\in\CVal^\tau\,|\,\locns{v} = L\}$ \\
          $\I^\tau(R) = \{\vec{v}\in(D^\tau)^n\,|\,\vec{v}\in R^\val\}$

    \item $D^\tassn_L = \{ \phi:\Store\to\Bool_\bot\,|\,
                              \forall R\in\Sigma^L_n.\,\phi\,(R^\sto) \subseteq \delta^n\Bool_\bot \\
                              \hspace*{5cm} \wedge \dom{\phi} = \Store_L
                         \}$ \\
          $\I^\tassn(R) = \{ \vec{\phi} \in (D^\tassn)^n\,|\,
                                  \vec{\phi}\,(R^\sto) \subseteq \delta^n \Bool_\bot \}$

    \item $D^{\ttarrow{\theta}{\pi}}_L = \{ \psi:D^\theta \to D^\pi\,|\,
                              \forall L'\in W.\, \psi\,(D^\theta_{L'}) \subseteq D^\pi_{L \cup L'} \\
                              \hspace*{4.2cm} \wedge \forall R\in\Sigma^L.\, \psi\,(\I^\theta(R))\subseteq\I^\pi(R)\}$ \\
          $\I^{\ttarrow{\theta}{\pi}}(R) = \{ \vec{\psi} \in (D^{\ttarrow{\theta}{\pi}})^n\,|\,
                                          \vec{\psi}\,(\I^\theta(R)) \subseteq \I^\pi(R) \}$
  \end{itemize}
\end{definition}

We follow the usual mathematical convention and use $\semantic{\pi}$ not only as a notation
for the $W$-$\Sigma$-set $(D^\pi,\I^\pi)$ but also for the underlying $W$-set $D^\pi$,
hence $\semantic{\pi}_L$ denotes the set $D^\pi_L$. Moreover, we use $R^\pi$ as an abbreviation
for $\I^\pi(R)$. However we still need to prove that the semantic domains $\semantic{\pi}$
are indeed $W$-$\Sigma$-sets.

\begin{lemma}
  Let $\pi\in\LType$ and $d \in \semantic{\pi}$.
  \begin{enumerate}
    \item $\forall L_1,L_2\in W.\, d \in \semantic{\pi}_{L_1} \cap \semantic{\pi}_{L_2} \Rightarrow L_1 = L_2$
    \item $\forall L \in W, n \in \N, R\in \OUT^L_n.\, d \in \semantic{\pi}_L \Rightarrow (d,\ldots,d) \in R^\pi$
  \end{enumerate}
\end{lemma}

\end{document}

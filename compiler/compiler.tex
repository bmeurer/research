\documentclass[12pt,a4paper]{article}

\usepackage{amsmath}
\usepackage{amssymb}
\usepackage{amstext}
\usepackage{array}
\usepackage[american]{babel}
\usepackage{color}
\usepackage{enumerate}
\usepackage[a4paper,%
            colorlinks=false,%
            final,%
            pdfkeywords={},%
            pdftitle={},%
            pdfauthor={Benedikt Meurer},%
            pdfsubject={},%
            pdfdisplaydoctitle=true]{hyperref}
\usepackage{ifthen}
\usepackage[latin1]{inputenc}
\usepackage{latexsym}
\usepackage[final]{listings}
\usepackage{makeidx}
\usepackage{ngerman}
\usepackage[standard,thmmarks]{ntheorem}
\usepackage{stmaryrd}

%% LaTeX macros
%%
%% macros.tex - Useful LaTeX macros
%%
%% Copyright (c) 2006-2008 Benedikt Meurer <benedikt.meurer@googlemail.com>
%%


%%
%% Generic style macros
%%

\newcommand{\nstyle}[1]{\ensuremath{\mathit{#1}}}
\newcommand{\pstyle}[1]{\ensuremath{\mathsf{#1}}}


%%
%% Misc
%%

\newcommand{\coloncolon}{\mathrel{::}}
\newcommand{\pto}{\hookrightarrow}
\newcommand{\tto}{\rightarrow_t}


%%
%% Sets
%%

\newcommand{\N}{\ensuremath{\mathbb N}}
\newcommand{\Z}{\ensuremath{\mathbb Z}}

\newcommand{\Bool}{\nstyle{Bool}}
\newcommand{\Const}{\nstyle{Const}}
\newcommand{\Exp}{\nstyle{Exp}}
\newcommand{\Member}{\ensuremath{\mathcal{M}}}
\newcommand{\Var}{\ensuremath{\mathcal{X}}}
\newcommand{\Int}{\Z}
\newcommand{\Loc}{\nstyle{Loc}}
\newcommand{\Locfin}{\ensuremath{\Loc_{fin}}}
\newcommand{\Op}{\nstyle{Op}}
\newcommand{\Store}{\nstyle{Store}}
\newcommand{\Type}{\nstyle{Type}}
\newcommand{\Val}{\nstyle{Val}}

\newcommand{\AType}{\nstyle{AType}}
\newcommand{\DType}{\nstyle{DType}}
\newcommand{\LType}{\nstyle{LType}}

\newcommand{\Assn}{\nstyle{Assn}}
\newcommand{\Formula}{\nstyle{Formula}}
\newcommand{\Term}{\nstyle{Term}}

\newcommand{\Perm}{\nstyle{Perm}}


%%
%% Functions (TODO - DeclareMathOperator)
%%

\newcommand{\dom}[1]{\ensuremath{\nstyle{dom}\,(#1)}}
\newcommand{\free}[1]{\ensuremath{\nstyle{free}\,(#1)}}
\newcommand{\locns}[1]{\ensuremath{\nstyle{locns}\,(#1)}}
\newcommand{\grph}[1]{\ensuremath{\nstyle{graph}\,(#1)}}
\newcommand{\supp}[1]{\ensuremath{\nstyle{supp}\,(#1)}}
\newcommand{\reach}[2]{\ensuremath{\nstyle{reach}\,(#1,#2)}}
\newcommand{\reachn}[3]{\ensuremath{\nstyle{reach}_{#1}\,(#2,#3)}}
\newcommand{\semantic}[1]{\ensuremath{\llbracket#1\rrbracket}}

\newcommand{\powerset}[1]{\ensuremath{\wp\,(#1)}}
\newcommand{\powersetfin}[1]{\ensuremath{\wp_{fin}\,(#1)}}


%%
%% Names
%%

\newcommand{\op}{\nstyle{op}}


%%
%% Types
%%

\newcommand{\tassn}{\pstyle{assn}}
\newcommand{\tunit}{\pstyle{unit}}
\newcommand{\tbool}{\pstyle{bool}}
\newcommand{\tint}{\pstyle{int}}
\newcommand{\tarrow}[2]{#1\to#2}
\newcommand{\ttarrow}[2]{#1\tto#2}
\newcommand{\trecord}[1]{\left\{#1\right\}}
\newcommand{\tref}[1]{#1\,\pstyle{ref}}


%%
%% Constants
%%

\newcommand{\true}{\pstyle{true}}
\newcommand{\false}{\pstyle{false}}
\newcommand{\assign}{\pstyle{:=}}
\newcommand{\fix}{\pstyle{fix}}
\newcommand{\cref}{\pstyle{ref}}
\newcommand{\unit}{\pstyle{()}}


%%
%% Expressions
%%

\newcommand{\abstr}[2]{\lambda #1.#2}
\newcommand{\app}[2]{#1\,#2}
\newcommand{\ifte}[3]{\pstyle{if}\,#1\,\pstyle{then}\,#2\,\pstyle{else}\,#3}
\newcommand{\record}[1]{\left\{#1\right\}}
\newcommand{\proj}[2]{#1.#2}

\newcommand{\triple}[3]{\{#1\}\,#2\,\{#3\}}


%%
%% Grammars
%%

\newcommand{\GRbeg}{\begin{array}{rrlll}}
\newcommand{\GRend}{\end{array}}

\newcommand{\GRis}{& ::= &}
\newcommand{\GRal}{\\ & \mid &}
\newcommand{\GRnl}{\vspace{2mm}\\}
\newcommand{\GRmid}{\,\mid\,}
\newcommand{\GRtext}[1]{\hfill & \text{#1}}


%%
%% Rules
%%

\newcommand{\tj}[2]{#1\,\coloncolon\,#2}
\newcommand{\Tj}[3]{#1\,\triangleright\,#2\coloncolon#3}



\begin{document}


\section{Rekursion und Iteration}

Seien $A,B$ disjunkte Mengen und sei $g: A \times B^* \pto A \uplus B$. Dann definieren wir
drei bin"are Relationen
\[\begin{array}{rcl}
  \Downarrow & \subseteq & A \times B \\
  \multimap  & \subseteq & A \times A \\
  \step      & \subseteq & (A \uplus B)^* \times (A \uplus B)^*
\end{array}\]
induktiv durch folgende Regeln (f"ur alle $n \ge 0$):
\begin{enumerate}
\item Wenn $a_i \Downarrow b_i$ f"ur $i=1,\ldots,n$ \\
  und $g(a,b_1 \ldots b_{i-1}) = a_i$ f"ur $i = 1,\ldots,n$ \\
  und $g(a,b_1 \ldots b_n) = b$ \\
  dann gilt $a \Downarrow b$.
\item Wenn $a_i \Downarrow b_i$ f"ur $i=1,\ldots,n$ \\
  und $g(a,b_1 \ldots b_{i-1})=a_i$ f"ur $i=1,\ldots,n$ \\
  dann gilt $a \multimap a_n$.
\item Wenn $g(a,b_1 \ldots b_n) = a'$ \\
  dann gilt $w\,a\,b_1\,\ldots\,b_n \step w\,a\,b_1\,\ldots\,b_n\,a'$.
\item Wenn $g(a,b_1 \ldots b_n) = b$ \\
  dann gilt $w\,a\,b_1\,\ldots\,b_n \step w\,b$.
\end{enumerate}
Um im Folgenden ausschlie"slich \emph{sinnvolle} Semantiken zu erhalten, fordern wir, dass $g$
keine \emph{L"ucken} aufwei"st.
\begin{definition}
  $g: A \times B^* \pto A \uplus B$ hei"st \emph{vollst"andig}, wenn f"ur alle $n \in \N$, $a,a'\in A$
  und $b,b_1,\ldots,b_{n+1} \in B$ gilt: 
  Wenn $g(a,b_1 \ldots b_{n+1}) = a'$ oder $g(a,b_1 \ldots b_{n+1}) = b$ dann
  existieren $a_1',\ldots,a_n' \in A$ so dass $g(a,b_1 \ldots b_i) = a_i'$ f"ur alle $i = 1,\ldots,n$.
\end{definition}


\section{Anwendung auf eine Programmiersprache}

Sei $A = \Exp \times \Env$, $B = \Val$ und $g$ wie folgt definiert:
\[\begin{array}{lcl}
  g(c,\eta) &=& c \\
  g(\abstr{x}{e},\eta) &=& \clov{\abstr{x}{e}}{\eta} \\
  g(\app{e_1}{e_2},\eta) &=& (e_1,\eta) \\
  g(\app{e_1}{e_2},\eta,\clov{\abstr{x'}{e'}}{\eta'}) &=& (e_2,\eta) \\
  g(\app{e_1}{e_2},\eta,\clov{\abstr{x'}{e'}}{\eta'}\,v') &=& (e',x':v';\eta') \\
  g(\app{e_1}{e_2},\eta,\clov{\abstr{x'}{e'}}{\eta'}\,v'\,v) &=& v \\
  g(\ifte{e_0}{e_1}{e_2},\eta) &=& (e_0,\eta) \\
  g(\ifte{e_0}{e_1}{e_2},\eta,\true) &=& (e_1,\eta) \\
  g(\ifte{e_0}{e_1}{e_2},\eta,\true\,v) &=& v \\
  g(\ifte{e_0}{e_1}{e_2},\eta,\false) &=& (e_2,\eta) \\
  g(\ifte{e_0}{e_1}{e_2},\eta,\false\,v) &=& v \\
  \vdots \\
\end{array}\]
Daraus ergibt sich unmittelbar nach obigem Schema eine operationelle Semantik, und zwar
sowohl in rekursiver als auch in iterativer Form.

Die rekursive Form, also die Relation $\Downarrow$ eignet sich hervorragend als Grundlage
f"ur die Implementation eines Interpreters f"ur die Programmiersprache. Wir interessieren
uns an dieser allerdings mehr f"ur die iterative Form und die Anwendung letzterer f"ur
die automatische Erzeugung eines Compilers f"ur die Programmiersprache.

Aus dem oben definierten $g$ ergeben sich unter anderem folgende Regeln f"ur Reduktionen:
\[\begin{array}{lcl}
  w\,(\app{e_1}{e_2},\eta)
  &\step& w\,(\app{e_1}{e_2},\eta)\,(e_1,\eta) \\
  w\,(\app{e_1}{e_2},\eta)\,\clov{\abstr{x'}{e'}}{\eta'}
  &\step& w\,(\app{e_1}{e_2},\eta)\,\clov{\abstr{x'}{e'}}{\eta'}\,(e_2,\eta) \\
  w\,(\app{e_1}{e_2},\eta)\,\clov{\abstr{x'}{e'}}{\eta'}\,v'
  &\step& w\,(\app{e_1}{e_2},\eta)\,\clov{\abstr{x'}{e'}}{\eta'}\,v'\,(e',x':v';\eta') \\
\end{array}\]
Diese Umsetzung einer operationellen Programmiersprachensemantik in Konfigurations"uberg"ange einer Stackmaschine
ist zwar sehr universell und beeindruckend einfach, jedoch gleichzeitig auch sehr naiv und ineffizient.

Die gemeinsame Verwaltung von Ausdruck und Umgebung macht es schwierig bis unm"oglich, sinnvolle Optimierungen,
die sich aus bestimmten Eigenschaften von $g$ ergeben (wie zum Beispiel, dass $e_1$ und $e_2$ mit demselben
$\eta$ ausgewertet werden), anzuwenden. Der erste logische Schritt w"are dementsprechend, Ausdr"ucke und
Umgebungen getrennt voneinander zu verwalten.


\subsection{Umgebungsinvarianz}

Eine einleuchtende L"osung w"are es beispielsweise die Ausdr"ucke, Umgebungen und Werte jeweils auf einem
eigenen Stack zu verwalten, allerdings funktioniert dies nicht so einfach, da man f"ur den Resultatstack eine
Art \emph{Framing}-Mechanismus braucht, um zu wissen, wieviele Werte bereits als Zwischenresultate zum
aktuellen Ausdruck geh"oren. Entsprechend werden wir zun"achst Ausdr"ucke und Resultate auf einem gemeinsamen
Stack verwalten. Wir definieren die Reduktionsrelation
\[\begin{array}{rcl}
  \step & \subseteq & ((\Exp \uplus \Val)^* \times \Env^*) \times ((\Exp \uplus \Val)^* \times \Env^*)
\end{array}\]
als die kleinste solche Relation f"ur die gilt (f"ur alle $n \ge 0$):
\begin{enumerate}
\item Wenn $g(e,\eta,v_1 \ldots v_n) = (e',\eta')$ \\
  dann gilt $(R\,e\,v_1\,\ldots\,v_n,E\,\eta) \step (R\,e\,v_1\,\ldots\,v_n\,e',E\,\eta\,\eta')$.
\item Wenn $g(e,\eta,v_1 \ldots v_n) = v$ \\
  dann gilt $(R\,e\,v_1\,\ldots\,v_n,E\,\eta) \step (R\,v,E)$.
\end{enumerate}
Betrachtet man nun die Regeln f"ur die Applikation
\[\begin{array}{rcl}
  (R\,(\app{e_1}{e_2}),E\,\eta)
  &\step& (R\,(\app{e_1}{e_2})\,e_1,E\,\eta\,\eta) \\
  (R\,(\app{e_1}{e_2})\,\clov{\abstr{x'}{e'}}{\eta'},E\,\eta)
  &\step& (R\,(\app{e_1}{e_2})\,\clov{\abstr{x'}{e'}}{\eta'}\,e_2,E\,\eta\,\eta) \\
  (R\,(\app{e_1}{e_2})\,\clov{\abstr{x'}{e'}}{\eta'}\,v',E\,\eta)
  &\step& (R\,(\app{e_1}{e_2})\,\clov{\abstr{x'}{e'}}{\eta'}\,v'\,e',E\,\eta\,(x':v';\eta')) \\
\end{array}\]
so wird schnell offensichtlich, dass die Verwaltung der Umgebungen $\eta$ keinesfalls optimal ist,
da sowohl f"ur $e_1$ als auch f"ur $e_2$ stets die letzte Umgebung auf dem Umgebungsstack dupliziert
werden muss ("ahnliches zeigt sich bei anderen Programmkonstrukten).

Selbst wenn man davon ausgeht, dass die Verwaltung der Umgebungen auf einem Heap mit geschickter \emph{Garbage
Collection} erfolgt, und somit das Duplizieren von $\eta$ im obigen Beispiel lediglich einem Kopieren eines
Pointers entspricht, so ist der Overhead trotzdem nicht unbedeutend, da in einer realistischen
Programmiersprachensemantik jede zweite Regel ein $\eta$ dupliziert.

Der bisherige Ansatz erlaubt es nicht, das Duplizieren von Umgebungen durch geschickte Optimierung zu
verhindern, da die Invariante f"ur die Auswertung eines Ausdrucks $e$ in einer Umgebung $\eta$ wie
folgt lautet:
\[\begin{array}{rcl}
  (e,\eta) \Downarrow v & \text{gdw.} & (R\,e,E\,\eta) \step^* (R\,v,E)
\end{array}\]
Das hei"st eine Konfigurationsfolge f"ur die Berechnung von $e$ in $\eta$ terminiert mit einem Wert $v$
und entfernt die urspr"ungliche Umgebung $\eta$ vom Stack. Unabh"angig davon, ob $\eta$ schon das letzte
Element in $E$ ist, muss $\eta$ trotzalledem auf den Stack, da sonst die Invariante verletzt w"urde.

Will man also das unn"otige Duplizieren von Umgebungen verhindern, muss man eine andere Invariante formulieren.
Statt einem Ausdruck $e$ das \emph{Aufr"aumen} seiner Umgebung $\eta$ zu "uberlassen, verschiebt man diese
Aufgabe stattdessen in den Ausdruck, der die Auswertung von $e$ angestossen hat. Die Berechnung eines Ausdrucks
$e$ in $\eta$ soll lediglich $e$ durch das Resultat ersetzen, aber den Umgebungsstack in seinem urspr"unglichen
Zustand hinterlassen. Es sollte sich folgende Invariante ergeben:
\[\begin{array}{rcl}
  (e,\eta) \Downarrow v & \text{gdw.} & (R\,e,E\,\eta) \step^* (R\,v,E\,\eta)
\end{array}\]
Eine Relation $\step$, welche diese Invariante erf"ullt, zu definieren, gestaltet sich etwas trickreicher,
da nun zusammengesetzte Ausdr"ucke, wie zum Beispiel Applikationen $\app{e_1}{e_2}$, selbst daf"ur Sorge
tragen m"ussen, dass Umgebungen zur Auswertung der Teilausdr"ucke wieder vom Stack verschwinden (was zuvor
die Berechnung der Teilausdr"ucke erledigt hatte). Wir definieren die neue Reduktionsrelation
\[\begin{array}{rcl}
  \step & \subseteq & ((\Exp \uplus \Val)^* \times \Env^*) \times ((\Exp \uplus \Val)^* \times \Env^*)
\end{array}\]
als die kleinste solche Relation f"ur die gilt (f"ur alle $n \ge 1$):
\begin{enumerate}
\item Wenn $g(e,\eta) = v$ \\
  dann gilt $(R\,e,E\,\eta) \step (R\,v,E\,\eta)$.
\item Wenn $g(e,\eta) = (e',\eta')$ \\
  dann gilt $(R\,e,E\,\eta) \step (R\,e\,e',E\,\eta\,\eta')$.
\item Wenn $g(e,\eta,v_1 \ldots v_n) = v$ \\
  dann gilt $(R\,e\,v_1\,\ldots\,v_n,E\,\eta\,\eta'') \step (R\,v,E\,\eta)$.
\item Wenn $g(e,\eta,v_1 \ldots v_n) = (e',\eta')$ \\
  dann gilt $(R\,e\,v_1\,\ldots\,v_n,E\,\eta\,\eta'') \step (R\,e\,v_1\,\ldots\,v_n\,e',E\,\eta\,\eta')$.
\end{enumerate}
Die erste Regel ist klar und entspricht trivialerweise der Invariante. Die zweite Regel sorgt daf"ur, dass
bei zusammengesetzten Ausdr"ucken die Umgebung f"ur den ersten Teilausdruck auf dem Stack landet, die vierte
Regel tut dies f"ur die weiteren Teilausdr"ucke, indem zun"achst die Umgebung des vorangegangenen Teilausdrucks
entfernt wird und anschliessend die f"ur den n"achsten Teilausdruck auf den Stack gelegt wird. Die dritte
Regel entfernt nach der Berechnung des letzten Teilausdrucks die zugeh"orige Umgebung und liefert das
Ergebnis, so dass am Ende das Ergebnis von $e$ und die urspr"ungliche Umgebung von $e$ oben liegen.

Basierend auf dieser Semantik kann man nun daran gehen, "Uberlegungen anzustellen, wie es anhand 
bestimmter Eigenschaften von $g$ m"oglich sein k"onnte, unn"otiges Duplizieren von Umgebungen zu
vermeiden.

Betrachtet man nun noch einmal beispielhaft die Regeln f"ur die Applikation so ergibt sich folgendes
Bild:
\[\begin{array}{rcl}
  (R\,(\app{e_1}{e_2}),E\,\eta)
  &\step& (R\,(\app{e_1}{e_2})\,e_1,E\,\eta\,\eta) \\
  (R\,(\app{e_1}{e_2})\,\clov{\abstr{x'}{e'}}{\eta'},E\,\eta\,\eta'')
  &\step& (R\,(\app{e_1}{e_2})\,\clov{\abstr{x'}{e'}}{\eta'}\,e_2,E\,\eta\,\eta) \\
  (R\,(\app{e_1}{e_2})\,\clov{\abstr{x'}{e'}}{\eta'}\,v',E\,\eta\,\eta'')
  &\step& (R\,(\app{e_1}{e_2})\,\clov{\abstr{x'}{e'}}{\eta'}\,v'\,e',E\,\eta\,(x':v';\eta')) \\
  (R\,(\app{e_1}{e_2})\,\clov{\abstr{x'}{e'}}{\eta'}\,v'\,v,E\,\eta\,\eta'')
  &\step& (R\,v,E\,\eta) \\
\end{array}\]
Zun"achst erscheint es, als h"atten wir keinen Blumentopf gewonnen. Ruft man sich aber die Invariante
\[\begin{array}{rcl}
  (e,\eta) \Downarrow v & \text{gdw.} & (R\,e,E\,\eta) \step^* (R\,v,E\,\eta)
\end{array}\]
wieder in Erinnerung, so wird sofort ersichtlich, dass in der zweiten und dritten Regel $\eta'' = \eta$
gelten muss, da die erste das urspr"ungliche $\eta$ dupliziert hat, und die Berechnungen f"ur $e_1$ und
$e_2$ den Umgebungsstack im urspr"unglichen Zustand hinterlassen. Das bedeutet, die ersten beiden
Teilberechnungen jeder Applikation erfolgen mit einem Stack, der $\eta$ unn"otigerweise zumindest
doppelt enth"alt. Erst die dritte Teilberechnung erfolgt dann wirklich mit einer neuen Umgebung.
Diese Beobachtung l"asst sich zur"uckverfolgen zur Definition von $g$ f"ur die Applikation. Hier gilt
n"amlich:
\[\begin{array}{lcl}
  g(\app{e_1}{e_2},\eta) &=& (e_1,\eta) \\
  g(\app{e_1}{e_2},\eta,\clov{\abstr{x'}{e'}}{\eta'}) &=& (e_2,\eta) \\
  g(\app{e_1}{e_2},\eta,\clov{\abstr{x'}{e'}}{\eta'}\,v') &=& (e',x':v';\eta') \\
\end{array}\]
$g$ ist also -- Applikationen $\app{e_1}{e_2}$ betreffend -- f"ur $0$ bis $1$ Resultate in gewissem Sinne
\emph{invariant} bzgl. $\eta$. In diesen F"allen wird stets das urspr"ungliche $\eta$ f"ur den n"achsten
auszuwertenden Teilausdruck geliefert. Dies l"asst sich leicht auf beliebige Ausdr"ucke verallgemeinern:
\begin{definition}[Umgebungsinvarianz]
  Zu jedem $g$ und jedem $n$ ist die Menge $\Delta^n(g)$ aller \emph{umgebungsinvarianten Ausdr"ucke $n$-ter Stufe}
  wie folgt definiert:
  \[\begin{array}{c}
    \Delta^n(g) = \{e \mid \forall \eta,\eta',e',v_1,\ldots,v_n.\,g(e,\eta,v_1 \ldots v_n)=(e',\eta') \Rightarrow \eta = \eta'\}
  \end{array}\]
\end{definition}
Ist das $g$ klar aus dem Zusammenhang, so schreiben wir kurz $\Delta^n$ statt $\Delta^n(g)$.
Formal ausgedr"uckt gilt f"ur Applikationen $\app{e_1}{e_2}$ also $(\app{e_1}{e_2}) \in (\Delta^0 \cap \Delta^1)$.

Wie aber k"onnen wir nun die Umgebungsinvarianz ausnutzen, um das Duplizieren von Umgebungen zu vermeiden?
Die Idee hierbei ist im Grunde recht simpel:
\begin{itemize}
\item Liegt ein Ausdruck $e$ nicht in $\Delta^n$, so wird f"ur die $n$-te Teilberechnung wie bisher
  vorgegangen.
\item Liegt ein Ausdruck $e$ in $\Delta^n$, so k"onnen wir f"ur die $n$-te Teilberechnung das urspr"ungliche
  $\eta$ verwenden, welche sich bereits auf dem Stack befindet. Allerdings muss hier beachtet werden, dass
  $e$ nicht zwangsl"aufig in $\Delta^{n-1}$ liegen muss und somit noch die Umgebung der vorherigen
  Teilberechnung auf dem Stack liegen k"onnte, welche dann zun"achst entfernt werden m"usste.
\end{itemize}
Unter Beachtung dieser beiden Punkte l"asst sich die Optimierung wie folgt in der Relation $\step$ umsetzen
(f"ur alle $n \ge 0$):
\begin{enumerate}
\item Wenn $g(e,\eta) = v$ \\
  dann gilt $(R\,e,E\,\eta) \step (R\,v,E\,\eta)$.
\item Wenn $g(e,\eta) = (e',\eta)$ und $e\in\Delta^0$ \\
  dann gilt $(R\,e,E\,\eta) \step (R\,e\,e',E\,\eta)$.
\item Wenn $g(e,\eta) = (e',\eta')$ und $e\not\in\Delta^0$ \\
  dann gilt $(R\,e,E\,\eta) \step (R\,e\,e',E\,\eta\,\eta')$.
\item Wenn $g(e,\eta,v_1 \ldots v_{n+1}) = v$ und $e\in\Delta^n$ \\
  dann gilt $(R\,e\,v_1\,\ldots\,v_{n+1},E\,\eta) \step (R\,v,E\,\eta)$.
\item Wenn $g(e,\eta,v_1 \ldots v_{n+1}) = v$ und $e\not\in\Delta^n$ \\
  dann gilt $(R\,e\,v_1\,\ldots\,v_{n+1},E\,\eta\,\eta'') \step (R\,v,E\,\eta)$.
\item Wenn $g(e,\eta,v_1 \ldots v_{n+1}) = (e',\eta)$ und $e\in\Delta^n$, $e\in\Delta^{n+1}$ \\
  dann gilt $(R\,e\,v_1\,\ldots\,v_{n+1},E\,\eta) \step (R\,e\,v_1\,\ldots\,v_{n+1}\,e',E\,\eta)$.
\item Wenn $g(e,\eta,v_1 \ldots v_{n+1}) = (e',\eta')$ und $e\in\Delta^n$, $e\not\in\Delta^{n+1}$ \\
  dann gilt $(R\,e\,v_1\,\ldots\,v_{n+1},E\,\eta) \step (R\,e\,v_1\,\ldots\,v_{n+1}\,e',E\,\eta\,\eta')$.
\item Wenn $g(e,\eta,v_1 \ldots v_{n+1}) = (e',\eta)$ und $e\not\in\Delta^n$, $e\in\Delta^{n+1}$ \\
  dann gilt $(R\,e\,v_1\,\ldots\,v_{n+1},E\,\eta\,\eta'') \step (R\,e\,v_1\,\ldots\,v_{n+1}\,e',E\,\eta)$.
\item Wenn $g(e,\eta,v_1 \ldots v_{n+1}) = (e',\eta')$ und $e\not\in\Delta^n$, $e\not\in\Delta^{n+1}$ \\
  dann gilt $(R\,e\,v_1\,\ldots\,v_{n+1},E\,\eta\,\eta'') \step (R\,e\,v_1\,\ldots\,v_{n+1}\,e',E\,\eta\,\eta')$.
\end{enumerate}
Betrachten wir wiederum die Regeln f"ur die Applikation, so erhalten wir nun folgendes Regelwerk:
\[\begin{array}{rcl}
  (R\,(\app{e_1}{e_2}),E\,\eta)
  &\step& (R\,(\app{e_1}{e_2})\,e_1,E\,\eta) \\
  (R\,(\app{e_1}{e_2})\,\clov{\abstr{x'}{e'}}{\eta'},E\,\eta)
  &\step& (R\,(\app{e_1}{e_2})\,\clov{\abstr{x'}{e'}}{\eta'}\,e_2,E\,\eta) \\
  (R\,(\app{e_1}{e_2})\,\clov{\abstr{x'}{e'}}{\eta'}\,v',E\,\eta)
  &\step& (R\,(\app{e_1}{e_2})\,\clov{\abstr{x'}{e'}}{\eta'}\,v'\,e',E\,\eta\,(x':v';\eta')) \\
  (R\,(\app{e_1}{e_2})\,\clov{\abstr{x'}{e'}}{\eta'}\,v'\,v,E\,\eta\,\eta'')
  &\step& (R\,v,E\,\eta)
\end{array}\]
Wie sofort zu ersehend ist, findet nun keine Duplikation von Umgebungen mehr statt\footnote{Tats"achlich
k"onnte dies nach wie vor der Fall sein, denn es k"onnte $\eta = (x':v';\eta')$ gelten, aber dies kann
nicht allgemein verhindert werden, denn es l"asst sich nicht unbedingt voraussagen.}. Noch deutlicher
wird dies im Fall von bedingten Ausdr"ucken $\ifte{e_0}{e_1}{e_2}$. Hier ergeben sich folgende Regeln:
\[\begin{array}{rcl}
  (R\,(\ifte{e_0}{e_1}{e_2}),E\,\eta)
  &\step& (R\,(\ifte{e_0}{e_1}{e_2})\,e_0,E\,\eta) \\
  (R\,(\ifte{e_0}{e_1}{e_2})\,\true,E\,\eta)
  &\step& (R\,(\ifte{e_0}{e_1}{e_2})\,\true\,e_1,E\,\eta) \\
  (R\,(\ifte{e_0}{e_1}{e_2})\,\true\,v,E\,\eta)
  &\step& (R\,v,E\,\eta) \\
  (R\,(\ifte{e_0}{e_1}{e_2})\,\true,E\,\eta)
  &\step& (R\,(\ifte{e_0}{e_1}{e_2})\,\false\,e_2,E\,\eta) \\
  (R\,(\ifte{e_0}{e_1}{e_2})\,\false\,v,E\,\eta)
  &\step& (R\,v,E\,\eta) \\
\end{array}\]
Es wird ausschliesslich mit der bereits auf dem Stack befindlichen Umgebung $\eta$ gearbeitet.


\subsubsection{Anmerkungen}

Die "Uberlegungen in diesem Abschnitt betreffen nicht nur die Anwendung auf Programmiersprachen mit einer
Umgebungssemantik, sondern sind allgemein anwendbar f"ur beliebige $g$, sofern die Argumente Paare aus
$A \times C$ sind, und sich im Fall von $g(a,c,b_1 \ldots b_n) = (a',c')$ die zweite Komponente des
Arguments nur selten "andert.


\subsection{Baummethode}

Wie zuvor beschrieben, lassen sich die Ausdr"ucke $e$ und die Resultate $v$ nicht auf getrennten Stacks
verwalten, da man f"ur den Resultatstack eine Art \emph{Framing}-Mechanismus ben"otigt, um zu wissen,
wieviele Werte bereits als Zwischenresultate zum aktuellen Ausdruck geh"oren. Diese Abh"angigkeit
verhindert aber einige interessante Optimierungen. Abhilfe w"urde es schaffen, wenn anhand des aktuellen
Teilausdrucks bestimmt werden k"onnte, wieviele Werte des Resultatstacks bereits als Zwischenergebnisse
zum "ubergeordneten Ausdruck geh"oren.

Betrachten wir dies einmal am Beispiel von bedingten Ausdr"ucken. Wir lassen dabei die Umgebungsinvarianz
zun"achst beiseite um uns auf diesen neuen Sachverhalt konzentrieren zu k"onnen. Hier hatten wir folgende
Regeln:
\[\begin{array}{rcl}
  (R\,(\ifte{e_0}{e_1}{e_2}),E\,\eta)
  &\step& (R\,(\ifte{e_0}{e_1}{e_2})\,e_0,E\,\eta\,\eta) \\
  (R\,(\ifte{e_0}{e_1}{e_2})\,\true,E\,\eta\,\eta')
  &\step& (R\,(\ifte{e_0}{e_1}{e_2})\,\true\,e_1,E\,\eta\,\eta) \\
  (R\,(\ifte{e_0}{e_1}{e_2})\,\true\,v,E\,\eta\,\eta')
  &\step& (R\,v,E\,\eta) \\
  (R\,(\ifte{e_0}{e_1}{e_2})\,\true,E\,\eta\,\eta')
  &\step& (R\,(\ifte{e_0}{e_1}{e_2})\,\false\,e_2,E\,\eta\,\eta) \\
  (R\,(\ifte{e_0}{e_1}{e_2})\,\false\,v,E\,\eta\,\eta')
  &\step& (R\,v,E\,\eta) \\
\end{array}\]
Intuitiv ergibt sich folgender Zusammenhang zwischen dem aktuellen Ausdruck und den Zwischenergebnissen
f"ur bedingte Ausdr"ucke: Steht man vor der Auswertung eines $\ifte{e_0}{e_1}{e_2}$, so beginnt man mit
der Auswertung von $e_0$ und weiss, dass noch kein Zwischenergebnis f"ur den bedingten Ausdruck
vorliegt. Endet nun die Auswertung von $e_0$, d.h. steht man hinter $e_0$, so weiss man, dass nun das
erste Zwischenergebnis vorliegt. Ist dieses $\true$, so beginnt man mit der Auswertung von $e_1$ und
weiss nach wie vor, dass nun ein Zwischenergebnis vorliegt. Sobald $e_1$ ausgewertet ist, man also
hinter $e_1$ steht, weiss man, dass die obersten beiden Zwischenergebnisse zum bedingten Ausdruck
geh"oren. Dann wird das oberste dieser Zwischenergebnisse als Endergebnis geliefert und man steht
effektiv hinter dem bedingten Ausdruck. Sollte $e_0$ mit $\false$ enden, verl"auft es analog.

An dieser Stelle versagt allerdings die abstrakte Syntax. Angenommen $e_0$ und $e_1$ sind beide $\true$,
dann ist es unm"oglich zu sagen, ob man nun hinter $e_0$ oder $e_1$ steht, und somit ist unbestimmt,
ob nun das oberste Resultat bereits ein Zwischenergebnis f"ur den bedingten Ausdruck ist, oder nicht.
Dahinter verbirgt sich das Problem, dass einzelne Knoten im Syntaxbaum durch die abstrakte Syntax
ausschliesslich durch den Ausdruck, den sie repr"asentieren identifizierbar sind, nicht aber durch
die Position, an der sie im Programm stehen. Abhilfe l"asst sich nur dadurch schaffen, dass man
statt mit abstrakter Syntax direkt mit dem Syntaxbaum arbeitet.

\begin{definition}[Markierter Wurzelbaum]
  Ein \emph{markierter Wurzelbaum} \linebreak
  $\mathcal{T} = (\mathcal{K}, \mathcal{E}, \omega, \ell)$ ist ein gerichteter
  Graph $(\mathcal{K}, \mathcal{E})$ mit einem ausgezeichneten Knoten $\omega \in \mathcal{K}$, der
  sogenannten \emph{Wurzel}, f"ur den gilt, dass jeder Knoten durch genau einen gerichteten Pfad von
  $\omega$ aus erreichbar ist. Die Funktion $\ell: \mathcal{K} \to \mathcal{M}$ ordnet dabei jedem
  Knoten eine \emph{Markierung} aus $\mathcal{M}$ zu.
\end{definition}

Ein \emph{Syntaxbaum} $T_e = (\Node,\to,\omega,\ell)$ ist ein spezieller markierter Wurzelbaum, der
eindeutig ein Programm $e$, welches
der abstrakten Syntax gen"ugt, darstellt. Ohne hier auf die konkrete Darstellung einzugehen, nehmen
wir an, dass zu jedem Programmkonstrukt eine eigene Markierung existiert, also z.B. $\textsc{App}$
f"ur Applikationen, $\textsc{Abstr}$ f"ur Abstraktionen, usw. Im Folgenden gehen wir immer von einem
konkreten Syntaxbaum f"ur einen gegebenen Ausdruck $e$ aus, und nehmen an, dass $g$ in geeigneter
Weise auf Knoten des Syntaxbaums definiert ist. Insbesondere nehmen wir an, dass wann immer $g$
ein Paar $(k,\eta)$ liefert, dieses $k$ Bestandteil des Syntaxbaums ist. Wenden wir die naive
Methode an, so erhalten wir dann
\[\begin{array}{rcl}
  \step &\subseteq& ((\Node \uplus \Val)^* \times \Env^*) \times ((\Node \uplus \Val)^* \times \Env^*)
\end{array}\]
als kleinste solche Relation f"ur die gilt (f"ur alle $n \ge 1$):
\begin{enumerate}
\item Wenn $g(k,\eta) = v$ \\
  dann gilt $(R\,k,E\,\eta) \step (R\,v,E\,\eta)$.
\item Wenn $g(k,\eta) = (k',\eta')$ \\
  dann gilt $(R\,k,E\,\eta) \step (R\,k\,k',E\,\eta\,\eta')$.
\item Wenn $g(k,\eta,v_1 \ldots v_n) = v$ \\
  dann gilt $(R\,k\,v_1\,\ldots\,v_n,E\,\eta\,\eta'') \step (R\,v,E\,\eta)$.
\item Wenn $g(k,\eta,v_1 \ldots v_n) = (k',\eta')$ \\
  dann gilt $(R\,k\,v_1\,\ldots\,v_n,E\,\eta\,\eta'') \step (R\,k\,v_1\,\ldots\,v_n\,k',E\,\eta\,\eta')$.
\end{enumerate}
Die Umsetzung entspricht der naiven Umsetzung mittels abstrakter Syntax, insofern haben wir noch keinen
Blumentopf gewonnen. Kommen wir also zur"uck auf die zuvor formulierte Idee, welche uns als Motivation
f"ur die \emph{Baummethode} dient. Statt Knoten (oder Ausdr"ucke) zum Framing im Resultatstack einzusetzen,
sollte es m"oglich sein, anhand der Position des aktuellen Knotens im Syntaxbaum bestimmen zu k"onnen,
wieviel Zwischenergebnisse f"ur den \emph{"ubergeordneten} Knoten bereits zur Verf"ugung stehen. Obwohl
dies im obigen Beispiel von bedingten Ausdr"ucken trivial erscheint, ist dies tats"achlich nicht f"ur alle
denkbaren Semantiken der Fall, wie wir im Folgenden sehen werden.

Nimmt man beispielsweise an, dass ein bestimmter Knoten $k$ im Syntaxbaum einmal zur Auswertung des ersten
Zwischenresultats eines anderen Knotens $k'$ und einmal zur Auswertung des vierten Zwischenresultats von
besagtem Knoten $k'$ verwendet wird, so ist f"ur diesen Knoten $k$ nicht klar, wieviele Zwischenresultate
sich bereits f"ur $k'$ auf dem Resultatstack befinden. Nimmt man aber an, dass dies nicht der Fall ist, und
$g$ dahingehend eindeutig ist, dass f"ur jedes $k$ nur h"ochstens ein $n$ ex., so dass
$g(k,\eta,v_1 \ldots v_n) = (k',\eta')$, so ist anhand des aktuellen Knotens $k$ immer genau zu bestimmen,
wieviele Zwischenergebnisse bereits f"ur $k'$ auf dem Resultatstack liegen.

\begin{definition}
  $g: \Node \times \Env \times \Val^* \pto (\Node \times \Env) \uplus \Val$ hei"st
  \emph{deterministisch}, wenn f"ur alle $n,m,k,k_0,\eta,\eta_0,\eta',\eta_0',v_1,\ldots,v_n,v_1',\ldots,v_m'$
  gilt: \\
  \[g(k,\eta,v_1 \ldots v_n) = (k_0,\eta_0)\]
  und
  \[g(k,\eta',v_1' \ldots v_m') = (k_0,\eta_0')\]
  impliziert $n = m$.
\end{definition}

F"ur ein solch deterministisches $g$ l"asst sich sogar noch mehr sagen. Und zwar ist die maximale Anzahl
von Zwischenresultaten f"ur jeden Knoten beschr"ankt durch die Anzahl der Knoten im Syntaxbaum, welche
wiederum endlich ist.

Im Folgenden gehen wir immer von einem $g$ aus, welches deterministisch ist.\footnote{Man kann sich leicht
"uberlegen, dass das $g$ f"ur die bisher betrachtete Programmiersprachensemantik offensichtlich deterministisch
ist.}. F"ur ein solches $g$ l"asst sich nun leicht die oben beschriebene Idee umsetzen, indem nun Resultate und
Knoten auf zwei unterschiedlichen Stacks verwaltet werden. Hierzu m"ussen wir allerdings zun"achst formalisieren,
was es bedeutet \emph{vor} oder \emph{hinter} einem Knoten $k$ zu sein. Sei dazu
\[\begin{array}{rcl}
  \Node_p &=& \{\cdot k \mid k \in \Node\} \cup \{k \cdot \mid k \in \Node\}
\end{array}\]
eine Menge von \emph{punktierten Knoten} "uber $\Node$. $\cdot k$ gibt hierbei an, dass sich die Maschine
\emph{vor} $k$, und $k \cdot$, dass sie sich \emph{hinter} $k$ befindet. Eine Konfiguration besteht nun
aus einem \emph{Programcounter}, der stets eine punktierten Knoten enth"alt, einem Umgebungsstack,
einem Resultatstack und einem R"ucksprungstack mit Knoten. Die Relation
\[\begin{array}{rcl}
  \step &\subseteq& (\Node_p \times \Env^* \times \Val^* \times \Node^*) \times (\Node_p \times \Env^* \times \Val^* \times \Node^*)
\end{array}\]
ist die kleinste solche Relation f"ur die gilt (f"ur alle $n \ge 0$):
\begin{enumerate}
\item Wenn $g(k,\eta) = v$ \\
  dann gilt $(\cdot k,E\,\eta,R,K) \step (k\cdot,E\,\eta,R\,v,K)$.
\item Wenn $g(k,\eta) = (k',\eta')$ \\
  dann gilt $(\cdot k,E\,\eta,R,K) \step (\cdot k',E\,\eta\,\eta',R,K\,k)$.
\item Wenn $g(k,\eta,v_1 \ldots v_n) = (k',\eta')$ \\
  und $g(k,\eta,v_1 \ldots v_{n+1}) = v$ \\
  dann gilt $(k'\cdot,E\,\eta\,\eta',R\,v_1\,\ldots\,v_{n+1},K\,k) \step (k\cdot,E\,\eta,R\,v,K)$.
\item Wenn $g(k,\eta,v_1 \ldots v_n) = (k',\eta')$ \\
  und $g(k,\eta,v_1 \ldots v_{n+1}) = (k'',\eta'')$ \\
  dann gilt $(k'\cdot,E\,\eta\,\eta',R\,v_1\,\ldots\,v_{n+1},K\,k) \step (\cdot k'',E\,\eta\,\eta'',R\,v_1\,\ldots\,v_{n+1},K\,k)$.
\end{enumerate}
Wie man leicht sieht, sind f"ur deterministische $g$ nun keinerlei Markierung im Resultatstack mehr erforderlich.
Die Anzahl der Zwischenergebnisse zu einem bestimmten $k$ ist eindeutig f"ur jedes $k'$, welches als Teilberechnung
von $k$ auftritt.


\subsubsection{Anmerkungen}

Tats"achlich l"asst sich diese Optimierung auf beliebige $g$ anwenden, die \emph{deterministisch} sind und f"ur
die die Argumente aufgeteilt sind in zwei Teile, wie zuvor bei der Umgebungsinvarianz beschrieben.


\end{document}
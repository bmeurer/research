\documentclass[12pt,a4paper,draft]{article}

\usepackage{amsmath}
\usepackage{amssymb}
\usepackage{amstext}
\usepackage{array}
\usepackage[american]{babel}
\usepackage{color}
\usepackage{enumerate}
\usepackage[a4paper,%
            colorlinks=false,%
            final,%
            pdfkeywords={},%
            pdftitle={},%
            pdfauthor={Benedikt Meurer},%
            pdfsubject={},%
            pdfdisplaydoctitle=true]{hyperref}
\usepackage{ifthen}
\usepackage[latin1]{inputenc}
\usepackage{latexsym}
\usepackage[final]{listings}
\usepackage{makeidx}
\usepackage{ngerman}
\usepackage[standard,thmmarks]{ntheorem}
\usepackage{stmaryrd}

%% LaTeX macros
%%
%% macros.tex - Useful LaTeX macros
%%
%% Copyright (c) 2006-2011 Benedikt Meurer <benedikt.meurer@googlemail.com>
%% 


%%
%% Styles
%%

\newcommand{\nstyle}[1]{\ensuremath{\mathsf{#1}}}
\newcommand{\sstyle}[1]{\ensuremath{\mathit{#1}}}


%%
%% Misc
%%

\newcommand{\abort}{\ensuremath{\mathbf{abort}}}
\newcommand{\pto}{\rightharpoonup}
\newcommand{\step}{\ensuremath{\rightsquigarrow}}

\newcommand{\sem}[1]{\ensuremath{[\![#1]\!]}}


%%
%% Names
%%

\newcommand{\arity}{\ensuremath{\mathit{arity}}}
\newcommand{\cl}{\ensuremath{\mathit{cl}}}
\newcommand{\fr}{\ensuremath{\mathit{fr}}}
\newcommand{\free}{\ensuremath{\mathit{free}}}
\newcommand{\graph}{\ensuremath{\mathit{graph}}}
\newcommand{\id}{\ensuremath{\mathit{id}}}


%%
%% Sets
%%

\newcommand{\I}{\ensuremath{\mathcal I}}
\newcommand{\N}{\ensuremath{\mathbb N}}
\renewcommand{\O}{\ensuremath{\mathcal O}}
\newcommand{\Z}{\ensuremath{\mathbb Z}}
\newcommand{\Cl}{\sstyle{Cl}}
\newcommand{\Env}{\sstyle{Env}}
\newcommand{\Exp}{\sstyle{Exp}}
\newcommand{\Frame}{\sstyle{Frame}}
\newcommand{\Id}{\sstyle{Id}}
\newcommand{\Node}{\sstyle{Node}}
\newcommand{\Val}{\sstyle{Val}}


%%
%% Expressions
%%

\newcommand{\app}[2]{{#1}\,{#2}}
\newcommand{\abstr}[2]{\lambda{#1}.\,{#2}}
\newcommand{\ifte}[3]{\mathbf{if}\,{#1}\,\mathbf{then}\,{#2}\,\mathbf{else}\,{#3}}


%%
%% Values
%%

\newcommand{\clov}[2]{\langle{#1},{#2}\rangle}
\newcommand{\false}{\mathbf{false}}
\newcommand{\true}{\mathbf{true}}


%%
%% Grammars
%%

\newenvironment{grammar}{\begin{array}{rrlll}}{\end{array}}

\newcommand{\is}{& ::= &}
\newcommand{\al}{\\ & \mid &}
\newcommand{\nl}{\vspace{2mm}\\}


%%
%% Other environments
%%

\newenvironment{case}{\left\{\!\!\!\begin{array}{ll}}{\end{array}\right.}


%%% Local Variables: 
%%% mode: latex
%%% TeX-master: "compiler"
%%% End: 


\begin{document}


\section{Rekursion und Iteration}

Seien $A,B$ disjunkte Mengen und sei $g: A \times B^* \pto A \uplus B$. Dann definieren wir
drei bin"are Relationen
\[\begin{array}{rcl}
  \Downarrow & \subseteq & A \times B \\
  \multimap  & \subseteq & A \times A \\
  \step      & \subseteq & (A \uplus B)^* \times (A \uplus B)^*
\end{array}\]
induktiv durch folgende Regeln (f"ur alle $n \ge 0$):
\begin{enumerate}
\item Wenn $a_i \Downarrow b_i$ f"ur $i=1,\ldots,n$ \\
  und $g(a,b_1 \ldots b_{i-1}) = a_i$ f"ur $i = 1,\ldots,n$ \\
  und $g(a,b_1 \ldots b_n) = b$ \\
  dann gilt $a \Downarrow b$.
\item Wenn $a_i \Downarrow b_i$ f"ur $i=1,\ldots,n$ \\
  und $g(a,b_1 \ldots b_{i-1})=a_i$ f"ur $i=1,\ldots,n$ \\
  dann gilt $a \multimap a_n$.
\item Wenn $g(a,b_1 \ldots b_n) = a'$ \\
  dann gilt $w\,a\,b_1\,\ldots\,b_n \step w\,a\,b_1\,\ldots\,b_n\,a'$.
\item Wenn $g(a,b_1 \ldots b_n) = b$ \\
  dann gilt $w\,a\,b_1\,\ldots\,b_n \step w\,b$.
\end{enumerate}
Um im Folgenden ausschlie"slich \emph{sinnvolle} Semantiken zu erhalten, fordern wir, dass $g$
keine \emph{L"ucken} aufwei"st.
\begin{definition}
  $g: A \times B^* \pto A \uplus B$ hei"st \emph{vollst"andig}, wenn f"ur alle $n \in \N$, $a,a'\in A$
  und $b,b_1,\ldots,b_{n+1} \in B$ gilt: 
  Wenn $g(a,b_1 \ldots b_{n+1}) = a'$ oder $g(a,b_1 \ldots b_{n+1}) = b$ dann
  existieren $a_1',\ldots,a_n' \in A$ so dass $g(a,b_1 \ldots b_i) = a_i'$ f"ur alle $i = 1,\ldots,n$.
\end{definition}


\section{Anwendung auf eine Programmiersprache}

Sei $A = \Exp \times \Env$, $B = \Val$ und $g$ wie folgt definiert:
\[\begin{array}{lcl}
  g(c,\eta) &=& c \\
  g(\abstr{x}{e},\eta) &=& \clov{\abstr{x}{e}}{\eta} \\
  g(\app{e_1}{e_2},\eta) &=& (e_1,\eta) \\
  g(\app{e_1}{e_2},\eta,\clov{\abstr{x'}{e'}}{\eta'}) &=& (e_2,\eta) \\
  g(\app{e_1}{e_2},\eta,\clov{\abstr{x'}{e'}}{\eta'}\,v') &=& (e',x':v';\eta') \\
  g(\app{e_1}{e_2},\eta,\clov{\abstr{x'}{e'}}{\eta'}\,v'\,v) &=& v \\
  g(\ifte{e_0}{e_1}{e_2},\eta) &=& (e_0,\eta) \\
  g(\ifte{e_0}{e_1}{e_2},\eta,\true) &=& (e_1,\eta) \\
  g(\ifte{e_0}{e_1}{e_2},\eta,\true\,v) &=& v \\
  g(\ifte{e_0}{e_1}{e_2},\eta,\false) &=& (e_2,\eta) \\
  g(\ifte{e_0}{e_1}{e_2},\eta,\false\,v) &=& v \\
  \vdots \\
\end{array}\]
Daraus ergibt sich unmittelbar nach obigem Schema eine operationelle Semantik, und zwar
sowohl in rekursiver als auch in iterativer Form.

Die rekursive Form, also die Relation $\Downarrow$ eignet sich hervorragend als Grundlage
f"ur die Implementation eines Interpreters f"ur die Programmiersprache. Wir interessieren
uns an dieser allerdings mehr f"ur die iterative Form und die Anwendung letzterer f"ur
die automatische Erzeugung eines Compilers f"ur die Programmiersprache.

Aus dem oben definierten $g$ ergeben sich unter anderem folgende Regeln f"ur Reduktionen:
\[\begin{array}{lcl}
  w\,(\app{e_1}{e_2},\eta)
  &\step& w\,(\app{e_1}{e_2},\eta)\,(e_1,\eta) \\
  w\,(\app{e_1}{e_2},\eta)\,\clov{\abstr{x'}{e'}}{\eta'}
  &\step& w\,(\app{e_1}{e_2},\eta)\,\clov{\abstr{x'}{e'}}{\eta'}\,(e_2,\eta) \\
  w\,(\app{e_1}{e_2},\eta)\,\clov{\abstr{x'}{e'}}{\eta'}\,v'
  &\step& w\,(\app{e_1}{e_2},\eta)\,\clov{\abstr{x'}{e'}}{\eta'}\,v'\,(e',x':v';\eta') \\
\end{array}\]
Diese Umsetzung einer operationellen Programmiersprachensemantik in Konfigurations"uberg"ange einer Stackmaschine
ist zwar sehr universell und beeindruckend einfach, jedoch gleichzeitig auch sehr naiv und ineffizient.

Die gemeinsame Verwaltung von Ausdruck und Umgebung macht es schwierig bis unm"oglich, sinnvolle Optimierungen,
die sich aus bestimmten Eigenschaften von $g$ ergeben (wie zum Beispiel, dass $e_1$ und $e_2$ mit demselben
$\eta$ ausgewertet werden), anzuwenden. Der erste logische Schritt w"are dementsprechend, Ausdr"ucke und
Umgebungen getrennt voneinander zu verwalten.

Eine einleuchtende L"osung w"are es beispielsweise die Ausdr"ucke, Umgebungen und Werte jeweils auf einem
eigenen Stack zu verwalten, allerdings funktioniert dies nicht so einfach, da man f"ur den Resultatstack eine
Art \emph{Framing}-Mechanismus braucht, um zu wissen, wieviele Werte bereits als Zwischenresultate zum
aktuellen Ausdruck geh"oren. Entsprechend werden wir zun"achst Ausdr"ucke und Resultate auf einem gemeinsamen
Stack verwalten. Wir definieren die Reduktionsrelation
\[\begin{array}{rcl}
  \step & \subseteq & ((\Exp \uplus \Val)^* \times \Env^*) \times ((\Exp \uplus \Val)^* \times \Env^*)
\end{array}\]
als die kleinste solche Relation f"ur die gilt (f"ur alle $n \ge 0$):
\begin{enumerate}
\item Wenn $g(e,\eta,v_1 \ldots v_n) = (e',\eta')$ \\
  dann gilt $(R\,e\,v_1\,\ldots\,v_n,E\,\eta) \step (R\,e\,v_1\,\ldots\,v_n\,e',E\,\eta\,\eta')$.
\item Wenn $g(e,\eta,v_1 \ldots v_n) = v$ \\
  dann gilt $(R\,e\,v_1\,\ldots\,v_n,E\,\eta) \step (R\,v,E)$.
\end{enumerate}
Betrachtet man nun die Regeln f"ur die Applikation
\[\begin{array}{rcl}
  (R\,(\app{e_1}{e_2}),E\,\eta)
  &\step& (R\,(\app{e_1}{e_2})\,e_1,E\,\eta\,\eta) \\
  (R\,(\app{e_1}{e_2})\,\clov{\abstr{x'}{e'}}{\eta'},E\,\eta)
  &\step& (R\,(\app{e_1}{e_2})\,\clov{\abstr{x'}{e'}}{\eta'}\,e_2,E\,\eta\,\eta) \\
  (R\,(\app{e_1}{e_2})\,\clov{\abstr{x'}{e'}}{\eta'}\,v',E\,\eta)
  &\step& (R\,(\app{e_1}{e_2})\,\clov{\abstr{x'}{e'}}{\eta'}\,v'\,e',E\,\eta\,(x':v';\eta')) \\
\end{array}\]
so wird schnell offensichtlich, dass die Verwaltung der Umgebungen $\eta$ keinesfalls optimal ist,
da sowohl f"ur $e_1$ als auch f"ur $e_2$ stets die letzte Umgebung auf dem Umgebungsstack dupliziert
werden muss ("ahnliches zeigt sich bei anderen Programmkonstrukten).

Selbst wenn man davon ausgeht, dass die Verwaltung der Umgebungen auf einem Heap mit geschickter \emph{Garbage
Collection} erfolgt, und somit das Duplizieren von $\eta$ im obigen Beispiel lediglich einem Kopieren eines
Pointers entspricht, so ist der Overhead trotzdem nicht unbedeutend, da in einer realistischen
Programmiersprachensemantik jede zweite Regel ein $\eta$ dupliziert.

Der bisherige Ansatz erlaubt es nicht, das Duplizieren von Umgebungen durch geschickte Optimierung zu
verhindern, da die Invariante f"ur die Auswertung eines Ausdrucks $e$ in einer Umgebung $\eta$ wie
folgt lautet:
\[\begin{array}{rcl}
  (e,\eta) \Downarrow v & \text{gdw.} & (R\,e,E\,\eta) \step^* (R\,v,E)
\end{array}\]
Das hei"st eine Konfigurationsfolge f"ur die Berechnung von $e$ in $\eta$ terminiert mit einem Wert $v$
und entfernt die urspr"ungliche Umgebung $\eta$ vom Stack. Unabh"angig davon, ob $\eta$ schon das letzte
Element in $E$ ist, muss $\eta$ trotzalledem auf den Stack, da sonst die Invariante verletzt w"urde.

Will man also das unn"otige Duplizieren von Umgebungen verhindern, muss man eine andere Invariante formulieren.
Statt einem Ausdruck $e$ das \emph{Aufr"aumen} seiner Umgebung $\eta$ zu "uberlassen, verschiebt man diese
Aufgabe stattdessen in den Ausdruck, der die Auswertung von $e$ angestossen hat. Die Berechnung eines Ausdrucks
$e$ in $\eta$ soll lediglich $e$ durch das Resultat ersetzen, aber den Umgebungsstack in seinem urspr"unglichen
Zustand hinterlassen. Es sollte sich folgende Invariante ergeben:
\[\begin{array}{rcl}
  (e,\eta) \Downarrow v & \text{gdw.} & (R\,e,E\,\eta) \step^* (R\,v,E\,\eta)
\end{array}\]
Eine Relation $\step$, welche diese Invariante erf"ullt, zu definieren, gestaltet sich etwas trickreicher,
da nun zusammengesetzte Ausdr"ucke, wie zum Beispiel Applikationen $\app{e_1}{e_2}$, selbst daf"ur Sorge
tragen m"ussen, dass Umgebungen zur Auswertung der Teilausdr"ucke wieder vom Stack verschwinden (was zuvor
die Berechnung der Teilausdr"ucke erledigt hatte). Wir definieren die neue Reduktionsrelation
\[\begin{array}{rcl}
  \step & \subseteq & ((\Exp \uplus \Val)^* \times \Env^*) \times ((\Exp \uplus \Val)^* \times \Env^*)
\end{array}\]
als die kleinste solche Relation f"ur die gilt (f"ur alle $n \ge 1$):
\begin{enumerate}
\item Wenn $g(e,\eta) = v$ \\
  dann gilt $(R\,e,E\,\eta) \step (R\,v,E\,\eta)$.
\item Wenn $g(e,\eta) = (e',\eta')$ \\
  dann gilt $(R\,e,E\,\eta) \step (R\,e\,e',E\,\eta\,\eta')$.
\item Wenn $g(e,\eta,v_1 \ldots v_n) = v$ \\
  dann gilt $(R\,e\,v_1\,\ldots\,v_n,E\,\eta\,\eta'') \step (R\,v,E\,\eta)$.
\item Wenn $g(e,\eta,v_1 \ldots v_n) = (e',\eta')$ \\
  dann gilt $(R\,e\,v_1\,\ldots\,v_n,E\,\eta\,\eta'') \step (R\,e\,v_1\,\ldots\,v_n\,e',E\,\eta\,\eta')$.
\end{enumerate}
Die erste Regel ist klar und entspricht trivialerweise der Invariante. Die zweite Regel sorgt daf"ur, dass
bei zusammengesetzten Ausdr"ucken die Umgebung f"ur den ersten Teilausdruck auf dem Stack landet, die vierte
Regel tut dies f"ur die weiteren Teilausdr"ucke, indem zun"achst die Umgebung des vorangegangenen Teilausdrucks
entfernt wird und anschliessend die f"ur den n"achsten Teilausdruck auf den Stack gelegt wird. Die dritte
Regel entfernt nach der Berechnung des letzten Teilausdrucks die zugeh"orige Umgebung und liefert das
Ergebnis, so dass am Ende das Ergebnis von $e$ und die urspr"ungliche Umgebung von $e$ oben liegen.

Basierend auf dieser Semantik kann man nun daran gehen, "Uberlegungen anzustellen, wie es anhand 
bestimmter Eigenschaften von $g$ m"oglich sein k"onnte, unn"otiges Duplizieren von Umgebungen zu
vermeiden.

Betrachtet man nun noch einmal beispielhaft die Regeln f"ur die Applikation so ergibt sich folgendes
Bild:
\[\begin{array}{rcl}
  (R\,(\app{e_1}{e_2}),E\,\eta)
  &\step& (R\,(\app{e_1}{e_2})\,e_1,E\,\eta\,\eta) \\
  (R\,(\app{e_1}{e_2})\,\clov{\abstr{x'}{e'}}{\eta'},E\,\eta\,\eta'')
  &\step& (R\,(\app{e_1}{e_2})\,\clov{\abstr{x'}{e'}}{\eta'}\,e_2,E\,\eta\,\eta) \\
  (R\,(\app{e_1}{e_2})\,\clov{\abstr{x'}{e'}}{\eta'}\,v',E\,\eta\,\eta'')
  &\step& (R\,(\app{e_1}{e_2})\,\clov{\abstr{x'}{e'}}{\eta'}\,v'\,e',E\,\eta\,(x':v';\eta')) \\
\end{array}\]
Zun"achst erscheint es, als h"atten wir keinen Blumentopf gewonnen. Ruft man sich aber die Invariante
\[\begin{array}{rcl}
  (e,\eta) \Downarrow v & \text{gdw.} & (R\,e,E\,\eta) \step^* (R\,v,E\,\eta)
\end{array}\]
wieder in Erinnerung, so wird sofort ersichtlich, dass in der zweiten und dritten Regel $\eta'' = \eta$
gelten muss, da die erste das urspr"ungliche $\eta$ dupliziert hat, und die Berechnungen f"ur $e_1$ und
$e_2$ den Umgebungsstack im urspr"unglichen Zustand hinterlassen. Das bedeutet, die ersten beiden
Teilberechnungen jeder Applikation erfolgen mit einem Stack, der $\eta$ unn"otigerweise zumindest
doppelt enth"alt. Erst die dritte Teilberechnung erfolgt dann wirklich mit einer neuen Umgebung.
Diese Beobachtung l"asst sich zur"uckverfolgen zur Definition von $g$ f"ur die Applikation. Hier gilt
n"amlich:
\[\begin{array}{lcl}
  g(\app{e_1}{e_2},\eta) &=& (e_1,\eta) \\
  g(\app{e_1}{e_2},\eta,\clov{\abstr{x'}{e'}}{\eta'}) &=& (e_2,\eta) \\
  g(\app{e_1}{e_2},\eta,\clov{\abstr{x'}{e'}}{\eta'}\,v') &=& (e',x':v';\eta') \\
\end{array}\]
$g$ ist also -- Applikationen $\app{e_1}{e_2}$ betreffend -- f"ur $0$ bis $1$ Resultate in gewissem Sinne
\emph{invariant} bzgl. $\eta$. In diesen F"allen wird stets das urspr"ungliche $\eta$ f"ur den n"achsten
auszuwertenden Teilausdruck geliefert. Dies l"asst sich leicht auf beliebige Ausdr"ucke verallgemeinern:
\begin{definition}[Umgebungsinvarianz]
  Zu jedem $g$ und jedem $n$ ist die Menge $\Delta^n(g)$ aller \emph{umgebungsinvarianten Ausdr"ucke $n$-ter Stufe}
  wie folgt definiert:
  \[\begin{array}{c}
    \Delta^n(g) = \{e \mid \forall \eta,\eta',e',v_1,\ldots,v_n.\,g(e,\eta,v_1 \ldots v_n)=(e',\eta') \Rightarrow \eta = \eta'\}
  \end{array}\]
\end{definition}
Ist das $g$ klar aus dem Zusammenhang, so schreiben wir kurz $\Delta^n$ statt $\Delta^n(g)$.
Formal ausgedr"uckt gilt f"ur Applikationen $\app{e_1}{e_2}$ also $(\app{e_1}{e_2}) \in (\Delta^0 \cap \Delta^1)$.

Wie aber k"onnen wir nun die Umgebungsinvarianz ausnutzen, um das Duplizieren von Umgebungen zu vermeiden?
Die Idee hierbei ist im Grunde recht simpel:
\begin{itemize}
\item Liegt ein Ausdruck $e$ nicht in $\Delta^n$, so wird f"ur die $n$-te Teilberechnung wie bisher
  vorgegangen.
\item Liegt ein Ausdruck $e$ in $\Delta^n$, so k"onnen wir f"ur die $n$-te Teilberechnung das urspr"ungliche
  $\eta$ verwenden, welche sich bereits auf dem Stack befindet. Allerdings muss hier beachtet werden, dass
  $e$ nicht zwangsl"aufig in $\Delta^{n-1}$ liegen muss und somit noch die Umgebung der vorherigen
  Teilberechnung auf dem Stack liegen k"onnte, welche dann zun"achst entfernt werden m"usste.
\end{itemize}
Unter Beachtung dieser beiden Punkte l"asst sich die Optimierung wie folgt in der Relation $\step$ umsetzen
(f"ur alle $n \ge 0$):
\begin{enumerate}
\item Wenn $g(e,\eta) = v$ \\
  dann gilt $(R\,e,E\,\eta) \step (R\,v,E\,\eta)$.
\item Wenn $g(e,\eta) = (e',\eta)$ und $e\in\Delta^0$ \\
  dann gilt $(R\,e,E\,\eta) \step (R\,e\,e',E\,\eta)$.
\item Wenn $g(e,\eta) = (e',\eta')$ und $e\not\in\Delta^0$ \\
  dann gilt $(R\,e,E\,\eta) \step (R\,e\,e',E\,\eta\,\eta')$.
\item Wenn $g(e,\eta,v_1 \ldots v_{n+1}) = v$ und $e\in\Delta^n$ \\
  dann gilt $(R\,e\,v_1\,\ldots\,v_{n+1},E\,\eta) \step (R\,v,E\,\eta)$.
\item Wenn $g(e,\eta,v_1 \ldots v_{n+1}) = v$ und $e\not\in\Delta^n$ \\
  dann gilt $(R\,e\,v_1\,\ldots\,v_{n+1},E\,\eta\,\eta'') \step (R\,v,E\,\eta)$.
\item Wenn $g(e,\eta,v_1 \ldots v_{n+1}) = (e',\eta)$ und $e\in\Delta^n$, $e\in\Delta^{n+1}$ \\
  dann gilt $(R\,e\,v_1\,\ldots\,v_{n+1},E\,\eta) \step (R\,e\,v_1\,\ldots\,v_{n+1}\,e',E\,\eta)$.
\item Wenn $g(e,\eta,v_1 \ldots v_{n+1}) = (e',\eta')$ und $e\in\Delta^n$, $e\not\in\Delta^{n+1}$ \\
  dann gilt $(R\,e\,v_1\,\ldots\,v_{n+1},E\,\eta) \step (R\,e\,v_1\,\ldots\,v_{n+1}\,e',E\,\eta\,\eta')$.
\item Wenn $g(e,\eta,v_1 \ldots v_{n+1}) = (e',\eta)$ und $e\not\in\Delta^n$, $e\in\Delta^{n+1}$ \\
  dann gilt $(R\,e\,v_1\,\ldots\,v_{n+1},E\,\eta\,\eta'') \step (R\,e\,v_1\,\ldots\,v_{n+1}\,e',E\,\eta)$.
\item Wenn $g(e,\eta,v_1 \ldots v_{n+1}) = (e',\eta')$ und $e\not\in\Delta^n$, $e\not\in\Delta^{n+1}$ \\
  dann gilt $(R\,e\,v_1\,\ldots\,v_{n+1},E\,\eta\,\eta'') \step (R\,e\,v_1\,\ldots\,v_{n+1}\,e',E\,\eta\,\eta')$.
\end{enumerate}


\end{document}
\documentclass[12pt,a4paper]{article}

\usepackage{amsmath}
\usepackage{amssymb}
\usepackage{amstext}
\usepackage{array}
\usepackage[american]{babel}
\usepackage{color}
\usepackage{enumerate}
\usepackage[a4paper,%
            colorlinks=false,%
            final,%
            pdfkeywords={},%
            pdftitle={},%
            pdfauthor={Benedikt Meurer},%
            pdfsubject={},%
            pdfdisplaydoctitle=true]{hyperref}
\usepackage{ifthen}
\usepackage[latin1]{inputenc}
\usepackage{latexsym}
\usepackage[final]{listings}
\usepackage{makeidx}
\usepackage{ngerman}
\usepackage[standard,thmmarks]{ntheorem}
\usepackage{stmaryrd}

%% LaTeX macros
%%
%% macros.tex - Useful LaTeX macros
%%
%% Copyright (c) 2006-2008 Benedikt Meurer <benedikt.meurer@googlemail.com>
%%


%%
%% Generic style macros
%%

\newcommand{\nstyle}[1]{\ensuremath{\mathit{#1}}}
\newcommand{\pstyle}[1]{\ensuremath{\mathsf{#1}}}


%%
%% Misc
%%

\newcommand{\coloncolon}{\mathrel{::}}
\newcommand{\pto}{\hookrightarrow}
\newcommand{\tto}{\rightarrow_t}


%%
%% Sets
%%

\newcommand{\N}{\ensuremath{\mathbb N}}
\newcommand{\Z}{\ensuremath{\mathbb Z}}

\newcommand{\Bool}{\nstyle{Bool}}
\newcommand{\Const}{\nstyle{Const}}
\newcommand{\Exp}{\nstyle{Exp}}
\newcommand{\Member}{\ensuremath{\mathcal{M}}}
\newcommand{\Var}{\ensuremath{\mathcal{X}}}
\newcommand{\Int}{\Z}
\newcommand{\Loc}{\nstyle{Loc}}
\newcommand{\Locfin}{\ensuremath{\Loc_{fin}}}
\newcommand{\Op}{\nstyle{Op}}
\newcommand{\Store}{\nstyle{Store}}
\newcommand{\Type}{\nstyle{Type}}
\newcommand{\Val}{\nstyle{Val}}

\newcommand{\AType}{\nstyle{AType}}
\newcommand{\DType}{\nstyle{DType}}
\newcommand{\LType}{\nstyle{LType}}

\newcommand{\Assn}{\nstyle{Assn}}
\newcommand{\Formula}{\nstyle{Formula}}
\newcommand{\Term}{\nstyle{Term}}

\newcommand{\Perm}{\nstyle{Perm}}


%%
%% Functions (TODO - DeclareMathOperator)
%%

\newcommand{\dom}[1]{\ensuremath{\nstyle{dom}\,(#1)}}
\newcommand{\free}[1]{\ensuremath{\nstyle{free}\,(#1)}}
\newcommand{\locns}[1]{\ensuremath{\nstyle{locns}\,(#1)}}
\newcommand{\grph}[1]{\ensuremath{\nstyle{graph}\,(#1)}}
\newcommand{\supp}[1]{\ensuremath{\nstyle{supp}\,(#1)}}
\newcommand{\reach}[2]{\ensuremath{\nstyle{reach}\,(#1,#2)}}
\newcommand{\reachn}[3]{\ensuremath{\nstyle{reach}_{#1}\,(#2,#3)}}
\newcommand{\semantic}[1]{\ensuremath{\llbracket#1\rrbracket}}

\newcommand{\powerset}[1]{\ensuremath{\wp\,(#1)}}
\newcommand{\powersetfin}[1]{\ensuremath{\wp_{fin}\,(#1)}}


%%
%% Names
%%

\newcommand{\op}{\nstyle{op}}


%%
%% Types
%%

\newcommand{\tassn}{\pstyle{assn}}
\newcommand{\tunit}{\pstyle{unit}}
\newcommand{\tbool}{\pstyle{bool}}
\newcommand{\tint}{\pstyle{int}}
\newcommand{\tarrow}[2]{#1\to#2}
\newcommand{\ttarrow}[2]{#1\tto#2}
\newcommand{\trecord}[1]{\left\{#1\right\}}
\newcommand{\tref}[1]{#1\,\pstyle{ref}}


%%
%% Constants
%%

\newcommand{\true}{\pstyle{true}}
\newcommand{\false}{\pstyle{false}}
\newcommand{\assign}{\pstyle{:=}}
\newcommand{\fix}{\pstyle{fix}}
\newcommand{\cref}{\pstyle{ref}}
\newcommand{\unit}{\pstyle{()}}


%%
%% Expressions
%%

\newcommand{\abstr}[2]{\lambda #1.#2}
\newcommand{\app}[2]{#1\,#2}
\newcommand{\ifte}[3]{\pstyle{if}\,#1\,\pstyle{then}\,#2\,\pstyle{else}\,#3}
\newcommand{\record}[1]{\left\{#1\right\}}
\newcommand{\proj}[2]{#1.#2}

\newcommand{\triple}[3]{\{#1\}\,#2\,\{#3\}}


%%
%% Grammars
%%

\newcommand{\GRbeg}{\begin{array}{rrlll}}
\newcommand{\GRend}{\end{array}}

\newcommand{\GRis}{& ::= &}
\newcommand{\GRal}{\\ & \mid &}
\newcommand{\GRnl}{\vspace{2mm}\\}
\newcommand{\GRmid}{\,\mid\,}
\newcommand{\GRtext}[1]{\hfill & \text{#1}}


%%
%% Rules
%%

\newcommand{\tj}[2]{#1\,\coloncolon\,#2}
\newcommand{\Tj}[3]{#1\,\triangleright\,#2\coloncolon#3}



\begin{document}


\section{Rekursion und Iteration}

Seien $A,B$ disjunkte Mengen und sei $g: A \times B^* \pto A \uplus B$. Dann definieren wir
drei bin"are Relationen
\[\begin{array}{rcl}
  \Downarrow & \subseteq & A \times B \\
  \multimap  & \subseteq & A \times A \\
  \step      & \subseteq & (A \uplus B)^* \times (A \uplus B)^*
\end{array}\]
induktiv durch folgende Regeln (f"ur alle $n \ge 0$):
\begin{enumerate}
\item Wenn $a_i \Downarrow b_i$ f"ur $i=1,\ldots,n$ \\
  und $g(a,b_1 \ldots b_{i-1}) = a_i$ f"ur $i = 1,\ldots,n$ \\
  und $g(a,b_1 \ldots b_n) = b$ \\
  dann $a \Downarrow b$.
\item Wenn $a_i \Downarrow b_i$ f"ur $i=1,\ldots,n$ \\
  und $g(a,b_1 \ldots b_{i-1})=a_i$ f"ur $i=1,\ldots,n$ \\
  dann $a \multimap a_n$.
\item Wenn $g(a,b_1 \ldots b_n) = a'$ \\
  dann $w\,a\,b_1 \ldots b_n \step w\,a\,b_1 \ldots b_n\,a'$.
\item Wenn $g(a,b_1 \ldots b_n) = b$ \\
  dann $w\,a\,b_1 \ldots b_n \step w\,b$.
\end{enumerate}
Um im Folgenden ausschlie"slich \emph{sinnvolle} Semantiken zu erhalten, fordern wir, dass $g$
keine \emph{L"ucken} aufwei"st.
\begin{definition}
  $g: A \times B^* \pto A \uplus B$ hei"st \emph{vollst"andig}, wenn f"ur alle $n \in \N$, $a,a'\in A$
  und $b,b_1,\ldots,b_{n+1} \in B$ gilt: 
  Wenn $g(a,b_1 \ldots b_{n+1}) = a'$ oder $g(a,b_1 \ldots b_{n+1}) = b$ dann
  existieren $a_1',\ldots,a_n' \in A$ so dass $g(a,b_1 \ldots b_i) = a_i'$ f"ur alle $i = 1,\ldots,n$.
\end{definition}


\section{Offset-eindeutigkeit}

Die naive iterative Umsetzung, die anfangs beschrieben wurde, verwaltet Argumente und Resultate auf einem einzigen
Stack. Grund hierf"ur ist unter anderem das die Argumente f"ur das sogenannte \emph{Framing} ben"otigt werden.
Hinter jedem Argument $a$ findet sich jeweils eine Liste von Resultaten $b_1 \ldots b_n$ bevor sich das n"achste
Argument $a'$ anschlie"st (oder das Ende des Stacks erreicht ist). Diese Resultate $b_1 \ldots b_n$ bilden die
bereits berechneten Zwischenresultate von $a$. Der Teilabschnitt $a\,b_1 \ldots b_n$ des Stacks wird dann als
\emph{Frame von $a$} bezeichnet.

Mittels dieses Mechanismus ist stets eindeutig zu erkennen, wieviele Zwischenresultate von $a$ bereichts
berechnet wurden, und wie anhand dieser weiter vorzugehen ist. W"urde man die Argumente auf einem separaten
Stack verwalten, ohne weitere Markierungen f"ur den Resultatstack, die das Erkennen von Frames erm"oglichen,
einzuf"uhren, so w"are im Allgemeinen eine Zuordnung der Zwischenresultate unm"oglich. Aber erst durch eine
(weitgehend) getrennte Verwaltung von Argumenten und Resultaten werden interessante Optimierungen, wie z.B.
das Weglassen von "uberfl"ussigen Argumenten oder Resultaten, m"oglich.

Betrachten wir dazu zun"achst ein einfaches Beispiel. Sei $g$ wie folgt definiert:
\[\begin{array}{lclclclclcllclc}
  a &\mapsto& a_1 &\quad& a,b &\mapsto& a_2 &\quad& && &\quad& \\
  a_1 &\mapsto& b \\
  a_2 &\mapsto& a_3 && a_2,b &\mapsto& b_1 && a_2,b_2 &\mapsto& a_4 && a_2,b\,b_2 &\mapsto& a_5 \\
  a_3 &\mapsto& b_2 \\
  &\vdots&
\end{array}\]
Die Berechnung von $a$ in der naiven iterativen Umsetzung sieht dann wie folgt aus:
\[\begin{array}{c}
a \step a\,a_1 \step a\,b \step a\,b\,a_2 \step a\,b\,a_2\,a_3 \step a\,b\,a_2\,b_2 \step a\,b\,a_2\,a_4 \step \ldots
\end{array}\]
W"urde man nun Argumente und Resultate auf zwei unabh"angigen Stacks verwalten, so w"urde die
Berechnung f"ur $a$ wie zuvor mit
\[\begin{array}{c}
(a) \step (a\,a_1) \step (a,b) \step (a\,a_2,b)
\end{array}\]
beginnen, aber ab hier w"are es nicht mehr eindeutig, wie weiter zu verfahren ist. Denn f"ur $b$ ist
unklar ob es ein Zwischenresultat von $a$ oder $a_2$ ist. Ersteres w"urde bedeuten, dass
\[\begin{array}{c}
(a\,a_2,b) \step (a\,a_2\,a_3,b)
\end{array}\]
der n"achste Schritt w"are, denn $g(a_2)=a_3$, w"ahrend letzteres bedeuten w"urde, dass mit
\[\begin{array}{c}
(a\,a_2,b) \step (a,b_1)
\end{array}\]
fortgefahren werden m"usste, denn $g(a_2,b)=b_1$. Ausgehend von der Annahme, dass die Berechnung den \emph{richtigen}
Weg -- also den ersten -- nehmen w"urde, erg"abe sich dann nach
\[\begin{array}{c}
(a\,a_2,b) \step (a\,a_2\,a_3,b) \step (a\,a_2,b\,b_2)
\end{array}\]
das n"achste Problem. Hier existieren sogar drei verschiedene M"oglichkeiten fortzufahren, abh"angig
davon, ob man animmt, dass f"ur $a_2$ noch keine Zwischenresultat, ein Zwischenresultat $b_2$ oder zwei
Zwischenresultate $b\,b_2$ vorliegen.

Intuitiv ist jedoch stets klar, wie weiter vorgegangen werden m"usste. F"ur das erste Problem wissen wir,
dass wir in der Berechnung \emph{vor} $a_3$ stehen, d.h. es k"onnen noch keine Zwischenergebnisse f"ur $a_3$
vorliegen. Beim zweiten Problem ist klar, dass sich die Berechnung nun \emph{hinter} $a_3$ befindet und das
Ergebnis ist $b_2$. Da $a_3$ nur dann als Nachfolger von $a_2$ auftreten kann, wenn vorher keine
Zwischenresultate f"ur $a_2$ vorlagen, ergibt sich daraus sofort, dass genau ein Zwischenresultat f"ur $a_2$
vorliegen muss, wenn man \emph{hinter} $a_3$ steht. Formalisieren wir zun"achst, was wir unter Nachfolgern
eines Arguments verstehen:

\begin{definition}[Nachfolger]
  Sei $g: A \times B^* \pto A \uplus B$ und sei $a \in A$. Die Menge aller \emph{Nachfolger} von $a$ ist
  definiert durch
  \[\begin{array}{rcl}
    \children_g(a) = \{a' \in A \mid \exists n \in \N,b_1,\ldots,b_n \in B.\,g(a,b_1 \ldots b_n) = a'\}.
  \end{array}\]
\end{definition}

F"ur jeden Nachfolger $a'$ eines Arguments $a$ existiert also mindestens ein $n \in N$, welches die Anzahl
der zuvor bestimmten Zwischenergebnisse angibt. Diese $n$ bezeichnen wir als die \emph{relativen Offsets}
von $a'$ bzgl. $a$.

\begin{definition}[Relative und absolute Offsets]
  Sei $g: A \times B^* \pto A \uplus B$ und seien $a,a' \in A$. Die Menge aller \emph{relativen Offsets}
  von $a'$ bzgl. $a$ ist definiert durch
  \[\begin{array}{rcl}
    \offsets_g(a,a') &=& \{n \in \N \mid \exists b_1,\ldots,b_n \in B.\, g(a,b_1 \ldots b_n) = a'\},
  \end{array}\]
  und die Menge aller \emph{absoluten Offsets} von $a'$ ist definiert durch
  \[\begin{array}{rcl}
    \offsets_g(a') &=& \bigcup_{a \in A} \offsets_g(a,a').
  \end{array}\]
  Die Funktion $g$ hei"st
  \begin{enumerate}
  \item \emph{relativ offset-eindeutig} falls $|\offsets_g(a,a')| \le 1$ f.a. $a,a' \in A$, bzw.
  \item \emph{(absolut) offset-eindeutig} falls $|\offsets_g(a')| \le 1$ f.a. $a' \in A$.
  \end{enumerate}
\end{definition}

Im obigen Beispiel wissen wir, dass $0$ das einzige relative Offset f"ur $a_3$ bzgl. $a_2$ ist. Dies
erlaubt uns den R"uckschluss, dass nach der Berechnung f"ur $a_3$ exakt ein Zwischenresultat f"ur
$a_2$ vorliegt. Genaugenommen wissen wir sogar, dass die Funktion $g$ aus dem Beispiel -- zumindest
soweit sie angegeben ist -- absolut offset-eindeutig ist, denn f"ur jedes Argument existiert
nur h"ochstens ein absolutes Offset.

Weiterhin m"ussen wir nun noch formalisieren, was es bedeutet, in der Berechnung \emph{vor} oder
\emph{hinter} einem Argument zu stehen. Intuitiv haben wir diese Information aus den vorangegangenen
Konfigurationen abgeleitet. Es ist jedoch im Allgemeinen nicht m"oglich diese Information -- d.h.
ob man \emph{vor} oder \emph{hinter} einem Argument steht -- aus einer einzigen Konfiguration zu
gewinnen. Dementsprechend muss diese Information in Konfigurationen explizit gemacht werden. Dazu
markieren wir jeweils das aktuelle Argument mit einem Punkt vor dem Argument, um deutlich zu machen,
dass die Berechnung \emph{vor} selbigem steht, bzw. hinter dem Argument, um deutlich zu machen, dass
die Berechnung \emph{hinter} dem Argument steht.

\begin{definition}[Punktierte Mengen]
  Sei $M$ eine beliebige Menge. Die Menge $\dot{M}$ aller \emph{punktierten Elemente} aus
  $M$ ist definiert durch
  \[\begin{array}{rcl}
    \dot{M} &=& \{\cdot m \mid m \in M\} \cup \{m \cdot \mid m \in M\}.
  \end{array}\]
\end{definition}

Als Konfigurationen w"ahlen wir Tripel, wobei das erste Element das punktierte aktuelle Argument
enth"alt, das zweite Element den Stack der vorangegangenen Argumente und das dritte Element den
Resultatstack. D.h. $\dot{A} \times A^* \times B^*$ ist die Menge aller m"oglichen Konfigurationen.
Auf dieser Menge definieren wir nun zu jedem relativ offset-eindeutigen $g$ eine bin"are Relation
$\step$ als die kleinste Relation, f"ur die gilt:
\begin{enumerate}
\item Wenn $g(a) = b$ \\
  dann $(\cdot a,v,w) \step (a \cdot,v,w\,b)$.
\item Wenn $g(a) = a'$ \\
  dann $(\cdot a,v,w) \step (\cdot a',v\,a,w)$.
\item Wenn $g(a,b_1 \ldots b_n) = a'$ und $g(a,b_1 \ldots b_{n+1}) = b$ \\
  dann $(a' \cdot,v\,a,w\,b_1 \ldots b_{n+1}) \step (a \cdot,v,w\,b)$.
\item Wenn $g(a,b_1 \ldots b_n) = a'$ und $g(a,b_1 \ldots b_{n+1}) = a''$ \\
  dann $(a' \cdot,v\,a,w\,b_1 \ldots b_{n+1}) \step (\cdot a'',v\,a,w\,b_1 \ldots b_{n+1})$.
\end{enumerate}

Die ersten beiden Regeln beschreiben die m"oglichen Konfigurations"uberg"ange, wenn sich die Berechnung
\emph{vor} einem Argument $a$ befindet. Hier existieren zwei M"oglichkeiten: Entweder liefert $g$ sofort
ein Resultat $b$, dann wird der Punkt hinter das $a$ gesetzt und das Resultat auf den Stack gelegt. Oder
$g$ liefert einen Nachfolger $a'$ f"ur $a$, dann wird $a$ auf den Vorg"angerstack gelegt und die Berechnung
f"ahrt mit $\cdot a'$ fort.

Steht nun die Berechnung hinter $a'$ und ist $a$ der $n$-te Vorg"anger von $a'$, so m"ussen $n+1$
Zwischenresultate f"ur $a$ auf dem Resultatstack liegen, denn $g$ ist nach Voraussetzung relativ
offset-eindeutig. Liefert nun $g(a,b_1 \ldots b_{n+1})$ ein Resultat $b$, so greift die dritte
Regel: Hier werden die Zwischenresultate f"ur $a$ auf dem Resultatstack durch $b$ ersetzt, das
Argument $a$ vom Vorg"angerstack entfernt und anschlie"send die Berechnung hinter $a$ fortgesetzt.
Liefert andererseits $g(a,b_1 \ldots b_{n+1})$ einen weiteren Nachfolger $a''$, so wird die Berechnung
vor $a''$ fortgesetzt.

Anhand dieser Erl"auterungen ist unmittelbar ersichtlich, dass die so definierte "Ubergangsrelation
\emph{deterministisch} ist, d.h. dass zu jeder Konfiguration $\zeta$ nur h"ochstens eine Konfiguration
$\zeta'$ existiert, so dass $\zeta \step \zeta'$ gilt. Weiterhin gilt folgende Berechnungsinvariante:
\begin{lemma}
  Sei $a \in A$ und $b \in B$. Dann gilt
  $(\cdot a,\varepsilon,\varepsilon) \step^* (a \cdot, b, \varepsilon)$
  genau dann wenn $a \Downarrow b$.
\end{lemma}


\section{Vorg"anger-eindeutigkeit}

Die zuvor beschriebene iterative Umsetzung f"ur offset-eindeutige Funktionen $g$ trennt die Argumente
und Resultate, so dass diese weitgehend unabh"angig von einander verwaltet werden k"onnen. Dies schafft
Raum f"ur weitere Optimierungen. Betrachten wir dazu n"amlich noch einmal die Beispielfunktion $g$ aus
dem vorangegangenen Kapitel
\[\begin{array}{lclclclclcllclc}
  a &\mapsto& a_1 &\quad& a,b &\mapsto& a_2 &\quad& && &\quad& \\
  a_1 &\mapsto& b \\
  a_2 &\mapsto& a_3 && a_2,b &\mapsto& b_1 && a_2,b_2 &\mapsto& a_4 && a_2,b\,b_2 &\mapsto& a_5 \\
  a_3 &\mapsto& b_2 \\
  &\vdots&
\end{array}\]
so stellen wir fest, dass zu jedem Nachfolger $a'$ nur h"ochstens ein $a$ existiert, f"ur welches
$a'$ als Ergebnis von $g(a,b_1 \ldots b_n)$ auftreten kann. Dieses $a$ nennen wir dann den
\emph{Vorg"anger} von $a'$. Allgemein definieren wir dazu:

\begin{definition}[Vorg"anger]
  Sei $g: A \times B^* \pto A \uplus B$ und sei $a'\in A$. Die Menge aller \emph{Vorg"anger} von
  $a'$ ist definiert durch
  \[\begin{array}{rcl}
    \parents_g(a') &=& \{a \in A \mid \offsets_g(a,a') \ne \emptyset \}.
  \end{array}\]
  $g$ hei"st \emph{vorg"anger-eindeutig} falls $|\parents_g(a')| \le 1$ f.a. $a' \in A$.
\end{definition}

Im obigen Beispiel existiert -- wie beschrieben -- zu jedem $a'$ h"ochstens ein Vorg"anger $a$.
Die Beispielfunktion ist also offensichtlich vorg"anger-eindeutig (zumindest soweit sie angegeben
ist). Rufen wir uns nun noch einmal die Umsetzung einer offset-eindeutigen Funktion $g$ in eine
Konfigurations"ubergangsrelation $\step$ ins Ged"achtnis, so zeigt sich, dass die Buchf"uhrung
"uber die Vorg"angerargumente f"ur \emph{vorg"anger-eindeutige Funktionen} unn"otig wird, da
die Vorg"anger f"ur alle Argumente eindeutig bestimmt sind.

Dementsprechend kann zu jeder relativ offset-eindeutigen und vorg"anger-eindeutigen Funktion
$g$ eine bin"are Relation $\step$ als die kleinste Relation "uber
$(\dot{A} \times B^*) \times (\dot{A} \times B^*)$ definiert werden, f"ur die gilt:
\begin{enumerate}
\item Wenn $g(a) = b$ \\
  dann $(\cdot a,w) \step (a \cdot,w\,b)$.
\item Wenn $g(a) = a'$ \\
  dann $(\cdot a,w) \step (\cdot a',w)$.
\item Wenn $g(a,b_1 \ldots b_n) = a'$ und $g(a,b_1 \ldots b_{n+1})=b$ \\
  dann $(a' \cdot,w\,b_1 \ldots b_{n+1}) \step (a \cdot,w\,b)$.
\item Wenn $g(a,b_1 \ldots b_n) = a'$ und $g(a,b_1 \ldots b_{n+1})=b$ \\
  dann $(a' \cdot,w\,b_1 \ldots b_{n+1}) \step (\cdot a'',w\,b_1 \ldots b_{n+1})$.
\end{enumerate}
Hierf"ur gen"ugt es tats"achlich zu fordern, dass die Funktion $g$ relativ offset-eindeutig ist, denn
zwischen Vorg"anger- und Offset-eindeutigkeit besteht folgender Zusammenhang:

\begin{lemma}
  Ist eine Funktion $g$ \emph{relativ offset-eindeutig} und \emph{vorg"anger-eindeutig},
  so ist $g$ auch \emph{absolut offset-eindeutig}.
\end{lemma}

Betrachtet man die so definierte Konfigurations"ubergangsrelation $\step$, so ist diese Umsetzung
-- bezogen auf die Verwaltung der Argumente -- optimal. Es wird keinerlei Buchf"uhrung f"ur die
vorangegangenen Argumente ben"otigt, da f"ur jedes Argument $a'$,
\begin{itemize}
\item durch die Vorg"anger-eindeutigkeit von $g$, h"ochstens ein eindeutiger Vorg"anger $a$ existiert, und
\item durch die Offset-eindeutigkeit von $g$, eindeutig bestimmt ist, wieviele Zwischenresultate des
  (eindeutigen) Vorg"angers $a$ bereits auf dem Resultatstack liegen.
\end{itemize}
Insgesamt gen"ugt diese Umsetzung der folgenden Berechnungsinvariante:
\begin{lemma}
  Sei $a \in A$ und $b \in B$. Dann gilt
  $(\cdot a,\varepsilon) \step^* (a \cdot, b)$
  genau dann wenn $a \Downarrow b$.
\end{lemma}

So verlockend diese optimale Umsetzung aussehen mag, sie kommt f"ur die praktische Anwendung, z.B. auf die
Semantik einer Programmiersprache, nicht in Frage, denn die interessanten Funktionen $g$ sind allesamt
nicht vorg"anger-eindeutig. Im Falle der Semantik einer Programmiersprache, scheitert dies schon alleine
an Funktions- bzw. Unterprogrammaufrufen. Dies sind Spr"unge zu einem Programmteil (in der allgemeinen
Terminilogie, zu einem Argument), die von verschiedenen anderen Stellen im Programm erfolgen k"onnen.
F"ur Funktionen und Unterprogramme kann demzufolge im Allgemeinen kein eindeutiger Vorg"anger existieren.

Ein vollst"andiger Verzicht auf Vorg"angerverwaltung ist dementsprechend nicht praktikabel. Allerdings
hat man es in der Semantik von Programmiersprachen trotzdem h"aufig mit Programmteilen zu tun, deren
Vorg"anger eindeutig ist. Dementsprechend w"are eine Hybridl"osung angebracht, die nur diejenigen
Vorg"anger auf einem Argumentstack verwaltet, die nicht eindeutig sind. F"ur die "ubrigen Argumente,
deren Vorg"anger eindeutig ist, also die Menge
\[\begin{array}{c}
  \{a' \in A \mid \exists a \in A.\,\{a\} = \parents_g(a')\}
\end{array}\]
kann die zuvor beschriebene Umsetzung verwendet werden.

Zu jeder absolut offset-eindeutigen Funktion $g$ wird eine bin"are Relation $\step$ als kleinste
"uber $(\dot{A} \times A^* \times B^*) \times (\dot{A} \times A^* \times B^*)$ definiert, f"ur
die gilt:
\begin{enumerate}
\item Wenn $g(a) = b$ \\
  dann $(\cdot a,v,w) \step (a \cdot,v,w\,b)$.
\item Wenn $g(a) = a'$ \\
  und $\{a\} = \parents_g(a')$ \\
  dann $(\cdot a,v,w) \step (\cdot a',v,w)$.
\item Wenn $g(a) = a'$ \\
  und $\{a\} \subset \parents_g(a')$ \\
  dann $(\cdot a,v,w) \step (\cdot a',v\,a,w)$.
\item Wenn $g(a,b_1 \ldots b_n) = a'$ \\
  und $g(a,b_1 \ldots b_{n+1})=b$ \\
  und $\{a\} = \parents_g(a')$ \\
  dann $(a' \cdot,v,w\,b_1 \ldots b_{n+1}) \step (a \cdot,v,w\,b)$.
\item Wenn $g(a,b_1 \ldots b_n) = a'$ \\
  und $g(a,b_1 \ldots b_{n+1}) = b$ \\
  und $\{a\} \subset \parents_g(a')$ \\
  dann $(a' \cdot,v\,a,w\,b_1 \ldots b_{n+1}) \step (a \cdot,v,w\,b)$.
\item Wenn $g(a,b_1 \ldots b_n) = a'$ \\
  und $g(a,b_1 \ldots b_{n+1}) = a''$ \\
  und $\{a\} = \parents_g(a')$ \\
  und $\{a\} = \parents_g(a'')$ \\
  dann $(a' \cdot,v,w\,b_1 \ldots b_{n+1}) \step (a'' \cdot,v,w\,b_1 \ldots b_{n+1})$.
\item Wenn $g(a,b_1 \ldots b_n) = a'$ \\
  und $g(a,b_1 \ldots b_{n+1}) = a''$ \\
  und $\{a\} \subset \parents_g(a')$ \\
  und $\{a\} = \parents_g(a'')$ \\
  dann $(a' \cdot,v\,a,w\,b_1 \ldots b_{n+1}) \step (a'' \cdot,v,w\,b_1 \ldots b_{n+1})$.
\item Wenn $g(a,b_1 \ldots b_n) = a'$ \\
  und $g(a,b_1 \ldots b_{n+1}) = a''$ \\
  und $\{a\} = \parents_g(a')$ \\
  und $\{a\} \subset \parents_g(a'')$ \\
  dann $(a' \cdot,v,w\,b_1 \ldots b_{n+1}) \step (a'' \cdot,v\,a,w\,b_1 \ldots b_{n+1})$.
\item Wenn $g(a,b_1 \ldots b_n) = a'$ \\
  und $g(a,b_1 \ldots b_{n+1}) = a''$ \\
  und $\{a\} \subset \parents_g(a')$ \\
  und $\{a\} \subset \parents_g(a'')$ \\
  dann $(a' \cdot,v\,a,w\,b_1 \ldots b_{n+1}) \step (a'' \cdot,v\,a,w\,b_1 \ldots b_{n+1})$.
\end{enumerate}


\section{Zusammengesetzte Argumente}

In den bisherigen Betrachtungen haben wir Argumente stets als \emph{atomar} betrachtet. In der praktischen
Anwendung hat man es aber h"aufig mit komplexen, \emph{zusammengesetzten} Argumenten zu tun, die aus mehreren
Komponenten bestehen. Hier verhindert die atomare Behandlung interessante Optimierungen f"ur Argumente. Im
Folgenden betrachten wir einige Beispiele, wie durch Auftrennung der Argumentkomponenten die iterative
Umsetzung verbessert werden kann.


\subsection{Seiteneffekte}

Angenommen $A,B$ sind wie zuvor disjunkte Mengen, $C$ sei eine beliebige Menge und
$g: (A \times C) \times (B \times C)^* \pto (A \times C) \uplus (B \times C)$ sei ein
Rekursionsschema. Argumente sind in diesem Fall Paare aus $A \times C$ und Resultate sind
Paare aus $B \times C$, das bedeutet, Elemente aus $C$ treten sowohl als Bestandteil der
Argumente wie auch als Bestandteil der Resultate auf.

\begin{definition}
  $g: (A \times C) \times (B \times C)^* \pto (A \times C) \uplus (B \times C)$ hei"st
  \emph{$C$-seiteneffektbehaftet}, wenn f"ur alle $n \ge 0$ gilt:
  \begin{enumerate}
  \item Wenn $g((a_0,c_0),(b_1,c_1) \ldots (b_n,c_n)) = (b,c)$ \\
    und $g((a_0,c_0'),(b_1,c_1') \ldots (b_n,c_n')) = (b',c')$ \\
    und $c_n = c_n'$, \\
    dann gilt $(b,c) = (b',c')$.
  \item Wenn $g((a_0,c_0),(b_1,c_1) \ldots (b_n,c_n)) = (a,c)$ \\
    und $g((a_0,c_0'),(b_1,c_1') \ldots (b_n,c_n')) = (a',c')$ \\
    und $c_n = c_n'$, \\
    dann gilt $(a,c) = (a',c')$.
  \end{enumerate}
\end{definition}

F"ur eine solche \emph{$C$-seiteneffektbehaftete} Funktion $g$ wird f"ur die weitere Berechnung
immer nur das aktuelle $c$ ben"otigt. Dementsprechend ist es unn"otig, bei der iterativen Umsetzung
f"ur $C$ einen Stack vorzusehen. Als Konfigurationen k"onnen wir stattdessen Tupel aus
$(A \uplus B)^* \times C$ verwenden, wobei die zweite Komponente das aktuelle $c$ enth"alt. Dann
definieren wir $\step$ als die kleinste bin"are Relation zwischen Konfigurationen, f"ur die gilt
$(n \ge 0)$:
\begin{enumerate}
\item Wenn $g((a,c),(b_1,c) \ldots (b_n,c)) = (b',c')$ \\
  dann $(w\,a\,b_1 \ldots b_n,c) \step (w\,b',c')$.
\item Wenn $g((a,c),(b_1,c) \ldots (b_n,c)) = (a',c')$ \\
  dann $(w\,a\,b_1 \ldots b_n,c) \step (w\,a\,b_1 \ldots b_n\,a',c')$.
\end{enumerate}
Offensichtlich gilt f"ur diese Umsetzung nachfolgende Berechnungsinvariante.

\begin{lemma}
  Sei $a \in A$, $b' \in B$ und $c,c' \in C$. Dann gilt $(a,c) \step^* (b',c')$
  genau dann wenn $(a,c) \Downarrow (b',c')$.
\end{lemma}


\section{Zwischenresultatoptimierung}

In den vorangegangenen Abschnitten haben wir unterschiedliche M"oglichkeiten zur effizienten Verwaltung
von Argumenten dargestellt. Dabei haben wir aber die Betrachtung der Zwischenresultate weitgehend ausseracht
gelassen. Dies wollen wir nun in diesem Abschnitt nachholen.

Zun"achst stellen wir dabei fest, dass bisher f"ur jedes Argument $a$ stets alle Zwischenresultate
$b_1,\ldots,b_n$ aufgesammelt werden, bis schliesslich $g(a,b_1 \ldots b_n)$ ein Resultat $b$ liefert.
Dies ungeachtet der Tatsache, ob $b$ nun tats"achlich von allen $b_1,\ldots,b_n$ abh"angt. Ebensowenig
wird ber"ucksichtigt, ob die Nachfolger $a_{i+1}$ von $a$ tats"achlich von allen bis dahin bekannten
Zwischenresultaten $b_1,\ldots,b_i$ abh"angen.

Ist dies nicht der Fall, so k"onnte man Speicherplatz sparen, indem man Zwischenresultate nur solange
aufhebt, wie diese f"ur die weitere Berechnung von Belang sind. Angenommen $b_1$ wird nur ben"otigt,
um zu entscheiden, dass $g(a,b_1)=a_1$ liefert, dann k"onnte man in der iterativen Umsetzung, $b_1$
nach diesem Schritt vom Zwischenresultatstack entfernen. Dies darf nat"urlich nur dann geschehen,
wenn alle folgenden Entscheidungen invariant unter der Wahl von $b_1$ sind, d.h. wenn
$g(a,b_1 b_2\ldots b_k) = g(a,b_1' b_2 \ldots b_k)$ f"ur alle $k \ge 2$. In diesem Fall wissen wir,
dass das erste Zwischenresultat f"ur $a$ nur im ersten Entscheidungsschritt f"ur $a$ ben"otigt
wird\footnote{Die Sprechweise \emph{$n$-ter Entscheidungsschritt f"ur $a$} bezieht sich auf das
Ergebnis von $g(a,b_1 \ldots b_n)$.}.

Formal l"asst sich dieser Sachverhalt durch eine Funktion $\gamma$ beschreiben, die zu jedem Argument $a$
und jedem Zwischenresultatindex $i$ den Index $j$ eines Entscheidungsschrittes f"ur $a$ liefert,
bis zu dem das $i$-te Zwischenresultat (m"oglicherweise) noch ben"otigt wird. Eine solche Funktion
$\gamma$ nennen wir dann \emph{(Zwischenresultat-)Absch"atzung f"ur g}.

\begin{definition}[Zwischenresultatabsch"atzung]
  Sei $g: A \times B^* \pto A \uplus B$. $\gamma: A \times \N \to \N$ hei"st
  \emph{(Zwischenresultat-)Absch"atzung f"ur $g$}, wenn f"ur alle $a \in A$ und $i \ge 1$ gilt:
  \begin{enumerate}
  \item Wenn $\gamma(a,i) = 0$, dann gilt
    \[\begin{array}{rcl}
      g(a,b_1 \ldots b_n) &=& g(a,b_1' \ldots b_n')
    \end{array}\]
    f"ur alle $n \ge i$, $b_1,\ldots,b_n,b_1',\ldots,b_n' \in B$, wobei $b_k = b_k'$ wenn $k \ne i$.
  \item Wenn $\gamma(a,i) = j$, dann gilt $j \ge i$ und
    \[\begin{array}{rcl}
      g(a,b_1 \ldots b_n) &=& g(a,b_1' \ldots b_n')
    \end{array}\]
    f"ur alle $n > j$, $b_1,\ldots,b_n,b_1',\ldots,b_n' \in B$, wobei $b_k = b_k'$ wenn $k \ne i$.
  \end{enumerate}
\end{definition}

Der erste Fall beschreibt dabei Zwischenresultate, die f"ur die weitere Berechnung vollst"andig
irrelevant sind, w"ahrend der zweite Fall Zwischenresultate $b_i$ beschreibt, die nur bis zum
$j$-ten Entscheidungsschritt ben"otigt werden. Vollst"andig irrelevante Zwischenresultate deuten
auf Fehler in $g$ hin und treten bei einer \emph{sinnvoll} formulierten Funktion $g$ nicht auf.
Wir werden diesen deshalb im Folgende keine weitere Beachtung schenken.

Anhand einer Absch"atzung $\gamma$ erh"alt man f"ur jedes Zwischenresultat eine Art \emph{Lebenszeit}.
Diese entspricht dem Entscheidungsschritt, bis zu dem das jeweilige Zwischenresultat laut $\gamma$
mindestens vorzuhalten ist. Umgekehrt lassen sich daraus f"ur jeden Entscheidungsschritt $i$ eines
Arguments $a$ die Indizes derjenigen Zwischenresultate angeben, die einerseits f"ur $i$ vorliegen
m"ussen und andererseits f"ur $i+1$ beibehalten werden m"ussen. Die ersteren nennen wir dabei
die \emph{Eingabe f"ur $(a,i)$} w"ahrend wir die letzteren als \emph{Ausgabe von $(a,i)$} bezeichnen.
Trivialerweise muss die Ausgabe von $(a,i)$ der Eingabe von $(a,i+1)$ entsprechen. Formal l"asst sich
dies wie folgt beschreiben.

\begin{definition}[Ein- und Ausgabe]
  Sei $g: A \times B^* \pto A \uplus B$ und sei weiter $\gamma: A \times \N \to \N$ eine Absch"atzung
  f"ur $g$. Dann sind f"ur alle $a \in A$ und $i \ge 0$ die Mengen $\I_{g,\gamma}(a,i) \subseteq \N$ der
  \emph{Eingaben von $(a,i)$} und $\O_{g,\gamma}(a,i) \subseteq \N$ der \emph{Ausgaben von $(a,i)$} wie folgt
  induktiv definiert:
  \[\begin{array}{lcl}
    \I_{g,\gamma}(a,0)
    &=& \emptyset \\
    \I_{g,\gamma}(a,i+1)
    &=& \O_{g,\gamma}(a,i) \\
    \O_{g,\gamma}(a,i)
    &=& \{j\in(\I_{g,\gamma}(a,i)\cup\{i+1\}) \mid \gamma(a,j) > i \} \\
  \end{array}\]
\end{definition}

Basierend hierauf l"asst sich nun eine \emph{Ein-/Ausgabe-Relation} definieren, die zu jedem Paar $(a,a')$
Eingaben in Form einer Liste von Zwischenresultaten f"ur $a$ auf Ausgaben in Form eines Resultats $b$ oder
eines Nachfolgers $a''$ und einer Liste von Zwischenresultaten, die beizubehalten sind, abbildet.

\begin{definition}[Ein-/Ausgabe-Relation]
  Sei $g: A \times B^* \pto A \uplus B$ und sei $\gamma: A \times \N \to \N$ eine Absch"atzung f"ur $g$.
  Dann ist f"ur alle $a,a' \in A$ mit $a \in \parents_g(a')$ die \emph{Ein-/Ausgabe-Relation f"ur $(a,a')$}
  definiert als die kleinste Relation $\IO_{g,\gamma}(a,a') \subseteq B^* \times (B \uplus (A \times B^*))$
  f"ur die gilt:
  \begin{enumerate}
  \item Wenn $k \in \offsets_g(a,a')$, $g(a,b_1 \ldots b_{k+1}) = b$ und $\I_{g,\gamma}(a,k+1) = \{i_1,\ldots,i_n\}$ mit
    $i_1 < \ldots < i_n$, dann gilt
    \[(b_{i_1} \ldots b_{i_n},b) \in \IO_{g,\gamma}(a,a').\]
  \item Wenn $k \in \offsets_g(a,a')$, $g(a,b_1 \ldots b_{k+1}) = a''$, $\I_{g,\gamma}(a,k) = \{i_1,\ldots,i_n\}$ mit
    $i_1 < \ldots < i_n$, und $\O_{g,\gamma}(a,k+1) = \{j_1,\ldots,j_m\}$ mit
    $j_1 < \ldots < j_m$, dann gilt
    \[(b_{i_1} \ldots b_{i_n},(a'',b_{j_1} \ldots b_{j_m})) \in \IO_{g,\gamma}(a,a').\]
  \end{enumerate}
\end{definition}

Zu jeder relativ offset-eindeutigen Funktion $g$ und jeder Absch"atzung $\gamma$ f"ur $g$ wird dann eine
bin"are Relation $\step$ als kleinste solche Relation "uber
$(\dot{A} \times A^* \times B^*) \times (\dot{A} \times A^* \times B^*)$ definiert, f"ur die gilt
(mit $n \ge 1, m \ge 0$):
\begin{enumerate}
\item Wenn $g(a) = b$ \\
  dann $(\cdot a,v,w) \step (a \cdot,v,w\,b)$.
\item Wenn $g(a) = a'$ \\
  und $\{a\} = \parents_g(a')$ \\
  dann $(\cdot a,v,w) \step (\cdot a',v,w)$.
\item Wenn $g(a) = a'$ \\
  und $\{a\} \subset \parents_g(a')$ \\
  dann $(\cdot a,v,w) \step (\cdot a',v\,a,w)$.
\item Wenn $(b_1 \ldots b_n,b) \in \IO_{g,\gamma}(a,a')$ \\
  und $\{a\} = \parents_g(a')$ \\
  dann $(a' \cdot,v,w\,b_1 \ldots b_n) \step (a \cdot,v,w\,b)$.
\item Wenn $(b_1 \ldots b_n,b) \in \IO_{g,\gamma}(a,a')$ \\
  und $\{a\} \subset \parents_g(a')$ \\
  dann $(a' \cdot,v\,a,w\,b_1 \ldots b_n) \step (a \cdot,v,w\,b)$.
\item Wenn $(b_1 \ldots b_n,(a'',b_1' \ldots b_m')) \in \IO_{g,\gamma}(a,a')$ \\
  und $\{a\} = \parents_g(a')$ \\
  und $\{a\} = \parents_g(a'')$ \\
  dann $(a' \cdot,v,w\,b_1 \ldots b_n) \step (a'' \cdot,v,w\,b_1' \ldots b_m')$.
\item Wenn $(b_1 \ldots b_n,(a'',b_1' \ldots b_m')) \in \IO_{g,\gamma}(a,a')$ \\
  und $\{a\} \subset \parents_g(a')$ \\
  und $\{a\} = \parents_g(a'')$ \\
  dann $(a' \cdot,v\,a,w\,b_1 \ldots b_n) \step (a'' \cdot,v,w\,b_1' \ldots b_m')$.
\item Wenn $(b_1 \ldots b_n,(a'',b_1' \ldots b_m')) \in \IO_{g,\gamma}(a,a')$ \\
  und $\{a\} = \parents_g(a')$ \\
  und $\{a\} \subset \parents_g(a'')$ \\
  dann $(a' \cdot,v,w\,b_1 \ldots b_n) \step (a'' \cdot,v\,a,w\,b_1' \ldots b_m')$.
\item Wenn $(b_1 \ldots b_n,(a'',b_1' \ldots b_m')) \in \IO_{g,\gamma}(a,a')$ \\
  und $\{a\} \subset \parents_g(a')$ \\
  und $\{a\} \subset \parents_g(a'')$ \\
  dann $(a' \cdot,v\,a,w\,b_1 \ldots b_n) \step (a'' \cdot,v\,a,w\,b_1' \ldots b_m')$.
\end{enumerate}


\end{document}
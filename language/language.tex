\documentclass[12pt,a4paper]{article}

\usepackage{amsmath}
\usepackage{amssymb}
\usepackage{amstext}
\usepackage{array}
\usepackage[american]{babel}
\usepackage{color}
\usepackage{enumerate}
\usepackage[a4paper,%
            colorlinks=false,%
            final,%
            pdfkeywords={},%
            pdftitle={},%
            pdfauthor={Benedikt Meurer},%
            pdfsubject={},%
            pdfdisplaydoctitle=true]{hyperref}
\usepackage{ifthen}
\usepackage[latin1]{inputenc}
\usepackage{latexsym}
\usepackage[final]{listings}
\usepackage{makeidx}
\usepackage{ngerman}
\usepackage[standard,thmmarks]{ntheorem}
\usepackage{stmaryrd}

%% LaTeX macros
%%
%% macros.tex - Useful LaTeX macros
%%
%% Copyright (c) 2006-2011 Benedikt Meurer <benedikt.meurer@googlemail.com>
%% 


%%
%% Styles
%%

\newcommand{\nstyle}[1]{\ensuremath{\mathsf{#1}}}
\newcommand{\sstyle}[1]{\ensuremath{\mathit{#1}}}


%%
%% Misc
%%

\newcommand{\abort}{\ensuremath{\mathbf{abort}}}
\newcommand{\pto}{\rightharpoonup}
\newcommand{\step}{\ensuremath{\rightsquigarrow}}

\newcommand{\sem}[1]{\ensuremath{[\![#1]\!]}}


%%
%% Names
%%

\newcommand{\arity}{\ensuremath{\mathit{arity}}}
\newcommand{\cl}{\ensuremath{\mathit{cl}}}
\newcommand{\fr}{\ensuremath{\mathit{fr}}}
\newcommand{\free}{\ensuremath{\mathit{free}}}
\newcommand{\graph}{\ensuremath{\mathit{graph}}}
\newcommand{\id}{\ensuremath{\mathit{id}}}


%%
%% Sets
%%

\newcommand{\I}{\ensuremath{\mathcal I}}
\newcommand{\N}{\ensuremath{\mathbb N}}
\renewcommand{\O}{\ensuremath{\mathcal O}}
\newcommand{\Z}{\ensuremath{\mathbb Z}}
\newcommand{\Cl}{\sstyle{Cl}}
\newcommand{\Env}{\sstyle{Env}}
\newcommand{\Exp}{\sstyle{Exp}}
\newcommand{\Frame}{\sstyle{Frame}}
\newcommand{\Id}{\sstyle{Id}}
\newcommand{\Node}{\sstyle{Node}}
\newcommand{\Val}{\sstyle{Val}}


%%
%% Expressions
%%

\newcommand{\app}[2]{{#1}\,{#2}}
\newcommand{\abstr}[2]{\lambda{#1}.\,{#2}}
\newcommand{\ifte}[3]{\mathbf{if}\,{#1}\,\mathbf{then}\,{#2}\,\mathbf{else}\,{#3}}


%%
%% Values
%%

\newcommand{\clov}[2]{\langle{#1},{#2}\rangle}
\newcommand{\false}{\mathbf{false}}
\newcommand{\true}{\mathbf{true}}


%%
%% Grammars
%%

\newenvironment{grammar}{\begin{array}{rrlll}}{\end{array}}

\newcommand{\is}{& ::= &}
\newcommand{\al}{\\ & \mid &}
\newcommand{\nl}{\vspace{2mm}\\}


%%
%% Other environments
%%

\newenvironment{case}{\left\{\!\!\!\begin{array}{ll}}{\end{array}\right.}


%%% Local Variables: 
%%% mode: latex
%%% TeX-master: "compiler"
%%% End: 


\begin{document}


\section{Semantik von Programmiersprachen}

\begin{definition}[Programmiersprache]
  Eine \emph{Programmiersprache} ist ein Tupel $\L = (\Prog,\Obs,\beh)$. Hierbei ist
  \begin{enumerate}
  \item $\Prog$ die Menge der syntaktisch korrekten \emph{Programme} von $\L$,
  \item $\Obs$ eine Menge von \emph{beobachtbaren Verhalten} f"ur $\L$, und
  \item $\beh: \Prog \to \powerset{\Obs}$ eine Funktion, die jedem $\L$-Programm sein
    beobachtbares Verhalten zuordnet.
  \end{enumerate}
\end{definition}

\begin{definition}[Semantik]
  Sei $\Obs$ eine beliebige Menge von beobachtbaren Verhalten. Eine \emph{$\Obs$-Semantik} ist
  ein Tupel $S = (\Exp,D,\semantic{\cdot},\obs)$, bestehend aus
  \begin{enumerate}
  \item der Menge $\Exp$ von \emph{Ausdr"ucken},
  \item dem \emph{semantischen Bereich} $D$,
  \item der \emph{Semantikfunktion} $\semantic{\cdot}:\Exp \to D$, welche jedem Ausdruck
    ein Element des semantischen Bereichs zuordnet, und
  \item der \emph{Beobachtungsfunktion} $\obs:D \to \powerset{\Obs}$, welche jedem Element
    des semantischen Bereichs sein beobachtbares Verhalten zuordnet.
  \end{enumerate}
\end{definition}

\begin{definition}
  Sei $\L=(\Prog,\Obs,\beh)$ eine beliebige Programmiersprache und $S=(\Exp,D,\semantic{\cdot},\obs)$
  eine $\Obs$-Semantik mit $\Prog \subseteq \Exp$. $S$ hei"st \emph{Semantik f"ur $\L$}, wenn 
  $\obs(\semantic{e}) = \beh(e)$ f"ur alle $e \in \Prog$.
  Die Menge aller Semantiken einer Programmiersprache $\L$ bezeichnen wir mit $\Sem(\L)$.
\end{definition}

\begin{definition}
  Zwei Semantiken $S_1$ und $S_2$ hei"sen \emph{"aquivalent}, geschrieben $S_1 \sim S_2$, wenn
  f"ur alle Programmiersprachen $\L$ gilt: $S_1 \in \Sem(\L)$ genau dann wenn $S_2 \in \Sem(\L)$.
\end{definition}

\begin{lemma}
  Zwei Semantiken $S_1=(\Exp,D_1,\semantic{\cdot}_1,\obs_1)$ und $S_2=(\Exp,D_2,\semantic{\cdot}_2,\obs_2)$,
  welche die gleiche Menge von Ausdr"ucken beschreiben, sind offensichtlich genau dann "aquivalent, wenn
  $\obs_1(\semantic{e}_1)=\obs_2(\semantic{e}_2)$ f"ur alle $e \in \Exp$.
\end{lemma}

\begin{definition}
  Eine Menge $A$ hei"st \emph{induktiv definiert} durch eine Familie $C = (C_n)_{n\in\N}$ von
  \emph{Konstruktoren} $\zeta_n$, wenn $A$ der induktive Abschluss von $C$ ist.
\end{definition}

\begin{definition}[Kompositionalit"at]
  Sei $\Exp$ induktiv definiert durch $C=(C_n)_{n\in\N}$.
  Eine Semantik $S=(\Exp,D,\semantic{\cdot},\obs)$ hei"st \emph{$C$-kompositional},
  wenn zu jedem $\zeta_n \in C_n$ eine Funktion $\semantic{\zeta_n}: D^n \to D$ existiert, so dass gilt:
  Wenn $e = \zeta_n (e_1 \ldots e_n)$, dann 
  $\semantic{e} = \semantic{\zeta_n}(\semantic{e_1} \ldots \semantic{e_n})$.
\end{definition}

\begin{theorem}
  Sei $\Exp$ induktiv definiert durch $C=(C_n)_{n\in\N}$, und seien \linebreak
  $S_1 = (\Exp,D_1,\semantic{\cdot}_1,\obs_1)$ und $S_2 = (\Exp,D_2,\semantic{\cdot}_2,\obs_2)$
  $C$-kompositionale Semantiken. $S_1$ und $S_2$ sind "aquivalent, wenn eine Relation
  $\sim\ \subseteq D_1 \times D_2$ existiert, so dass gilt:
  \begin{enumerate}
  \item Wenn $d_{1,i} \sim d_{2,i}$ f"ur $i=1,\ldots,n$, dann gilt
    \[\semantic{\zeta_n}_1 (d_{1,1} \ldots d_{1,n}) \sim \semantic{\zeta_n}_2 (d_{2,1} \ldots d_{2,n})\]
    f"ur alle $n \ge 0$ und $\zeta_n \in C_n$.
  \item Wenn $d_1 \sim d_2$, dann gilt $\obs(d_1) = \obs(d_2)$ f"ur alle $d_1 \in D_1, d_2 \in D_2$.
  \end{enumerate}
\end{theorem}

\end{document}
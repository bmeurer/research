\documentclass[12pt,a4paper]{article}

\usepackage{amsmath}
\usepackage{amssymb}
\usepackage{amstext}
\usepackage{array}
\usepackage[american]{babel}
\usepackage{color}
\usepackage{enumerate}
\usepackage[a4paper,%
            colorlinks=false,%
            final,%
            pdfkeywords={},%
            pdftitle={},%
            pdfauthor={Benedikt Meurer},%
            pdfsubject={},%
            pdfdisplaydoctitle=true]{hyperref}
\usepackage{ifthen}
\usepackage[latin1]{inputenc}
\usepackage{latexsym}
\usepackage[final]{listings}
\usepackage{makeidx}
\usepackage{ngerman}
\usepackage[standard,thmmarks]{ntheorem}
\usepackage{stmaryrd}

%% LaTeX macros
%%
%% macros.tex - Useful LaTeX macros
%%
%% Copyright (c) 2006-2008 Benedikt Meurer <benedikt.meurer@googlemail.com>
%%


%%
%% Generic style macros
%%

\newcommand{\nstyle}[1]{\ensuremath{\mathit{#1}}}
\newcommand{\pstyle}[1]{\ensuremath{\mathsf{#1}}}


%%
%% Misc
%%

\newcommand{\coloncolon}{\mathrel{::}}
\newcommand{\pto}{\hookrightarrow}
\newcommand{\tto}{\rightarrow_t}


%%
%% Sets
%%

\newcommand{\N}{\ensuremath{\mathbb N}}
\newcommand{\Z}{\ensuremath{\mathbb Z}}

\newcommand{\Bool}{\nstyle{Bool}}
\newcommand{\Const}{\nstyle{Const}}
\newcommand{\Exp}{\nstyle{Exp}}
\newcommand{\Member}{\ensuremath{\mathcal{M}}}
\newcommand{\Var}{\ensuremath{\mathcal{X}}}
\newcommand{\Int}{\Z}
\newcommand{\Loc}{\nstyle{Loc}}
\newcommand{\Locfin}{\ensuremath{\Loc_{fin}}}
\newcommand{\Op}{\nstyle{Op}}
\newcommand{\Store}{\nstyle{Store}}
\newcommand{\Type}{\nstyle{Type}}
\newcommand{\Val}{\nstyle{Val}}

\newcommand{\AType}{\nstyle{AType}}
\newcommand{\DType}{\nstyle{DType}}
\newcommand{\LType}{\nstyle{LType}}

\newcommand{\Assn}{\nstyle{Assn}}
\newcommand{\Formula}{\nstyle{Formula}}
\newcommand{\Term}{\nstyle{Term}}

\newcommand{\Perm}{\nstyle{Perm}}


%%
%% Functions (TODO - DeclareMathOperator)
%%

\newcommand{\dom}[1]{\ensuremath{\nstyle{dom}\,(#1)}}
\newcommand{\free}[1]{\ensuremath{\nstyle{free}\,(#1)}}
\newcommand{\locns}[1]{\ensuremath{\nstyle{locns}\,(#1)}}
\newcommand{\grph}[1]{\ensuremath{\nstyle{graph}\,(#1)}}
\newcommand{\supp}[1]{\ensuremath{\nstyle{supp}\,(#1)}}
\newcommand{\reach}[2]{\ensuremath{\nstyle{reach}\,(#1,#2)}}
\newcommand{\reachn}[3]{\ensuremath{\nstyle{reach}_{#1}\,(#2,#3)}}
\newcommand{\semantic}[1]{\ensuremath{\llbracket#1\rrbracket}}

\newcommand{\powerset}[1]{\ensuremath{\wp\,(#1)}}
\newcommand{\powersetfin}[1]{\ensuremath{\wp_{fin}\,(#1)}}


%%
%% Names
%%

\newcommand{\op}{\nstyle{op}}


%%
%% Types
%%

\newcommand{\tassn}{\pstyle{assn}}
\newcommand{\tunit}{\pstyle{unit}}
\newcommand{\tbool}{\pstyle{bool}}
\newcommand{\tint}{\pstyle{int}}
\newcommand{\tarrow}[2]{#1\to#2}
\newcommand{\ttarrow}[2]{#1\tto#2}
\newcommand{\trecord}[1]{\left\{#1\right\}}
\newcommand{\tref}[1]{#1\,\pstyle{ref}}


%%
%% Constants
%%

\newcommand{\true}{\pstyle{true}}
\newcommand{\false}{\pstyle{false}}
\newcommand{\assign}{\pstyle{:=}}
\newcommand{\fix}{\pstyle{fix}}
\newcommand{\cref}{\pstyle{ref}}
\newcommand{\unit}{\pstyle{()}}


%%
%% Expressions
%%

\newcommand{\abstr}[2]{\lambda #1.#2}
\newcommand{\app}[2]{#1\,#2}
\newcommand{\ifte}[3]{\pstyle{if}\,#1\,\pstyle{then}\,#2\,\pstyle{else}\,#3}
\newcommand{\record}[1]{\left\{#1\right\}}
\newcommand{\proj}[2]{#1.#2}

\newcommand{\triple}[3]{\{#1\}\,#2\,\{#3\}}


%%
%% Grammars
%%

\newcommand{\GRbeg}{\begin{array}{rrlll}}
\newcommand{\GRend}{\end{array}}

\newcommand{\GRis}{& ::= &}
\newcommand{\GRal}{\\ & \mid &}
\newcommand{\GRnl}{\vspace{2mm}\\}
\newcommand{\GRmid}{\,\mid\,}
\newcommand{\GRtext}[1]{\hfill & \text{#1}}


%%
%% Rules
%%

\newcommand{\tj}[2]{#1\,\coloncolon\,#2}
\newcommand{\Tj}[3]{#1\,\triangleright\,#2\coloncolon#3}



\begin{document}


\section{Semantik von Programmiersprachen}

\begin{definition}[Syntax]
  Eine \emph{Syntax} ist eine Menge $\L$ von syntaktisch korrekten Programmen.
\end{definition}

\begin{definition}[Semantik]
  Sei $\L$ eine Syntax. Eine \emph{$\L$-Semantik} ist ein Paar $S = (\Obs,\beh)$.
  Hierbei ist
  \begin{enumerate}
  \item $\Obs$ eine Menge von \emph{beobachtbaren Verhalten}, und
  \item $\beh: \L \to \powerset{\Obs}$ eine \emph{Beobachtungsfunktion}, die jedem $\L$-Programm
    beobachtbares Verhalten zuordnet.
  \end{enumerate}
\end{definition}

\begin{definition}[Implementierung]
  Sei $\L$ eine Syntax und sei $\Obs$ eine Menge von beobachtbaren Verhalten.
  Eine \emph{$(\L,\Obs)$-Implementierung} ist ein Tupel $M = (\Exp,D,\semantic{\cdot},\obs)$,
  bestehend aus
  \begin{enumerate}
  \item der Menge $\Exp \supseteq \L$ von \emph{Ausdr"ucken},
  \item dem \emph{semantischen Bereich} $D$,
  \item der \emph{Semantikfunktion} $\semantic{\cdot}:\Exp \to D$, welche jedem Ausdruck
    ein Element des semantischen Bereichs zuordnet, und
  \item der \emph{Bewertungsfunktion} $\obs:D \to \powerset{\Obs}$, welche jedes Element
    des semantischen Bereichs als beobachtbares Verhalten bewertet.
  \end{enumerate}
  Eine Implementierung hei"st \emph{ad"aquat} f"ur eine $\L$-Semantik $S=(\Obs,\beh)$,
  wenn $\obs(\semantic{e}) = \beh(e)$ f"ur alle $e \in \L$ gilt.
\end{definition}

\begin{lemma}
  Sei $(\L,S)$ eine Programmiersprache mit $S = (\Obs,\beh)$ und $M_1 = (\Exp,D_1,\semantic{\cdot}_1,\obs_1)$
  eine ad"aquate $(\L,\Obs)$-Implementierung f"ur $S$.
  Eine weitere $(\L,\Obs)$-Implementierung $M_2=(\Exp,D_2,\semantic{\cdot}_2,\obs_2)$
  ist ad"aquat f"ur $S$, wenn $\obs_1(\semantic{e}_1) = \obs_2(\semantic{e}_2)$ f"ur alle $e \in \Exp$.
\end{lemma}

\begin{definition}
  Eine Menge $A$ hei"st \emph{induktiv definiert} durch eine Familie $C = (C_n)_{n\in\N}$ von
  \emph{Konstruktoren} $\zeta_n$, wenn $A$ der induktive Abschluss von $C$ ist.
\end{definition}

\begin{definition}[Kompositionalit"at]
  Sei $\Exp$ induktiv definiert durch die Familie $C=(C_n)_{n\in\N}$.
  Eine Implementierung $M=(\Exp,D,\semantic{\cdot},\obs)$ hei"st \emph{$C$-kompositional},
  wenn zu jedem $\zeta_n \in C_n$ eine Funktion $\semantic{\zeta_n}: D^n \to D$ existiert, so dass gilt:
  Wenn $e = \zeta_n (e_1 \ldots e_n)$, dann 
  $\semantic{e} = \semantic{\zeta_n}(\semantic{e_1} \ldots \semantic{e_n})$.
\end{definition}

\begin{theorem}
  Sei $(\L,S)$ eine Programmiersprache mit $S=(\Obs,\beh)$, $\Exp$ induktiv definiert durch
  $C=(C_n)_{n\in\N}$, und seien $M_1=(\Exp,D_1,\semantic{\cdot}_1,\obs_1)$ und
  $M_2 = (\Exp,D_2,\semantic{\cdot}_2,\obs_2)$ $C$-kompositionale $(\L,\Obs)$-Implementierungen.
  $M_2$ ist ad"aquat f"ur $S$, wenn $M_1$ ad"aquat f"ur $S$ und eine Relation $\sim\ \subseteq D_1 \times D_2$
  existiert, so dass gilt:
  \begin{enumerate}
  \item Wenn $d_{1,i} \sim d_{2,i}$ f"ur $i=1,\ldots,n$, dann gilt
    \[\semantic{\zeta_n}_1 (d_{1,1} \ldots d_{1,n}) \sim \semantic{\zeta_n}_2 (d_{2,1} \ldots d_{2,n})\]
    f"ur alle $n \ge 0$ und $\zeta_n \in C_n$.
  \item Wenn $d_1 \sim d_2$, dann gilt $\obs_1(d_1) = \obs_2(d_2)$.
  \end{enumerate}
\end{theorem}

\end{document}
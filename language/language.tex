\documentclass[12pt,a4paper]{article}

\usepackage{amsmath}
\usepackage{amssymb}
\usepackage{amstext}
\usepackage{array}
\usepackage[american]{babel}
\usepackage{color}
\usepackage{enumerate}
\usepackage[a4paper,%
            colorlinks=false,%
            final,%
            pdfkeywords={},%
            pdftitle={},%
            pdfauthor={Benedikt Meurer},%
            pdfsubject={},%
            pdfdisplaydoctitle=true]{hyperref}
\usepackage{ifthen}
\usepackage[latin1]{inputenc}
\usepackage{latexsym}
\usepackage[final]{listings}
\usepackage{makeidx}
\usepackage{ngerman}
\usepackage[standard,thmmarks]{ntheorem}
\usepackage{stmaryrd}

%% LaTeX macros
%%
%% macros.tex - Useful LaTeX macros
%%
%% Copyright (c) 2006-2008 Benedikt Meurer <benedikt.meurer@googlemail.com>
%%


%%
%% Generic style macros
%%

\newcommand{\nstyle}[1]{\ensuremath{\mathit{#1}}}
\newcommand{\pstyle}[1]{\ensuremath{\mathsf{#1}}}


%%
%% Misc
%%

\newcommand{\coloncolon}{\mathrel{::}}
\newcommand{\pto}{\hookrightarrow}
\newcommand{\tto}{\rightarrow_t}


%%
%% Sets
%%

\newcommand{\N}{\ensuremath{\mathbb N}}
\newcommand{\Z}{\ensuremath{\mathbb Z}}

\newcommand{\Bool}{\nstyle{Bool}}
\newcommand{\Const}{\nstyle{Const}}
\newcommand{\Exp}{\nstyle{Exp}}
\newcommand{\Member}{\ensuremath{\mathcal{M}}}
\newcommand{\Var}{\ensuremath{\mathcal{X}}}
\newcommand{\Int}{\Z}
\newcommand{\Loc}{\nstyle{Loc}}
\newcommand{\Locfin}{\ensuremath{\Loc_{fin}}}
\newcommand{\Op}{\nstyle{Op}}
\newcommand{\Store}{\nstyle{Store}}
\newcommand{\Type}{\nstyle{Type}}
\newcommand{\Val}{\nstyle{Val}}

\newcommand{\AType}{\nstyle{AType}}
\newcommand{\DType}{\nstyle{DType}}
\newcommand{\LType}{\nstyle{LType}}

\newcommand{\Assn}{\nstyle{Assn}}
\newcommand{\Formula}{\nstyle{Formula}}
\newcommand{\Term}{\nstyle{Term}}

\newcommand{\Perm}{\nstyle{Perm}}


%%
%% Functions (TODO - DeclareMathOperator)
%%

\newcommand{\dom}[1]{\ensuremath{\nstyle{dom}\,(#1)}}
\newcommand{\free}[1]{\ensuremath{\nstyle{free}\,(#1)}}
\newcommand{\locns}[1]{\ensuremath{\nstyle{locns}\,(#1)}}
\newcommand{\grph}[1]{\ensuremath{\nstyle{graph}\,(#1)}}
\newcommand{\supp}[1]{\ensuremath{\nstyle{supp}\,(#1)}}
\newcommand{\reach}[2]{\ensuremath{\nstyle{reach}\,(#1,#2)}}
\newcommand{\reachn}[3]{\ensuremath{\nstyle{reach}_{#1}\,(#2,#3)}}
\newcommand{\semantic}[1]{\ensuremath{\llbracket#1\rrbracket}}

\newcommand{\powerset}[1]{\ensuremath{\wp\,(#1)}}
\newcommand{\powersetfin}[1]{\ensuremath{\wp_{fin}\,(#1)}}


%%
%% Names
%%

\newcommand{\op}{\nstyle{op}}


%%
%% Types
%%

\newcommand{\tassn}{\pstyle{assn}}
\newcommand{\tunit}{\pstyle{unit}}
\newcommand{\tbool}{\pstyle{bool}}
\newcommand{\tint}{\pstyle{int}}
\newcommand{\tarrow}[2]{#1\to#2}
\newcommand{\ttarrow}[2]{#1\tto#2}
\newcommand{\trecord}[1]{\left\{#1\right\}}
\newcommand{\tref}[1]{#1\,\pstyle{ref}}


%%
%% Constants
%%

\newcommand{\true}{\pstyle{true}}
\newcommand{\false}{\pstyle{false}}
\newcommand{\assign}{\pstyle{:=}}
\newcommand{\fix}{\pstyle{fix}}
\newcommand{\cref}{\pstyle{ref}}
\newcommand{\unit}{\pstyle{()}}


%%
%% Expressions
%%

\newcommand{\abstr}[2]{\lambda #1.#2}
\newcommand{\app}[2]{#1\,#2}
\newcommand{\ifte}[3]{\pstyle{if}\,#1\,\pstyle{then}\,#2\,\pstyle{else}\,#3}
\newcommand{\record}[1]{\left\{#1\right\}}
\newcommand{\proj}[2]{#1.#2}

\newcommand{\triple}[3]{\{#1\}\,#2\,\{#3\}}


%%
%% Grammars
%%

\newcommand{\GRbeg}{\begin{array}{rrlll}}
\newcommand{\GRend}{\end{array}}

\newcommand{\GRis}{& ::= &}
\newcommand{\GRal}{\\ & \mid &}
\newcommand{\GRnl}{\vspace{2mm}\\}
\newcommand{\GRmid}{\,\mid\,}
\newcommand{\GRtext}[1]{\hfill & \text{#1}}


%%
%% Rules
%%

\newcommand{\tj}[2]{#1\,\coloncolon\,#2}
\newcommand{\Tj}[3]{#1\,\triangleright\,#2\coloncolon#3}



\begin{document}


\section{Semantik von Programmiersprachen}

\begin{definition}[Programmiersprache]
  Eine \emph{Programmiersprache} ist ein Tupel $\L = (\Prog,\Obs,\beh)$. Hierbei ist
  \begin{enumerate}
  \item $\Prog$ die Menge der syntaktisch korrekten \emph{Programme} von $\L$,
  \item $\Obs$ eine Menge von \emph{beobachtbaren Verhalten} f"ur $\L$, und
  \item $\beh: \Prog \to \powerset{\Obs}$ eine Funktion, die jedem $\L$-Programm sein
    beobachtbares Verhalten zuordnet.
  \end{enumerate}
\end{definition}

\begin{definition}[Semantik]
  Sei $\Obs$ eine beliebige Menge von beobachtbaren Verhalten. Eine \emph{$\Obs$-Semantik} ist
  ein Tupel $S = (\Exp,D,\semantic{\cdot},\obs)$, bestehend aus
  \begin{enumerate}
  \item der Menge $\Exp$ von \emph{Ausdr"ucken},
  \item dem \emph{semantischen Bereich} $D$,
  \item der \emph{Semantikfunktion} $\semantic{\cdot}:\Exp \to D$, welche jedem Ausdruck
    ein Element des semantischen Bereichs zuordnet, und
  \item der \emph{Beobachtungsfunktion} $\obs:D \to \powerset{\Obs}$, welche jedem Element
    des semantischen Bereichs sein beobachtbares Verhalten zuordnet.
  \end{enumerate}
\end{definition}

\begin{definition}
  Sei $\L=(\Prog,\Obs,\beh)$ eine beliebige Programmiersprache und $S=(\Exp,D,\semantic{\cdot},\obs)$
  eine $\Obs$-Semantik mit $\Prog \subseteq \Exp$. $S$ hei"st \emph{Semantik f"ur $\L$}, wenn 
  $\obs(\semantic{e}) = \beh(e)$ f"ur alle $e \in \Prog$.
  Die Menge aller Semantiken einer Programmiersprache $\L$ bezeichnen wir mit $\Sem(\L)$.
\end{definition}

\begin{definition}
  Zwei Semantiken $S_1$ und $S_2$ hei"sen \emph{"aquivalent}, geschrieben $S_1 \sim S_2$, wenn
  f"ur alle Programmiersprachen $\L$ gilt: $S_1 \in \Sem(\L)$ genau dann wenn $S_2 \in \Sem(\L)$.
\end{definition}

\begin{definition}[Kompositionalit"at]
  Eine Semantik $S=(\Exp,D,\semantic{\cdot},\obs)$ hei"st \emph{kompositional}, wenn es eine
  Familie $C = (C_n)_{n\in\N}$ von \emph{Konstruktoren}
  $\zeta_n:\Exp^n \to \Exp$ gibt und f"ur jedes $\zeta_n \in C_n$ eine
  Funktion $\semantic{\zeta_n}: D^n \to D$ existiert, so dass gilt:
  \begin{enumerate}
  \item $\Exp$ ist der induktive Abschluss von $C$.
  \item Wenn $e = \zeta_n (e_1 \ldots e_n)$, dann gilt
    $\semantic{e} = \semantic{\zeta_n}(\semantic{e_1} \ldots \semantic{e_n})$.
  \end{enumerate}
\end{definition}


\end{document}
\documentclass[12pt,a4paper]{article}

\usepackage{amsmath}
\usepackage{amssymb}
\usepackage{amstext}
\usepackage{array}
\usepackage[american]{babel}
\usepackage{color}
\usepackage{enumerate}
\usepackage[a4paper,%
            colorlinks=false,%
            final,%
            pdfkeywords={},%
            pdftitle={},%
            pdfauthor={Benedikt Meurer},%
            pdfsubject={},%
            pdfdisplaydoctitle=true]{hyperref}
\usepackage{ifthen}
\usepackage[latin1]{inputenc}
\usepackage{latexsym}
\usepackage[final]{listings}
\usepackage{makeidx}
\usepackage{ngerman}
\usepackage[standard,thmmarks]{ntheorem}
\usepackage{stmaryrd}

%% Silbentrennung
\hyphenation{%
Re-kur-sions-sche-ma%
}

%% LaTeX macros
%%
%% macros.tex - Useful LaTeX macros
%%
%% Copyright (c) 2006-2011 Benedikt Meurer <benedikt.meurer@googlemail.com>
%% 


%%
%% Styles
%%

\newcommand{\nstyle}[1]{\ensuremath{\mathsf{#1}}}
\newcommand{\sstyle}[1]{\ensuremath{\mathit{#1}}}


%%
%% Misc
%%

\newcommand{\abort}{\ensuremath{\mathbf{abort}}}
\newcommand{\pto}{\rightharpoonup}
\newcommand{\step}{\ensuremath{\rightsquigarrow}}

\newcommand{\sem}[1]{\ensuremath{[\![#1]\!]}}


%%
%% Names
%%

\newcommand{\arity}{\ensuremath{\mathit{arity}}}
\newcommand{\cl}{\ensuremath{\mathit{cl}}}
\newcommand{\fr}{\ensuremath{\mathit{fr}}}
\newcommand{\free}{\ensuremath{\mathit{free}}}
\newcommand{\graph}{\ensuremath{\mathit{graph}}}
\newcommand{\id}{\ensuremath{\mathit{id}}}


%%
%% Sets
%%

\newcommand{\I}{\ensuremath{\mathcal I}}
\newcommand{\N}{\ensuremath{\mathbb N}}
\renewcommand{\O}{\ensuremath{\mathcal O}}
\newcommand{\Z}{\ensuremath{\mathbb Z}}
\newcommand{\Cl}{\sstyle{Cl}}
\newcommand{\Env}{\sstyle{Env}}
\newcommand{\Exp}{\sstyle{Exp}}
\newcommand{\Frame}{\sstyle{Frame}}
\newcommand{\Id}{\sstyle{Id}}
\newcommand{\Node}{\sstyle{Node}}
\newcommand{\Val}{\sstyle{Val}}


%%
%% Expressions
%%

\newcommand{\app}[2]{{#1}\,{#2}}
\newcommand{\abstr}[2]{\lambda{#1}.\,{#2}}
\newcommand{\ifte}[3]{\mathbf{if}\,{#1}\,\mathbf{then}\,{#2}\,\mathbf{else}\,{#3}}


%%
%% Values
%%

\newcommand{\clov}[2]{\langle{#1},{#2}\rangle}
\newcommand{\false}{\mathbf{false}}
\newcommand{\true}{\mathbf{true}}


%%
%% Grammars
%%

\newenvironment{grammar}{\begin{array}{rrlll}}{\end{array}}

\newcommand{\is}{& ::= &}
\newcommand{\al}{\\ & \mid &}
\newcommand{\nl}{\vspace{2mm}\\}


%%
%% Other environments
%%

\newenvironment{case}{\left\{\!\!\!\begin{array}{ll}}{\end{array}\right.}


%%% Local Variables: 
%%% mode: latex
%%% TeX-master: "compiler"
%%% End: 


\begin{document}


\section{Rekursion}

Wir betrachten im Folgenden Funktionen als spezielle Relationen, d.h. $\subseteq$ ordnet
Funktionen basierend auf ihren Graphen.

\begin{definition}[Rekursionsschema]
  Seien $A,B$ disjunkte Mengen. Ein \emph{$(A,B)$-Rekursionsschema} ist eine Funktion
  $\sigma: A \times B^* \pto A \uplus B$, geschrieben $\sigma: A \Rightarrow B$.
\end{definition}

\begin{definition}[Rekursion]
  Sei $\sigma: A \Rightarrow B$ ein Rekursionsschema. Die durch $\sigma$ definierte \emph{Rekursion}
  ist die kleinste Relation $\Lambda \sigma \subseteq A \times B$ f"ur die gilt:
  \begin{enumerate}
  \item Wenn $\sigma(a,\varepsilon) = b$, dann $(a,b) \in \Lambda \sigma$.
  \item Wenn $n \ge 1$, $(a_i,b_i) \in \Lambda \sigma$ und $\sigma(a,b_1 \ldots b_{i-1})=a_i$ f"ur $i=1,\ldots,n$,
    und $\sigma(a,b_1 \ldots b_n) = b$, dann $(a,b) \in \Lambda \sigma$.
  \end{enumerate}
\end{definition}
Diese induktive Definition von $\Lambda \sigma$ l"asst sich auch k"urzer notieren:
\begin{quote}
  F"ur alle $n \in \N$, $a, a_1,\ldots,a_n \in A$ und $b,b_1,\ldots,b_n\in B$ gilt: 
  Wenn $(a_i,b_i)\in\Lambda\sigma$ und $\sigma(a,b_1 \ldots b_{i-1}) = a_i$ f"ur $i = 1,\ldots,n$ und
  $\sigma(a,b_1 \ldots b_n) = b$, dann $(a,b)\in\Lambda \sigma$.
\end{quote}
Wie man leicht sieht, ist $\Lambda g$ wohldefiniert und eine Funktion.
Eine Funktion $f : A \pto B$ hei"st \emph{rekursiv definierbar}, wenn ein Rekursions\-schema
$\sigma: A \Rightarrow B$ existiert, so dass $f = \Lambda \sigma$.

\begin{lemma}[Monotonie] \label{lemma:Monotonie}
  F"ur alle $\sigma,\sigma': A \Rightarrow B$ gilt:
  Wenn $\sigma \subseteq \sigma'$, dann $\Lambda \sigma \subseteq \Lambda \sigma'$.
\end{lemma}

\begin{proof}
  Gelte also $\sigma \subseteq \sigma'$. Via Induktion "uber den Aufbau von $\Lambda\sigma$ l"asst sich nun zeigen,
  dass $(a,b) \in \Lambda \sigma'$ wenn $(a,b) \in \Lambda \sigma$:
  \begin{enumerate}
  \item Wenn $\sigma(a,\varepsilon) = b$, dann gilt auch $\sigma'(a,\varepsilon)=b$ wegen
    $\sigma\subseteq\sigma'$. Also folgt $(a,b) \in \Lambda \sigma'$.
  \item Wenn $n \ge 1$, $(a_i,b_i) \in \Lambda \sigma$ und $\sigma(a,b_1 \ldots b_{i-1})=a_i$ f"ur $i=1,\ldots,n$,
    und $\sigma(a,b_1 \ldots b_n) = b$, dann folgt nach Induktionsannahme $(a_i,b_i)\in\Lambda \sigma'$
    f"ur $i=1,\ldots,n$ und wegen $\sigma \subseteq \sigma'$ folgt $\sigma'(a,b_1 \ldots b_{i-1})=a_i$
    f"ur $i=1,\ldots,n$ und $\sigma'(a,b_1 \ldots b_n) =b$. Also folgt insgesamt $(a,b) \in \Lambda \sigma'$.
  \end{enumerate}
\end{proof}

\begin{definition}
  Ein Rekursionsschema $\sigma: A \Rightarrow B$ hei"st \emph{kanonisch}, wenn
  f"ur alle $n \in \N$, $a \in A$ und $b_1,\ldots,b_{n+1} \in B$ gilt:
  \begin{quote}
    Wenn $\sigma(a,b_1 \ldots b_{n+1}) \in A \uplus B$, dann gilt
    $\sigma(a,b_1 \ldots b_i) \in A$ f"ur alle $i = 0,\ldots,n$.
  \end{quote}
\end{definition}

\begin{lemma}
  Zu jedem $\sigma: A \Rightarrow B$ existiert ein kanonisches $\sigma':A \Rightarrow B$ mit
  $\sigma' \subseteq \sigma$ und $\Lambda \sigma' = \Lambda \sigma$.
\end{lemma}

\begin{proof}
  W"ahle $\sigma' = \Can(\sigma)$ mit
  \[\begin{array}{rll}
    \Can(\sigma):& (A \times B^*) \pto A \uplus B, \\
    & (a,\varepsilon) \mapsto \sigma(a,\varepsilon) \\
    & (a,b_1 \ldots b_{n+1}) \mapsto \sigma(a,b_1 \ldots b_{n+1}) & \text{falls $\sigma(a,b_1 \ldots b_i)\in A$} \\
    && \text{f"ur alle $i =0,\ldots,n$.}
  \end{array}\]
  Trivialerweise gilt $\sigma' = \Can(\sigma) \subseteq \sigma$ und $\sigma' = \Can(\sigma)$ ist nach
  Konstruktion kanonisch. Weiterhin folgt $\Lambda \sigma' \subseteq \Lambda \sigma$ nach
  Lemma~\ref{lemma:Monotonie}, und per Induktion "uber $\Lambda \sigma$ zeigen wir, dass auch
  $\Lambda \sigma \subseteq \Lambda \sigma'$ gilt. Sei dazu $(a,b) \in \Lambda \sigma$, dann
  gilt entweder:
  \begin{enumerate}
  \item $\sigma(a,\varepsilon)=b$, also auch $\sigma'(a,\varepsilon)=b$. Dann gilt insb. $(a,b)\in\Lambda\sigma'$.
  \item Es gilt $(a_i,b_i)\in\Lambda \sigma$ und $\sigma(a,b_1 \ldots b_{i-1})=a_i$ f"ur $i=1,\ldots,n$ und
    $\sigma(a,b_1 \ldots b_n) = b$. Dann folgt nach Induktionsannahme $(a_i,b_i) \in \Lambda \sigma'$ f"ur
    $i=1,\ldots,n$. Nach Definition von $\sigma' = \Can(\sigma)$ folgt weiterhin $\sigma'(a,b_1 \ldots b_{i-1})=a_i$
    f"ur $i=1,\ldots,n$ und $\sigma'(a,b_1 \ldots b_n)=b$. Also folgt insgesamt $(a,b) \in \Lambda\sigma'$.
  \end{enumerate}
\end{proof}

\begin{corollary}
  F"ur alle $\sigma,\sigma':A \Rightarrow B$ gilt:
  \begin{enumerate}
  \item $\Can(\sigma) \subseteq \sigma$.
  \item $\Can(\Can(\sigma)) = \Can(\sigma)$.
  \item Wenn $\sigma \subseteq \sigma'$, dann auch $\Can(\sigma) \subseteq \Can(\sigma')$.
  \end{enumerate}
\end{corollary}

\begin{lemma}
  Zu jeder rekursiv definierbaren Funktion $f$ existiert ein kanonisches Rekursionsschema $\sigma$,
  so dass $f = \Lambda \sigma$ gilt.
\end{lemma}
Interessant ist anzumerken, dass dieses Rekursionsschema im Allgemeinen nicht eindeutig ist. Zum
Beispiel besitzt die Fibonacci-Funktion mindestens zwei verschiedene, kanonische Rekursionsschemata;
eines, welches zun"achst $x-1$ als n"achstes Argument liefert und ein anderes, welches zun"achst
$x-2$ liefert.

\begin{definition}
  Zwei Rekursionsschemata $\sigma,\sigma':A \Rightarrow B$ hei"sen \emph{isomorph},
  geschrieben $\sigma \equiv \sigma'$, wenn sie die gleiche Rekursion beschreiben,
  d.h. wenn $\Lambda \sigma = \Lambda \sigma'$ gilt.
\end{definition}
Dann sind trivialerweise alle Rekursionsschemata rekursiv definierbarer Funktionen eindeutig
bis auf Isomorphie; insbesondere sind die kanonischen Rekursionsschemata eindeutig bis auf
Isomorphie.

\begin{corollary}
  Wenn $\sigma \equiv \sigma'$, dann $\Can(\sigma) \equiv \sigma'$.
\end{corollary}


\subsection{Implementierung}

\begin{definition}[Implementierung]
  Seien $A',B'$ disjunkte Mengen. Eine \emph{$(A,B)$-Implementierung} ist ein Tripel
  $\gamma = (\phi,\sigma,\psi)$ mit
  \begin{enumerate}
  \item $\phi:A \pto A'$,
  \item $\sigma: A' \Rightarrow B'$, und
  \item $\psi:B' \pto B$.
  \end{enumerate}
  Die durch $\gamma$ \emph{implementierte Funktion $\Lambda\gamma: A \pto B$} ist definiert
  durch\footnote{Hierbei ist $\circ$ die auf partielle Funktionen verallgemeinerte Komposition.}
  \[\begin{array}{rcl}
    \Lambda \gamma &=& \psi \circ (\Lambda \sigma) \circ \phi.
  \end{array}\]
\end{definition}
Sei $f: A \pto B$. $\gamma$ hei"st \emph{Implementierung von $f$}, wenn $f = \Lambda \gamma$.
Offensichtlich definiert jedes Rekursionsschema $\sigma:A \Rightarrow B$ eine triviale
Implementierung $\gamma_\sigma = (\id_A,\sigma,\id_B)$.

\begin{definition}
  Zwei $(A,B)$-Implementierungen $\gamma$ und $\gamma'$ hei"sen \emph{isomorph}, geschrieben
  $\gamma \equiv \gamma'$, wenn sie die gleiche Funktion implementieren, d.h. wenn
  $\Lambda \gamma = \Lambda \gamma'$ gilt.
\end{definition}


\subsection{"Aquivalenz}

\begin{definition}
  Seien $f:A \pto B$ und $f':A' \pto B'$ rekursiv definierbare Funktionen und
  sei $\sim\ \subseteq (A \times A') \uplus (B \times B')$. $f$ und $f'$ hei"sen
  \emph{"aquivalent bzgl. $\sim$}, geschrieben $f \sim f'$, wenn f"ur
  alle $a \sim a'$ entweder
  \begin{enumerate}
  \item $f(a)$ und $f'(a')$ sind undefiniert, oder
  \item $f(a) \sim f'(a')$
  \end{enumerate}
  gilt.
\end{definition}

\begin{definition}
  Seien $\sigma:A \Rightarrow B$, $\sigma': A' \Rightarrow B'$ und $\sim\ \subseteq (A \times A') \uplus (B \times B')$.
  $\sigma$ und $\sigma'$ hei"sen \emph{"aquivalent bzgl. $\sim$}, geschrieben $\sigma \sim \sigma'$,
  wenn f"ur alle $n \in \N$ gilt:
  \begin{quote}
    Wenn $a \sim a'$ und $b_i \sim b_i'$ f"ur $i=1,\ldots,n$, dann gilt entweder
    \begin{enumerate}
    \item $\sigma(a,b_1 \ldots b_n)$ und $\sigma'(a',b_1' \ldots b_n')$ sind undefiniert, oder
    \item $\sigma(a,b_1 \ldots b_n) \sim \sigma'(a',b_1' \ldots b_n')$.
    \end{enumerate}
  \end{quote}
\end{definition}

\begin{corollary}
  Wenn $\sigma \sim \sigma'$, dann $\Can(\sigma) \sim \Can(\sigma')$.
\end{corollary}

\begin{corollary} \label{corollary:Equiv}
  Wenn $\sigma \sim \sigma'$, dann $\Lambda \sigma \sim \Lambda \sigma'$.
\end{corollary}


\section{Anwendungen}

\subsection{"Aquivalenz zwischen Substitutions- und Umgebungssemantik}

Sei $S: \Exp \Rightarrow \Val$ das Rekursionsschema der Substitutionssemantik und
$U: \Exp \times \Env \Rightarrow \Val \times \Env$ das Rekursionsschema der Umgebungssemantik,
wobei $\Val \subseteq \Exp$.
Der intuitive Zusammenhang ist, dass jedes Paar $(e,\eta)$ Darstellung eines
(abgeschlossenen) Ausdrucks ist. Sei dazu
\[\begin{array}{rcl}
  \tr(e,\eta) &=& e[\tr(e_1,\eta_1),\ldots,\tr(e_n,\eta_n)]
\end{array}\]
f"ur $\eta = [(e_1,\eta_1);\ldots;(e_n,\eta_n)]$. Dann definieren wir die "Aquivalenzrelation
durch
\[\begin{array}{rcl}
  e_0 \sim (e,\eta) &\text{gdw.}& e_0 = \tr(e,\eta).
\end{array}\]
Nach Korollar~\ref{corollary:Equiv} gilt $\Lambda S \sim \Lambda U$, wenn $S \sim U$. Auf diese
Weise bleibt uns die Induktion erspart und wir m"ussen lediglich zeigen, dass 
\[\begin{array}{rcl}
  S(\tr(e,\eta)) = \tr(U(e,eta))
\end{array}\]
f"ur alle $e,\eta$ gilt. Dann n"amlich folgt
\[\begin{array}{rcl}
  \Lambda S \circ \tr = \tr \circ \Lambda U.
\end{array}\]
Dies l"a"st sich per Fallunterscheidung nach der Form von $e$ leicht
zeigen, ist aber immer noch recht m"uhsam, vorallem wenn $e$ ein Bezeichner ist.


\end{document}
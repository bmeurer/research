\documentclass[12pt,a4paper]{article}

\usepackage{amsmath}
\usepackage{amssymb}
\usepackage{amstext}
\usepackage{array}
\usepackage[american]{babel}
\usepackage{color}
\usepackage{enumerate}
\usepackage[a4paper,%
            colorlinks=false,%
            final,%
            pdfkeywords={},%
            pdftitle={},%
            pdfauthor={Benedikt Meurer},%
            pdfsubject={},%
            pdfdisplaydoctitle=true]{hyperref}
\usepackage{ifthen}
\usepackage[latin1]{inputenc}
\usepackage{latexsym}
\usepackage[final]{listings}
\usepackage{makeidx}
\usepackage{ngerman}
\usepackage[standard,thmmarks]{ntheorem}
\usepackage{stmaryrd}

%% LaTeX macros
%%
%% macros.tex - Useful LaTeX macros
%%
%% Copyright (c) 2006-2008 Benedikt Meurer <benedikt.meurer@googlemail.com>
%%


%%
%% Generic style macros
%%

\newcommand{\nstyle}[1]{\ensuremath{\mathit{#1}}}
\newcommand{\pstyle}[1]{\ensuremath{\mathsf{#1}}}


%%
%% Misc
%%

\newcommand{\coloncolon}{\mathrel{::}}
\newcommand{\pto}{\hookrightarrow}
\newcommand{\tto}{\rightarrow_t}


%%
%% Sets
%%

\newcommand{\N}{\ensuremath{\mathbb N}}
\newcommand{\Z}{\ensuremath{\mathbb Z}}

\newcommand{\Bool}{\nstyle{Bool}}
\newcommand{\Const}{\nstyle{Const}}
\newcommand{\Exp}{\nstyle{Exp}}
\newcommand{\Member}{\ensuremath{\mathcal{M}}}
\newcommand{\Var}{\ensuremath{\mathcal{X}}}
\newcommand{\Int}{\Z}
\newcommand{\Loc}{\nstyle{Loc}}
\newcommand{\Locfin}{\ensuremath{\Loc_{fin}}}
\newcommand{\Op}{\nstyle{Op}}
\newcommand{\Store}{\nstyle{Store}}
\newcommand{\Type}{\nstyle{Type}}
\newcommand{\Val}{\nstyle{Val}}

\newcommand{\AType}{\nstyle{AType}}
\newcommand{\DType}{\nstyle{DType}}
\newcommand{\LType}{\nstyle{LType}}

\newcommand{\Assn}{\nstyle{Assn}}
\newcommand{\Formula}{\nstyle{Formula}}
\newcommand{\Term}{\nstyle{Term}}

\newcommand{\Perm}{\nstyle{Perm}}


%%
%% Functions (TODO - DeclareMathOperator)
%%

\newcommand{\dom}[1]{\ensuremath{\nstyle{dom}\,(#1)}}
\newcommand{\free}[1]{\ensuremath{\nstyle{free}\,(#1)}}
\newcommand{\locns}[1]{\ensuremath{\nstyle{locns}\,(#1)}}
\newcommand{\grph}[1]{\ensuremath{\nstyle{graph}\,(#1)}}
\newcommand{\supp}[1]{\ensuremath{\nstyle{supp}\,(#1)}}
\newcommand{\reach}[2]{\ensuremath{\nstyle{reach}\,(#1,#2)}}
\newcommand{\reachn}[3]{\ensuremath{\nstyle{reach}_{#1}\,(#2,#3)}}
\newcommand{\semantic}[1]{\ensuremath{\llbracket#1\rrbracket}}

\newcommand{\powerset}[1]{\ensuremath{\wp\,(#1)}}
\newcommand{\powersetfin}[1]{\ensuremath{\wp_{fin}\,(#1)}}


%%
%% Names
%%

\newcommand{\op}{\nstyle{op}}


%%
%% Types
%%

\newcommand{\tassn}{\pstyle{assn}}
\newcommand{\tunit}{\pstyle{unit}}
\newcommand{\tbool}{\pstyle{bool}}
\newcommand{\tint}{\pstyle{int}}
\newcommand{\tarrow}[2]{#1\to#2}
\newcommand{\ttarrow}[2]{#1\tto#2}
\newcommand{\trecord}[1]{\left\{#1\right\}}
\newcommand{\tref}[1]{#1\,\pstyle{ref}}


%%
%% Constants
%%

\newcommand{\true}{\pstyle{true}}
\newcommand{\false}{\pstyle{false}}
\newcommand{\assign}{\pstyle{:=}}
\newcommand{\fix}{\pstyle{fix}}
\newcommand{\cref}{\pstyle{ref}}
\newcommand{\unit}{\pstyle{()}}


%%
%% Expressions
%%

\newcommand{\abstr}[2]{\lambda #1.#2}
\newcommand{\app}[2]{#1\,#2}
\newcommand{\ifte}[3]{\pstyle{if}\,#1\,\pstyle{then}\,#2\,\pstyle{else}\,#3}
\newcommand{\record}[1]{\left\{#1\right\}}
\newcommand{\proj}[2]{#1.#2}

\newcommand{\triple}[3]{\{#1\}\,#2\,\{#3\}}


%%
%% Grammars
%%

\newcommand{\GRbeg}{\begin{array}{rrlll}}
\newcommand{\GRend}{\end{array}}

\newcommand{\GRis}{& ::= &}
\newcommand{\GRal}{\\ & \mid &}
\newcommand{\GRnl}{\vspace{2mm}\\}
\newcommand{\GRmid}{\,\mid\,}
\newcommand{\GRtext}[1]{\hfill & \text{#1}}


%%
%% Rules
%%

\newcommand{\tj}[2]{#1\,\coloncolon\,#2}
\newcommand{\Tj}[3]{#1\,\triangleright\,#2\coloncolon#3}



\begin{document}

\section{Domains und Patterns}

Vorgegeben sei eine (unendliche) Menge von Domainnamen $\alpha$, eine beliebige Menge von Konstruktoren
$\zeta$ und eine beliebige Menge von Basisdomains $\beta$. Die Menge $\Dom$ aller Domains $\delta$ ist
durch die kontextfreie Grammatik
\[\begin{grammar}
  \delta \in \Dom \is \beta
  \al \alpha \mid \mu \alpha.\,\delta
  \al \delta_1 * \ldots * \delta_n & (n \ge 2)
  \al \zeta_1\,\delta_1+\ldots+\zeta_n\,\delta_n & (n \ge 2)
\end{grammar}\]
definiert, wobei in der letzten Produktion angenommen wird, dass die $\zeta_1$ bis $\zeta_n$ paarweise verschieden
sind. Die Menge $\free{\delta}$ aller \emph{freien Domainnamen von $\delta$} ist induktiv definiert durch:
\[\begin{array}{rcl}
  \free{\beta} &=& \emptyset \\
  \free{\alpha} &=& \{\alpha\} \\
  \free{\mu\alpha.\,\delta} &=& \free{\delta} \setminus \{\alpha\} \\
  \free{\delta_1 * \ldots * \delta_n} &=& \free{\delta_1} \cup \ldots \cup \free{\delta_n} \\
  \free{\zeta_1\,\delta_1+\ldots+\zeta_n\,\delta_n} &=& \free{\delta_1} \cup \ldots \cup \free{\delta_n} \\
\end{array}\]
Eine Domain $\delta$ hei"st \emph{abgeschlossen}, wenn $\free{\delta} = \emptyset$.


\subsection{Patterns}

Weiter sei vorgegeben eine Menge $\Var$ von Variablen $x$ und zu jeder Basisdomain $\beta$ eine beliebige
Menge von Konstanten $c^\beta$. Die Menge $\Pat$ aller Patterns $p$ ist durch die kontextfreie Grammatik
\[\begin{grammar}
  p \in \Pat \is \any \mid x
  \al c \mid \zeta\,p
  \al p_1,\ldots,p_n & (n \ge 2)
\end{grammar}\]
definiert. Die Menge $\free{p}$ aller \emph{freien Variablen von $p$} ist induktiv definiert durch:
\[\begin{array}{rcl}
  \free{\any} &=& \emptyset \\
  \free{x} &=& \{x\} \\
  \free{c} &=& \emptyset \\
  \free{\zeta\,p} &=& \free{p} \\
  \free{p_1,\ldots,p_n} &=& \free{p_1} \cup \ldots \cup \free{p_n} \\
\end{array}\]

Eine \emph{Domainisierung} ist eine partielle Funktion $\Gamma: \Var \pto \Dom$ mit endlichem
Definitionsbereich. Eine \emph{Patterndomainzuordnung} ist eine Formel der Gestalt $\da{\Gamma}{p}{\delta}$.
Eine solche Patterndomainzuordnung hei"st g"ultig, wenn sie sich mit den folgenden Regeln herleiten l"asst: \\[5mm]
\begin{tabular}{cccc}
  $\da{[\,]}{\any}{\delta}$
  & $\da{[x \mapsto \delta]}{x}{\delta}$
  & $\RULE{\da{\Gamma}{c}{\delta[\mu\alpha.\,\delta/\alpha]}}{\da{\Gamma}{c}{\mu\alpha.\,\delta}}$
  & $\da{[\,]}{c^\beta}{\beta}$ \\[3mm]
  \multicolumn{2}{c}{$\RULE{\da{\Gamma}{\zeta\,p}{\delta[\mu\alpha.\,\delta/\alpha]}}{\da{\Gamma}{\zeta\,p}{\mu\alpha.\,\delta}}$}
  & \multicolumn{2}{c}{$\RULE{1 \le i \le n \quad \da{\Gamma}{p}{\delta_i}}{\da{\Gamma}{\zeta_i\,p}{\zeta_1\,\delta_1+\ldots+\zeta_n\,\delta_n}}$} \\[3mm]
  \multicolumn{2}{c}{$\RULE{\da{\Gamma}{p_1,\ldots,p_n}{\delta[\mu\alpha.\,\delta/\alpha]}}{\da{\Gamma}{p_1,\ldots,p_n}{\mu\alpha.\,\delta}}$}
  & \multicolumn{2}{c}{$\RULE{\da{\Gamma_1}{p_1}{\delta_1} \quad \ldots \quad \da{\Gamma_n}{p_n}{\delta_n}}{\da{\Gamma_1 \circ \ldots \circ \Gamma_n}{p_1,\ldots,p_n}{\delta_1 * \ldots * \delta_n}}$}
\end{tabular} \\[3mm]
Wir sagen \emph{das Pattern $p$ geh"ort zur Domain $\delta$}, wenn ein $\Gamma$ existiert,
so dass $\da{\Gamma}{p}{\delta}$. Eine Domain $\delta$ hei"st \emph{azyklisch} oder \emph{g"ultig},
wenn ein $p \in \Pat$ existiert, welches zu $\delta$ geh"ort.


\subsection{Ausdr"ucke}

Die Mengen $\Exp$ aller Ausdr"ucke $e$ und $\Node \subset \Exp$ aller Knoten $\kappa$ sind durch die
kontextfreie Grammatik
\[\begin{grammar}
  e \in \Exp \is x \mid c \mid \zeta\,e
  \al e_1,\ldots,e_n & (n \ge 2)
  \nl
  \kappa \in \Node \is c \mid \zeta\,\kappa
  \al \kappa_1,\ldots,\kappa_n & (n \ge 2)
\end{grammar}\]
definiert. Die Menge $\free{e}$ aller \emph{freien Variablen von $e$} ist induktiv definiert durch:
\[\begin{array}{rcl}
  \free{x} &=& \{x\} \\
  \free{c} &=& \emptyset \\
  \free{\zeta\,e} &=& \free{e} \\
  \free{e_1,\ldots,e_n} &=& \free{e_1} \cup \ldots \cup \free{e_n} \\
\end{array}\]

Ein \emph{Ausdrucksurteil} ist eine Formel der Gestalt $\ej{\Gamma}{e}{\delta}$. Ein
solches Ausdrucksurteil hei"st g"ultig, wenn sie sich mit den folgenden Regeln herleiten
l"asst: \\[5mm]
\begin{tabular}{ccc}
  $\RULE{\Gamma(x) = \delta}{\ej{\Gamma}{x}{\delta}}$
  & $\RULE{\ej{\Gamma}{c}{\delta[\mu\alpha.\,\delta/\alpha]}}{\ej{\Gamma}{c}{\mu\alpha.\,\delta}}$
  & $\ej{\Gamma}{c^\beta}{\beta}$ \\[3mm]
  $\RULE{\ej{\Gamma}{\zeta\,e}{\delta[\mu\alpha.\,\delta/\alpha]}}{\ej{\Gamma}{\zeta\,e}{\mu\alpha.\,\delta}}$
  & $\RULE{1 \le i \le n \quad \ej{\Gamma}{e}{\delta_i}}{\ej{\Gamma}{\zeta_i\,e}{\zeta_1\,\delta_1+\ldots+\zeta_n\,\delta_n}}$ \\[3mm]
  $\RULE{\ej{\Gamma}{e_1,\ldots,e_n}{\delta[\mu\alpha.\,\delta/\alpha]}}{\ej{\Gamma}{e_1,\ldots,e_n}{\mu\alpha.\,\delta}}$
  & $\RULE{\ej{\Gamma}{e_1}{\delta_1} \quad\ldots\quad \ej{\Gamma}{e_n}{\delta_n}}{\ej{\Gamma}{e_1,\ldots,e_n}{\delta_1 * \ldots * \delta_n}}$ \\[3mm]
\end{tabular} \\[3mm]
Wir sagen \emph{der Ausdruck $e$ geh"ort zur Domain $\delta$}, wenn ein $\Gamma$ existiert, so dass
$\ej{\Gamma}{e}{\delta}$.


\subsection{Patternmatching}

Eine \emph{Variablenbelegung} ist eine partielle Funktion $\gamma:\Var \pto \Node$ mit endlichem
Definitionsbereich. Ein \emph{Patternmatching} ist eine Formel der Gestalt $\match{p}{\kappa}{\gamma}$
und hei"st g"ultig, wenn es sich mit den folgenden Regeln herleiten l"asst: \\[5mm]
\begin{tabular}{ccc}
  $\match{\any}{\kappa}{[\,]}$
  & $\match{x}{\kappa}{[\kappa/x]}$
  & $\match{c}{c}{[\,]}$ \\[1mm]
  $\RULE{\match{p}{\kappa}{\gamma}}{\match{\zeta\,p}{\zeta\,\kappa}{\gamma}}$
  & \multicolumn{2}{c}{$\RULE{\match{p_1}{\kappa_1}{\gamma_1} \quad\ldots\quad \match{p_n}{\kappa_n}{\gamma_n}}{\match{p_1,\ldots,p_n}{\kappa_1,\ldots,\kappa_n}{\gamma_1 \circ\ldots\circ \gamma_n}}$} \\[3mm]
\end{tabular} \\[3mm]

\begin{lemma}
  Wenn $\match{p}{\kappa}{\gamma}$ und $\match{p}{\kappa}{\gamma'}$, dann gilt $\gamma = \gamma'$.
\end{lemma}

Der Ausdruck $(e\,\gamma) \in \Exp$, welcher durch Anwendung der Variablenbelegung $\gamma$ auf den Ausdruck
$e$ entsteht, ist induktiv definiert durch:
\[\begin{array}{rcl}
  x\,\gamma &=& \begin{case} \gamma(x) & \text{falls } x \in \dom{\gamma} \\ x & \text{sonst} \end{case} \\
  c\,\gamma &=& c \\
  (\zeta\,e)\,\gamma &=& \zeta\,(e\,\gamma) \\
  (e_1,\ldots,e_n)\,\gamma &=& (e_1\,\gamma),\ldots,(e_n\,\gamma) \\
\end{array}\]

Sei $\gamma$ eine Variablenbelegung und $\Gamma$ eine Domainisierung. $\gamma$ hei"st \emph{wohldefiniert bzgl.
$\Gamma$}, geschrieben $\Gamma \models \gamma$, wenn $\dom{\gamma} = \dom{\Gamma}$ und
$\ej{[\,]}{\gamma(x)}{\Gamma(x)}$ f"ur alle $x \in \dom{\gamma}$.


\subsection{Zusammenhang}

\begin{proposition}
  Wenn $\da{\Gamma}{p}{\delta}$ und $\ej{\Gamma}{\kappa}{\delta}$, dann gilt:
  \begin{enumerate}
  \item Es existiert ein $\gamma$ mit $\match{p}{\kappa}{\gamma}$.
  \item F"ur alle $\gamma$ mit $\match{p}{\kappa}{\gamma}$ gilt $\Gamma \models \gamma$.
  \end{enumerate}
\end{proposition}

\begin{proposition}
  Wenn $\Gamma \models \gamma$ und $\ej{\Gamma}{e}{\delta}$, dann gilt
  $\ej{[\,]}{e\,\gamma}{\delta}$.
\end{proposition}


\end{document}
\documentclass[%
  12pt,%
  a4paper,%
%  draft,%
]{article}

\usepackage{amsmath}
\usepackage{amssymb}
\usepackage{amstext}
\usepackage{array}
\usepackage[american]{babel}
\usepackage{color}
\usepackage{enumerate}
\usepackage[a4paper,%
            colorlinks=false,%
            final,%
            pdfkeywords={},%
            pdftitle={},%
            pdfauthor={Benedikt Meurer},%
            pdfsubject={},%
            pdfdisplaydoctitle=true]{hyperref}
\usepackage{ifthen}
\usepackage[latin1]{inputenc}
\usepackage{latexsym}
\usepackage[final]{listings}
\usepackage{makeidx}
\usepackage{mathpartir}
\usepackage{ngerman}
\usepackage[standard,thmmarks]{ntheorem}
\usepackage{stmaryrd}
\usepackage{varwidth}

%% LaTeX macros
%%
%% macros.tex - Useful LaTeX macros
%%
%% Copyright (c) 2006-2011 Benedikt Meurer <benedikt.meurer@googlemail.com>
%% 


%%
%% Styles
%%

\newcommand{\nstyle}[1]{\ensuremath{\mathsf{#1}}}
\newcommand{\sstyle}[1]{\ensuremath{\mathit{#1}}}


%%
%% Misc
%%

\newcommand{\abort}{\ensuremath{\mathbf{abort}}}
\newcommand{\pto}{\rightharpoonup}
\newcommand{\step}{\ensuremath{\rightsquigarrow}}

\newcommand{\sem}[1]{\ensuremath{[\![#1]\!]}}


%%
%% Names
%%

\newcommand{\arity}{\ensuremath{\mathit{arity}}}
\newcommand{\cl}{\ensuremath{\mathit{cl}}}
\newcommand{\fr}{\ensuremath{\mathit{fr}}}
\newcommand{\free}{\ensuremath{\mathit{free}}}
\newcommand{\graph}{\ensuremath{\mathit{graph}}}
\newcommand{\id}{\ensuremath{\mathit{id}}}


%%
%% Sets
%%

\newcommand{\I}{\ensuremath{\mathcal I}}
\newcommand{\N}{\ensuremath{\mathbb N}}
\renewcommand{\O}{\ensuremath{\mathcal O}}
\newcommand{\Z}{\ensuremath{\mathbb Z}}
\newcommand{\Cl}{\sstyle{Cl}}
\newcommand{\Env}{\sstyle{Env}}
\newcommand{\Exp}{\sstyle{Exp}}
\newcommand{\Frame}{\sstyle{Frame}}
\newcommand{\Id}{\sstyle{Id}}
\newcommand{\Node}{\sstyle{Node}}
\newcommand{\Val}{\sstyle{Val}}


%%
%% Expressions
%%

\newcommand{\app}[2]{{#1}\,{#2}}
\newcommand{\abstr}[2]{\lambda{#1}.\,{#2}}
\newcommand{\ifte}[3]{\mathbf{if}\,{#1}\,\mathbf{then}\,{#2}\,\mathbf{else}\,{#3}}


%%
%% Values
%%

\newcommand{\clov}[2]{\langle{#1},{#2}\rangle}
\newcommand{\false}{\mathbf{false}}
\newcommand{\true}{\mathbf{true}}


%%
%% Grammars
%%

\newenvironment{grammar}{\begin{array}{rrlll}}{\end{array}}

\newcommand{\is}{& ::= &}
\newcommand{\al}{\\ & \mid &}
\newcommand{\nl}{\vspace{2mm}\\}


%%
%% Other environments
%%

\newenvironment{case}{\left\{\!\!\!\begin{array}{ll}}{\end{array}\right.}


%%% Local Variables: 
%%% mode: latex
%%% TeX-master: "compiler"
%%% End: 


\DeclareMathOperator{\as}{\textbf{as}}
\newcommand{\any}{\textbf{any}}
\newcommand{\match}[3]{\textbf{match}\,{#1}\,\textbf{with}\,{#2}\,\textbf{in}\,{#3}}

\newcommand{\tj}[2]{{#1}\div{#2}}
\newcommand{\Tj}[3]{{#1}\vdash{#2}\div{#3}}

\begin{document}


\section*{Notationen}

\begin{mathpar}
  A^n = \underbrace{A \times \ldots \times A}_n
  \and
  A^* = \bigcup_{n \ge 0} A^n
  \and
  A^+ = \bigcup_{n \ge 1} A^n
\end{mathpar}
wobei $A^0 = \{\varepsilon\}$


\section*{Abstrakte Syntax}

Die Mengen $\Spec$ aller \emph{(Bindungs-)Spezifikationen} $\sigma$ und
$\Type$ aller \emph{(Bindungs-)Typen} $\tau$ sind definiert durch die
folgende kontextfreie Grammatik:
\[\begin{grammar}
  \sigma \in \Spec
  \is \Dynamic
  \al \Static{i} & i \ge 1
  \nl
  \tau \in \Type
  \is \langle \sigma_1,\ldots,\sigma_n \rangle & n \ge 0
  \nl
\end{grammar}\]
Eine Bindungsspezifikation $\gamma$ bzw. ein Bindungstyp $\tau$ hei"st \emph{g\"ultig bzgl. $k \in \N$},
geschrieben $\sigma \preceq k$ bzw. $\tau \preceq k$, wenn sich dies mit den folgenden Regeln herleiten
l"a"st:
\begin{mathpar}
  \inferrule{%
  }{%
    \Dynamic \preceq k
  }%
  \and
  \inferrule{%
    i \le k
  }{%
    \Static{i} \preceq k
  }%
  \and
  \inferrule{%
    \sigma_1 \preceq k \\
    \ldots \\
    \sigma_n \preceq k
  }{%
    \langle \sigma_1,\ldots,\sigma_n \rangle \preceq k
  }%
\end{mathpar}
Intuitiv bedeutet $\tau \preceq k$, dass alle statischen Spezifikationen in $\tau$ innerhalb des
Intervalls $[1,k]$ liegen.

Eine \emph{Syntaxsignatur} $\Psi = \langle \mathcal{C}, \delta \rangle$ besteht
aus einer Menge $\mathcal{C}$ von \emph{Konstruktoren} $\zeta$ und einer
(totalen) Funktion $\delta: \mathcal{C} \to \Type^*$, welche jedem Konstruktor
eine Stelligkeit und damit verbundene Bindungstypen zuordnet.

\subsection*{Terme und Regeln}

Sei $\Var$ eine (unendliche) Menge von \emph{Variablen}.
Die Mengen $\Term(\Psi)$ aller \emph{$\Psi$-Terme} $t$ und $\Case(\Psi)$ aller
\emph{$\Psi$-F"alle} $\alpha$ sind definiert durch die folgende kontextfreie Grammatik
\[\begin{grammar}
  t \in \Term(\Psi)
  \is X
  \al t \langle t_1,\ldots,t_n \rangle & n \ge 1
  \al \zeta(t_1,\ldots,t_n)
  \al \match{t}{\zeta(X_1,\ldots,X_n)}{t'}
  \nl
  \alpha \in \Case(\Psi)
  \is X_1,\ldots,X_n \Uparrow t & n \ge 1
  \al X_1,\ldots,X_n \Downarrow t & n \ge 1
  \nl
\end{grammar}\]
wobei vorausgesetzt wird, dass die $X_1,\ldots,X_n$ paarweise verschieden sind.
Eine \emph{$\Psi$-Regelspezifikation} $r$ ist eine (mglw. unendliche)
Menge von $\Psi$-F"allen.

Eine \emph{Typbelegung} ist eine partielle, endliche Abbildung $\Gamma:\Var \pto \Type$.
Ein \emph{Typurteil f"ur $\Psi$-Terme} ist eine Formel der Gestalt $\Tj{\Gamma}{t}{\tau}$, und
hei"st \emph{g"ultig}, wenn es sich mit den folgenden Regeln herleiten l"a"st:
\begin{mathpar}
  \inferrule[(Var)]{%
    \Gamma(X) = \tau
  }{%
    \Tj{\Gamma}{X}{\tau}
  }%
  \and
  \inferrule[(Cons)]{%
    \delta(\zeta)=(\tau_1,\ldots,\tau_n) \\
    \Tj{\Gamma}{t_i}{\tau_i} \\
    \tau_i \preceq n \\
    i=1,\ldots,n
  }{%
    \Tj{\Gamma}{\zeta(t_1,\ldots,t_n)}{\langle \rangle}
  }%
  \and
  \inferrule[(App)]{%
    \Tj{\Gamma}{t}{\langle\sigma_1,\ldots,\sigma_n\rangle} \\
    \Tj{\Gamma}{t_i}{\langle \rangle} \\
    \sigma_i \preceq n \\
    i=1,\ldots,n
  }{%
    \Tj{\Gamma}{t\langle t_1,\ldots,t_n \rangle}{\langle \rangle}
  }%
  \and
  \inferrule[(Match)]{%
    \Tj{\Gamma}{t}{\langle \rangle} \\
    \delta(\zeta)=(\tau_1,\ldots,\tau_n) \\
    \Tj{\Gamma[\tau_1/X_1]\ldots[\tau_n/X_n]}{t'}{\tau}
  }{%
    \Tj{\Gamma}{\match{t}{\zeta(X_1,\ldots,X_n)}{t'}}{\tau}
  }%
\end{mathpar}
Ein Fall $\alpha$ hei"st \emph{g"ultig}, wenn gilt:
\begin{mathpar}
  \inferrule[(Arg)]{%
    \Tj{[X_1:\langle \rangle,\ldots,X_n:\langle \rangle]}{t}{\langle\rangle}
  }{%
    X_1,\ldots,X_n \Uparrow t
  }%
  \and
  \inferrule[(Res)]{%
    \Tj{[X_1:\langle \rangle,\ldots,X_n:\langle \rangle]}{t}{\langle\rangle}
  }{%
    X_1,\ldots,X_n \Downarrow t
  }%
\end{mathpar}
Und eine Regelspezifikation $r$ hei"st \emph{g"ultig}, wenn alle $\alpha \in r$
g"ultig sind.


\section*{Semantik}

Eine \emph{$\Psi$-Implementierung} $\mathcal{D}$ ist eine Struktur $(D,\sem{\cdot}_D)$, bestehend aus
\begin{itemize}
\item einer nicht-leeren \emph{Tr\"agermenge $D$} mit Teilmengen $D^\tau \subseteq D$ f\"ur jeden Typ
  $\tau \in \Type$, und
\item einer \emph{Funktionsbelegung $\sem{\cdot}_D$}, welche den (Familien von) Funktionszeichen
  $\App_n$ $(n \ge 1)$, $\Child_n^i$ $(n \ge i \ge 1)$, $\Compose_n$ $(n \ge 0)$ und $\Label$ Funktionen
  nach folgendem Schema zuordnet:
  \[\begin{array}{ll}
    \sem{\App_n}_D: D \times \underbrace{D \times \ldots \times D}_n \pto D &
    \sem{\Child_n^i}_D: D \pto D \\
    \sem{\Compose_n}_D: \zeta \times \underbrace{D \times \ldots \times D}_n \pto D &
    \sem{\Label}_D: D \pto \zeta \\
  \end{array}\]
\end{itemize}
Eine Implementierung $\mathcal{D}$ hei"st \emph{g\"ultig}, wenn gilt:
\begin{enumerate}
\item Wenn $d_0 \in D^{\langle \sigma_1,\ldots,\sigma_n \rangle}$ und $d_i \in D^{\langle \rangle}$ f\"ur $i=1,\ldots,n$,
  dann existiert ein $d \in D^{\langle \rangle}$, so dass $d = \sem{\App_n}_D(d_0,d_1,\ldots,d_n)$.
\item Wenn $\delta(\zeta) = (\tau_1,\ldots,\tau_n)$ und $d_i \in D^{\tau_i}$ f\"ur $i=1,\ldots,n$,
  dann existiert ein $d \in D^{\langle \rangle}$, so dass $d = \sem{\Compose_n}_D(\zeta,d_1,\ldots,d_n)$.
\item Wenn $\delta(\zeta) = (\tau_1,\ldots,\tau_n)$, $d \in D^{\langle \rangle}$ und $\zeta = \sem{\Label}_D(d)$, dann
  existieren $d_1 \in D^{\tau_1},\ldots,d_n \in D^{\tau_n}$, so dass $\sem{\Child_n^i}_D(d) = d_i$ f\"ur $i=1,\ldots,n$.
\end{enumerate}
Wir identifizieren $\mathcal{D}$ mit $D$.

\subsection*{Semantik von Termen und F"allen}

Eine \emph{$D$-Belegung} ist eine partielle, endliche Abbildung $\varrho: \Var \pto D$. Eine Belegung
$\varrho$ hei"st \emph{g"ultig bzgl. einer Typbelegung $\Gamma$}, geschrieben $\Gamma \vdash \varrho$,
wenn gilt:
\begin{enumerate}
\item $\dom(\Gamma) = \dom(\varrho)$
\item $\varrho(X) \in D^{\Gamma(X)}$ f"ur alle $X \in \dom(\Gamma)$
\end{enumerate}
Die Menge aller g"ultigen Belegungen bzgl. $\Gamma$ bezeichnen wir mit $\Assn(\Gamma)$.
\begin{lemma}
  Wenn $\Gamma \vdash \varrho$ und $d \in D^\tau$, dann $\Gamma[\tau/X] \vdash \varrho[d/X]$.
\end{lemma}
\begin{proof}
  Trivial.
\end{proof}
Jedem g"ultigen Typurteil $\Tj{\Gamma}{t}{\tau}$ wird wie folgt induktiv
eine Semantik $D\sem{\Tj{\Gamma}{t}{\tau}}: \Assn(\Gamma) \pto D$ zugeordnet:
\begin{itemize}
\item $D\sem{\Tj{\Gamma}{X}{\tau}}\,\varrho = \varrho(X)$
\item $D\sem{\Tj{\Gamma}{t \langle t_1,\ldots,t_n \rangle}{\langle \rangle}}\,\varrho
  = \sem{\App_n}_D(d, d_1, \ldots, d_n)$ \\
  falls $d = D\sem{\Tj{\Gamma}{t}{\langle\sigma_1,\ldots,\sigma_n\rangle}}\,\varrho$ \\
  und $d_i = D\sem{\Tj{\Gamma}{t_i}{\langle \rangle}}\,\varrho$ f\"ur $i=1,\ldots,n$
\item $D\sem{\Tj{\Gamma}{\zeta(t_1,\ldots,t_n)}{\langle \rangle}}\,\varrho
  = \sem{\Compose_n}_D(\zeta,d_1,\ldots,d_n)$ \\
  falls $d_i = D\sem{\Tj{\Gamma}{t_i}{\tau_i}}\,\varrho$ f\"ur $i=1,\ldots,n$
\item $D\sem{\Tj{\Gamma}{\match{t}{\zeta(X_1,\ldots,X_n)}{t'}}{\tau}}\,\varrho
  = d'$ \\
  falls $d = D\sem{\Tj{\Gamma}{t}{\langle \rangle}}\,\varrho$, \\
  $\sem{\Label}_D(d) = \zeta$, $d_i = \sem{\Child_n^i}_D(d)$ f"ur $i=1,\ldots,n$, \\
  und $d' = D\sem{\Tj{\Gamma[\tau_1/X_1]\ldots[\tau_n/X_n]}{t'}{\tau}}\,\bigl(\varrho[d_1/X_1]\ldots[d_n/X_n]\bigr)$
\end{itemize}
Jedem g"ultigen Fall $\alpha$ wird wie folgt eine Semantik
$D\sem{\alpha} \subseteq D^+ \times \{\Uparrow,\Downarrow\} \times D$ zugeordnet:
\begin{itemize}
\item $D\sem{X_1,\ldots,X_n \Uparrow t} \\
  =
  \{(d_1,\ldots,d_n,\Uparrow,D\sem{\Tj{[X_i:\langle\rangle]_{i=1}^n}{t}{\langle \rangle}}\,[X_i:d_i]_{i=1}^n) \mid
    \text{$d_1,\ldots,d_n \in D^{\langle \rangle}$ }
  \}$
\item $D\sem{X_1,\ldots,X_n \Downarrow t} \\
  =
  \{(d_1,\ldots,d_n,\Downarrow,D\sem{\Tj{[X_i:\langle\rangle]_{i=1}^n}{t}{\langle \rangle}}\,[X_i:d_i]_{i=1}^n) \mid
    \text{$d_1,\ldots,d_n \in D^{\langle \rangle}$ }
  \}$
\end{itemize}
Analog wird jeder g"ultigen Regelspezifikation $r$ eine Semantik $D\sem{r} = \bigcup_{\alpha\in r}D\sem{\alpha}$
zugeordnet.

\begin{lemma}[Wohldefiniertheit der Semantik] \label{lem:Wohldefiniertheit_der_Semantik}
  Sei $D$ eine g"ultige Implementierung.
  \begin{enumerate}
  \item Wenn $\Tj{\Gamma}{t}{\tau}$ und $\varrho \in \Assn(\Gamma)$, dann gilt:
    \begin{enumerate}
    \item Wenn $D\sem{\Tj{\Gamma}{t}{\tau}}\,\varrho = d$, dann $d \in D^\tau$.
    \item Wenn $D\sem{\Tj{\Gamma}{t}{\tau}}\,\varrho = d$ und $D\sem{\Tj{\Gamma}{t}{\tau}}\,\varrho = d'$,
      dann $d = d'$.
    \end{enumerate}

  \item Wenn $\alpha$ g"ultig ist, dann gilt:
    \begin{enumerate}
    \item Wenn $(d_1,\ldots,d_n,\Uparrow,d) \in D\sem{\alpha}$,
      dann $d_1,\ldots,d_n,d \in D^{\langle\rangle}$.
    \item Wenn $(d_1,\ldots,d_n,\Downarrow,d) \in D\sem{\alpha}$,
      dann $d_1,\ldots,d_n,d \in D^{\langle\rangle}$.
    \end{enumerate}

  \item Wenn $r$ g"ultig ist, dann gilt:
    \begin{enumerate}
    \item Wenn $(d_1,\ldots,d_n,\Uparrow,d) \in D\sem{r}$,
      dann $d_1,\ldots,d_n,d \in D^{\langle\rangle}$.
    \item Wenn $(d_1,\ldots,d_n,\Downarrow,d) \in D\sem{r}$,
      dann $d_1,\ldots,d_n,d \in D^{\langle\rangle}$.
    \end{enumerate}
  \end{enumerate}
\end{lemma}

\begin{proof}
  Sollte klar sein.
\end{proof}

\subsection*{"Aquivalenz}

\begin{definition}["Aquivalenz von Implementierungen] \label{def:Aequivalenz_von_Implementierungen}
  Seien $D$ und $E$ $\Psi$-Implementierungen und $\Delta \subseteq D \times E$. $D$ und $E$ hei"sen
  \emph{$\Delta$-"aquivalent}, geschrieben $D \sim_\Delta E$, wenn $d \in D^\tau$ gdw. $e \in E^\tau$
  f"ur alle $(d,e) \in \Delta$ und $\tau \in \Type$,
  und f"ur alle $(\hat{d},\hat{e}),(d_1,e_1),\ldots,(d_n,e_n) \in \Delta$ gilt:
  \begin{enumerate}
  \item Wenn $\sem{\App_n}_D(\hat{d},d_1,\ldots,d_n) = d$ und $\sem{\App_n}_E(\hat{e},e_1,\ldots,e_n) = e$,
    dann $(d,e) \in \Delta$.
  \item Wenn $\sem{\Compose_n}_D(\zeta,d_1,\ldots,d_n) = d$ und $\sem{\Compose_n}_E(\zeta,e_1,\ldots,e_n) = e$,
    dann $(d,e) \in \Delta$.
  \item Wenn $\sem{\Child_n^i}_D(\hat{d}) = d$ und $\sem{\Child_n^i}_E(\hat{e}) = e$,
    dann $(d,e) \in \Delta$.
  \item $\sem{\Label}_D(\hat{d}) = \zeta$ gdw. $\sem{\Label}_E(\hat{e}) = \zeta$.
  \end{enumerate}
\end{definition}
Zwei Belegungen $\varrho_D: \Var \pto D$ und $\varrho_E: \Var \pto E$ hei"sen \emph{$\Delta$-"aquivalent},
geschrieben $\varrho_D \sim_\Delta \varrho_E$, wenn gilt:
\begin{enumerate}
\item $\dom(\varrho_D) = \dom(\varrho_E)$
\item $\bigl(\varrho_D(X),\varrho_E(X)\bigr) \in \Delta$ f"ur alle $X \in \dom(\varrho_D)$
\end{enumerate}

\begin{lemma} \label{lem:Aequivalenz_von_Termen}
  Seien $D$, $E$ g"ultige, $\Delta$-"aquivalente $\Psi$-Implementierungen,
  und $\varrho_D \sim_\Delta \varrho_E$ mit $\Gamma \vdash \varrho_D$ und $\Gamma \vdash \varrho_E$.
  Wenn $\Tj{\Gamma}{t}{\tau}$, dann gilt genau eine der folgenden Aussagen:
  \begin{enumerate}
  \item
    $\bigl( D\sem{\Tj{\Gamma}{t}{\tau}}\,\varrho_D, E\sem{\Tj{\Gamma}{t}{\tau}}\,\varrho_E \bigr) \in \Delta$
  \item
    $\bigl( D\sem{\Tj{\Gamma}{t}{\tau}}\,\varrho_D \bigr) \not\in D$
    und $\bigl( E\sem{\Tj{\Gamma}{t}{\tau}}\,\varrho_E \bigr) \not\in E$
  \end{enumerate}
\end{lemma}

\begin{proof}
  Via Induktion "uber die L"ange der Herleitung des Typurteils $\Tj{\Gamma}{t}{\tau}$ und Fallunterscheidung
  nach der zuletzt angewandten Typregel:
  \begin{itemize}
  \item $\Tj{\Gamma}{X}{\tau}$ mit \textsc{(Var)}

    Klar, denn $\bigl(\varrho_D(X), \varrho_E(X)\bigr) \in \Delta$ nach Voraussetzung.

  \item $\Tj{\Gamma}{\zeta(t_1,\ldots,t_n)}{\tau}$ mit \textsc{(Cons)}

    Dann gilt $\delta(\zeta) = (\tau_1,\ldots,\tau_n)$ und $\Tj{\Gamma}{t_i}{\tau_i}$ 
    f"ur $i=1,\ldots,n$. Nach I.V. gilt die Behauptung f"ur jedes $\Tj{\Gamma}{t_i}{\tau_i}$.
    Angenommen es ex. $(d_1,e_1),\ldots,(d_n,e_n) \in \Delta$ mit
    \begin{mathpar}
      D\sem{\Tj{\Gamma}{t_i}{\tau_i}}\,\varrho_D = d_i
      \and
      E\sem{\Tj{\Gamma}{t_i}{\tau_i}}\,\varrho_E = e_i
    \end{mathpar}
    f"ur $i=1,\ldots,n$, dann folgt wegen Lemma~\ref{lem:Wohldefiniertheit_der_Semantik}, dass
    $d_i \in D^{\tau_i}$ und $e_i \in E^{\tau_i}$
    f"ur $i=1,\ldots,n$, und da $D$ nach Voraussetzung g"ultig ist, existieren dar"uberhinaus
    $d \in D^{\langle \rangle}$ und $e \in E^{\langle \rangle}$ mit
    \begin{mathpar}
      d
      = \sem{\Compose_n}_D(\zeta,d_1,\ldots,d_n)
      = D\sem{\Tj{\Gamma}{\zeta(t_1,\ldots,t_n)}{\langle \rangle}}\,\varrho_D
      \and
      e
      = \sem{\Compose_n}_E(\zeta,e_1,\ldots,e_n)
      = E\sem{\Tj{\Gamma}{\zeta(t_1,\ldots,t_n)}{\langle \rangle}}\,\varrho_E
    \end{mathpar}
    und gem"a"s Definition~\ref{def:Aequivalenz_von_Implementierungen} folgt $(d,e) \in \Delta$.

    Gilt andererseits
    $\bigl(D\sem{\Tj{\Gamma}{t_i}{\tau_i}}\,\varrho_D\bigr) \not\in D$
    und
    $\bigl(E\sem{\Tj{\Gamma}{t_i}{\tau_i}}\,\varrho_E\bigr) \not\in E$
    f"ur mindestens ein $i \in \{1,\ldots,n\}$, so folgt unmittelbar, dass auch
    \begin{mathpar}
      \bigl(D\sem{\Tj{\Gamma}{\zeta(t_1,\ldots,t_n)}{\langle \rangle}}\,\varrho_D\bigr) \not\in D
      \and
      \bigl(E\sem{\Tj{\Gamma}{\zeta(t_1,\ldots,t_n)}{\langle \rangle}}\,\varrho_E\bigr) \not\in E
    \end{mathpar}
    gilt.

  \item $\Tj{\Gamma}{t \langle t_1,\ldots,t_n \rangle}{\tau}$ mit \textsc{(App)}

    Analog zum vorangegangenen Fall der \textsc{(Cons)} Regel.

  \item $\Tj{\Gamma}{\match{t}{\zeta(X_1,\ldots,X_n)}{t'}}{\tau}$ mit \textsc{(Match)}

    Dann gilt $\Tj{\Gamma}{t}{\langle \rangle}$, $\delta(\zeta) = (\tau_1,\ldots,\tau_n)$
    und $\Tj{\Gamma[\tau_1/X_1]\ldots[\tau_n/X_n]}{t'}{\tau}$. Nach I.V. gilt die Behauptung
    f"ur $\Tj{\Gamma}{t}{\langle \rangle}$. Angenommen ex. $(d,e) \in \Delta$ mit
    \begin{mathpar}
      d = D\sem{\Tj{\Gamma}{t}{\langle \rangle}}\,\varrho_D
      \and
      e = E\sem{\Tj{\Gamma}{t}{\langle \rangle}}\,\varrho_E
    \end{mathpar}
    dann gilt $d \in D^{\langle \rangle}$ und $e \in E^{\langle \rangle}$
    wegen Lemma~\ref{lem:Wohldefiniertheit_der_Semantik}. Nehmen wir weiterhin an, dass
    $\sem{Label}_D(d) = \zeta$, dann gilt gem"a"s Definition~\ref{def:Aequivalenz_von_Implementierungen}
    auch $\sem{Label}_E(e) = \zeta$. Da $D$ nach Voraussetzung g"ultig ist, ex.
    $d_1 \in D^{\tau_1},\ldots,d_n \in D^{\tau_n}$ und $e_1 \in E^{\tau_1},\ldots,e_n \in E^{\tau_n}$, so dass
    \begin{mathpar}
      d_i = \sem{\Child_n^i}_D(d)
      \and
      e_i = \sem{\Child_n^i}_E(e)
    \end{mathpar}
    f"ur $i=1,\ldots,n$. Wiederum wg. Definition~\ref{def:Aequivalenz_von_Implementierungen} folgt
    daraus $(d_i,e_i) \in \Delta$ f"ur $i=1,\ldots,n$, also gilt insbesondere
    \begin{mathpar}
      \bigl(\varrho_D[d_1/X_1]\ldots[d_n/X_n]\bigr)
      \sim_\Delta
      \bigl(\varrho_E[e_1/X_1]\ldots[e_n/X_n]\bigr)
    \end{mathpar}
    und nach I.V. gilt die Behauptung auch f"ur $\Tj{\Gamma[\tau_1/X_1]\ldots[\tau_n/X_n]}{t'}{\tau}$.
    Angenommen ex. $(d',e')\in\Delta$ mit
    \begin{mathpar}
      d' = D\sem{\Tj{\Gamma[\tau_1/X_1]\ldots[\tau_n/X_n]}{t'}{\tau}}\,\bigl(\varrho_D[d_1/X_1]\ldots[d_n/X_n]\bigr)
      \and
      e' = E\sem{\Tj{\Gamma[\tau_1/X_1]\ldots[\tau_n/X_n]}{t'}{\tau}}\,\bigl(\varrho_E[e_1/X_1]\ldots[e_n/X_n]\bigr)
    \end{mathpar}
    dann folgt unmittelbar die Behauptung.

    Trifft eine der obigen Annahmen nicht zu, gilt trivialerweise
    \begin{mathpar}
      \bigl(D\sem{\Tj{\Gamma}{\match{t}{\zeta(X_1,\ldots,X_n)}{t'}}{\tau}}\,\varrho_D\bigr) \not\in D
      \and
      \bigl(E\sem{\Tj{\Gamma}{\match{t}{\zeta(X_1,\ldots,X_n)}{t'}}{\tau}}\,\varrho_E\bigr) \not\in E
    \end{mathpar}
    was zu zeigen war.
  \end{itemize}
\end{proof}

\begin{lemma} \label{lem:Aequivalenz_von_Faellen}
  Seien $D, E$ g"ultige, $\Delta$-"aquivalente $\Psi$-Implementierungen, $\alpha$ ein
  g"ultiger Fall, und $(d_1,e_1),\ldots,(d_n,e_n) \in \Delta$. Dann gilt:
  \begin{enumerate}
  \item Wenn $(d_1,\ldots,d_n,\Uparrow,d) \in D\sem{\alpha}$, dann ex. $e \in E$ so dass
    $(d,e) \in \Delta$ und $(e_1,\ldots,e_n,\Uparrow,e) \in E\sem{\alpha}$.
  \item Wenn $(e_1,\ldots,e_n,\Uparrow,e) \in E\sem{\alpha}$, dann ex. $d \in D$ so dass
    $(d,e) \in \Delta$ und $(d_1,\ldots,d_n,\Uparrow,d) \in D\sem{\alpha}$.
  \item Wenn $(d_1,\ldots,d_n,\Downarrow,d) \in D\sem{\alpha}$, dann ex. $e \in E$ so dass
    $(d,e) \in \Delta$ und $(e_1,\ldots,e_n,\Downarrow,e) \in E\sem{\alpha}$.
  \item Wenn $(e_1,\ldots,e_n,\Downarrow,e) \in E\sem{\alpha}$, dann ex. $d \in D$ so dass
    $(d,e) \in \Delta$ und $(d_1,\ldots,d_n,\Downarrow,d) \in D\sem{\alpha}$.
  \end{enumerate}
\end{lemma}

\begin{proof}
  Wir beweisen beispielhaft die erste Behauptung, die "ubrigen folgen analog.
  Hier muss $\alpha$ von der Form $X_1,\ldots,X_n \Uparrow t$ sein, und nach Voraussetzung
  gilt $d_1,\ldots,d_n \in D^{\langle \rangle}$ und
  $d = D\sem{\Tj{[X_i:\langle\rangle]_{i=1}^n}{t}{\langle \rangle}}\,[X_i:d_i]_{i=1}^n$.
  Gem"a"s Definition~\ref{def:Aequivalenz_von_Implementierungen} gilt $e_1,\ldots,e_n \in E^{\langle \rangle}$
  und wegen Lemma~\ref{lem:Aequivalenz_von_Termen} ex. $e \in E$, so dass
  $(d,e) \in \Delta$ mit $e = E\sem{\Tj{[X_i:\langle\rangle]_{i=1}^n}{t}{\langle \rangle}}\,[X_i:e_i]_{i=1}^n$.
  Also gilt insbesondere $(e_1,\ldots,e_n,\Uparrow,e) \in E\sem{\alpha}$.
\end{proof}

\begin{lemma} \label{lem:Aqeuivalenz_von_Regeln}
  Seien $D,E$ g"ultige $\Delta$-"aquivalente $\Psi$-Implementierungen, $r$ eine
  g"ultige Regelspezifikation, und $(d_1,e_1),\ldots,(d_n,e_n) \in \Delta$. Dann gilt:
  \begin{enumerate}
  \item Wenn $(d_1,\ldots,d_n,\Uparrow,d) \in D\sem{r}$, dann ex. $e \in E$ so dass
    $(d,e) \in \Delta$ und $(e_1,\ldots,e_n,\Uparrow,e) \in E\sem{r}$.
  \item Wenn $(e_1,\ldots,e_n,\Uparrow,e) \in E\sem{r}$, dann ex. $d \in D$ so dass
    $(d,e) \in \Delta$ und $(d_1,\ldots,d_n,\Uparrow,d) \in D\sem{r}$.
  \item Wenn $(d_1,\ldots,d_n,\Downarrow,d) \in D\sem{r}$, dann ex. $e \in E$ so dass
    $(d,e) \in \Delta$ und $(e_1,\ldots,e_n,\Downarrow,e) \in E\sem{r}$.
  \item Wenn $(e_1,\ldots,e_n,\Downarrow,e) \in E\sem{r}$, dann ex. $d \in D$ so dass
    $(d,e) \in \Delta$ und $(d_1,\ldots,d_n,\Downarrow,d) \in D\sem{r}$.
  \end{enumerate}
\end{lemma}

\begin{proof}
  Trivial.
\end{proof}


\section*{Anwendung: Subst.- $\leadsto$ Umgeb.-Modell}

Sei $\Psi$ eine beliebige Syntaxsignatur, und $\Id$ eine (unendliche) Menge von
\emph{Bezeichnern} $\id$. Die Menge $\Exp(\Psi)$ aller $\Psi$-Ausdr\"ucke $e$
ist definiert durch:
\[\begin{grammar}
  e \in \Exp(\Psi)
  \is \id
  \al \zeta(e_1,\ldots,e_n)
  \al \Lambda \id_1:\sigma_1,\ldots,\id_n:\sigma_n.e & n \ge 1
  \nl
\end{grammar}\]
Ein \emph{Typurteil f\"ur $\Psi$-Ausdr\"ucke} ist eine Formel der Gestalt $\tj{e}{\tau}$. Ein solches
Typurteil hei"st \emph{g\"ultig}, wenn es sich mit den folgenden Regeln herleiten
l"a"st:
\begin{mathpar}
  \inferrule[(Id)]{%
  }{%
    \tj{\id}{\langle \rangle}
  }%
  \and
  \inferrule[(Bind)]{%
    \tj{e}{\langle \rangle}
  }{%
    \tj{\Lambda\id_1:\sigma_1,\ldots,\id_n:\sigma_n.e}{\langle \sigma_1,\ldots,\sigma_n \rangle}
  }%
  \and
  \inferrule[(Cons)]{%
    \delta(\zeta)=(\tau_1,\ldots,\tau_n) \\
    \tj{e_i}{\tau_i} \\
    \tau_i \preceq n \\
    i=1,\ldots,n
  }{%
    \tj{\zeta(e_1,\ldots,e_n)}{\langle \rangle}
  }%
\end{mathpar}
$\free(e)$ ist wie \"ublich definiert; ein Ausdruck $e$ hei"st \emph{abgeschlossen},
wenn $\free(e) = \emptyset$. Ein \emph{Programm} ist ein abgeschlossener Ausdruck
vom Typ $\langle \rangle$.

\subsection*{Substitutionsmodell}


\end{document}
\documentclass[12pt,a2paper,draft]{article}

\usepackage{amssymb}
\usepackage{amsthm}
\usepackage[blocks]{authblk}
\usepackage[english]{babel}
\usepackage{color}
\usepackage[colorlinks=false,%
            pdfkeywords={TODO},%
            pdftitle={TODO},%
            pdfauthor={TODO},%
            pdfsubject={},%
            pdfdisplaydoctitle=true]{hyperref}
\usepackage{mathpartir}
\usepackage{ngerman} %TODO
\usepackage{varwidth}

\newcommand{\abstr}[2]{\ensuremath{\lambda{#1}.\,{#2}}}
\newcommand{\app}[2]{\ensuremath{{#1}\,{#2}}}
\newcommand{\rec}[2]{\ensuremath{{\normalfont\textsf{rec}}\,{#1}.\,{#2}}}
\newcommand{\unit}{\ensuremath{\normalfont\textsf{unit}}}
\newcommand{\Unit}{\ensuremath{\normalfont\textsf{Unit}}}

\newcommand{\Tj}[3]{\mbox{\ensuremath{{#1}\vdash{#2}:{#3}}}}
\newcommand{\tj}[2]{\Tj{\emptyset}{#1}{#2}}
\newcommand{\Pj}[3]{\mbox{\ensuremath{{#1}\vdash{#2}\rightarrow{#3}}}}
\newcommand{\tree}[1]{\mathcal{D}(#1)}
\newcommand{\rn}[1]{\mbox{\ensuremath{\textsc{(#1)}}}}

\newtheorem{lemma}{Lemma}
\newtheorem{theorem}{Theorem}
\newtheorem{definition}{Definition}

\begin{document}

\title{%
  Typsicherheit in Rekursionssystemen
}
\author{Benedikt Meurer}
\author{Kurt Sieber}
\affil{%
  Compilerbau und Softwareanalyse\\
  Universit\"at Siegen\\
  D-57072 Siegen, Germany\\
  {\tt \{meurer,sieber\}@informatik.uni-siegen.de}
}
\date{}
\maketitle
\begin{abstract}
  TODO
\end{abstract}


%% Metatheorie
\section{Metatheorie}

\begin{definition}[Rekursionssystem]
  Seien $A$ und $B$ beliebige disjunkte Mengen. Eine \emph{$(A,B)$-Rekursionsregel}
  ist eine Relation $R \subseteq A \times B^* \times (A \uplus B)$. Ein
  \emph{$(A,B)$-Rekursionssystem} ist eine Menge $S$
  von $(A,B)$-Rekursionsregeln.
  Die durch $S$ definierte \emph{nat\"urliche Semantik} ist die kleinste Relation
  $\Downarrow_S\ \subseteq A \times B$, f\"ur die gilt:
  \begin{quote}
    Wenn $R \in S$, $(a,b_1 \ldots b_{i-1},a_i) \in R$ und
    $a_i \Downarrow_S b_i$ f\"ur $i=1,\ldots,n$, und $(a,b_1 \ldots b_n,b) \in R$,
    dann gilt auch $a \Downarrow_S b$.
  \end{quote}
\end{definition}

\begin{definition}[Typsystem]
  Ein \emph{$(A,B)$-Typsystem} $T$ besteht aus
  \begin{itemize}
  \item einer Menge \textsf{Type} von Typen $\tau$, und
  \item Mengen $A^\tau \subseteq A$ und $B^\tau \subseteq B$ f\"ur jeden Typ $\tau \in \textsf{Type}$.
  \end{itemize}
\end{definition}

\begin{definition}
  Sei $S$ ein $(A,B)$-Rekursionssystem. Die Menge $\textsf{State}(S)$ aller
  \emph{Rekursionszust\"ande} $st$ von $S$ ist definiert als
  \begin{quote}
    $\textsf{State}(S) = (A \cup B \cup S)^+$.
  \end{quote}
  Auf $\textsf{State}(S)$ sei eine \"Ubergangsrelation $\vdash_S$ definiert durch:
  \begin{mathpar}
    \inferrule[(AA)]{%
      (a,\varepsilon,a') \in R
    }{%
      w\,a \vdash_S w\,a\,R\,a'
    }
    \and
    \inferrule[(AB)]{%
      (a,\varepsilon,b) \in R
    }{%
      w\,a \vdash_S w\,b
    }
    \and
    \inferrule[(BB)]{%
      (a,b_1 \ldots b_n,b) \in R
    }{%
      w\,a\,R\,b_1 \ldots b_n \vdash_S w\,b
    }
    \and
    \inferrule[(BA)]{%
      (a,b_1 \ldots b_n,a_{n+1}) \in R
    }{%
      w\,a\,R\,b_1 \ldots b_n \vdash_S w\,a\,R\,b_1 \ldots b_n\,a_{n+1}
    }
  \end{mathpar}
  Eine \emph{Berechnung f\"ur $a$} ist eine maximale Folge $a \vdash_S \ldots$. Ist die Folge unendlich,
  so sagen wir Berechnung \emph{divergiert}, ist die Folge endlich und der letzte Rekursionszustand ein
  Resultat, so sagen wir die Berechnung \emph{terminiert}, ansonsten sagen wir die Berechnung \emph{bleibt
  stecken}. $S$ hei"st \emph{deterministisch}, wenn zu jedem $a \in A$ exakt eine Berechnung existiert.
\end{definition}

\begin{definition}[Zwischenresultate]
  $b_1,\ldots,b_n$ ($n \ge 0$) hei"sen \emph{Zwischenresultate f\"ur $a$ bzgl. $R$}, wenn
  $a_1,\ldots,a_n$ existieren, so dass gilt:
  \begin{itemize}
  \item $a_i \Downarrow_S b_i$ f\"ur $i=1,\ldots,n$, und
  \item $(a,b_1 \ldots b_{i-1}, a_i) \in R$ f\"ur $i = 1,\ldots,n$.
  \end{itemize}
\end{definition}

\noindent
(Intuitiv sind $b_1,\ldots,b_n$ die Resultate der ersten $n$ Pr\"amissen f\"ur $a$ in der
Regel $R$).

\begin{lemma}[Reachable states] \label{lemma:Reachable_states}
  Wenn $a_0 \vdash^* st$ und $st'$ nicht-leeres Pr\"afix von $st$, dann gilt eine
  der folgenden Aussagen:
  \begin{enumerate}
  \item[(1)] $st' = a_0$
  \item[(2)] $st' \in B$
  \item[(3)] $st'$ endet auf $a\,R\,b_1 \ldots b_n$ und $b_1,\ldots,b_n$ sind Zwischenresultate
    f\"ur $a$ bzgl. $R$.
  \item[(4)] $st'$ endet auf $a\,R\,b_1 \ldots b_n\,a'$, wobei $b_1,\ldots,b_n$ Zwischenresultate
    f\"ur $a$ bzgl. $R$ sind und $(a,b_1 \ldots b_n,a') \in R$.
  \end{enumerate}
\end{lemma}

\begin{proof}
  Induktion \"uber die L\"ange der Berechnung. I.A. trivial, f\"ur den
  I.S. Fallunterscheidung nach der zuletzt angewendeten Regel:
  \begin{itemize}
  \item \rn{AA} $a_0 \vdash^* w\,a \vdash w\,a\,R\,a'$ mit $(a,\varepsilon,a') \in R$
    
    Nach I.V. hat jedes nicht-leere Pr\"afix von $w\,a$ die passende Form, f\"ur $w\,a\,R$
    gilt trivialerweise $(3)$, f\"ur $w\,a\,R\,a'$ gilt $(4)$ wegen $(a,\varepsilon,a') \in R$.

  \item \rn{BB} $a_0 \vdash^* w\,a\,R\,b_1 \ldots b_n \vdash w\,b$ mit $(a,b_1 \ldots b_n,b) \in R$

    Nach I.V. hat jedes nicht-leere Pr\"afix von $w\,a$ die passende Form, also insbesondere auch
    jedes nicht-leere Pr\"afix von $w$, wobei gilt entweder $w = \varepsilon$ oder $w$ endet
    auf $a'\,R'\,b_1' \ldots b_m'$ und $b_1',\ldots,b_m'$ sind Zwischenresultate f\"ur $a'$ bzgl.
    $R'$.
    Der erste Fall ist klar, f\"ur den zweiten Fall gilt nach Voraussetzung $a \Downarrow b$,
    also hat $w\,b$ die Form $(4)$.

  \item \rn{AB} $a_0 \vdash^* w\,a \vdash w\,b$ mit $(a,\varepsilon,b) \in R$

    analog zu \rn{BB}

  \item \rn{BA} $a_0 \vdash^* w\,a\,R\,b_1 \ldots b_n \vdash w\,a\,R\,b_1 \ldots b_n\,a_{n+1}$

    Folgt unmittelbar mit $(4)$ wegen $(a,b_1 \ldots b_n,a_{n+1}) \in R$.
  \end{itemize}
\end{proof}

\begin{definition}[Programmiersprache]
  Eine \emph{Programmiersprache} $\mathcal{L}$ ist ein Quadrupel $(A,B,S,T)$, wobei $S$ ein
  $(A,B)$-Rekursionssystem und $T$ ein $(A,B)$-Typsystem ist. $\mathcal{L}$ hei"st \emph{typsicher},
  wenn f\"ur jedes $a \in \bigcup_{\tau}A^\tau$ gilt: Es existiert keine Berechnung f\"ur $a$,
  die stecken bleibt.
\end{definition}

Formal ist eine Programmiersprache typsicher, wenn aus $a \in A^\tau$ und $a \vdash^*_S st$ folgt,
dass entweder $st \in B$ oder es ex. $st'$ mit $st \vdash_S st'$.

\begin{lemma}[Typerhaltung f"ur Argumente] \label{lemma:Argument_preservation}
  Wenn f\"ur alle $a \in A^\tau$, $(a,b_1 \ldots b_n,a') \in R$ mit $b_1,\ldots,b_n$ Zwischenresultate
  f\"ur $a$ bzgl. $R$ ein $\tau'$ mit $a' \in A^{\tau'}$ existiert, dann gilt:
  \begin{quote}
    Wenn $a \vdash^* v\,a'\,w$ und $a \in A^\tau$, dann existiert ein $\tau'$ so dass $a' \in A^{\tau'}$.
  \end{quote}
\end{lemma}

\begin{proof}
  Straight-forward induction.
\end{proof}

\begin{theorem}[Meta-Theorem]
  F\"ur Typsicherheit gen\"ugt zu zeigen:
  \begin{enumerate}
  \item \emph{Local Progress}

    Wenn $a \in A^\tau$, dann ex. $R$ und $x \in A \cup B$ mit $(a,\varepsilon,x) \in R$.

    Wenn $a \in A^\tau$ und $b_1,\ldots,b_n$ Zwischenresultate f\"ur $a$ bzgl. $R$ $(n \ge 1)$,
    dann ex. $x \in A \cup B$ mit $(a,b_1 \ldots b_n,x) \in R$.

  \item \emph{Local Preservation}

    Wenn $a \in A^\tau$, $(a,b_1 \ldots b_n,a') \in R$ und $b_1,\ldots,b_n$ Zwischenresultate
    f\"ur $a$ bzgl. $R$, dann ex. $\tau'$ mit $a' \in A^{\tau'}$.
  \end{enumerate}
\end{theorem}

\noindent
Zum Beweis von \emph{Local Preservation} im konkreten Anwendungsfall ben\"otigt man mit ziemlicher
Sicherheit \emph{Preservation} \"uber big steps, also: Wenn $a \in A^\tau$ und $a \Downarrow b$,
dann $b \in B^\tau$.

\begin{proof}
  Sei $a_0 \in A^{\tau_0}$ und $a_0 \vdash^* st$. Nach Lemma~\ref{lemma:Reachable_states} unterscheiden wir
  nach der Form von $st$:
  \begin{enumerate}
  \item[(1,4)] $st = wa$

    Wegen Lemma~\ref{lemma:Argument_preservation} ex. $\tau$ mit $a \in A^\tau$. Nach \emph{Local Progress}
    ex. $R$, $x$ mit $(a,\varepsilon,x) \in R$, also in weiterer Schritt mit \rn{AB} oder \rn{AA} je nach Form
    von $x$.

  \item[(2)] $st \in B$

    Klar.

  \item[(3)] $st = w\,a\,R\,b_1 \ldots b_n$ und $b_1,\ldots,b_n$ sind Zwischenresultate f\"ur $a$ bzgl.
    $R$.

    Wg. Lemma~\ref{lemma:Argument_preservation} ex. $\tau$ mit $a \in A^\tau$, ex. nach \emph{Local Progress}
    ein $x$ mit $(a,b_1 \ldots b_n,x) \in R$, und je nach Form von $x$ ein weiterer Schritt mit \rn{BA} oder \rn{BB}.
  \end{enumerate}
\end{proof}


%% Anwendungsbeispiele
\section{Anwendungsbeispiele}

\begin{enumerate}
\item $\mathcal{L}_2$ (ohne exceptions) $A = \textsf{Exp}$, $B = \textsf{Val}$
\item $\mathcal{L}_2$ (mit exceptions) $A = \textsf{Exp}$, $B = \textsf{Exn} \cup \textsf{Val}$
\item $\mathcal{L}_2$ (mit exceptions und states) $A = \textsf{Exp} \times \textsf{State}$,
  $B = (\textsf{Exn} \cup \textsf{Val}) \times \textsf{State}$ \\
  dazu: Schreibweisen, die es erlauben eine Relation $R$ durch exceptions und states
  zu ``erweitern''.
\end{enumerate}

\subsection{Simply-typed $\lambda$-calculus with unit and recursion}

This section describes the programming language considered in this paper,
the simply typed $\lambda$-calculus with $\unit$ and general recursion, which
presents the simplest possible functional language that exhibits run-time
errors (closed expressions that ``go wrong'') and divergence. Figure~\ref{fig:Basic_syntax}
shows the basic syntax of the programming language.

\begin{figure}[htb]
  \centering
  \begin{tabular}{llcl}
    Variables   & \multicolumn{3}{l}{$x,y,z,\ldots$} \\
    Expressions & $e$ & $::=$ & $\unit \mid x \mid \abstr{x}{e} \mid \app{e_1}{e_2} \mid \rec{x}{e}$ \\
    Values      & $v$ & $::=$ & $\unit \mid \abstr{x}{e}$
  \end{tabular}
  \caption{Basic syntax}
  \label{fig:Basic_syntax}
\end{figure}

We write $e'[x \mapsto e]$ for the capture-avoiding substitution of $e$ for all free occurrences
of $x$ in $e'$. The standard call-by-value semantics for this language is shown in
figure~\ref{fig:Call_by_value_recursion_rules} as set of recursion rules.

\begin{figure}[htb]
  \centering
  \[\begin{array}{rcl}
    \textsc{Val} &:=& \{(v,\varepsilon,\underline{v})\} \\
    \textsc{App} &:=& \{(\app{e_1}{e_2},\varepsilon,e_1)\} \\
                 &\cup& \{(\app{e_1}{e_2},\underline{\abstr{x}{e}},e_2)\} \\
                 &\cup& \{(\app{e_1}{e_2},\underline{\abstr{x}{e}}\,\underline{v},e[x \mapsto v])\} \\
                 &\cup& \{(\app{e_1}{e_2},\underline{\abstr{x}{e}}\,\underline{v}\,\underline{v'},\underline{v'})\} \\
    \textsc{Rec} &:=& \{(\rec{x}{e},\varepsilon,e[x \mapsto \rec{x}{e}])\} \\
                 &\cup& \{(\rec{x}{e},\underline{v},\underline{v})\}
  \end{array}\]
  \caption{Call-by-value recursion rules}
  \label{fig:Call_by_value_recursion_rules}
\end{figure}

\begin{figure}[htb]
  \centering
  \begin{mathpar}
    \inferrule[(T-Unit)]{}{\Tj{\Gamma}{\unit}{\Unit}}
    \and
    \inferrule[(T-Var)]{%
      \Gamma(x) = \tau
    }{%
      \Tj{\Gamma}{x}{\tau}
    }
    \and
    \inferrule[(T-Abstr)]{%
      \Tj{\Gamma + \{x : \tau \}}{e}{\tau'}
    }{%
      \Tj{\Gamma}{\abstr{x}{e}}{\tau \to \tau'}
    }
    \\
    \inferrule[(T-Rec)]{%
      \Tj{\Gamma + \{x:\tau\}}{e}{\tau}
    }{%
      \Tj{\Gamma}{\rec{x}{e}}{\tau}
    }
    \and
    \inferrule[(T-App)]{%
      \Tj{\Gamma}{e_1}{\tau' \to \tau} \\
      \Tj{\Gamma}{e_2}{\tau'}
    }{%
      \Tj{\Gamma}{\app{e_1}{e_2}}{\tau}
    }
  \end{mathpar}
  \caption{Typing rules}
  \label{fig:Typing_rules}
\end{figure}

\begin{lemma}[Preservation of types under substitution] \label{lemma:Preservation_of_types_under_substitution}
  If $\Tj{\Gamma + \{x:\tau\}}{e'}{\tau'}$ and $\Tj{\Gamma}{e}{\tau}$,
  then $\Tj{\Gamma}{e'[x \mapsto e]}{\tau'}$.
\end{lemma}

\begin{lemma}[Canonical Forms] \label{lemma:Canonical_Forms} \
  \begin{enumerate}
  \item If $\tj{v}{\Unit}$, then $v = \unit$.
  \item If $\tj{v}{\tau' \to \tau}$, then $v = \abstr{x}{e}$.
  \end{enumerate}
\end{lemma}

\begin{lemma}[Preservation of types across big steps] \label{lemma:Preservation_of_types_across_big_steps}
  If $\Tj{\Gamma}{e}{\tau}$ and $e \Downarrow \underline{v}$, then $\Tj{\Gamma}{v}{\tau}$.
\end{lemma}

\begin{proof}
  Yep.
\end{proof}

\begin{theorem}[Local Progress] \
  \begin{enumerate}
  \item Wenn $\tj{e}{\tau}$ dann ex. $R$ und entweder $e'$ mit $(e,\varepsilon,e') \in R$
    oder $v$ mit $(e,\varepsilon,\underline{v}) \in R$.
  \item Wenn $\tj{e}{\tau}$ und $\underline{v_1},\ldots,\underline{v_n}$ Zwischenresultate
    f\"ur $e$ bzgl. $R$ $(n \ge 1)$, dann ex. entweder $e'$ mit $(e,\underline{v_1}\ldots\underline{v_n},e') \in R$
    oder ein $v$ mit $(e,\underline{v_1}\ldots\underline{v_n},v) \in R$.
  \end{enumerate}
\end{theorem}

\begin{proof} \
  \begin{enumerate}
  \item Klar.
  \item Hier kommen nur \rn{T-App} und \rn{T-Rec} in Frage. F\"ur Applikationen braucht man
    Lemma~\ref{lemma:Preservation_of_types_across_big_steps} (Preservation).
  \end{enumerate}
\end{proof}

\begin{theorem}[Local Preservation]
  Wenn $\tj{e}{\tau}$, $(e,\underline{v_1}\ldots\underline{v_n},e') \in R$ und
  $\underline{v_1},\ldots,\underline{v_n}$ Zwischenresultate f\"ur $e$ bzgl. $R$, dann
  ex. $\tau'$ so dass $\tj{e'}{\tau'}$.
\end{theorem}

\begin{proof}
  Ebenfalls nur \rn{T-App} und \rn{T-Rec} m\"oglich, ben\"otigt
  Lemma~\ref{lemma:Preservation_of_types_under_substitution}
  und Lemma~\ref{lemma:Preservation_of_types_across_big_steps}.
\end{proof}


%% References
\bibliographystyle{abbrv}
\bibliography{citations}


\end{document}
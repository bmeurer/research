\documentclass[12pt,a4paper]{article}

\usepackage{amsmath}
\usepackage{amssymb}
\usepackage{amstext}
\usepackage{array}
\usepackage[american]{babel}
\usepackage{color}
\usepackage{enumerate}
\usepackage[a4paper,%
            colorlinks=false,%
            final,%
            pdfkeywords={},%
            pdftitle={},%
            pdfauthor={Benedikt Meurer},%
            pdfsubject={},%
            pdfdisplaydoctitle=true]{hyperref}
\usepackage{ifthen}
\usepackage[latin1]{inputenc}
\usepackage{latexsym}
\usepackage[final]{listings}
\usepackage{makeidx}
\usepackage{ngerman}
\usepackage[standard,thmmarks]{ntheorem}
\usepackage{stmaryrd}

%% LaTeX macros
%%
%% macros.tex - Useful LaTeX macros
%%
%% Copyright (c) 2006-2011 Benedikt Meurer <benedikt.meurer@googlemail.com>
%% 


%%
%% Styles
%%

\newcommand{\nstyle}[1]{\ensuremath{\mathsf{#1}}}
\newcommand{\sstyle}[1]{\ensuremath{\mathit{#1}}}


%%
%% Misc
%%

\newcommand{\abort}{\ensuremath{\mathbf{abort}}}
\newcommand{\pto}{\rightharpoonup}
\newcommand{\step}{\ensuremath{\rightsquigarrow}}

\newcommand{\sem}[1]{\ensuremath{[\![#1]\!]}}


%%
%% Names
%%

\newcommand{\arity}{\ensuremath{\mathit{arity}}}
\newcommand{\cl}{\ensuremath{\mathit{cl}}}
\newcommand{\fr}{\ensuremath{\mathit{fr}}}
\newcommand{\free}{\ensuremath{\mathit{free}}}
\newcommand{\graph}{\ensuremath{\mathit{graph}}}
\newcommand{\id}{\ensuremath{\mathit{id}}}


%%
%% Sets
%%

\newcommand{\I}{\ensuremath{\mathcal I}}
\newcommand{\N}{\ensuremath{\mathbb N}}
\renewcommand{\O}{\ensuremath{\mathcal O}}
\newcommand{\Z}{\ensuremath{\mathbb Z}}
\newcommand{\Cl}{\sstyle{Cl}}
\newcommand{\Env}{\sstyle{Env}}
\newcommand{\Exp}{\sstyle{Exp}}
\newcommand{\Frame}{\sstyle{Frame}}
\newcommand{\Id}{\sstyle{Id}}
\newcommand{\Node}{\sstyle{Node}}
\newcommand{\Val}{\sstyle{Val}}


%%
%% Expressions
%%

\newcommand{\app}[2]{{#1}\,{#2}}
\newcommand{\abstr}[2]{\lambda{#1}.\,{#2}}
\newcommand{\ifte}[3]{\mathbf{if}\,{#1}\,\mathbf{then}\,{#2}\,\mathbf{else}\,{#3}}


%%
%% Values
%%

\newcommand{\clov}[2]{\langle{#1},{#2}\rangle}
\newcommand{\false}{\mathbf{false}}
\newcommand{\true}{\mathbf{true}}


%%
%% Grammars
%%

\newenvironment{grammar}{\begin{array}{rrlll}}{\end{array}}

\newcommand{\is}{& ::= &}
\newcommand{\al}{\\ & \mid &}
\newcommand{\nl}{\vspace{2mm}\\}


%%
%% Other environments
%%

\newenvironment{case}{\left\{\!\!\!\begin{array}{ll}}{\end{array}\right.}


%%% Local Variables: 
%%% mode: latex
%%% TeX-master: "compiler"
%%% End: 


\begin{document}


\section{Semantik von Programmiersprachen}

\begin{definition}[Syntax]
  Eine \emph{Syntax} ist eine Menge $\L$ von syntaktisch korrekten Programmen.
\end{definition}

\begin{definition}[Semantik]
  Sei $\L$ eine Syntax. Eine \emph{$\L$-Semantik} ist ein Paar $S = (\Obs,\beh)$.
  Hierbei ist
  \begin{enumerate}
  \item $\Obs$ eine Menge von \emph{beobachtbaren Verhalten}, und
  \item $\beh: \L \to \powerset{\Obs}$ eine \emph{Beobachtungsfunktion}, die jedem $\L$-Programm
    beobachtbares Verhalten zuordnet.
  \end{enumerate}
\end{definition}

\begin{definition}[Implementierung]
  Sei $\L$ eine Syntax und sei $\Obs$ eine Menge von beobachtbaren Verhalten.
  Eine \emph{$(\L,\Obs)$-Implementierung} ist ein Tupel $M = (\Exp,D,\semantic{\cdot},\obs)$,
  bestehend aus
  \begin{enumerate}
  \item der Menge $\Exp \supseteq \L$ von \emph{Ausdr"ucken},
  \item dem \emph{semantischen Bereich} $D$,
  \item der \emph{Semantikfunktion} $\semantic{\cdot}:\Exp \to D$, welche jedem Ausdruck
    ein Element des semantischen Bereichs zuordnet, und
  \item der \emph{Bewertungsfunktion} $\obs:D \to \powerset{\Obs}$, welche jedes Element
    des semantischen Bereichs als beobachtbares Verhalten bewertet.
  \end{enumerate}
  Eine Implementierung hei"st \emph{ad"aquat} f"ur eine $\L$-Semantik $S=(\Obs,\beh)$,
  wenn $\obs(\semantic{e}) = \beh(e)$ f"ur alle $e \in \L$ gilt.
\end{definition}

\begin{lemma} \label{lemma1}
  Sei $(\L,S)$ eine Programmiersprache mit $S = (\Obs,\beh)$ und $M_1 = (\Exp,D_1,\semantic{\cdot}_1,\obs_1)$
  eine ad"aquate $(\L,\Obs)$-Implementierung f"ur $S$.
  Eine weitere $(\L,\Obs)$-Implementierung $M_2=(\Exp,D_2,\semantic{\cdot}_2,\obs_2)$
  ist ad"aquat f"ur $S$, wenn $\obs_1(\semantic{e}_1) = \obs_2(\semantic{e}_2)$ f"ur alle $e \in \Exp$.
\end{lemma}

\begin{definition}
  Eine Menge $A$ hei"st \emph{induktiv definiert} durch eine Familie $C = (C_n)_{n\in\N}$ von
  \emph{Konstruktoren} $\zeta_n$, wenn $A$ der induktive Abschluss von $C$ ist.
\end{definition}

\begin{definition}[Kompositionalit"at]
  Sei $\Exp$ induktiv definiert durch die Familie $C=(C_n)_{n\in\N}$.
  Eine Implementierung $M=(\Exp,D,\semantic{\cdot},\obs)$ hei"st \emph{$C$-kompositional},
  wenn zu jedem $\zeta_n \in C_n$ eine Funktion $\semantic{\zeta_n}: D^n \to D$ existiert, so dass gilt:
  Wenn $e = \zeta_n (e_1 \ldots e_n)$, dann 
  $\semantic{e} = \semantic{\zeta_n}(\semantic{e_1} \ldots \semantic{e_n})$.
\end{definition}

\begin{theorem} \label{theorem1}
  Sei $(\L,S)$ eine Programmiersprache mit $S=(\Obs,\beh)$, $\Exp$ induktiv definiert durch
  $C=(C_n)_{n\in\N}$, und seien $M_1=(\Exp,D_1,\semantic{\cdot}_1,\obs_1)$ und
  $M_2 = (\Exp,D_2,\semantic{\cdot}_2,\obs_2)$ $C$-kompositionale $(\L,\Obs)$-Implementierungen.
  $M_2$ ist ad"aquat f"ur $S$, wenn $M_1$ ad"aquat f"ur $S$ und eine Relation $\sim\ \subseteq D_1 \times D_2$
  existiert, so dass gilt:
  \begin{enumerate}
  \item Wenn $d_{1,i} \sim d_{2,i}$ f"ur $i=1,\ldots,n$, dann gilt
    \[\semantic{\zeta_n}_1 (d_{1,1} \ldots d_{1,n}) \sim \semantic{\zeta_n}_2 (d_{2,1} \ldots d_{2,n})\]
    f"ur alle $n \ge 0$ und $\zeta_n \in C_n$.
  \item Wenn $d_1 \sim d_2$, dann gilt $\obs_1(d_1) = \obs_2(d_2)$.
  \end{enumerate}
\end{theorem}

\begin{proof}
  Wir zeigen zun"achst, dass $\semantic{e}_1 \sim \semantic{e}_2$ f"ur alle $e\in\Exp$. Da $M_1$ und $M_2$
  nach Voraussetzung $C$-kompositional sind, l"a"st sich dies leicht durch Induktion "uber die Struktur
  von $e$ zeigen:
  \begin{itemize}
  \item F"ur $e = \zeta_0$ folgt nach Voraussetzung unmittelbar $\semantic{\zeta_0}_1 \sim \semantic{\zeta_0}_2$.
  \item F"ur $e = \zeta_n(e_1 \ldots e_n)$ folgt zun"achst nach Induktionsvoraussetzung, dass
    $\semantic{e_i}_1 \sim \semantic{e_i}_2$ f"ur alle $i=1,\ldots,n$ gilt. Daraus folgt dann unmittelbar,
    dass auch $\semantic{\zeta_n(e_1 \ldots e_n)}_1 \sim \semantic{\zeta_n(e_1 \ldots e_n)}_2$ gilt.
  \end{itemize}
  $\semantic{e}_1 \sim \semantic{e}_2$ wiederum impliziert $\obs_1(\semantic{e}_1)=\obs_2(\semantic{e}_2)$,
  also ist nach Lemma~\ref{lemma1} $M_2$ ebenfalls ad"aquat.
\end{proof}

Dies bedeutet, dass es, gegeben eine ad"aquate Implementierung einer Programmiersprache, f"ur den
Beweis der Ad"aquatheit einer weiteren Implementierung derselben Programmiersprache, ausreicht zu
zeigen, dass eine Relation zwischen den verschiedenen semantischen Bereichen existiert, welche
genau diejenigen Bereiche enth"alt, deren beobachtbares Verhalten "ubereinstimmt, und dass die
Semantik der semantisch relevanten Bestandteile der Programme diese Relation erh"alt.

Interessant ist inbesondere der Fall, dass der semantische Bereich $D$ aus Funktionen $A \to B$ besteht,
f"ur zun"achst beliebige $A,B$. Dies entspricht den "ublichen Substitutions- und Umgebungsmodellen, welche
Ausdr"ucke in Abh"angigkeit einer Substitution bzw. einer Umgebung auf einen Wert abbilden (bzw. in einer
denotationellen Semantik auf seine Bedeutung abbilden).

\begin{theorem}
  Sei $(\L,S)$ eine Programmiersprache mit $S = (\Obs,\beh)$, $\Exp$ induktiv definiert durch
  $C = (C_n)_{n\in\N}$, und seien $M_1 = (\Exp,A_1 \to B_1,\semantic{\cdot}_1,\obs_1)$ und
  $M_2 = (\Exp,A_2 \to B_2,\semantic{\cdot}_2,\obs_2)$ $C$-kompositionale $(\L,\Obs)$-Implementierungen.
  $M_2$ ist ad"aquat f"ur $S$, wenn $M_1$ ad"aquat f"ur $S$ ist und Relationen
  $\sim\ \subseteq A_1 \times A_2$ und $\sim\ \subseteq B_1 \times B_2$ existieren, so dass gilt:
  \begin{enumerate}
  \item Wenn $d_{1,i}\,a_1 \sim d_{2,i}\,a_2$ f"ur alle $a_1 \in A_1,a_2 \in A_2$ mit
    $a_1 \sim a_2$ und $i = 1,\ldots,n$ dann gilt
    \[\begin{array}{c}
      \semantic{\zeta_n}_1\,(d_{1,1} \ldots d_{1,n})\,a_1
      \sim
      \semantic{\zeta_n}_2\,(d_{2,1} \ldots d_{2,n})\,a_2
    \end{array}\]
    f"ur alle $a_1 \sim a_2$, $n \ge 0$ und $\zeta_n \in C_n$.
  \item Wenn $d_1\,a_1 \sim d_2\,a_2$ f"ur alle $a_1 \sim a_2$, dann gilt
    $\obs_1(d_1) = \obs_2(d_2)$.
  \end{enumerate}
\end{theorem}

\begin{proof}
  Die Relationen $\sim\ \subseteq A_1 \times A_2$ und $\sim\ \subseteq B_1 \times B_2$ definieren
  wie folgt eine Relation $\sim$ auf $D_1 \times D_2$:
  \[\begin{array}{rcl}
    d_1 \sim d_2
    &:\Leftrightarrow&
    \forall a_1,a_2.\,a_1 \sim a_2 \Rightarrow d_1\,a_1 \sim d_2\,a_2
  \end{array}\]
  Seien nun also $d_{1,i} \sim d_{2,i}$ f"ur $i=1,\ldots,n$, dann folgt nach der ersten Voraussetzung, dass
  \[\begin{array}{c}
      \semantic{\zeta_n}_1\,(d_{1,1} \ldots d_{1,n})\,a_1
      \sim
      \semantic{\zeta_n}_2\,(d_{2,1} \ldots d_{2,n})\,a_2
  \end{array}\]
  f"ur alle $a_1 \sim a_2$ gilt, also gilt auch insbesondere
  \[\begin{array}{c}
      \semantic{\zeta_n}_1\,(d_{1,1} \ldots d_{1,n})
      \sim
      \semantic{\zeta_n}_2\,(d_{2,1} \ldots d_{2,n}).
  \end{array}\]
  Nach Theorem~\ref{theorem1} ist demzufolge auch $M_2$ ad"aquat f"ur $S$.
\end{proof}

\end{document}
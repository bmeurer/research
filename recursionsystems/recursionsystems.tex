\documentclass[12pt,a2paper,draft]{article}

\usepackage{amssymb}
\usepackage{amsthm}
\usepackage[blocks]{authblk}
\usepackage[english]{babel}
\usepackage{color}
\usepackage[colorlinks=false,%
            pdfkeywords={TODO},%
            pdftitle={TODO},%
            pdfauthor={TODO},%
            pdfsubject={},%
            pdfdisplaydoctitle=true]{hyperref}
\usepackage{mathpartir}
\usepackage{ngerman} %TODO
\usepackage{varwidth}

\newcommand{\abstr}[2]{\ensuremath{\lambda{#1}.\,{#2}}}
\newcommand{\app}[2]{\ensuremath{{#1}\,{#2}}}
\newcommand{\rec}[2]{\ensuremath{{\normalfont\textsf{rec}}\,{#1}.\,{#2}}}
\newcommand{\unit}{\ensuremath{\normalfont\textsf{unit}}}
\newcommand{\Unit}{\ensuremath{\normalfont\textsf{Unit}}}

\newcommand{\Tj}[3]{\mbox{\ensuremath{{#1}\vdash{#2}:{#3}}}}
\newcommand{\tj}[2]{\Tj{\emptyset}{#1}{#2}}
\newcommand{\Pj}[3]{\mbox{\ensuremath{{#1}\vdash{#2}\rightarrow{#3}}}}
\newcommand{\tree}[1]{\mathcal{D}(#1)}
\newcommand{\rn}[1]{\mbox{\ensuremath{\textsc{(#1)}}}}

\newtheorem{lemma}{Lemma}
\newtheorem{theorem}{Theorem}
\newtheorem{definition}{Definition}

\begin{document}

\title{%
  Typsicherheit in Rekursionssystemen
}
\author{Benedikt Meurer}
\author{Kurt Sieber}
\affil{%
  Compilerbau und Softwareanalyse\\
  Universit\"at Siegen\\
  D-57072 Siegen, Germany\\
  {\tt \{meurer,sieber\}@informatik.uni-siegen.de}
}
\date{}
\maketitle
\begin{abstract}
  TODO
\end{abstract}


\begin{definition}[Rekursionssystem]
  Seien $A$ und $B$ beliebige disjunkte Mengen. Eine \emph{$(A,B)$-Rekursionsregel}
  ist eine Relation $R \subseteq A \times B^* \times (A \uplus B)$. Ein
  \emph{$(A,B)$-Rekursionssystem} ist eine Menge $S$
  von $(A,B)$-Rekursionsregeln.
  Die durch $S$ definierte \emph{nat\"urliche Semantik} ist die kleinste Relation
  $\Downarrow_S\ \subseteq A \times B$, f\"ur die gilt:
  \begin{quote}
    Wenn $R \in S$, $(a,b_1 \ldots b_{i-1},a_i) \in R$ und
    $a_i \Downarrow_S b_i$ f\"ur $i=1,\ldots,n$, und $(a,b_1 \ldots b_n,b) \in R$,
    dann gilt auch $a \Downarrow_S b$.
  \end{quote}
\end{definition}

\begin{definition}[Typsystem]
  Ein \emph{$(A,B)$-Typsystem} $T$ besteht aus
  \begin{itemize}
  \item einer Menge \textsf{Type} von Typen $\tau$, und
  \item Mengen $A^\tau \subseteq A$ und $B^\tau \subseteq B$ f\"ur jeden Typ $\tau \in \textsf{Type}$.
  \end{itemize}
\end{definition}

\begin{definition}
  Sei $S$ ein $(A,B)$-Rekursionssystem. Die Menge $\textsf{State}(S)$ aller
  \emph{Rekursionszust\"ande} $st$ von $S$ ist definiert als
  \begin{quote}
    $\textsf{State}(S) = (A \cup B \cup S)^+$.
  \end{quote}
  Auf $\textsf{State}(S)$ sei eine \"Ubergangsrelation $\vdash_S$ definiert durch:
  \begin{mathpar}
    \inferrule[(AA)]{%
      (a,\varepsilon,a') \in R
    }{%
      w\,a \vdash_S w\,a\,R\,a'
    }
    \and
    \inferrule[(AB)]{%
      (a,\varepsilon,b) \in R
    }{%
      w\,a \vdash_S w\,b
    }
    \and
    \inferrule[(BB)]{%
      (a,b_1 \ldots b_n,b) \in R
    }{%
      w\,a\,R\,b_1 \ldots b_n \vdash_S w\,b
    }
    \and
    \inferrule[(BA)]{%
      (a,b_1 \ldots b_n,a_{n+1}) \in R
    }{%
      w\,a\,R\,b_1 \ldots b_n \vdash_S w\,a\,R\,b_1 \ldots b_n\,a_{n+1}
    }
  \end{mathpar}
  Eine \emph{Berechnung f\"ur $a$} ist eine maximale Folge $a \vdash_S \ldots$. Ist die Folge unendlich,
  so sagen wir Berechnung \emph{divergiert}, ist die Folge endlich und der letzte Rekursionszustand ein
  Resultat, so sagen wir die Berechnung \emph{terminiert}, ansonsten sagen wir die Berechnung \emph{bleibt
  stecken}. $S$ hei"st \emph{deterministisch}, wenn zu jedem $a \in A$ exakt eine Berechnung existiert.
\end{definition}

\begin{lemma}[Reachable states]
  Wenn $a_0 \vdash^* st$ und $st'$ nicht-leeres Pr\"afix von $st$, dann gilt eine
  der folgenden Aussagen:
  \begin{enumerate}
  \item[(1)] $st' = a_0$
  \item[(2)] $st' \in B$
  \item[(3)] $st'$ endet auf $a\,R\,b_1 \ldots b_n$, wobei ex. $a_1,\ldots,a_n$ mit
    \begin{itemize}
    \item $(a,b_1 \ldots b_i,a_{i+1}) \in R$ f\"ur $i=1,\ldots,n-1$, und
    \item $a_i \Downarrow b_i$ f\"ur $i=1,\ldots,n$.
    \end{itemize}
  \item[(4)] $st'$ endet auf $a\,R\,b_1 \ldots b_n\,a_{n+1}$, wobei ex. $a_1,\ldots,a_n$ mit
    \begin{itemize}
    \item $(a,b_1 \ldots b_i,a_{i+1}) \in R$ f\"ur $i=1,\ldots,n$, und
    \item $a_i \Downarrow b_i$ f\"ur $i=1,\ldots,n$.
    \end{itemize}
  \end{enumerate}
\end{lemma}

\begin{proof}
  Induktion \"uber die L\"ange der Berechnung. I.A. trivial, f\"ur den
  I.S. Fallunterscheidung nach der zuletzt angewendeten Regel:
  \begin{itemize}
  \item \rn{AA} $a_0 \vdash^* w\,a \vdash w\,a\,R\,a'$ mit $(a,\varepsilon,a') \in R$
    
    Nach I.V. hat jedes nicht-leere Pr\"afix von $w\,a$ die passende Form, f\"ur $w\,a\,R$
    gilt trivialerweise $(3)$, f\"ur $w\,a\,R\,a'$ gilt $(4)$ wegen $(a,\varepsilon,a') \in R$.

  \item \rn{BB} $a_0 \vdash^* w\,a\,R\,b_1 \ldots b_n \vdash w\,b$ mit $(a,b_1 \ldots b_n,b) \in R$

    Nach I.V. hat jedes nicht-leere Pr\"afix von $w\,a$ die passende Form, also insbesondere auch
    jedes nicht-leere Pr\"afix von $w$, wobei gilt entweder $w = \varepsilon$ oder $w$ endet
    auf $\hat{a}\,\hat{R}\,\hat{b}_1 \ldots \hat{b}_m$ und es ex. $\hat{a}_1,\ldots,\hat{a}_m$
    so dass gilt
    \begin{itemize}
    \item $(\hat{a},\hat{b}_1 \ldots \hat{b}_m,\hat{a}_{i+1}) \in \hat{R}$ f\"ur $i=1,\ldots,m-1$, und
    \item $\hat{a}_i \Downarrow \hat{b}_i$ f\"ur $i=1,\ldots,m$.
    \end{itemize}
    Der erste Fall ist klar, f\"ur den zweiten Fall gilt nach Voraussetzung $a \Downarrow b$,
    also hat $w\,b$ die Form $(4)$.

  \item \rn{AB} $a_0 \vdash^* w\,a \vdash w\,b$ mit $(a,\varepsilon,b) \in R$

    analog zu \rn{BB}

  \item \rn{BA} $a_0 \vdash^* w\,a\,R\,b_1 \ldots b_n \vdash w\,a\,R\,b_1 \ldots b_n\,a_{n+1}$
    mit $(a,b_1 \ldots b_n,a_{n+1}) \in R$

    folgt unmittelbar mit $(4)$
  \end{itemize}
\end{proof}

\begin{definition}[Programmiersprache]
  Eine \emph{Programmiersprache} $\mathcal{L}$ ist ein Quadrupel $(A,B,S,T)$, wobei $S$ ein
  $(A,B)$-Rekursionssystem und $T$ ein $(A,B)$-Typsystem ist. $\mathcal{L}$ hei"st \emph{typsicher},
  wenn f\"ur jedes $a \in \bigcup_{\tau}A^\tau$ gilt: Es existiert keine Berechnung f\"ur $a$,
  die stecken bleibt.
\end{definition}

Formal ist eine Programmiersprache typsicher, wenn aus $a \in A^\tau$ und $a \vdash^*_S st$ folgt,
dass entweder $st \in B$ oder es ex. $st'$ mit $st \vdash_S st'$.

\begin{definition}[Zwischenresultate]
  $b_1,\ldots,b_n$ ($n \ge 1$) hei"sen \emph{Zwischenresultate} f\"ur $a$ bzgl. $R$, wenn
  $a_1,\ldots,a_n$ existieren, so dass gilt:
  \begin{itemize}
  \item $a_i \Downarrow_S b_i$ f\"ur $i=1,\ldots,n$, und
  \item $(a,b_1 \ldots b_{i-1}, a_i) \in R$ f\"ur $i = 1,\ldots,n$.
  \end{itemize}
\end{definition}

\noindent
(Intuitiv sind $b_1,\ldots,b_n$ die Resultate der ersten $n$ Pr\"amissen f\"ur $a$ in der
Regel $R$).

\begin{theorem}[Meta-Theorem]
  F\"ur Typsicherheit gen\"ugt zu zeigen:
  \begin{enumerate}
  \item \emph{Local Progress}

    Wenn $a \in A^\tau$, dann ex. $R$ und $x \in A \cup B$ mit $(a,\varepsilon,x) \in R$.

    Wenn $a \in A^\tau$ und $b_1,\ldots,b_n$ Zwischenresultate f\"ur $a$ bzgl. $R$ $(n \ge 1)$,
    dann ex. $x \in A \cup B$ mit $(a,b_1 \ldots b_n,x) \in R$.

  \item \emph{Local Preservation}

    Wenn $a \in A^\tau$ und $(a,b_1 \ldots b_n,b) \in R$, dann $b \in B^\tau$.

    Wenn $a \in A^\tau$ und $(a,b_1 \ldots b_n,a') \in R$, dann ex. $\tau'$ mit $a' \in A^{\tau'}$.
  \end{enumerate}
\end{theorem}

\noindent
\textbf{Anwendungsbeispiele:}
\begin{enumerate}
\item $\mathcal{L}_2$ (ohne exceptions) $A = \textsf{Exp}$, $B = \textsf{Val}$
\item $\mathcal{L}_2$ (mit exceptions) $A = \textsf{Exp}$, $B = \textsf{Exn} \cup \textsf{Val}$
\item $\mathcal{L}_2$ (mit exceptions und states) $A = \textsf{Exp} \times \textsf{State}$,
  $B = (\textsf{Exn} \cup \textsf{Val}) \times \textsf{State}$ \\
  dazu: Schreibweisen, die es erlauben eine Relation $R$ durch exceptions und states
  zu ``erweitern''.
\end{enumerate}

%% References
%\bibliographystyle{abbrv}
%\bibliography{citations}


\end{document}
\documentclass[12pt,a2paper,draft]{article}

\usepackage{amssymb}
\usepackage{amsthm}
\usepackage[blocks]{authblk}
\usepackage[english]{babel}
\usepackage{color}
\usepackage[colorlinks=false,%
            pdfkeywords={TODO},%
            pdftitle={TODO},%
            pdfauthor={TODO},%
            pdfsubject={},%
            pdfdisplaydoctitle=true]{hyperref}
\usepackage{mathpartir}
\usepackage{ngerman} %TODO
\usepackage{varwidth}

\newcommand{\abstr}[2]{\ensuremath{\lambda{#1}.\,{#2}}}
\newcommand{\app}[2]{\ensuremath{{#1}\,{#2}}}
\newcommand{\rec}[2]{\ensuremath{{\normalfont\textsf{rec}}\,{#1}.\,{#2}}}
\newcommand{\unit}{\ensuremath{\normalfont\textsf{unit}}}
\newcommand{\Unit}{\ensuremath{\normalfont\textsf{Unit}}}

\newcommand{\Tj}[3]{\mbox{\ensuremath{{#1}\vdash{#2}:{#3}}}}
\newcommand{\tj}[2]{\Tj{\emptyset}{#1}{#2}}
\newcommand{\Pj}[3]{\mbox{\ensuremath{{#1}\vdash{#2}\rightarrow{#3}}}}
\newcommand{\tree}[1]{\mathcal{D}(#1)}
\newcommand{\rn}[1]{\mbox{\ensuremath{\textsc{(#1)}}}}

\newtheorem{lemma}{Lemma}
\newtheorem{theorem}{Theorem}
\newtheorem{definition}{Definition}

\begin{document}

\title{%
  Typsicherheit in Rekursionssystemen
}
\author{Benedikt Meurer}
\author{Kurt Sieber}
\affil{%
  Compilerbau und Softwareanalyse\\
  Universit\"at Siegen\\
  D-57072 Siegen, Germany\\
  {\tt \{meurer,sieber\}@informatik.uni-siegen.de}
}
\date{}
\maketitle
\begin{abstract}
  TODO
\end{abstract}


\begin{definition}[Rekursionssystem]
  Seien $A$ und $B$ beliebige disjunkte Mengen. Eine \emph{$(A,B)$-Rekursionsregel}
  ist eine Relation $R \subseteq A \times B^* \times (A \uplus B)$. Ein
  \emph{$(A,B)$-Rekursionssystem} ist eine Menge $S$
  von $(A,B)$-Rekursionsregeln.
  Die durch $S$ definierte \emph{nat\"urliche Semantik} ist die kleinste Relation
  $\Downarrow_S\ \subseteq A \times B$, f\"ur die gilt:
  \begin{quote}
    Wenn $R \in S$, $(a,b_1 \ldots b_{i-1},a_i) \in R$ und
    $a_i \Downarrow_S b_i$ f\"ur $i=1,\ldots,n$, und $(a,b_1 \ldots b_n,b) \in R$,
    dann gilt auch $a \Downarrow_S b$.
  \end{quote}
\end{definition}

\begin{definition}[Typsystem]
  Ein \emph{$(A,B)$-Typsystem} $T$ besteht aus
  \begin{itemize}
  \item einer Menge \textsf{Type} von Typen $\tau$, und
  \item Mengen $A^\tau \subseteq A$ und $B^\tau \subseteq B$ f\"ur jeden Typ $\tau \in \textsf{Type}$.
  \end{itemize}
\end{definition}

\begin{definition}[Programmiersprache]
  Eine \emph{Programmiersprache} $\mathcal{L}$ ist ein Quadrupel $(A,B,S,T)$, wobei $S$ ein
  $(A,B)$-Rekursionssystem und $T$ ein $(A,B)$-Typsystem ist. $\mathcal{L}$ hei\"st \emph{typsicher},
  wenn f\"ur jedes $a \in \bigcup_{\tau}A^\tau$ gilt: Es existiert keine Berechnung f\"ur $a$,
  die stecken bleibt.
\end{definition}

\begin{definition}[Zwischenresultate]
  $b_1,\ldots,b_n$ ($n \ge 0$) hei"sen \emph{Zwischenresultate} f\"ur $a$ bzgl. $R$, wenn
  $a_1,\ldots,a_n$ existieren, so dass gilt:
  \begin{itemize}
  \item $a_i \Downarrow_S b_i$ f\"ur $i=1,\ldots,n$, und
  \item $(a,b_1 \ldots b_{i-1}, a_i) \in R$ f\"ur $i = 1,\ldots,n$.
  \end{itemize}
\end{definition}

\noindent
(Intuitiv sind $b_1,\ldots,b_n$ die Resultate der ersten $n$ Pr\"amissen f\"ur $a$ in der
Regel $R$).

\begin{theorem}[Typsicherheit]
  F\"ur Typsicherheit gen\"ugt zu zeigen:
  \begin{enumerate}
  \item \emph{Local Progress}

    Wenn $a \in A^\tau$ und $b_1,\ldots,b_n$ Zwischenresultate f\"ur $a$ bzgl. $R$,
    dann ex. $x \in A \uplus B$ mit $(a,b_1 \ldots b_n,x) \in R$.

  \item \emph{Local Preservation}

    Wenn $a \in A^\tau$ und $(a,b_1 \ldots b_n,b) \in R$, dann $b \in B^\tau$.
    Wenn $a \in A^\tau$ und $(a,b_1 \ldots b_n,a') \in R$, dann ex. $\tau'$ mit $a' \in A^{\tau'}$.
  \end{enumerate}
\end{theorem}



%% References
%\bibliographystyle{abbrv}
%\bibliography{citations}


\end{document}
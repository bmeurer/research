\documentclass[12pt,a4paper]{article}

\usepackage{amsmath}
\usepackage{amssymb}
\usepackage{amstext}
\usepackage{array}
\usepackage[american]{babel}
\usepackage{color}
\usepackage{enumerate}
\usepackage[a4paper,%
            colorlinks=false,%
            final,%
            pdfkeywords={},%
            pdftitle={},%
            pdfauthor={Benedikt Meurer},%
            pdfsubject={},%
            pdfdisplaydoctitle=true]{hyperref}
\usepackage{ifthen}
\usepackage[latin1]{inputenc}
\usepackage{latexsym}
\usepackage[final]{listings}
\usepackage{makeidx}
\usepackage{ngerman}
\usepackage[standard,thmmarks]{ntheorem}
\usepackage{stmaryrd}

%% LaTeX macros
%%
%% macros.tex - Useful LaTeX macros
%%
%% Copyright (c) 2006-2008 Benedikt Meurer <benedikt.meurer@googlemail.com>
%%


%%
%% Generic style macros
%%

\newcommand{\nstyle}[1]{\ensuremath{\mathit{#1}}}
\newcommand{\pstyle}[1]{\ensuremath{\mathsf{#1}}}


%%
%% Misc
%%

\newcommand{\coloncolon}{\mathrel{::}}
\newcommand{\pto}{\hookrightarrow}
\newcommand{\tto}{\rightarrow_t}


%%
%% Sets
%%

\newcommand{\N}{\ensuremath{\mathbb N}}
\newcommand{\Z}{\ensuremath{\mathbb Z}}

\newcommand{\Bool}{\nstyle{Bool}}
\newcommand{\Const}{\nstyle{Const}}
\newcommand{\Exp}{\nstyle{Exp}}
\newcommand{\Member}{\ensuremath{\mathcal{M}}}
\newcommand{\Var}{\ensuremath{\mathcal{X}}}
\newcommand{\Int}{\Z}
\newcommand{\Loc}{\nstyle{Loc}}
\newcommand{\Locfin}{\ensuremath{\Loc_{fin}}}
\newcommand{\Op}{\nstyle{Op}}
\newcommand{\Store}{\nstyle{Store}}
\newcommand{\Type}{\nstyle{Type}}
\newcommand{\Val}{\nstyle{Val}}

\newcommand{\AType}{\nstyle{AType}}
\newcommand{\DType}{\nstyle{DType}}
\newcommand{\LType}{\nstyle{LType}}

\newcommand{\Assn}{\nstyle{Assn}}
\newcommand{\Formula}{\nstyle{Formula}}
\newcommand{\Term}{\nstyle{Term}}

\newcommand{\Perm}{\nstyle{Perm}}


%%
%% Functions (TODO - DeclareMathOperator)
%%

\newcommand{\dom}[1]{\ensuremath{\nstyle{dom}\,(#1)}}
\newcommand{\free}[1]{\ensuremath{\nstyle{free}\,(#1)}}
\newcommand{\locns}[1]{\ensuremath{\nstyle{locns}\,(#1)}}
\newcommand{\grph}[1]{\ensuremath{\nstyle{graph}\,(#1)}}
\newcommand{\supp}[1]{\ensuremath{\nstyle{supp}\,(#1)}}
\newcommand{\reach}[2]{\ensuremath{\nstyle{reach}\,(#1,#2)}}
\newcommand{\reachn}[3]{\ensuremath{\nstyle{reach}_{#1}\,(#2,#3)}}
\newcommand{\semantic}[1]{\ensuremath{\llbracket#1\rrbracket}}

\newcommand{\powerset}[1]{\ensuremath{\wp\,(#1)}}
\newcommand{\powersetfin}[1]{\ensuremath{\wp_{fin}\,(#1)}}


%%
%% Names
%%

\newcommand{\op}{\nstyle{op}}


%%
%% Types
%%

\newcommand{\tassn}{\pstyle{assn}}
\newcommand{\tunit}{\pstyle{unit}}
\newcommand{\tbool}{\pstyle{bool}}
\newcommand{\tint}{\pstyle{int}}
\newcommand{\tarrow}[2]{#1\to#2}
\newcommand{\ttarrow}[2]{#1\tto#2}
\newcommand{\trecord}[1]{\left\{#1\right\}}
\newcommand{\tref}[1]{#1\,\pstyle{ref}}


%%
%% Constants
%%

\newcommand{\true}{\pstyle{true}}
\newcommand{\false}{\pstyle{false}}
\newcommand{\assign}{\pstyle{:=}}
\newcommand{\fix}{\pstyle{fix}}
\newcommand{\cref}{\pstyle{ref}}
\newcommand{\unit}{\pstyle{()}}


%%
%% Expressions
%%

\newcommand{\abstr}[2]{\lambda #1.#2}
\newcommand{\app}[2]{#1\,#2}
\newcommand{\ifte}[3]{\pstyle{if}\,#1\,\pstyle{then}\,#2\,\pstyle{else}\,#3}
\newcommand{\record}[1]{\left\{#1\right\}}
\newcommand{\proj}[2]{#1.#2}

\newcommand{\triple}[3]{\{#1\}\,#2\,\{#3\}}


%%
%% Grammars
%%

\newcommand{\GRbeg}{\begin{array}{rrlll}}
\newcommand{\GRend}{\end{array}}

\newcommand{\GRis}{& ::= &}
\newcommand{\GRal}{\\ & \mid &}
\newcommand{\GRnl}{\vspace{2mm}\\}
\newcommand{\GRmid}{\,\mid\,}
\newcommand{\GRtext}[1]{\hfill & \text{#1}}


%%
%% Rules
%%

\newcommand{\tj}[2]{#1\,\coloncolon\,#2}
\newcommand{\Tj}[3]{#1\,\triangleright\,#2\coloncolon#3}



\begin{document}

\section{B"aume}

Die folgenden Definitionen basieren auf \emph{``Grundlagen der Programmiersprachen''} von Loeckx,
Mehlhorn und Willhelm.
\begin{definition}[Baumbereiche und B"aume] \
  \begin{enumerate}
  \item Die partiellen Funktionen $\mathit{parent}$ und $\mathit{brother}$ sind definiert durch
    \[\begin{array}{rl}
      \mathit{parent}: & \N^* \pto \N^*, \\
      & w \cdot n \mapsto w \\
      \\
      \mathit{brother}: & \N^* \pto \N^*, \\
      & w \cdot (n+1) \mapsto w \cdot n \\
    \end{array}\]
    Die reflexive, transitive H"ulle von $\mathit{parent}$ hei"st $\mathit{ancestor}$.
  \item Ein \emph{Baumbereich} $K$ ist eine endliche Teilmenge von $\N^*$, die abgeschlossen
    ist unter der Funktion $\mathit{parent}$, d.h. $\parent{K} \subseteq K$. Die Elemente des
    Baumbereichs hei"sen \emph{Knoten}.
  \item Sei $D$ eine Menge. Ein \emph{$D$-Baum} (kurz \emph{Baum}) ist ein Paar $(K,\ell)$,
    wobei $K$ ein Baumbereich ist, und $\ell$ eine totale Abbildung von $K$ nach $D$. Die
    Abbildung $\ell$ hei"st die \emph{Beschriftung} des Baumes und $K$ sein \emph{Bereich}.
    Die Menge aller $D$-B"aume bezeichnen wir mit $T_D$.
  \item Ein Baumbereich $K$ hei"st \emph{vollst"andig}, wenn $K$ abgeschlossen ist unter der
    Funktion $\mathit{brother}$, d.h. $\brother{K} \subseteq K$. Ein Baum $(K,\ell)$ hei"st
    \emph{vollst"andig}, wenn sein Baumbereich $K$ vollst"andig ist.
  \end{enumerate}
\end{definition}

\begin{definition}
  Sei $D$ eine Menge. Die Funktionen $\mathit{leaves}$ und $\mathit{leafword}$ sind definiert durch
  \[\begin{array}{rll}
    \mathit{leaves}: & \powersetfin{\N^*} \to \powersetfin{\N^*}, \\
    & \multicolumn{2}{l}{W \mapsto \{w \in W \mid \forall v \in \N^+.\,(w \cdot v) \not\in W\}.} \\
    \\
    \mathit{leafword}: & T_D \pto D^*, \\
    & (K,\ell) \mapsto \ell(w_1)\ldots\ell(w_k) & \text{wenn $\{w_1,\ldots,w_k\}=\leaves{K}$} \\
    && \text{mit $k \ge 1$ und $w_1 <_{\text{lex}} \ldots <_{\text{lex}} w_k$.} \\
    \end{array}\]
\end{definition}

\begin{definition}[Teilbaum]
  Sei $D$ eine Menge, $(K,\ell)$ ein $D$-Baum und $w \in K$ ein Knoten des Baums.
  Der \emph{$w$-Teilbaum} $(K',\ell') = \subtree{(K,\ell)}{w}$ von $(K,\ell)$ ist definiert
  durch
  \[\begin{array}{rl}
    \multicolumn{2}{l}{K' = \{v \in \N^* \mid w \cdot v \in K\}} \\
    \ell': & K '\to D, \\
    & v \mapsto \ell(w \cdot v).
  \end{array}\]
\end{definition}

\begin{lemma}
  Sei $D$ eine Menge, $(K,\ell)$ ein $D$-Baum und $w \in K$ ein Knoten des Baums. Dann gilt:
  \begin{enumerate}
  \item $\subtree{(K,\ell)}{w}$ ist ebenfalls ein $D$-Baum.
  \item Wenn $(K,\ell)$ vollst"andig ist, so ist auch $\subtree{(K,\ell)}{w}$ vollst"andig.
  \end{enumerate}
\end{lemma}


\section{Kontextfreie Grammatiken}

Eine \emph{kontextfreie Grammatik} $G$ ist ein Quadrupel $(N,\Sigma,\to,S)$ mit
\begin{enumerate}
\item $\Sigma$ ist eine endliche, nicht-leere Menge von \emph{Terminalzeichen},
\item $N$ ist eine endliche zu $\Sigma$ disjunkte Menge von \emph{Nichtterminalzeichen},
\item $\to\ \subseteq N \times (N \uplus \Sigma)^*$ ist eine Relation von Produktionen,
  wobei zu jedem $A \in S$ ein $w$ existieren muss, so dass $A \to w$, und
\item $S \subseteq N$ ist eine nichtleere Menge von sogenannten \emph{Startsymbolen}.
\end{enumerate}
Die Worte aus $\Sigma^*$ hei"sen \emph{Terminalworte}, die aus $(N \uplus \Sigma)^*$ \emph{Satzformen}. Eine
kontextfreie Grammatik hei"st \emph{$\varepsilon$-frei}, wenn sie keine Produktion der Form
$A \to \varepsilon$ enth"alt.

\begin{definition}[Ableitungen]
  Sei $G=(N,\Sigma,\to,S)$ eine kontextfreie Grammatik. Wir definieren die bin"aren Relationen $\Rightarrow$
  und $\Rightarrow^*$ auf der Menge der Satzformen $(\Sigma \uplus T)^*$ durch
  \begin{enumerate}
  \item $\alpha \Rightarrow \beta$ gdw. es ex. $\alpha_1,\alpha_2,\alpha_3\in (\Sigma \uplus T)^*$ und
    $A \to \alpha_3$, so dass $\alpha = \alpha_1 A \alpha_2$ und $\beta = \alpha_1\alpha_2\alpha_3$.
  \item $\Rightarrow^*$ ist die reflexive transitive H"ulle von $\Rightarrow$.
  \end{enumerate}
  Eine Folge $\alpha_0,\ldots,\alpha_n$ mit $\alpha_i \Rightarrow \alpha_{i+1}$ f"ur $0 \le i < n$ hei"st
  \emph{Ableitung} von $\alpha_0$ nach $\alpha_n$, und $n$ hei"st die \emph{L"ange} der Ableitung.
\end{definition}

$L(A) = \{\alpha \in \Sigma^* \mid A \Rightarrow^* \alpha\}$ bezeichnet die Menge aller
aus $A$ ableitbaren (Terminal-)W"orter.

\begin{definition}[Ableitungsbaum]
  Sei $G=(N,\Sigma,\to,S)$ eine kontextfreie Grammatik, $A \in N$ ein Nichtterminalzeichen und
  $\alpha \in (N \uplus \Sigma)^*$ eine Satzform. Ein vollst"andiger Baum
  $(K,\ell) \in T_{N \uplus \Sigma \uplus \{\varepsilon\}}$ hei"st
  \emph{Ableitungsbaum von $A$ nach $\alpha$}, wenn gilt:
  \begin{enumerate}
  \item Die Beschriftung der Wurzel ist $A$, d.h. $\ell(\varepsilon)=A$.
  \item F"ur jeden Knoten $w \in K$ mit Kindern $w1,\ldots,wn$, wobei $n \ge 1$, gilt
    $\ell(w) \to \ell(w1)\ldots\ell(wn)$ und entweder
    \begin{enumerate}
    \item $\ell(wi)\ne\varepsilon$ f"ur alle $i=1,\ldots,n$, oder
    \item $n = 1$ und $\ell(w1)=\varepsilon$.
    \end{enumerate}
  \item $\leafword{(K,\ell)} = \alpha$.
  \end{enumerate}
  Mit $T_A$ bezeichnen wir die Menge aller Ableitungsb"aume f"ur $A \Rightarrow^* \alpha$ mit
  $\alpha \in \Sigma^*$.
\end{definition}


\section{Patterns}

\begin{definition}[Patterns]
  Sei $G = (N,\Sigma,\to,S)$ eine $\varepsilon$-freie kontextfreie Grammatik und sei
  $V = (V_X)_{X\in N \uplus \Sigma}$ eine Familie von \emph{$G$-Variablenmengen}. F"ur
  jedes $X \in N \uplus \Sigma$ ist die Menge $P(X)$ aller \emph{$X$-Pattern}
  definiert als die kleinste Menge f"ur die gilt:
  \begin{enumerate}
  \item $X \in P(X)$ f"ur alle $X \in N \uplus \Sigma$.
  \item $\any \in P(X)$ f"ur alle $X \in N \uplus \Sigma$.
  \item $x^X \in P(X)$ f"ur alle $X \in N \uplus \Sigma$ und $x^X \in V_X$.
  \item Wenn $A \to X_1 \ldots X_n$ mit $n \ge 1$ und $p_i \in P(X_i)$ f"ur alle $i=1,\ldots,n$,
    dann gilt $(p_1 \ldots p_n) \in P(A)$.
  \end{enumerate}
\end{definition}

\begin{definition}[Patternmatching]
  Sei $G = (N,\Sigma,\to,S)$ eine $\varepsilon$-freie kontextfreie Grammatik und
  $A \in N$. Ein \emph{Patternmatching} ist eine Formel der Gestalt $(t,w) \vdash p \to s$,
  wobei $p \in P(A)$, $t = T_A$ und $s$ eine Substitution. Ein solches
  Patternmatching hei"st \emph{g"ultig}, wenn es sich mit den folgenden Regeln herleiten
  l"asst.
  \begin{enumerate}
  \item $((K,\ell),w) \vdash \any \to [\,]$.
  \item $((K,\ell),w) \vdash \ell(w) \to [\,]$.
  \item Wenn $x \in V_{\ell(w)}$, dann $((K,\ell),w) \vdash x \to [\subtree{(K,\ell)}{w}/x]$.
  \item Wenn $w1,\ldots,wn$ die Kindknoten von $w$ sind, mit $n \ge 1$, $((K,\ell),wi) \vdash p_i \to s_i$ f"ur alle
    $i=1,\ldots,n$ und $s = s_1 \oplus \ldots \oplus s_n$, dann gilt $((K,\ell),w) \vdash (p_1 \ldots p_n) \to s$.
  \end{enumerate}
\end{definition}


\section{Konstruktoren}

\begin{definition}[Konstruktoren]
  Sei $G = (N,\Sigma,\to,S)$ eine $\varepsilon$-freie kontextfreie Grammatik und sei
  $V = (V_X)_{X \in N \uplus \Sigma}$ eine Familie von \emph{$G$-Variablenmengen}. F"ur
  jedes $X \in N \uplus \Sigma$ ist die Menge $\zeta(X)$ aller \emph{$X$-Konstruktoren}
  definiert als die kleinste Menge f"ur die gilt:
  \begin{enumerate}
  \item $a \in \zeta(a)$ f"ur alle $a \in \Sigma$.
  \item $x^X \in \zeta(X)$ f"ur alle $X \in N \uplus \Sigma$.
  \item Wenn $A \to X_1 \ldots X_n$ mit $n \ge 1$ und $c_i \in \zeta(X_i)$ f"ur alle
    $i=1,\ldots,n$, dann gilt $(c_1 \ldots c_n) \in \zeta(A)$.
  \end{enumerate}
\end{definition}

\begin{definition}[Konstruktion]
  Sei $G = (N,\Sigma,\to,S)$ eine $\varepsilon$-freie kontextfreie Grammatik, $X \in N \uplus \Sigma$,
  $c \in \zeta(X)$ und $s$ eine $G$-Substitution. Die
  \emph{$(c,s)$-Konstruktion} $(K,\ell)$ ist induktiv definiert durch:
  \begin{enumerate}
  \item Wenn $c = a \in \Sigma$, dann
    \[\begin{array}{rcl}
      K &=& \{\varepsilon\} \\
      \ell: & \multicolumn{2}{l}{K \to N \uplus \Sigma,} \\
      & \multicolumn{2}{l}{\varepsilon \mapsto a.} \\
    \end{array}\]
  \item Wenn $c = x \in \dom{s}$, dann $(K,\ell) = s(x)$.
  \item Wenn $c = (c_1 \ldots c_n)$ und existieren $(c_i,s)$-Konstruktionen $(K_i,\ell_i)$ f"ur alle
    $i=1,\ldots,n$, dann
    \[\begin{array}{rcl}
      K &=& \{\varepsilon\} \cup \bigcup_{i=1,\ldots,n} \{i \cdot w \mid w \in K_i\} \\
      \ell: & \multicolumn{2}{l}{K \to N \uplus \Sigma,} \\
      & \multicolumn{2}{l}{\varepsilon \mapsto X,} \\
      & \multicolumn{2}{l}{(i \cdot w) \mapsto \ell_i(w) \text{ f"ur alle $i=1,\ldots,n$ und $w \in K_i$.}}
    \end{array}\]
  \end{enumerate}
\end{definition}

\begin{lemma}
  Sei $G$ eine $\varepsilon$-freie kontextfreie Grammatik, $c$ ein $G$-Konstruktor und
  $s$ eine $G$-Substitution mit $\var{c} \subseteq \dom{s}$.
  \begin{enumerate}
  \item Es existiert genau eine $(c,s)$-Konstruktion.
  \item Die $(c,s)$-Konstruktion ist ein $G$-Ableitungsbaum.
  \end{enumerate}
\end{lemma}


\section{Spezifikationen}

\begin{definition}[Regeln]
  Sei $G=(N,\Sigma,\to,S)$ eine $\varepsilon$-freie kontextfreie Grammatik und $A,B \in S$.
  \begin{enumerate}
  \item Eine \emph{$(A,B^n,A)$-Regel} ist eine Formel der Gestalt $(p_0,p_1 \ldots p_n) \Uparrow c$ mit
    $p_0 \in P(A)$, $p_1,\ldots,p_n \in P(B)$, $c \in \zeta(A)$ und $\var{c} \subseteq \bigcup_{i=0,\ldots,n} \var{p_i}$.
    Die Menge aller solchen Regeln f"ur gegebenes $n \ge 0$ bezeichnen wir mit $(A,B^n,A)$.
  \item Eine \emph{$(A,B^n,B)$-Regel} ist eine Formel der Gestaltl $(p_0,p_1 \ldots p_n) \Downarrow c$ mit
    $p_0 \in P(A)$, $p_1,\ldots,p_n \in P(B)$, $c \in \zeta(B)$ und $\var{c} \subseteq \bigcup_{i=0,\ldots,n} \var{p_i}$.
    Die Menge aller solchen Regeln f"ur gegebenes $n \ge 0$ bezeichnen wir mit $(A,B^n,B)$.
  \end{enumerate}
  Sei nun $(A,B^*,A)=\bigcup_{n \ge 0}(A,B^n,A)$ und $(A,B^*,B)=\bigcup_{n \ge 0}(A,B^n,B)$.
\end{definition}

\begin{definition}[Spezifikation]
  Sei $G=(N,\Sigma,\to,S)$ eine $\varepsilon$-freie kontextfreie Grammatik und $A,B \in S$.
  Eine \emph{$(A,B)$-Spezifikation} $\Pi$ ist eine nicht-leere Menge von $(A,B^*,A) \cup (A,B^*,B)$.
  Eine Spezifikation hei"st \emph{eindeutig}, wenn f"ur alle $p_0 \in P(A)$, $p_1,\ldots,p_n \in P(B)$
  h"ochstens ein $c \in \zeta(A) \cup \zeta(B)$ existiert, so dass entweder
  \begin{enumerate}
  \item $(p_0,p_1 \ldots p_n) \Uparrow c$ und $c \in \zeta(A)$, oder
  \item $(p_0,p_1 \ldots p_n) \Downarrow c$ und $c \in \zeta(B)$.
  \end{enumerate}
\end{definition}


\pagebreak


\section{Semantik}

\begin{definition}[Transformationen]
  Sei $G=(N,\Sigma,\to,S)$ eine $\varepsilon$-freie kontextfreie Grammatik, $A,B \in S$ und
  $\Pi$ eine $(A,B)$-Spezifikation.
  \begin{enumerate}
  \item Sei $(p_0,p_1 \ldots p_n) \Uparrow c$ eine Regel aus $\Pi$, $t_0 \in T_A$ und $t_1,\ldots,t_n \in T_B$
    mit $(t_i,\varepsilon) \vdash p_i \to s_i$ f"ur alle $i = 0,\ldots,n$ und $s = s_0 \oplus \ldots \oplus s_n$.
    Dann ist $(t_0,t_1 \ldots t_n) \Uparrow (c,s)$ eine \emph{$(T_A,T_B^n,T_A)_\Pi$-Transformation}. Die Menge aller
    solchen $\Pi$-Transformationen f"ur gegebenes $n \ge 0$ und $\Pi$ bezeichnen wir mit $(T_A,T_B^n,T_A)_\Pi$.
  \item Sei $(p_0,p_1 \ldots p_n) \Downarrow c$ eine Regel aus $\Pi$, $t_0 \in T_A$ und $t_1,\ldots,t_n \in T_B$
    mit $(t_i,\epsilon_i) \vdash p_i \to s_i$ f"ur alle $i = 0,\ldots,n$ und $s = s_0 \oplus \ldots \oplus s_n$.
    Dann ist $(t_0,t_1 \ldots t_n) \Downarrow (c,s)$ eine \emph{$(T_A,T_B^n,T_B)_\Pi$-Transformation}. Die Menge aller
    solchen $\Pi$-Transformationen f"ur gegebenes $n \ge 0$ und $\Pi$ bezeichnen wir mit $(T_A,T_B^n,T_B)_\Pi$.
  \end{enumerate}
  Wir definieren weiter $(T_A,T_B^*,T_A)_\Pi=\bigcup_{n \ge 0}(T_A,T_B^n,T_A)_\Pi$ und \\
  $(T_A,T_B^*,T_B)_\Pi=\bigcup_{n \ge 0}(T_A,T_B^n,T_B)_\Pi$.
\end{definition}

\begin{definition}[Semantik]
  Sei $G=(N,\Sigma,\to,S)$ eine $\varepsilon$-freie kontextfreie Grammatik, $A,B \in S$
  und $\Pi$ eine $(A,B)$-Spezifikation. Die \emph{$(A,B,\Pi)$-Semantik} ist die kleinste
  Menge $g \subseteq T_A \times T_B^* \times (T_A \uplus T_B)$, die alle $(T_A,T_B^*,T_A)_\Pi$- und
  $(T_a,T_B^*,T_B)_\Pi$-Transformationen enth"alt.
\end{definition}

\begin{lemma}
  Sei $G=(N,\Sigma,\to,S)$ eine $\varepsilon$-freie kontextfreie Grammatik, $A,B \in S$
  und $\Pi$ eine $(A,B)$-Spezifikation. Wenn $\Pi$ eindeutig ist, dann ist die
  $(A,B,\Pi)$-Semantik eine Funktion.
\end{lemma}


\end{document}
\documentclass[10pt,a4paper,final,twocolumn]{article}

% PDFLaTeX hint for arXiv to make sure breaklinks=true works for hyperref (must be within the first 5 lines)
\pdfoutput=1

\usepackage{amssymb}
\usepackage[english]{babel}
\usepackage{color}
\usepackage{enumerate}
\usepackage[breaklinks=true,%
            colorlinks=false,%
            pdfkeywords={OCaml, register allocation, linear scan},%
            pdftitle={Linear Scan Register Allocation for the OCaml Compiler},%
            pdfauthor={Marcell Fischbach, Benedikt Meurer},%
            pdfsubject={},%
            pdfdisplaydoctitle=true]{hyperref}
\usepackage{tikz}
\usepackage{varwidth}
\usepackage{xspace}

\newcommand{\etc}{etc.\@\xspace}
\newcommand{\ie}{i.e.\@\xspace}

\usetikzlibrary{arrows}
\usetikzlibrary{trees}
\usetikzlibrary{arrows,backgrounds,decorations.pathmorphing,fit,matrix,positioning}
\tikzset{
  phase/.style={
      rectangle,
      inner sep=1mm,
      rounded corners=1mm,
      minimum size=6mm,
      very thick,
      draw=blue!50!black!50,
      top color=white,
      bottom color=blue!50!black!20
  }
}

\definecolor{ocamljit2}{HTML}{FF0000}
\definecolor{ocamlnatext}{HTML}{00FF00}
\definecolor{ocamlnatjit}{HTML}{0000FF}

\begin{document}

\title{%
  Linear Scan Register Allocation for the OCaml Compiler
}
\author{%
  Marcell Fischbach \\
  Compilerbau und Softwareanalyse \\
  Fakult\"at IV, Universit\"at Siegen \\
  D-57068 Siegen, Germany \\
  \url{marcellfischbach@googlemail.com}
  \and
  Benedikt Meurer\thanks{Corresponding Author} \\
  Compilerbau und Softwareanalyse \\
  Fakult\"at IV, Universit\"at Siegen \\
  D-57068 Siegen, Germany \\
  \url{meurer@informatik.uni-siegen.de}
}
\date{}

\maketitle

\begin{abstract}
  \textcolor{red}{TODO}
\end{abstract}


%% Introduction
\section{Introduction}

The OCaml \cite{Leroy11,Remy02} system is the main implementation of the Caml language \cite{Caml11}, featuring
a powerful module system combined with a full-fledged object-oriented layer. It ships with an optimizing native
code compiler \texttt{ocamlopt}, for high performance; a byte code compiler \texttt{ocamlc} and interpreter
\texttt{ocamlrun}, for increased portability; and an interactive top-level \texttt{ocaml} based on the byte code
compiler and runtime, for interactive use of OCaml through a read-eval-print loop.

\textcolor{red}{TODO} In the remainder of this document we use the term \emph{OCaml Compiler} to refer to the
optimizing native code compiler \texttt{ocamlopt}.


%% Architecture of the OCaml compiler
\section{Architecture of the OCaml Compiler} \label{section:Architecture_of_the_OCaml_Compiler}

\textcolor{red}{TODO}


%% Linear Scan Register Allocation
\section{Linear Scan Register Allocation} \label{section:Linear_Scan_Register_Allocation}

\textcolor{red}{TODO}


%% Performance
\section{Performance} \label{section:Performance}

\textcolor{red}{TODO}


%% Related work
\section{Related work} \label{section:Related_work}

\textcolor{red}{TODO}


%% Conclusion
\section{Conclusion} \label{section:Conclusion}

\textcolor{red}{TODO}


%% References
\bibliographystyle{abbrv}
\bibliography{citations}

\end{document}

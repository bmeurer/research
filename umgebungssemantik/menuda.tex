\documentclass[12pt,fleqn,a4paper]{article}
\usepackage{ngerman}
\usepackage{hyperref,german,amssymb,amstext,amsmath,amsthm,array,stmaryrd,color,latexsym}

% TP Macros
% General

\newcommand{\name}[1]{{\text{\it #1\/}}}
%\newcommand{\name}[1]{\mathit{#1}}

%\newcommand{\bfbox}[1]{\mathbf{#1}}

\newcommand{\I}{{\cal I}}

% Proof trees

\newcounter{tree}
\newcounter{node}[tree]

\newlength{\treeindent}
\newlength{\nodeindent}
\newlength{\nodesep}

\newcommand{\refnode}[1]
 {\ref{\thetree.#1}}

\newcommand{\contextcolor}{\color{blue}}

\definecolor{darkgreen}{rgb}{0.1,0.5,0}
\definecolor{redexcolor}{rgb}{0.7,1,0}

\newcommand{\resulttypecolor}{\color{darkgreen}}

\newcommand{\resultcolor}{\color{blue}}

\newcommand{\byrulecolor}{\color{red}}

\newcommand{\marked}[1]{\colorbox{redexcolor}{$#1$}}

\newcommand{\smallsteparrow}[1]{\stackrel{\mbox{\scriptsize\byrulecolor (#1)}}{\longrightarrow}}

\newcommand{\byrule}[1]{\hspace{-5mm}\byrulecolor\mbox{\scriptsize\ #1}}

\newif\ifarrows 
\arrowsfalse 

\newcommand{\arrow}[3]
  {\ifarrows
   \ncangle[angleA=-90,angleB=#1]{<-}{\thetree.#2}{\thetree.#3}
   \else
   \fi}

\newcommand{\node}[4]
 {\ifarrows
   \else \refstepcounter{node}
         \noindent\hspace{\treeindent}\hspace{#2\nodeindent}
         \rnode{\thetree.#1}{\makebox[6mm]{(\thenode)}}\label{\thetree.#1}
         $\blong 
          #3 \\ 
          \byrule{#4} 
          \elong$
         \vspace{\nodesep}
   \fi}

\newcommand{\dummyarrow}[3]
  {\arrow{#1}{#2}{#3dummy}}

\newcommand{\dummynode}[2]
 {\ifarrows 
  \else \noindent\hspace{\treeindent}\hspace{#2\nodeindent}
         \rnode{\thetree.#1dummy}{\makebox[6mm]{(\refnode{#1})}}\label{\thetree.#1dummy}
         $\ldots$
         \vspace{\nodesep}
   \fi}

\newcommand{\mktree}[1]
 {\stepcounter{tree} #1 \arrowstrue #1 \arrowsfalse}

\fboxrule=0mm

\newcommand{\cenv}[1]{\fbox{$\begin{array}{|ll|}\hline #1 \\\hline\end{array}$}}


% Special Symbols

\newcommand{\uminus}{\widetilde{\ }}
\renewcommand{\uminus}{-\!}
\newcommand{\nop}{()}

% Im X-Symbol-Manual empfohlen

\newcommand{\nsubset}{\not\subset}
%\newcommand{\textflorin}{\textit{f}}
\newcommand{\setB}{{\mathord{\mathbb B}}}
\newcommand{\setC}{{\mathord{\mathbb C}}}
\newcommand{\setN}{{\mathord{\mathbb N}}}
\newcommand{\setQ}{{\mathord{\mathbb Q}}}
\newcommand{\setR}{{\mathord{\mathbb R}}}
\newcommand{\setZ}{{\mathord{\mathbb Z}}}
\newcommand{\coloncolon}{\mathrel{::}}

% Eigene (k\"urzere) Befehle

\newcommand{\pfi}{\varphi}
\newcommand{\eps}{\varepsilon}
\newcommand{\eval}{\Downarrow}
\newcommand{\pto}{\hookrightarrow}
\newcommand{\emp}{\emptyset}
\newcommand{\sleq}{\subseteq}
\newcommand{\sgeq}{\supseteq}
\newcommand{\sqleq}{\sqsubseteq}
\newcommand{\sqgeq}{\sqsupseteq}
\newcommand{\lub}{\bigsqcup}
\newcommand{\glb}{\bigsqcap}
\newcommand{\lsem}{\llbracket}
\newcommand{\rsem}{\rrbracket}
\newcommand{\sem}[1]{\lsem #1 \rsem}
\newcommand{\impl}{\models}
\newcommand{\step}{\vdash}
%\newcommand{\tr}{\triangleright}
\newcommand{\cc}{\coloncolon}


% Names (lower case italic)

\newcommand{\exn}{\name{exn}}
\newcommand{\id}{\name{id}}
\newcommand{\op}{\name{op}}

\newcommand{\true}{\name{true}}
\newcommand{\false}{\name{false}}
\newcommand{\Not}{\name{not}}

\newcommand{\sol}[3]{\name{solution}\,(#1,#2,#3)}
%\newcommand{\unify}{\name{unify}\,}

\newcommand{\Fst}{\name{fst}}
\newcommand{\Snd}{\name{snd}}

\newcommand{\Hd}{\name{hd}}
\newcommand{\Tl}{\name{tl}}
\newcommand{\Cons}{\name{cons}}
\newcommand{\Empty}{\name{is\_empty}}

\newcommand{\dom}[1]{\name{dom}(#1)}
\newcommand{\free}[1]{\name{free}\,(#1)}
\newcommand{\rank}[1]{\name{rank}\,(#1)}
\newcommand{\len}[1]{\name{len}\,(#1)}
\newcommand{\tr}[1]{\name{tr}(#1)}
\newcommand{\trdB}[1]{\name{tr}_{dB}(#1)}
\newcommand{\locns}[1]{\name{locns}\,(#1)}

% Names (upper case italic)

\newcommand{\Bool}{\name{Bool}}
\newcommand{\Btype}{\name{BType}}

\newcommand{\Conf}{\name{Conf}}
\newcommand{\Const}{\name{Const}}

\newcommand{\Dec}{\name{Dec}}

\newcommand{\Env}{\name{Env}}
\newcommand{\Exn}{\name{Exn}}
\newcommand{\EP}{\name{EP}}
\newcommand{\Exp}{\name{Exp}}

\newcommand{\Ncx}{\name{Ncx}}
\newcommand{\dbExp}{\name{dbExp}}
\newcommand{\dbVal}{\name{dbVal}}
\newcommand{\dbEnv}{\name{dbEnv}}
\newcommand{\dbCl}{\name{dbCl}}

\newcommand{\Id}{\name{Id}}
\newcommand{\Int}{\name{Int}}

\newcommand{\Lexp}{\name{LExp}}
\newcommand{\Loc}{\name{Loc}}

\newcommand{\Prog}{\name{Prog}}
\newcommand{\Instr}{\name{Instr}}
\newcommand{\Reg}{\name{Reg}}
\newcommand{\Stack}{\name{Stack}}
\newcommand{\State}{\name{State}}

\newcommand{\Op}{\name{Op}}

\newcommand{\Type}{\name{Type}}
\newcommand{\Tvar}{\name{TVar}}
\newcommand{\teqns}[3]{\name{teqns}\,(#1,#2,#3)}
\newcommand{\tvar}[1]{\name{tvar}\,(#1)}
\newcommand{\unify}[1]{\name{unify}\,(#1)}

\newcommand{\Ptype}{\name{PType}}

\newcommand{\Unit}{\name{Unit}}

\newcommand{\Val}{\name{Val}}

% keywords

\newcommand{\z}{\mathbf{int}}
\newcommand{\bool}{\mathbf{bool}}
\newcommand{\unit}{\mathbf{unit}}
\newcommand{\blist}{\mathbf{list}}
\newcommand{\bref}{\mathbf{ref}}
\newcommand{\ltype}[1]{#1\,\blist}
\newcommand{\reftype}[1]{#1\,\bref}

\renewcommand{\div}{\mathbin{\mathbf{div}}}
\newcommand{\sel}{\mathbin{.}}
%\newcommand{\mod}{\mathbin{\mathbf{mod}}}
\renewcommand{\mod}{\text{mod}}

\newcommand{\bif}{\mathbf{if}}
\newcommand{\bthen}{\mathbf{then}}
\newcommand{\belse}{\mathbf{else}}

\newcommand{\blet}{\mathbf{let}}
\newcommand{\bin}{\mathbf{in}}
\newcommand{\bend}{\mathbf{end}}

\newcommand{\bval}{\mathbf{val}}
\newcommand{\brec}{\mathbf{rec}}
\newcommand{\bfix}{\mathbf{fix}}
\newcommand{\bfun}{\mathbf{fun}}
\newcommand{\band}{\mathbf{and}}
\newcommand{\btype}{\mathbf{type}}

%\newcommand{\andalso}[2]{#1\,\&\&\,#2}
%\newcommand{\orelse}[2]{#1\,\|\,#2}
\newcommand{\andalso}{\&\&}
\newcommand{\orelse}{\|}
\newcommand{\bandalso}{\mathbf{andalso}}
\newcommand{\borelse}{\mathbf{orelse}}

\newcommand{\bwhile}{\mathbf{while}}
\newcommand{\bdo}{\mathbf{do}}
\newcommand{\bfor}{\mathbf{for}}

\newcommand{\brepeat}{\mathbf{repeat}}
\newcommand{\buntil}{\mathbf{until}}

\newcommand{\barray}{\mathbf{array}}
\newcommand{\bof}{\mathbf{of}}

\newcommand{\bclass}{\mathbf{class}}

\newcommand{\app}[2]{#1\,#2}
\newcommand{\bift}[2]{\bif\ #1\ \bthen\ #2}
\newcommand{\bifte}[3]{\bif\ #1\ \bthen\ #2\ \belse\ #3}
\newcommand{\Bifte}[3]{\blong\bif\ #1\\\bthen\ #2\\\belse\ #3\elong}
\newcommand{\bwd}[2]{\bwhile\ #1\ \bdo\ #2}
\newcommand{\bru}[2]{\brepeat\ #1\ \buntil\ #2}


\newcommand{\blie}[2]{\blet\ #1\ \bin\ #2 \ \bend}
\newcommand{\vdec}[2]{#1 = #2}
\newcommand{\rec}[2]{\brec\,#1.\,#2}
\newcommand{\recdots}[2]{\brec\ #1.\ \ldots}
\newcommand{\abstr}[2]{\lambda #1.\,#2}
\newcommand{\appl}[2]{#1\,#2}
\newcommand{\proj}[1]{\#_{#1}}

\newcommand{\Ref}{\name{ref}}
\newcommand{\Deref}{\,!\,}

\newcommand{\alloc}{\name{alloc}}
\newcommand{\Store}{\name{Store}}




% Typing rules 

\newcommand{\tj}[2]{#1\cc#2}
\newcommand{\ctj}[2]{\tj{#1}{{\resulttypecolor #2}}}
\newcommand{\Tj}[3]{#1 \, \triangleright \, #2\cc#3}
\newcommand{\cTj}[3]{\Tj{{\contextcolor #1}}{#2}{{\resulttypecolor #3}}}
\newcommand{\cbig}[2]{#1 \ \eval\ {\resultcolor #2}}
\newcommand{\cBig}[2]{#1 \\ \eval\ {\resultcolor #2}}
\newcommand{\Clos}[2]{\name{Closure}_{#1}(#2)}

\newcommand{\Tjl}[3]{#1 \,\triangleright_l\, #2\cc#3}
\newcommand{\Tjm}[3]{#1 \,\triangleright_m\, #2\cc#3}
\newcommand{\Tjh}[2]{#1 \triangleright  #2}

\newcommand{\Lj}[3]{#1\,\vdash\,#2\cc#3}

\newcommand{\brule}[1]{\begin{markiere}[#1]}
\newcommand{\erule}{\end{markiere}}

\newcommand{\regel}[2]{\ \begin{array}{@{}c@{}} #1 \\ \hline #2
 \end{array}\ }

\newcommand{\reason}[1]{\ \mbox{#1}}
\newcommand{\Reason}[1]{\vspace{1mm}\\ \mbox{ #1}}


% Program verification

\newcommand{\conj}{\,\land\,}
\newcommand{\Conj}{\bigwedge}
\newcommand{\disj}{\,\lor\,}
\newcommand{\Disj}{\bigvee\,}
\newcommand{\all}[1]{\forall{#1}.\,}
\newcommand{\ex}[1]{\exists{#1}.\,}

\newcommand{\power}[1]{\wp(#1)}

\newcommand{\disjoint}[2]{\name{disj}(#1,#2)}
\newcommand{\cont}[2]{#1 \mapsto #2}
\newcommand{\DEF}{\name{DEF}}

\newcommand{\ret}[2]{{\bf returns}\ #1.\, #2}
\newcommand{\tc}[2]{#1\,\{#2\}}
\newcommand{\triple}[3]{\{#1\}\,#2\,\{#3\}}

% Index

\newcommand{\define}[1]{{\em #1\/}\index{#1}}
\newcommand{\Define}[2]{{\em #1\/}\index{#2}}
\newcommand{\Index}[1]{\index{#1}}
\newcommand{\notation}[1]{#1\index{#1}}
\newcommand{\engl}[1]{(engl.: \define{#1})}
\newcommand{\Engl}[2]{(engl.: \Define{#1}{#2})}

% Theorems etc.

\newtheorem{theorem}{Satz}
\newtheorem{corollary}{Korollar}
\newtheorem{definition}{Definition:}
%\newtheorem{example}{Beispiel:}
%\newtheorem{examples}{Beispiele:}
\newtheorem{lemma}[theorem]{Lemma}
\newtheorem{proposition}[theorem]{Proposition}

%\renewcommand{\thedefinition}{}
%\renewcommand{\theexample}{}
%\renewcommand{\theexamples}{}
\renewcommand{\theenumi}{\rm (\alph{enumi})}
\renewcommand{\labelenumi}{\theenumi}

%\newcommand{\enumarabic}{\renewcommand{\theenumi}{\rm (\arabic{enumi})}}

%\newcommand{\bcoro}[1]{\begin{corollary}\label{cor:#1}}
%\newcommand{\ecoro}{\end{corollary}}

\newcommand{\brdef}[1]{\begin{definition}\label{def:#1}\rm}
\newcommand{\erdef}{\end{definition}}

%\newcommand{\blemm}[1]{\begin{lemma}\label{lem:#1}}
%\newcommand{\bLemm}[2]{\begin{lemma}[#2]\label{lem:#1}\index{#2}}
%\newcommand{\elemm}{\end{lemma}}

%\newcommand{\btheo}[1]{\begin{theorem}\label{th:#1}}
%\newcommand{\bTheo}[2]{\begin{theorem}[#2]\label{th:#1}\index{#2}}
%\newcommand{\etheo}{\end{theorem}}

%\newcommand{\litem}[1]{\item\label{it:#1}}
%\newcommand{\ritem}[1]{\ref{it:#1}}


% Grammars

\newcommand{\bgram}{\[\begin{array}{rrlll}}
\newcommand{\egram}{\end{array}\]}

\newcommand{\is}{& ::= &}
\newcommand{\al}{\\ & \mid &}
\newcommand{\n}{\vspace{2mm}\\}




% Other Environments

\newcommand{\bcase}{\left\{\!\!\!\begin{array}{ll}}
\newcommand{\ecase}{\end{array}\right.}
\newcommand{\benum}{\begin{enumerate}}
\newcommand{\eenum}{\end{enumerate}}
\newcommand{\beqns}{\[\begin{array}{rcll}}
\newcommand{\eeqns}{\end{array}\]}
\newcommand{\bitem}{\begin{itemize}}
\newcommand{\eitem}{\end{itemize}}
\newcommand{\blong}{\!\!\begin{array}[t]{l}}
\newcommand{\elong}{\end{array}}
\newcommand{\btabl}{\begin{tabular}}
\newcommand{\etabl}{\end{tabular}}
\newcommand{\brexa}{\begin{example}\enumarabic\rm}
\newcommand{\erexa}{\end{example}}
\newcommand{\brexs}{\begin{examples}\enumarabic\rm}
\newcommand{\erexs}{\end{examples}}


% German abbreviations

\newcommand{\abk}[1]{#1.\ }
\newcommand{\bzw}{\abk{bzw}}
\newcommand{\bzgl}{\abk{bzgl}}
\newcommand{\das}{\abk{d.h}}
\newcommand{\evtl}{\abk{evtl}}
\newcommand{\usw}{\abk{usw}}
\newcommand{\vgl}{\abk{vgl}}
\newcommand{\zb}{\abk{z.B}}


\newcommand{\infix}[3]{#2\mathbin{#1}#3}

\newcommand{\bli}[3]{\blet\ \vdec{#1}{#2}\ \bin\ #3}
\newcommand{\blidb}[2]{\blet\ {#1}\ \bin\ {#2}}
\newcommand{\blri}[3]{\blet\,\brec\ \vdec{#1}{#2}\ \bin\ #3}

\newcommand{\Bli}[3]{\begin{array}[t]{@{}l}
                     \blet\ \vdec{#1}{#2}\\\bin\ #3
                     \end{array}}

\newcommand{\Vdec}[2]{\begin{array}[t]{@{}l}
                      #1 = \\
                      \ #2
                      \end{array}}

\newcommand{\BLI}[3]{\begin{array}[t]{@{}l}
                     \blet\ \Vdec{#1}{#2}\\\bin\ #3
                     \end{array}}

\newcommand{\Abstr}[2]{\begin{array}[t]{@{}l}
                       \lambda #1.\\\ \ #2
                       \end{array}}


\newcommand{\RN}[1]{\mbox{\textsc{(#1)}}}
\newcommand{\Cl}{\name{Cl}}
\newcommand{\cl}{\name{cl}}
\newcommand{\Req}{\name{Req}}
\newcommand{\Ind}{\name{Index}}

\begin{document}

\section{Substitutionssemantik}

\begin{definition}[Syntax der Programmiersprache]
  Vorgegeben seien
  \begin{itemize}
  \item eine Menge $\Bool = \{\true,\false\}$ von booleschen Konstanten $b$,
  \item eine Menge $\Int = \setZ$ von Integerkonstanten $z$, und
  \item eine (unendliche) Menge $\Id$ von Namen $\id$.
  \end{itemize}
  Die Mengen $\Op$ aller {\em Operatoren} $\op$, $\Const$ aller {\em Konstanten} $c$, $\Exp$ aller 
  {\em Ausdr\"ucke} $e$ und $\Val$ aller {\em Werte} $v$ sind durch folgende kontextfreie Grammatik definiert:
  \bgram
  \op \is + \mid - \mid * \mid \le \mid \ge \mid < \mid > \mid = \\
  c \is b \mid z \mid \op \mid \proj{i} \\
  e \is c \mid \id \mid \abstr{\id}{e} \mid \app{e_1}{e_2} \mid \bli{\id}{e_1}{e_2} \mid \rec{\id}{e}
  \al \bifte{e_0}{e_1}{e_2} \mid (e_1,\ldots,e_n) \\
  v \is c \mid \abstr{\id}{e} \mid (v_1,\ldots,v_n)
  \egram
\end{definition}

\begin{definition}[Freie Namen] \label{definition:free}
  Die Menge $\free{e} \subseteq \Id$ aller \emph{im Ausdruck $e$ frei vorkommenden Namen}
  ist induktiv definiert durch:
  \[\begin{array}{rcl}
    \free{c} &=& \emptyset \\
    \free{\id} &=& \{\id\} \\
    \free{\abstr{\id}{e}} &=& \free{e} \setminus \{\id\} \\
    \free{\app{e_1}{e_2}} &=& \free{e_1} \cup \free{e_2} \\
    \free{\bli{\id}{e_1}{e_2}} &=& \free{e_1} \cup (\free{e_2} \setminus \{\id\}) \\
    \free{\rec{\id}{e}} &=& \free{e} \setminus \{\id\} \\
    \free{\bifte{e_0}{e_1}{e_2}} &=& \free{e_0} \cup \free{e_1} \cup \free{e_2} \\
    \free{(e_1,\ldots,e_n)} &=& \bigcup_{i=1 \ldots n} \free{e_i} \\
  \end{array}\]
\end{definition}

\begin{definition}[Simultane Substitution]
  Eine \emph{(simultane) Substitution} ist eine totale Funktion $s: \Id \to \Exp$, welche alle, bis auf endlich
  viele, Namen auf sich selbst abbildet. Diese endliche Menge von Namen
  \[\begin{array}{rcl}
    \dom{s} &=& \{\id\in\Id \mid s(\id) \ne \id\}
  \end{array}\]
  wird als der \emph{Definitionsbereich} von $s$ bezeichnet. Die Menge $\free{s}$ aller
  \emph{in $s$ frei vorkommenden Namen} ist definiert durch
  \[\begin{array}{rcl}
    \free{s} &=& \bigcup_{\id\in\dom{s}}\free{s(\id)}.
  \end{array}\]
\end{definition}

\noindent
Wir schreiben $[e_1/\id_1,\ldots,e_n/\id_n]$ f"ur die Substitution $s$, f"ur die gilt:
\begin{enumerate}
\item $s(\id_i) = e_i$ f"ur alle $i = 1,\ldots,n$ und
\item $s(\id) = \id$ f"ur alle $\id \not \in \{\id_1,\ldots,\id_n\}$.
\end{enumerate}
Weiterhin schreiben wir $s \setminus \id$ f"ur die Substitution $s'$, f"ur die gilt:
\begin{enumerate}
\item $s'(\id) = \id$ und
\item $s'(\id') = s(\id')$ f"ur alle $\id' \ne \id$.
\end{enumerate}

\begin{definition}
  Der Ausdruck $(e\,s)$, welcher durch \emph{Anwendung der (simultanen) Substitution $s$} auf den Ausdruck $e$
  entsteht, ist wie folgt induktiv definiert:
  \[\begin{array}{rcl}
    c\,s &=& c \\
    \id\,s &=& s(\id) \\
    (\abstr{\id}{e})\,s &=& \abstr{\id'}{(e[\id'/\id])\,(s \setminus \id)} \\
    && \text{\footnotesize{mit
        $\id'\not\in(\free{e}\setminus\{\id\}) \cup \free{s\setminus\id} \cup \dom{s\setminus\id}$}} \\
    (\app{e_1}{e_2})\,s &=& \app{(e_1\,s)}{(e_2\,s)} \\
    (\bli{\id}{e_1}{e_2})\,s &=& \bli{\id'}{e_1\,s}{(e_2[\id'/\id])\,(s \setminus\id)} \\
    && \text{\footnotesize{mit
        $\id'\not\in(\free{e_2}\setminus\{\id\}) \cup \free{s\setminus\id} \cup \dom{s\setminus\id}$}} \\
    (\rec{\id}{e})\,s &=& \rec{\id'}{(e[\id'/\id])\,(s\setminus\id)} \\
    && \text{\footnotesize{mit
        $\id'\not\in(\free{e}\setminus\{\id\}) \cup \free{s\setminus\id} \cup \dom{s\setminus\id}$}} \\
    (\bifte{e_0}{e_1}{e_2})\,s &=& \bifte{e_0\,s}{e_1\,s}{e_2\,s} \\
    (e_1,\ldots,e_n)\,s &=& (e_1\,s,\ldots,e_n\,s) \\
  \end{array}\]
\end{definition}

\begin{definition}[Big step Regeln]
Ein {\em big step} in der Substitutionssemantik ist eine Formel der Gestalt $e \Downarrow v$ mit $e\in\Exp$
und $v \in \Val$. Ein solcher big step hei"st {\em g"ultig}, wenn er sich mit den folgenden Regeln herleiten
l"asst: \\[5mm]
\begin{tabular}{ll}
  \RN{Const}      & $c \Downarrow c$ \\[1mm]
  \RN{Closure}    & $\abstr{\id}{e} \Downarrow \abstr{\id}{e}$ \\[1mm]
  \RN{Beta-V}     & $\regel{e_1 \Downarrow \abstr{\id}{e} \quad e_2 \Downarrow v \quad e[v/\id] \Downarrow v'}
                           {\app{e_1}{e_2} \Downarrow v'}$ \\[3mm]
  \RN{Op}         & $\regel{e_1 \Downarrow \op \quad e_2 \Downarrow (z_1,z_2)}
                           {\app{e_1}{e_2} \Downarrow \op^I(z_1,z_2)}$ \\[3mm]
  \RN{Proj}       & $\regel{e_1 \Downarrow \proj{i} \quad e_2 \Downarrow (v_1,\ldots,v_n) \quad 1 \le i \le n}
                           {\app{e_1}{e_2} \Downarrow v_i}$ \\[3mm]
  \RN{Let}        & $\regel{e_1 \Downarrow v \quad e_2[v/\id] \Downarrow v'}
                           {\bli{\id}{e_1}{e_2} \Downarrow v'}$ \\[3mm]
  \RN{Cond-True}  & $\regel{e_0 \Downarrow \true \quad e_1 \Downarrow v}
                           {\bifte{e_0}{e_1}{e_2} \Downarrow v}$ \\[3mm]
  \RN{Cond-False} & $\regel{e_0 \Downarrow \false \quad e_2 \Downarrow v}
                           {\bifte{e_0}{e_1}{e_2} \Downarrow v}$ \\[3mm]
  \RN{Tuple}      & $\regel{e_1 \Downarrow v_1 \quad \ldots \quad e_n \Downarrow v_n}
                           {(e_1,\ldots,e_n) \Downarrow (v_1,\ldots,v_n)}$ \\[3mm]
  \RN{Unfold}     & $\regel{e[\rec{\id}{e}/\id] \Downarrow v}
                           {\rec{\id}{e} \Downarrow v}$ \\[3mm]
\end{tabular}
\end{definition}

\begin{lemma}[Wohldefiniertheit der Substitutionssemantik] \label{lemma:WD_Substs}
  Wenn $e \Downarrow v$, dann gilt $\free{v} \subseteq \free{e}$.
\end{lemma}

\begin{proof}
  Sollte sich leicht durch Induktion "uber die L"ange der Herleitung des big steps zeigen lassen.
\end{proof}

\begin{corollary}
  Wenn $\free{e} = \emptyset$ und  $e \Downarrow v$, dann gilt auch $\free{v} = \emptyset$.
\end{corollary}

\begin{proof}
  Folgt unmittelbar aus Lemma~\ref{lemma:WD_Substs}.
\end{proof}

\section{Umgebungssemantik}

\begin{definition}[Closures und Umgebungen]
  Die Mengen $\Cl$ aller \emph{Closures} $\cl$ und $\Env$ aller \emph{Umgebungen} $\eta$ sind
  durch die folgende kontextfreie Grammatik definiert:
  \bgram
  \eta \is [\,]
  \al \id:\cl;\eta
  \n
  \cl \is (e,\eta)
  \al (\cl_1,\ldots,\cl_n) & \text{mit } n \ge 2
  \egram
  Der Definitionsbereich $\dom{\eta}$ einer Umgebung $\eta$ ist induktiv definiert durch:
  \[\begin{array}{rcl}
    \dom{[\,]} &=& \emptyset \\
    \dom{\id:\cl;\eta} &=& \{\id\} \cup \dom{\eta} \\
  \end{array}\]
\end{definition}

\noindent
F"ur $\id_1:\cl_1;\ldots;\id_n:\cl_n;[\,]$ schreiben wir kurz $[\id_1:\cl_1;\ldots;\id_n:\cl_n]$ und
f"ur $\eta = [\id_1:\cl_1;\ldots;\id_n:\cl_n]$ und $\id \in \dom{\eta}$ sei
\[\begin{array}{rcl}
  \eta(\id) &=& \cl_i
\end{array}\]
mit $i = \min\{j\in\{1,\ldots,n\}\mid\id = \id_j\}$. Weiterhin schreiben wir $\eta \setminus \id$ f"ur
diejenige Umgebung $\eta'$, die aus $\eta$ durch Entfernen aller Eintr"age f"ur $\id$ entsteht.

Der Begriff \emph{Closure} ist an dieser Stelle vielleicht etwas irref"uhrend, da eine Closure
$(e,\eta)$ nicht wirklich abgeschlossen sein muss, denn weder muss $\free{e} \subseteq \dom{\eta}$
gelten, noch m"ussen alle in $\eta$ vorkommenden Closures abgeschlossen sein. Stattdessen definieren
wir zus"atzlich den Begriff der \emph{g"ultigen Closures} und \emph{g"ultigen Umgebungen}.

\begin{definition}[Freie Namen in Closures und Umgebungen]
  Die Mengen $\free{\cl}\subseteq\Id$ und $\free{\eta}\subseteq\Id$ aller
  \emph{in der Closure $\cl$ bzw. der Umgebung $\eta$ vorkommenden freien Namen} sind wie folgt
  induktiv definiert:
  \[\begin{array}{rcl}
    \free{(e,\eta)} &=& (\free{e} \setminus \dom{\eta}) \cup \free{\eta} \\
    \free{(\cl_1,\ldots,\cl_n)} &=& \bigcup_{i=1 \ldots n} \free{\cl_i} \\
    \free{[\,]} &=& \emptyset \\
    \free{\id:\cl;\eta} &=& \free{\cl} \cup \free{\eta} \\
  \end{array}\]
\end{definition}

\begin{definition}[G"ultige Closures und Umgebungen] \
  \begin{enumerate}
  \item Eine Closure $\cl$ hei"st \emph{g"ultig}, wenn $\free{\cl} = \emptyset$.
  \item Eine Umgebung $\eta$ hei"st \emph{g"ultig}, wenn $\free{\eta} = \emptyset$.
  \end{enumerate}
\end{definition}

\begin{definition}[Werte der Umgebungssemantik]
  Die Menge $W \subseteq \Cl$ aller \emph{Werte der Umgebungssemantik} $w$ ist definiert durch
  folgende kontextfreie Grammatik:
  \bgram
  w \is (c,[\,])
  \al (\abstr{\id}{e},\eta)
  \al (w_1,\ldots,w_n) & \text{mit } n \ge 2
  \egram
\end{definition}

\begin{definition}[Big step Regeln der Umgebungssemantik]
Ein {\em big step} in der Umgebungssemantik ist eine Formel der Gestalt $\cl \Downarrow w$,
wobei $\cl\in\Cl$ und $w \in W$. Ein derartiger big step hei"st {\em g"ultig}, wenn er sich mit den
folgenden Regeln herleiten l"asst: \\[5mm]
\begin{tabular}{ll}
  \RN{Const}      & $(c,\eta) \Downarrow (c,[\,])$ \\[1mm]
  \RN{Closure}    & $(\abstr{\id}{e},\eta) \Downarrow (\abstr{\id}{e},\eta)$ \\[1mm]
  \RN{Tuple-V}    & $(w_1,\ldots,w_n) \Downarrow (w_1,\ldots,w_n)$ \\[1mm]
  \RN{Id}         & $\regel{\eta(\id) \Downarrow w}
                           {(\id,\eta) \Downarrow w}$ \\[3mm]
  \RN{Beta-V}     & $\regel{(e_1,\eta) \Downarrow (\abstr{\id'}{e'},\eta')
                            \quad (e_2,\eta) \Downarrow w'
                            \quad (e',\id':w';\eta') \Downarrow w}
                           {(\app{e_1}{e_2},\eta) \Downarrow w}$ \\[3mm]
  \RN{Op}         & $\regel{(e_1,\eta) \Downarrow (\op,[\,]) \quad (e_2,\eta) \Downarrow ((z_1,[\,]),(z_2,[\,]))}
                           {(\app{e_1}{e_2},\eta) \Downarrow (\op^I(z_1,z_2),[\,])}$ \\[3mm]
  \RN{Proj}       & $\regel{(e_1,\eta) \Downarrow \proj{i}
                            \quad (e_2,\eta) \Downarrow (w_1,\ldots,w_n)
                            \quad 1 \le i \le n}
                           {(\app{e_1}{e_2},\eta) \Downarrow w_i}$ \\[3mm]
  \RN{Let}        & $\regel{(e_1,\eta) \Downarrow w \quad (e_2,\id:w;\eta) \Downarrow w'}
                           {(\bli{\id}{e_1}{e_2},\eta) \Downarrow w'}$ \\[3mm]
  \RN{Cond-True}  & $\regel{(e_0,\eta) \Downarrow (\true,[\,]) \quad (e_1,\eta) \Downarrow w}
                           {(\bifte{e_0}{e_1}{e_2},\eta) \Downarrow w}$ \\[3mm]
  \RN{Cond-False} & $\regel{(e_0,\eta) \Downarrow (\false,[\,]) \quad (e_2,\eta) \Downarrow w}
                           {(\bifte{e_0}{e_1}{e_2},\eta) \Downarrow w}$ \\[3mm]
  \RN{Tuple}      & $\regel{(e_1,\eta) \Downarrow w_1 \quad \ldots \quad (e_n,\eta) \Downarrow w_n}
                           {((e_1,\ldots,e_n),\eta) \Downarrow (w_1,\ldots,w_n)}$ \\[3mm]
  \RN{Unfold}     & $\regel{(e,\id:(\rec{\id}{e},\eta);\eta) \Downarrow w}
                           {(\rec{\id}{e},\eta) \Downarrow w}$ \\[3mm]
\end{tabular}
\end{definition}

\begin{lemma}[Wohldefiniertheit der Umgebungssemantik] \label{lemma:WD_Umgeb}
  Wenn $\cl \Downarrow w$, dann gilt $\free{w} \subseteq \free{\cl}$.
\end{lemma}

\begin{proof}
  Sollte sich ebenfalls leicht durch Induktion "uber die L"ange der Herleitung des big steps zeigen lassen.
\end{proof}

\begin{corollary}
  Wenn $\cl$ g"ultig ist und $\cl \Downarrow w$, dann ist auch $w$ g"ultig.
\end{corollary}

\begin{proof}
  Folgt unmittelbar aus Lemma~\ref{lemma:WD_Umgeb}.
\end{proof}

\subsection{Zusammenhang}

Eine Closure kann als Darstellung eines Ausdrucks aufgefasst werden, wobei man sich die Umgebungen als
\emph{aufgesparte Substitutionen} vorstellt. Formal kann man dies durch eine \emph{"Ubersetzungsfunktion}
vorstellen, die nachfolgend definiert ist.

\begin{definition}["Ubersetzungsfunktion]
  Die "Ubersetzungsfunktion $\name{tr}:\Cl \to \Exp$ ist wie folgt induktiv "uber die Struktur von Closures
  definiert:
  \[\begin{array}{rcl}
%    \tr{e,[\id_1:\cl_1;\ldots;\id_n:\cl_n]} &=& e[\tr{\cl_1}/\id_1,\ldots,\tr{\cl_n}/\id_n] \\
    \tr{e,\eta} &=& e[\tr{\eta(\id_1)}/\id_1,\ldots,\tr{\eta(\id_n)}/\id_n] \\
    && \text{mit } \dom{\eta} = \{\id_1,\ldots,\id_n\} \\
    \tr{\cl_1,\ldots,\cl_n} &=& (\tr{\cl_1},\ldots,\tr{\cl_n}) \\
  \end{array}\]
\end{definition}

\begin{lemma} \
  \begin{enumerate}
  \item $\free{\tr{\cl}} \subseteq \free{\cl}$ f"ur alle $\cl \in \Cl$.
  \item $\tr{w} \in \Val$ f"ur alle $w \in W$.
  \end{enumerate}
\end{lemma}

\begin{proof}
  Vermutlich trivial.
\end{proof}

\begin{lemma}
  Die "Ubersetzungsfunktion $\name{tr}$ gen"ugt folgenden Gleichungen:
  \[\begin{array}{rcl}
    \tr{c,\eta} &=& c \\
    \tr{\id,\eta} &=& \bcase \tr{\eta(\id)}, & \text{falls } \id\in\dom{\eta} \\ \id, & \text{sonst} \ecase \\
    \tr{\abstr{\id}{e},\eta} &=& \abstr{\id'}{\tr{e[\id'/\id],\eta \setminus \id}} \\
      && \text{\footnotesize{mit
          $\id'\not\in(\free{e}\setminus\{\id\}) \cup \free{\eta\setminus\id} \cup \dom{\eta\setminus\id}$}} \\
    \tr{\app{e_1}{e_2},\eta} &=& \app{(\tr{e_1},\eta)}{(\tr{e_2},\eta)} \\
    \tr{\bli{\id}{e_1}{e_2},\eta} &=& \bli{\id'}{\tr{e_1,\eta}}{\tr{e_2[\id'/\id],\eta \setminus \id}} \\
      && \text{\footnotesize{mit
          $\id'\not\in(\free{e_2}\setminus\{\id\}) \cup \free{\eta\setminus\id} \cup \dom{\eta\setminus\id}$}} \\
    \tr{\rec{\id}{e},\eta} &=& \rec{\id'}{\tr{e[\id'/\id],\eta \setminus \id}} \\
      && \text{\footnotesize{mit
          $\id'\not\in(\free{e}\setminus\{\id\}) \cup \free{\eta\setminus\id} \cup \dom{\eta\setminus\id}$}} \\
    \tr{\bifte{e_0}{e_1}{e_2},\eta} &=& \bifte{(\tr{e_0},\eta)}{(\tr{e_1},\eta)}{(\tr{e_2},\eta)} \\
    \tr{(e_1,\ldots,e_n),\eta} &=& (\tr{e_1,\eta},\ldots,\tr{e_n,\eta}) \\
    \\
    \tr{e,[\,]} &=& e \\
    \tr{e,\id:\cl;\eta} &=& \tr{e,\id:\cl;(\eta \setminus \id)} \\
  \end{array}\]
\end{lemma}

\begin{proof}
  Induktion "uber die Struktur von Closures und Umgebungen.
\end{proof}

\begin{theorem}[Korrektheit der Umgebungssemantik] \label{theorem:Korr}
  Wenn $\cl \Downarrow w$ in der Umgebungssemantik, dann gilt $\tr{\cl} \Downarrow \tr{w}$ in der
  Substitutionssemantik.
\end{theorem}

\begin{proof}
  Induktion "uber die L"ange der Herleitung des big steps $\cl \Downarrow w$.
\end{proof}

\begin{theorem}[Vollst"andigkeit der Umgebungssemantik] \label{theorem:Voll}
  Wenn $\tr{\cl} \Downarrow v$ in der Substitutionssemantik, dann existiert ein $w \in W$, so dass
  \begin{enumerate}
  \item $\cl \Downarrow w$ in der Umgebungssemantik, und
  \item $\tr{w} = v$.
  \end{enumerate}
\end{theorem}

\begin{proof}
  Induktion "uber die L"ange der Herleitung des big steps $\tr{\cl} \Downarrow v$.
\end{proof}

\begin{corollary}["Aquivalenz der Modelle]
  F"ur einen abgeschlossenen Ausdruck $e$ gilt $e \Downarrow v$ genau dann, wenn ein $w \in W$ existiert mit
  \begin{enumerate}
  \item $(e,[\,]) \Downarrow w$ und
  \item $\tr{w} = v$.
  \end{enumerate}
\end{corollary}

\begin{proof}
  Folgt unmittelbar aus Satz~\ref{theorem:Voll} und Satz~\ref{theorem:Korr}.
\end{proof}

\section{De Bruijn-Semantik}

\begin{definition}[Syntax der De Bruijn-Programmiersprache]
  Vorgegeben seien
  \begin{itemize}
  \item eine Menge $\Bool = \{\true,\false\}$ von booleschen Konstanten $b$,
  \item eine Menge $\Int = \setZ$ von Integerkonstanten $z$, und
  \item eine (unendliche) Menge $\Ind = \{\underline{1},\underline{2},\ldots\}$ von Indizes $\underline{i}$.
  \end{itemize}
  Die Mengen $\Op$ aller {\em Operatoren} $\op$, $\Const$ aller {\em Konstanten} $c$, $\Exp_{dB}$ aller 
  {\em Ausdr\"ucke} $e$ und $\Val_{dB}$ aller {\em Werte} $v$ sind durch folgende kontextfreie Grammatik definiert:
  \bgram
  \op \is + \mid - \mid * \mid \le \mid \ge \mid < \mid > \mid = \\
  c \is b \mid z \mid \op \mid \proj{i} \\
  e \is c \mid \underline{i} \mid \abstr{}{e} \mid \app{e_1}{e_2} \mid \blidB{e_1}{e_2} \mid \rec{}{e}
  \al \bifte{e_0}{e_1}{e_2} \mid (e_1,\ldots,e_n) \\
  v \is c \mid \abstr{}{e} \mid (v_1,\ldots,v_n)
  \egram
\end{definition}

\begin{definition}[De Bruijn-Closures und -Umgebungen]
  Die Mengen $\Cl_{dB}$ aller \emph{De Bruijn-Closures} $\cl$, $\Env_{dB}$ aller \emph{De Bruijn-Umgebungen} $\eta$ und
  $W_{dB} \subseteq \Cl_{dB}$ aller \emph{De Bruijn-Umgebungswerte} $w$ sind durch die folgende kontextfreie
  Grammatik definiert:
  \bgram
  \eta \is [\,]
  \al \cl;\eta
  \n
  \cl \is (e,\eta)
  \al (\cl_1,\ldots,\cl_n) & \text{mit } n \ge 2
  \n
  w \is (c,[\,])
  \al (\abstr{}{e},\eta)
  \al (w_1,\ldots,w_n) & \text{mit } n \ge 2
  \egram
  Die \emph{L"ange} $|\eta|$ einer De Bruijn-Umgebung $\eta$ ist wie folgt induktiv definiert:
  \[\begin{array}{rcl}
    |[\,]| &=& 0 \\
    |\cl;\eta| &=& 1 + |\eta| \\
  \end{array}\]
\end{definition}

\noindent
Wie "ublich, schreiben wir kurz $[\cl_1;\ldots;\cl_n]$ statt $\cl_1;\ldots;\cl_n;[\,]$ und f"ur
eine Umgebung $\eta = [\cl_1;\ldots;\cl_n]$ und einen g"ultigen Index $i \in \{1,\ldots,|\eta|\}$ sei
\[\begin{array}{rcl}
  \eta(i) &=& \cl_i.
\end{array}\]

\begin{definition}[Big step Regeln]
Ein {\em big step} in der De Bruijn-Semantik ist eine Formel der Gestalt $\cl \Downarrow w$,
wobei $\cl\in\Cl_{dB}$ und $w \in W_{dB}$. Ein derartiger big step hei"st {\em g"ultig}, wenn er
sich mit den folgenden Regeln herleiten l"asst: \\[5mm]
\begin{tabular}{ll}
  \RN{Const}      & $(c,\eta) \Downarrow (c,[\,])$ \\[1mm]
  \RN{Closure}    & $(\abstr{}{e},\eta) \Downarrow (\abstr{}{e},\eta)$ \\[1mm]
  \RN{Tuple-V}    & $(w_1,\ldots,w_n) \Downarrow (w_1,\ldots,w_n)$ \\[1mm]
  \RN{Index}      & $\regel{\eta(i) \Downarrow w}
                           {(\underline{i},\eta) \Downarrow w}$ \\[3mm]
  \RN{Beta-V}     & $\regel{(e_1,\eta) \Downarrow (\abstr{}{e'},\eta')
                            \quad (e_2,\eta) \Downarrow w'
                            \quad (e',w';\eta') \Downarrow w}
                           {(\app{e_1}{e_2},\eta) \Downarrow w}$ \\[3mm]
  \RN{Op}         & $\regel{(e_1,\eta) \Downarrow (\op,[\,]) \quad (e_2,\eta) \Downarrow ((z_1,[\,]),(z_2,[\,]))}
                           {(\app{e_1}{e_2},\eta) \Downarrow (\op^I(z_1,z_2),[\,])}$ \\[3mm]
  \RN{Proj}       & $\regel{(e_1,\eta) \Downarrow \proj{i}
                            \quad (e_2,\eta) \Downarrow (w_1,\ldots,w_n)
                            \quad 1 \le i \le n}
                           {(\app{e_1}{e_2},\eta) \Downarrow w_i}$ \\[3mm]
  \RN{Let}        & $\regel{(e_1,\eta) \Downarrow w \quad (e_2,w;\eta) \Downarrow w'}
                           {(\blidB{e_1}{e_2},\eta) \Downarrow w'}$ \\[3mm]
  \RN{Cond-True}  & $\regel{(e_0,\eta) \Downarrow (\true,[\,]) \quad (e_1,\eta) \Downarrow w}
                           {(\bifte{e_0}{e_1}{e_2},\eta) \Downarrow w}$ \\[3mm]
  \RN{Cond-False} & $\regel{(e_0,\eta) \Downarrow (\false,[\,]) \quad (e_2,\eta) \Downarrow w}
                           {(\bifte{e_0}{e_1}{e_2},\eta) \Downarrow w}$ \\[3mm]
  \RN{Tuple}      & $\regel{(e_1,\eta) \Downarrow w_1 \quad \ldots \quad (e_n,\eta) \Downarrow w_n}
                           {((e_1,\ldots,e_n),\eta) \Downarrow (w_1,\ldots,w_n)}$ \\[3mm]
  \RN{Unfold}     & $\regel{(e,(\rec{}{e},\eta);\eta) \Downarrow w}
                           {(\rec{}{e},\eta) \Downarrow w}$ \\[3mm]
\end{tabular}
\end{definition}

\subsection{Zusammenhang}

\begin{definition}[Namenskontext]
  Die Menge $\Ncx$ aller \emph{Namenskontexte} $\Gamma$ ist definiert durch folgende kontextfreie Grammatik:
  \bgram
  \Gamma \is [\,]
  \al \id;\Gamma
  \egram
  Der \emph{Definitionsbereich} $\dom{\Gamma}$ eines Namenskontextes $\Gamma$ ist wie folgt induktiv definiert:
  \[\begin{array}{rcl}
    \dom{[\,]} &=& \emptyset \\
    \dom{\id;\Gamma} &=& \{\id\} \cup \dom{\Gamma} \\
  \end{array}\]
\end{definition}

\noindent
Statt $\id_1;\ldots;\id_n;[\,]$ schreiben wir kurz $[\id_1;\ldots;\id_n]$. F"ur $\Gamma = [\id_1;\ldots;\id_n]$
und $\id \in\{\id_1,\ldots,\id_n\}$ definieren wir
\[\begin{array}{rcl}
  \Gamma(\id) &=& \min \{i \in \{1,\ldots,n\} \mid \id_i = \id\}
\end{array}\]
d.h. $\Gamma(\id)$ ist der kleinste Index f"ur $\id$ in $\Gamma$.

\begin{definition}["Ubersetzungsfunktion] \
  \begin{enumerate}
  \item Sei $e \in \Exp$ und $\Gamma\in\Ncx$ mit $\free{e} \subseteq \dom{\Gamma}$.
    Der De Bruijn-Ausdruck $\trdB{\Gamma,e}$, der durch "Ubersetzung aus $e$ entsteht,
    ist wie folgt induktiv "uber die Struktur von $e$ definiert:
    \[\begin{array}{rcl}
      \trdB{\Gamma,c} &=& c \\
      \trdB{\Gamma,\id} &=& \underline{i} \quad\quad \text{ mit } i = \Gamma(\id) \\
      \trdB{\Gamma,\abstr{\id}{e}} &=& \abstr{}{\trdB{\id;\Gamma,e}} \\
      \trdB{\Gamma,\app{e_1}{e_2}} &=& \app{(\trdB{\Gamma,e_1})}{(\trdB{\Gamma,e_2})} \\
      \trdB{\Gamma,\bli{\id}{e_1}{e_2}} &=& \blidB{\trdB{\Gamma,e_1}}{\trdB{\id;\Gamma,e_2}} \\
      \trdB{\Gamma,\rec{\id}{e}} &=& \rec{}{\trdB{\id;\Gamma,e}} \\
      \trdB{\Gamma,\bifte{e_0}{e_1}{e_2}} &=& \bifte{\trdB{\Gamma,e_0}}{\trdB{\Gamma,e_1}}{\trdB{\Gamma,e_2}} \\
      \trdB{\Gamma,(e_1,\ldots,e_n)} &=& (\trdB{\Gamma,e_1},\ldots,\trdB{\Gamma,e_n}) \\
    \end{array}\]
  \item Sei $\cl \in \Cl$, $\eta \in \Env$ und $\Gamma \in \Ncx$ mit
    $\free{\cl} \cup \free{\eta} \subseteq \dom{\Gamma}$. Die De Bruijn-Closure $\trdB{\Gamma,\cl}$,
    die durch "Ubersetzung aus $\cl$ entsteht, und die De Bruijn-Umgebung $\trdB{\Gamma,\eta}$, die
    durch "Ubersetzung aus $\eta$ entsteht, sind wie folgt induktiv definiert:
    \[\begin{array}{rcl}
      \trdB{\Gamma,[\,]} &=& [\,] \\
      \trdB{\Gamma,\id:\cl;\eta} &=& (\trdB{\Gamma,\cl});(\trdB{\Gamma,\eta}) \\
      \trdB{\Gamma,(e,\underbrace{[\id_1:\cl_1;\ldots;\id_n:\cl_n]}_{\eta})} &=& (\trdB{\id_1;\ldots;\id_n;\Gamma,e},\trdB{\Gamma,\eta}) \\
      \trdB{\Gamma,(\cl_1,\ldots,\cl_n)} &=& (\trdB{\Gamma,\cl_1},\ldots,\trdB{\Gamma,\cl_n}) \\
    \end{array}\]
  \end{enumerate}
\end{definition}

\noindent
Offensichtlich gilt:
\[\begin{array}{rcl}
  \trdB{\Gamma,[\id_1:\cl_1;\ldots;\id_n:\cl_n]} &=& [\trdB{\Gamma,\cl_1};\ldots;\trdB{\Gamma,\cl_n}] \\
\end{array}\]

\begin{theorem}[Korrektheit der De Bruijn-Semantik]
  Wenn $\free{\cl} \subseteq \dom{\Gamma}$ und $\trdB{\Gamma,\cl} \Downarrow w$ in der
  De Bruijn-Semantik, dann existiert ein $w'$, so dass $\trdB{\Gamma,w'} = w$ und
  $\cl \Downarrow w'$ in der Umgebungssemantik.
\end{theorem}

\begin{theorem}[Vollst"andigkeit der De Bruijn-Semantik]
  Wenn $\free{\cl} \subseteq \dom{\Gamma}$ und $\cl \Downarrow w$ in der Umgebungssemantik,
  dann gilt $\trdB{\Gamma,\cl} \Downarrow \trdB{\Gamma,w}$ in der De Bruijn-Semantik.
\end{theorem}

\section{Pointersemantik}

\begin{definition}[Syntax der Pointer-Programmiersprache]
  Vorgegeben seien
  \begin{itemize}
  \item eine Menge $\Bool = \{\true,\false\}$ von booleschen Konstanten $b$,
  \item eine Menge $\Int = \setZ$ von Integerkonstanten $z$, und
  \item eine (unendliche) Menge $\Id$ von Namen $\id$.
  \end{itemize}
  Die Mengen $\Op$ aller {\em Operatoren} $\op$, $\Const$ aller {\em Konstanten} $c$ und $\Exp_{Ptr}$ aller 
  {\em Ausdr\"ucke} $e$ sind durch folgende kontextfreie Grammatik definiert:
  \bgram
  \op \is + \mid - \mid * \mid \le \mid \ge \mid < \mid > \mid = \\
  c \is b \mid z \mid \op \mid \proj{i} \\
  e \is c \mid \id \mid \abstr{\id}{e} \mid \app{e_1}{e_2} \mid \bli{\id}{e_1}{e_2} \mid \rec{\id}{v}
  \al \bifte{e_0}{e_1}{e_2} \mid (e_1,\ldots,e_n) \\
  v \is c \mid \abstr{\id}{e} \mid (v_1,\ldots,v_n)
  \egram
\end{definition}

\begin{definition}[Umgebungen und Umgebungswerte der Pointersemantik]
  Die Mengen $\Env_{Ptr}$ aller \emph{Umgebungen} $\eta$ und $W_{Ptr}$ aller \emph{Umgebungswerte} $w$ der
  Pointersemantik sind durch die folgende kontextfreie Grammatik definiert:
  \bgram
  \eta \is [\,]
  \al \id:(v,\uparrow);\eta
  \al \id:w;\eta
  \n
  w \is (c,[\,])
  \al (\abstr{\id}{e},\eta)
  \al (w_1,\ldots,w_n)
  \egram
\end{definition}

\begin{definition}
  Ein \emph{expand} ist eine Formel der Gestalt $\Lj{\eta}{v}{w}$, wobei $\eta\in\Env_{Ptr}$, $v\in\Val$ und
  $w\in W_{Ptr}$. Ein derartiger expand hei"st \emph{g"ultig}, wenn er sich mit den folgenden Regeln herleiten
  l"asst: \\[5mm]
  \begin{tabular}{ll}
    \RN{E-Const}   & $\Lj{\eta}{c}{(c,[\,])}$ \\[1mm]
    \RN{E-Closure} & $\Lj{\eta}{\abstr{\id}{e}}{(\abstr{\id}{e},\eta)}$ \\[1mm]
    \RN{E-Tuple}   & $\regel{\Lj{\eta}{v_1}{w_1} \quad\ldots\quad \Lj{\eta}{v_n}{w_n}}
                            {\Lj{\eta}{(v_1,\ldots,v_n)}{(w_1,\ldots,w_n)}}$ \\[3mm]
  \end{tabular} \\[2mm]
  Ein \emph{lookup} ist eine Formel der Gestalt $\Lj{\eta}{\id}{w}$, wobei $\eta\in\Env_{Ptr}$, $\id\in\Id$ und
  $w \in W_{Ptr}$. Ein derartiger lookup hei"st \emph{g"ultig}, wenn er sich mit den folgenden Regeln herleiten
  l"asst: \\[5mm]
  \begin{tabular}{ll}
    \RN{L-Immediate} & $\Lj{\id:w;\eta}{\id}{w}$ \\[1mm]
    \RN{L-Expand}    & $\regel{\Lj{\id:(v,\uparrow);\eta}{v}{w}}
                              {\Lj{\id:(v,\uparrow);\eta}{\id}{w}}$ \\[3mm]
    \RN{L-Skip}      & $\regel{\id \ne \id' \quad \Lj{\eta}{\id}{w}}
                              {\Lj{\id':\underline{\quad};\eta}{\id}{w}}$ \\[3mm]
  \end{tabular}
\end{definition}

\begin{definition}[Big step Regeln der Pointersemantik]
Ein {\em big step} in der Pointersemantik ist eine Formel der Gestalt $(e,\eta) \Downarrow w$,
wobei $e\in\Exp_{Ptr}$, $\eta \in \Env_{Ptr}$ und $w \in W_{Ptr}$. Ein derartiger big step hei"st
{\em g"ultig}, wenn er sich mit den folgenden Regeln herleiten l"asst: \\[5mm]
\begin{tabular}{ll}
  \RN{Const}      & $(c,\eta) \Downarrow (c,[\,])$ \\[1mm]
  \RN{Closure}    & $(\abstr{\id}{e},\eta) \Downarrow (\abstr{\id}{e},\eta)$ \\[1mm]
  \RN{Id}         & $\regel{\Lj{\eta}{\id}{w}}
                           {(\id,\eta) \Downarrow w}$ \\[3mm]
  \RN{Beta-V}     & $\regel{(e_1,\eta) \Downarrow (\abstr{\id'}{e'},\eta')
                            \quad (e_2,\eta) \Downarrow w'
                            \quad (e',\id':w';\eta') \Downarrow w}
                           {(\app{e_1}{e_2},\eta) \Downarrow w}$ \\[3mm]
  \RN{Op}         & $\regel{(e_1,\eta) \Downarrow (\op,[\,]) \quad (e_2,\eta) \Downarrow ((z_1,[\,]),(z_2,[\,]))}
                           {(\app{e_1}{e_2},\eta) \Downarrow (\op^I(z_1,z_2),[\,])}$ \\[3mm]
  \RN{Proj}       & $\regel{(e_1,\eta) \Downarrow \proj{i}
                            \quad (e_2,\eta) \Downarrow (w_1,\ldots,w_n)
                            \quad 1 \le i \le n}
                           {(\app{e_1}{e_2},\eta) \Downarrow w_i}$ \\[3mm]
  \RN{Let}        & $\regel{(e_1,\eta) \Downarrow w \quad (e_2,\id:w;\eta) \Downarrow w'}
                           {(\bli{\id}{e_1}{e_2},\eta) \Downarrow w'}$ \\[3mm]
  \RN{Cond-True}  & $\regel{(e_0,\eta) \Downarrow (\true,[\,]) \quad (e_1,\eta) \Downarrow w}
                           {(\bifte{e_0}{e_1}{e_2},\eta) \Downarrow w}$ \\[3mm]
  \RN{Cond-False} & $\regel{(e_0,\eta) \Downarrow (\false,[\,]) \quad (e_2,\eta) \Downarrow w}
                           {(\bifte{e_0}{e_1}{e_2},\eta) \Downarrow w}$ \\[3mm]
  \RN{Tuple}      & $\regel{(e_1,\eta) \Downarrow w_1 \quad \ldots \quad (e_n,\eta) \Downarrow w_n}
                           {((e_1,\ldots,e_n),\eta) \Downarrow (w_1,\ldots,w_n)}$ \\[3mm]
  \RN{Unfold}     & $(\rec{\id}{v},\eta) \Downarrow (v,\id:(v,\uparrow);\eta)$ \\[3mm]
\end{tabular}
\end{definition}

% Betrachtet man den wesentlichen Unterschied zwischen Pointer- und Umgebungssemantik, so wird die Idee, die
% hinter der Pointersemantik steckt schnell ersichtlich. Zun"achst mal existiert in der Pointersemantik kein
% allgemeiner $\brec$-Ausdruck mehr. Stattdessen hat man eingeschr"ankte $\blet\,\brec$-Ausdr"ucke, die sich
% als syntaktischer Zucker f"ur $\blet$- und $\brec$ auffassen lassen:
% \[\blri{\id}{v}{e}\]
% kann man als syntaktischen Zucker f"ur
% \[\bli{\id}{\rec{\id}{v}}{e}\]
% verstehen. In der Umgebungssemantik w"urde f"ur $(\bli{\id}{\rec{\id}{v}}{e},\eta)$ nun zun"achst ein
% \RN{Unfold} der Form
% \[(v,\id:(\rec{\id}{v},\eta);\eta)\]
% durchgef"uhrt, und das Ergebnis davon w"urde f"ur $e$ in $\eta$ eingetragen. D.h. es ergibt
% sich folgende Closure, wenn man mal annimmt, dass $v$ ein $\lambda$-Ausdruck ist:
% \[(e,\id:(v,\id:(\rec{\id}{v},\eta);\eta)\]
% Wird nun in der Berechnung von $e$ auf $\id$ Bezug genommen, ergibt sich wiederum die Closure
% \[(v,\id:(\rec{\id}{v},\eta);\eta)\]
% und wird nun $v$ (von dem wir angenommen haben, dass es ein $\lambda$-Ausdruck ist) auf ein
% Argument angewandt und w"ahrend der sich dann ergebenden Berechnung wiederum auf $\id$ Bezug
% genommen, so landen wir wieder bei
% \[(\rec{\id}{v},\eta)\]
% und m"ussen wieder direkt mit \RN{Unfold} auffalten zu:
% \[(v,\id:(\rec{\id}{v},\eta);\eta)\]
% Diese \RN{Unfold}-Schritte sind also stets gleich und eigentlich unn"otig, da $\brec$-Ausdr"ucke nur
% noch Werte als Teilausdr"ucke enthalten d"urfen, k"onnten wir das Auffalten auch direkt beim Lookup
% erledigen. D.h. man k"onnte die \RN{Id} und \RN{Unfold}-Regeln durch \\[5mm]
% \begin{tabular}{ll}
%   \RN{Id-Unfold} & $(\id,\id:(\rec{\id}{v},\eta);\eta) \Downarrow (v,\id:(\rec{\id}{v},\eta);\eta)$ \\[1mm]
% \end{tabular} \\[2mm]
% ersetzen.

\subsection{Zusammenhang}

\begin{definition}["Ubersetzungsfunktion]
  Seien $w \in W_{Ptr}$ und $\eta \in \Env_{Ptr}$. Der Umgebungswert $\trPtr{w}$, welcher durch "Ubersetzung
  aus $w$ entsteht, und die Umgebung $\trPtr{\eta}$, die durch "Ubersetzung aus $\eta$ entsteht, sind wie folgt
  induktiv definiert:
  \[\begin{array}{rcl}
    \trPtr{(c,[\,])} &=& (c,\trPtr{[\,]}) \\
    \trPtr{(\abstr{\id}{e},\eta)} &=& (\abstr{\id}{e},\trPtr{\eta}) \\
    \trPtr{(w_1,\ldots,w_n)} &=& (\trPtr{w_1},\ldots,\trPtr{w_n}) \\
    \\
    \trPtr{[\,]} &=& [\,] \\
    \trPtr{\id:(v,\uparrow);\eta} &=& \id:(\rec{\id}{v},\trPtr{\eta});(\trPtr{\eta}) \\
    \trPtr{\id:w;\eta} &=& \id:(\trPtr{w});(\trPtr{\eta}) \\
  \end{array}\]
\end{definition}

\noindent
Die "Ubersetzungsfunktion erweitert sich trivialerweise auf Closures durch:
\[\trPtr{e,\eta} = (e,\trPtr{\eta})\]

\begin{corollary} \
  \begin{enumerate}
  \item F"ur alle $w \in W_{Ptr}$ gilt: $\trPtr{w}\in W$.
  \item F"ur alle $\eta \in \Env_{Ptr}$ gilt: $\trPtr{\eta} \in \Env$.
  \end{enumerate}
\end{corollary}

\begin{theorem}[Korrektheit der Pointersemantik]
  Wenn $(e,\eta) \Downarrow w$ in der Pointersemantik, dann gilt
  $\trPtr{e,\eta} \Downarrow \trPtr{w}$ in der Umgebungssemantik.
\end{theorem}

\begin{theorem}[Vollst"andigkeit der Pointersemantik]
  Wenn $\trPtr{e,\eta} \Downarrow \trPtr{w}$ in der Umgebungssemantik,
  dann gilt $(e,\eta) \Downarrow w$ in der Pointersemantik.
\end{theorem}

\end{document}

% vi:set ts=2 sw=2 et:

\documentclass[12pt,fleqn,a4paper]{article}
\usepackage{ngerman}
\usepackage{hyperref,german,amssymb,amstext,amsmath,amsthm,array,stmaryrd,color,latexsym}

% TP Macros
% General

\newcommand{\name}[1]{{\text{\it #1\/}}}
%\newcommand{\name}[1]{\mathit{#1}}

%\newcommand{\bfbox}[1]{\mathbf{#1}}

\newcommand{\I}{{\cal I}}

% Proof trees

\newcounter{tree}
\newcounter{node}[tree]

\newlength{\treeindent}
\newlength{\nodeindent}
\newlength{\nodesep}

\newcommand{\refnode}[1]
 {\ref{\thetree.#1}}

\newcommand{\contextcolor}{\color{blue}}

\definecolor{darkgreen}{rgb}{0.1,0.5,0}
\definecolor{redexcolor}{rgb}{0.7,1,0}

\newcommand{\resulttypecolor}{\color{darkgreen}}

\newcommand{\resultcolor}{\color{blue}}

\newcommand{\byrulecolor}{\color{red}}

\newcommand{\marked}[1]{\colorbox{redexcolor}{$#1$}}

\newcommand{\smallsteparrow}[1]{\stackrel{\mbox{\scriptsize\byrulecolor (#1)}}{\longrightarrow}}

\newcommand{\byrule}[1]{\hspace{-5mm}\byrulecolor\mbox{\scriptsize\ #1}}

\newif\ifarrows 
\arrowsfalse 

\newcommand{\arrow}[3]
  {\ifarrows
   \ncangle[angleA=-90,angleB=#1]{<-}{\thetree.#2}{\thetree.#3}
   \else
   \fi}

\newcommand{\node}[4]
 {\ifarrows
   \else \refstepcounter{node}
         \noindent\hspace{\treeindent}\hspace{#2\nodeindent}
         \rnode{\thetree.#1}{\makebox[6mm]{(\thenode)}}\label{\thetree.#1}
         $\blong 
          #3 \\ 
          \byrule{#4} 
          \elong$
         \vspace{\nodesep}
   \fi}

\newcommand{\dummyarrow}[3]
  {\arrow{#1}{#2}{#3dummy}}

\newcommand{\dummynode}[2]
 {\ifarrows 
  \else \noindent\hspace{\treeindent}\hspace{#2\nodeindent}
         \rnode{\thetree.#1dummy}{\makebox[6mm]{(\refnode{#1})}}\label{\thetree.#1dummy}
         $\ldots$
         \vspace{\nodesep}
   \fi}

\newcommand{\mktree}[1]
 {\stepcounter{tree} #1 \arrowstrue #1 \arrowsfalse}

\fboxrule=0mm

\newcommand{\cenv}[1]{\fbox{$\begin{array}{|ll|}\hline #1 \\\hline\end{array}$}}


% Special Symbols

\newcommand{\uminus}{\widetilde{\ }}
\renewcommand{\uminus}{-\!}
\newcommand{\nop}{()}

% Im X-Symbol-Manual empfohlen

\newcommand{\nsubset}{\not\subset}
%\newcommand{\textflorin}{\textit{f}}
\newcommand{\setB}{{\mathord{\mathbb B}}}
\newcommand{\setC}{{\mathord{\mathbb C}}}
\newcommand{\setN}{{\mathord{\mathbb N}}}
\newcommand{\setQ}{{\mathord{\mathbb Q}}}
\newcommand{\setR}{{\mathord{\mathbb R}}}
\newcommand{\setZ}{{\mathord{\mathbb Z}}}
\newcommand{\coloncolon}{\mathrel{::}}

% Eigene (k\"urzere) Befehle

\newcommand{\pfi}{\varphi}
\newcommand{\eps}{\varepsilon}
\newcommand{\eval}{\Downarrow}
\newcommand{\pto}{\hookrightarrow}
\newcommand{\emp}{\emptyset}
\newcommand{\sleq}{\subseteq}
\newcommand{\sgeq}{\supseteq}
\newcommand{\sqleq}{\sqsubseteq}
\newcommand{\sqgeq}{\sqsupseteq}
\newcommand{\lub}{\bigsqcup}
\newcommand{\glb}{\bigsqcap}
\newcommand{\lsem}{\llbracket}
\newcommand{\rsem}{\rrbracket}
\newcommand{\impl}{\models}
\newcommand{\step}{\vdash}
%\newcommand{\tr}{\triangleright}
\newcommand{\cc}{\coloncolon}


% Names (lower case italic)

\newcommand{\exn}{\name{exn}}
\newcommand{\id}{\name{id}}
\newcommand{\op}{\name{op}}

\newcommand{\true}{\name{true}}
\newcommand{\false}{\name{false}}
\newcommand{\Not}{\name{not}}

\newcommand{\sol}[3]{\name{solution}\,(#1,#2,#3)}
%\newcommand{\unify}{\name{unify}\,}

\newcommand{\Fst}{\name{fst}}
\newcommand{\Snd}{\name{snd}}

\newcommand{\Hd}{\name{hd}}
\newcommand{\Tl}{\name{tl}}
\newcommand{\Cons}{\name{cons}}
\newcommand{\Empty}{\name{is\_empty}}

\newcommand{\dom}[1]{\name{dom}(#1)}
\newcommand{\free}[1]{\name{free}\,(#1)}
\newcommand{\rank}[1]{\name{rank}\,(#1)}
\newcommand{\len}[1]{\name{len}\,(#1)}
\newcommand{\tr}[1]{\name{tr}(#1)}

% Names (upper case italic)

\newcommand{\Bool}{\name{Bool}}
\newcommand{\Btype}{\name{BType}}

\newcommand{\Conf}{\name{Conf}}
\newcommand{\Const}{\name{Const}}

\newcommand{\Dec}{\name{Dec}}

\newcommand{\Env}{\name{Env}}
\newcommand{\Exn}{\name{Exn}}
\newcommand{\EP}{\name{EP}}
\newcommand{\Exp}{\name{Exp}}

\newcommand{\Ncx}{\name{Ncx}}
\newcommand{\dbExp}{\name{dbExp}}
\newcommand{\dbVal}{\name{dbVal}}
\newcommand{\dbEnv}{\name{dbEnv}}
\newcommand{\dbCl}{\name{dbCl}}

\newcommand{\Id}{\name{Id}}
\newcommand{\Int}{\name{Int}}

\newcommand{\Lexp}{\name{LExp}}
\newcommand{\Loc}{\name{Loc}}

\newcommand{\Op}{\name{Op}}

\newcommand{\Type}{\name{Type}}
\newcommand{\Tvar}{\name{TVar}}
\newcommand{\teqns}[3]{\name{teqns}\,(#1,#2,#3)}
\newcommand{\tvar}[1]{\name{tvar}\,(#1)}
\newcommand{\unify}[1]{\name{unify}\,(#1)}

\newcommand{\Ptype}{\name{PType}}

\newcommand{\Unit}{\name{Unit}}

\newcommand{\Val}{\name{Val}}

% keywords

\newcommand{\z}{\mathbf{int}}
\newcommand{\bool}{\mathbf{bool}}
\newcommand{\unit}{\mathbf{unit}}
\newcommand{\blist}{\mathbf{list}}
\newcommand{\bref}{\mathbf{ref}}
\newcommand{\ltype}[1]{#1\,\blist}
\newcommand{\reftype}[1]{#1\,\bref}

\renewcommand{\div}{\mathbin{\mathbf{div}}}
\newcommand{\sel}{\mathbin{.}}
%\newcommand{\mod}{\mathbin{\mathbf{mod}}}
\renewcommand{\mod}{\text{mod}}

\newcommand{\bif}{\mathbf{if}}
\newcommand{\bthen}{\mathbf{then}}
\newcommand{\belse}{\mathbf{else}}

\newcommand{\blet}{\mathbf{let}}
\newcommand{\bin}{\mathbf{in}}
\newcommand{\bend}{\mathbf{end}}

\newcommand{\bval}{\mathbf{val}}
\newcommand{\brec}{\mathbf{rec}}
\newcommand{\bfix}{\mathbf{fix}}
\newcommand{\bfun}{\mathbf{fun}}
\newcommand{\band}{\mathbf{and}}
\newcommand{\btype}{\mathbf{type}}

%\newcommand{\andalso}[2]{#1\,\&\&\,#2}
%\newcommand{\orelse}[2]{#1\,\|\,#2}
\newcommand{\andalso}{\&\&}
\newcommand{\orelse}{\|}
\newcommand{\bandalso}{\mathbf{andalso}}
\newcommand{\borelse}{\mathbf{orelse}}

\newcommand{\bwhile}{\mathbf{while}}
\newcommand{\bdo}{\mathbf{do}}
\newcommand{\bfor}{\mathbf{for}}

\newcommand{\brepeat}{\mathbf{repeat}}
\newcommand{\buntil}{\mathbf{until}}

\newcommand{\barray}{\mathbf{array}}
\newcommand{\bof}{\mathbf{of}}

\newcommand{\bclass}{\mathbf{class}}

\newcommand{\app}[2]{#1\,#2}
\newcommand{\bift}[2]{\bif\ #1\ \bthen\ #2}
\newcommand{\bifte}[3]{\bif\ #1\ \bthen\ #2\ \belse\ #3}
\newcommand{\Bifte}[3]{\blong\bif\ #1\\\bthen\ #2\\\belse\ #3\elong}
\newcommand{\bwd}[2]{\bwhile\ #1\ \bdo\ #2}
\newcommand{\bru}[2]{\brepeat\ #1\ \buntil\ #2}


\newcommand{\blie}[2]{\blet\ #1\ \bin\ #2 \ \bend}
\newcommand{\vdec}[2]{#1 = #2}
\newcommand{\rec}[2]{\brec\ #1.\ #2}
\newcommand{\recdots}[2]{\brec\ #1.\ \ldots}
\newcommand{\abstr}[2]{\lambda #1.\,#2}
\newcommand{\appl}[2]{#1\,#2}

\newcommand{\Ref}{\name{ref}}
\newcommand{\Deref}{\,!\,}

\newcommand{\alloc}{\name{alloc}}
\newcommand{\Store}{\name{Store}}




% Typing rules 

\newcommand{\tj}[2]{#1\cc#2}
\newcommand{\ctj}[2]{\tj{#1}{{\resulttypecolor #2}}}
\newcommand{\Tj}[3]{#1 \, \triangleright \, #2\cc#3}
\newcommand{\cTj}[3]{\Tj{{\contextcolor #1}}{#2}{{\resulttypecolor #3}}}
\newcommand{\cbig}[2]{#1 \ \eval\ {\resultcolor #2}}
\newcommand{\cBig}[2]{#1 \\ \eval\ {\resultcolor #2}}
\newcommand{\Clos}[2]{\name{Closure}_{#1}(#2)}

\newcommand{\Tjl}[3]{#1 \,\triangleright_l\, #2\cc#3}
\newcommand{\Tjm}[3]{#1 \,\triangleright_m\, #2\cc#3}
\newcommand{\Tjh}[2]{#1 \triangleright  #2}

\newcommand{\brule}[1]{\begin{markiere}[#1]}
\newcommand{\erule}{\end{markiere}}

\newcommand{\regel}[2]{\ \begin{array}{@{}c@{}} #1 \\ \hline #2
 \end{array}\ }

\newcommand{\reason}[1]{\ \mbox{#1}}
\newcommand{\Reason}[1]{\vspace{1mm}\\ \mbox{ #1}}


% Program verification

\newcommand{\conj}{\,\land\,}
\newcommand{\Conj}{\bigwedge}
\newcommand{\disj}{\,\lor\,}
\newcommand{\Disj}{\bigvee\,}
\newcommand{\all}[1]{\forall{#1}.\,}
\newcommand{\ex}[1]{\exists{#1}.\,}

\newcommand{\power}[1]{\wp(#1)}

\newcommand{\disjoint}[2]{\name{disj}(#1,#2)}
\newcommand{\cont}[2]{#1 \mapsto #2}
\newcommand{\DEF}{\name{DEF}}

\newcommand{\ret}[2]{{\bf returns}\ #1.\, #2}
\newcommand{\tc}[2]{#1\,\{#2\}}
\newcommand{\triple}[3]{\{#1\}\,#2\,\{#3\}}

% Index

\newcommand{\define}[1]{{\em #1\/}\index{#1}}
\newcommand{\Define}[2]{{\em #1\/}\index{#2}}
\newcommand{\Index}[1]{\index{#1}}
\newcommand{\notation}[1]{#1\index{#1}}
\newcommand{\engl}[1]{(engl.: \define{#1})}
\newcommand{\Engl}[2]{(engl.: \Define{#1}{#2})}

% Theorems etc.

\newtheorem{theorem}{Satz}
\newtheorem{corollary}{Korollar}
\newtheorem{definition}{Definition:}
%\newtheorem{example}{Beispiel:}
%\newtheorem{examples}{Beispiele:}
\newtheorem{lemma}[theorem]{Lemma}

%\renewcommand{\thedefinition}{}
%\renewcommand{\theexample}{}
%\renewcommand{\theexamples}{}
\renewcommand{\theenumi}{\rm (\alph{enumi})}
\renewcommand{\labelenumi}{\theenumi}

%\newcommand{\enumarabic}{\renewcommand{\theenumi}{\rm (\arabic{enumi})}}

%\newcommand{\bcoro}[1]{\begin{corollary}\label{cor:#1}}
%\newcommand{\ecoro}{\end{corollary}}

\newcommand{\brdef}[1]{\begin{definition}\label{def:#1}\rm}
\newcommand{\erdef}{\end{definition}}

%\newcommand{\blemm}[1]{\begin{lemma}\label{lem:#1}}
%\newcommand{\bLemm}[2]{\begin{lemma}[#2]\label{lem:#1}\index{#2}}
%\newcommand{\elemm}{\end{lemma}}

%\newcommand{\btheo}[1]{\begin{theorem}\label{th:#1}}
%\newcommand{\bTheo}[2]{\begin{theorem}[#2]\label{th:#1}\index{#2}}
%\newcommand{\etheo}{\end{theorem}}

%\newcommand{\litem}[1]{\item\label{it:#1}}
%\newcommand{\ritem}[1]{\ref{it:#1}}


% Grammars

\newcommand{\bgram}{\[\begin{array}{rrlll}}
\newcommand{\egram}{\end{array}\]}

\newcommand{\is}{& ::= &}
\newcommand{\al}{\\ & \mid &}
\newcommand{\n}{\vspace{2mm}\\}




% Other Environments

\newcommand{\bcase}{\left\{\!\!\!\begin{array}{ll}}
\newcommand{\ecase}{\end{array}\right.}
\newcommand{\benum}{\begin{enumerate}}
\newcommand{\eenum}{\end{enumerate}}
\newcommand{\beqns}{\[\begin{array}{rcll}}
\newcommand{\eeqns}{\end{array}\]}
\newcommand{\bitem}{\begin{itemize}}
\newcommand{\eitem}{\end{itemize}}
\newcommand{\blong}{\!\!\begin{array}[t]{l}}
\newcommand{\elong}{\end{array}}
\newcommand{\btabl}{\begin{tabular}}
\newcommand{\etabl}{\end{tabular}}
\newcommand{\brexa}{\begin{example}\enumarabic\rm}
\newcommand{\erexa}{\end{example}}
\newcommand{\brexs}{\begin{examples}\enumarabic\rm}
\newcommand{\erexs}{\end{examples}}


% German abbreviations

\newcommand{\abk}[1]{#1.\ }
\newcommand{\bzw}{\abk{bzw}}
\newcommand{\bzgl}{\abk{bzgl}}
\newcommand{\das}{\abk{d.h}}
\newcommand{\evtl}{\abk{evtl}}
\newcommand{\usw}{\abk{usw}}
\newcommand{\vgl}{\abk{vgl}}
\newcommand{\zb}{\abk{z.B}}


\newcommand{\infix}[3]{#2\mathbin{#1}#3}

\newcommand{\bli}[3]{\blet\ \vdec{#1}{#2}\ \bin\ #3}

\newcommand{\Bli}[3]{\begin{array}[t]{@{}l}
                     \blet\ \vdec{#1}{#2}\\\bin\ #3
                     \end{array}}

\newcommand{\Vdec}[2]{\begin{array}[t]{@{}l}
                      #1 = \\
                      \ #2
                      \end{array}}

\newcommand{\BLI}[3]{\begin{array}[t]{@{}l}
                     \blet\ \Vdec{#1}{#2}\\\bin\ #3
                     \end{array}}

\newcommand{\Abstr}[2]{\begin{array}[t]{@{}l}
                       \lambda #1.\\\ \ #2
                       \end{array}}


\newcommand{\RN}[1]{\mbox{\textsc{(#1)}}}
\newcommand{\Cl}{\name{Cl}}
\newcommand{\cl}{\name{cl}}
\newcommand{\Req}{\name{Req}}
\newcommand{\Ptr}{\name{Ptr}}

\begin{document}

\section{Substitutionssemantik}

\begin{definition}[Syntax der Programmiersprache]
  Vorgegeben seien
  \begin{itemize}
  \item eine Menge $\Bool = \{\true,\false\}$ von booleschen Konstanten $b$,
  \item eine Menge $\Int = \setZ$ von Integerkonstanten $z$, und
  \item eine (unendliche) Menge $\Id$ von Namen $\id$.
  \end{itemize}
  Die Mengen $\Op$ aller {\em Operatoren} $\op$, $\Const$ aller {\em Konstanten} $c$, $\Exp$ aller 
  {\em Ausdr\"ucke} $e$ und $\Val$ aller {\em Werte} $v$ sind durch folgende kontextfreie Grammatik definiert:
  \bgram
  \op \is + \mid - \mid * \mid \le \mid \ge \mid < \mid > \mid = \\
  c \is b \mid z \mid \op \mid \proj{i} \\
  e \is c \mid \id \mid \abstr{\id}{e} \mid \app{e_1}{e_2} \mid \bli{\id}{e_1}{e_2}
  \al \bifte{e_0}{e_1}{e_2} \mid (e_1,\ldots,e_n) \\
  v \is c \mid \abstr{\id}{e} \mid (v_1,\ldots,v_n)
  \egram
\end{definition}

\begin{definition}[Big step Regeln]
Ein {\em big step} in der Substitutionssemantik ist eine Formel der Gestalt $e \Downarrow v$ mit $e\in\Exp$
und $v \in \Val$. Ein solcher big step hei"st {\em g"ultig}, wenn er sich mit den folgenden Regeln herleiten
l"asst: \\[5mm]
\begin{tabular}{ll}
  \RN{Val}        & $v \Downarrow v$ \\[3mm]
  \RN{Beta-V}     & $\regel{e_1 \Downarrow \abstr{\id}{e} \quad e_2 \Downarrow v \quad e[v/\id] \Downarrow v'}
                           {\app{e_1}{e_2} \Downarrow v'}$ \\[3mm]
  \RN{Op}         & $\regel{e_1 \Downarrow \op \quad e_2 \Downarrow (z_1,z_2)}
                           {\app{e_1}{e_2} \Downarrow \op^I(z_1,z_2)}$ \\[3mm]
  \RN{Proj}       & $\regel{e_1 \Downarrow \proj{i} \quad e_2 \Downarrow (v_1,\ldots,v_n) \quad 1 \le i \le n}
                           {\app{e_1}{e_2} \Downarrow v_i}$ \\[3mm]
  \RN{Let}        & $\regel{e_1 \Downarrow v \quad e_2[v/\id] \Downarrow v'}
                           {\bli{\id}{e_1}{e_2} \Downarrow v'}$ \\[3mm]
  \RN{Cond-True}  & $\regel{e_0 \Downarrow \true \quad e_1 \Downarrow v}
                           {\bifte{e_0}{e_1}{e_2} \Downarrow v}$ \\[3mm]
  \RN{Cond-False} & $\regel{e_0 \Downarrow \false \quad e_2 \Downarrow v}
                           {\bifte{e_0}{e_1}{e_2} \Downarrow v}$ \\[3mm]
  \RN{Tuple}      & $\regel{e_1 \Downarrow v_1 \quad \ldots \quad e_n \Downarrow v_n}
                           {(e_1,\ldots,e_n) \Downarrow (v_1,\ldots,v_n)}$ \\[3mm]
\end{tabular}
\end{definition}


\section{Umgebungssemantik}

\begin{definition}[Umgebungen]
  Die Mengen $\Env$ aller {\em Umgebungen} $\eta$ und $W$ aller {\em Umgebungswerte} $w$ sind
  durch die folgende kontextfreie Grammatik definiert:
  \bgram
  \eta \is [\,]
  \al \id:v;\eta
  \al \id:w;\eta
  \n
  w \is c
  \al (\abstr{\id}{e},\eta)
  \al (w_1,\ldots,w_n)
  \egram
  Die Menge $\Cl$ aller Closures $\cl$ ist definiert durch:
  \bgram
  \cl \is (e,\eta)
  \egram
\end{definition}

% \begin{definition}[Freie Namen]
%   Die Mengen $\free{\eta}$, $\free{w}$ und $\free{\cl}$ sind wie folgt induktiv definiert:
%   \[\begin{array}{rcl}
%     \free{[\,]} &=& \emptyset \\
%     \free{\id:w;\eta} &=& \free{w} \cup \free{\eta} \\
%     \free{c} &=& \emptyset \\
%     \free{w_1,\ldots,w_n} &=& \bigcup_{i=1 \ldots n} \free{w_i} \\
%     \free{e,\eta} &=& \bigl( \free{e} \setminus \dom{\eta} \bigr) \cup \free{\eta} \\
%   \end{array}\]
%   Eine Umgebung $\eta$ hei"st \emph{abgeschlossen}, wenn $\free{\eta} = \emptyset$. Analog f"ur Werte $w$
%   und Closures $\cl$.
% \end{definition}

\begin{definition}
  Ein \emph{expand} ist eine Formel der Gestalt $\Lj{\eta}{v}{w}$, wobei $\eta\in\Env$, $v\in\Val$ und $w\in W$.
  Ein derartiger expand hei"st \emph{g"ultig}, wenn er sich mit den folgenden Regeln herleiten l"asst: \\[5mm]
  \begin{tabular}{ll}
    \RN{E-Const}   & $\Lj{\eta}{c}{c}$ \\[1mm]
    \RN{E-Closure} & $\Lj{\eta}{\abstr{\id}{e}}{(\abstr{\id}{e},\eta)}$ \\[1mm]
    \RN{E-Tuple}   & $\regel{\Lj{\eta}{v_1}{w_1} \quad\ldots\quad \Lj{\eta}{v_n}{w_n}}
                            {\Lj{\eta}{(v_1,\ldots,v_n)}{(w_1,\ldots,w_n)}}$ \\[3mm]
  \end{tabular} \\[2mm]
  Ein \emph{lookup} ist eine Formel der Gestalt $\Lj{\eta}{\id}{w}$, wobei $\eta\in\Env$, $\id\in\Id$ und
  $w \in W$. Ein derartiger lookup hei"st \emph{g"ultig}, wenn er sich mit den folgenden Regeln herleiten
  l"asst: \\[5mm]
  \begin{tabular}{ll}
    \RN{L-Immediate} & $\Lj{\id:w;\eta}{\id}{w}$ \\[1mm]
    \RN{L-Expand}    & $\regel{\Lj{\id:v;\eta}{v}{w}}
                              {\Lj{\id:v;\eta}{\id}{w}}$ \\[3mm]
    \RN{L-Skip}      & $\regel{\id \ne \id' \quad \Lj{\eta}{\id'}{w'}}
                              {\Lj{\id:\underline{\quad};\eta}{\id'}{w'}}$ \\[3mm]
  \end{tabular}
\end{definition}

\begin{definition}[Big step Regeln]
Ein {\em big step} in der Umgebungssemantik ist eine Formel der Gestalt $(e,\eta) \Downarrow w$,
wobei $e\in\Exp$, $\eta\in\Env$ und $w \in W$. Ein derartiger big step hei"st {\em g"ultig}, wenn er sich mit den
folgenden Regeln herleiten l"asst: \\[5mm]
\begin{tabular}{ll}
  \RN{Const}      & $(c,\eta) \Downarrow c$ \\[1mm]
  \RN{Closure}    & $(\abstr{\id}{e},\eta) \Downarrow (\abstr{\id}{e},\eta)$ \\[1mm]
  \RN{Id}         & $\regel{\Lj{\eta}{id}{w}}
                           {(\id,\eta) \Downarrow w}$ \\[3mm]
  \RN{Beta-V}     & $\regel{(e_1,\eta) \Downarrow (\abstr{\id'}{e'},\eta')
                            \quad (e_2,\eta) \Downarrow w'
                            \quad (e',\id':w';\eta') \Downarrow w}
                           {(\app{e_1}{e_2},\eta) \Downarrow w}$ \\[3mm]
  \RN{Op}         & $\regel{(e_1,\eta) \Downarrow \op \quad (e_2,\eta) \Downarrow (z_1,z_2)}
                           {(\app{e_1}{e_2},\eta) \Downarrow \op^I(z_1,z_2)}$ \\[3mm]
  \RN{Proj}       & $\regel{(e_1,\eta) \Downarrow \proj{i} \quad (e_2,\eta) \Downarrow (w_1,\ldots,w_n)
                            \quad 1 \le i \le n}
                           {\app{e_1}{e_2} \Downarrow w_i}$ \\[3mm]
  \RN{Let}        & $\regel{(e_1,\eta) \Downarrow w \quad (e_2,\id:w;\eta) \Downarrow w'}
                           {(\bli{\id}{e_1}{e_2},\eta) \Downarrow w'}$ \\[3mm]
  \RN{Let-Rec}    & $\regel{(e,\id:v;\eta) \Downarrow w}
                           {(\blri{\id}{v}{e},\eta) \Downarrow w}$ \\[3mm]
  \RN{Cond-True}  & $\regel{(e_0,\eta) \Downarrow \true \quad (e_1,\eta) \Downarrow w}
                           {(\bifte{e_0}{e_1}{e_2},\eta) \Downarrow w}$ \\[3mm]
  \RN{Cond-False} & $\regel{(e_0,\eta) \Downarrow \false \quad (e_2,\eta) \Downarrow w}
                           {(\bifte{e_0}{e_1}{e_2},\eta) \Downarrow w}$ \\[3mm]
  \RN{Tuple}      & $\regel{(e_1,\eta) \Downarrow w_1 \quad \ldots \quad (e_n,\eta) \Downarrow w_n}
                           {((e_1,\ldots,e_n),\eta) \Downarrow (w_1,\ldots,w_n)}$ \\[3mm]
\end{tabular}
\end{definition}

% \begin{lemma}[Wohldefiniertheit der Umgebungssemantik] \label{lemma:Wohldefiniertheit}
%   Wenn $\cl \Downarrow w$, dann gilt $\free{w} \subseteq \free{\cl}$.
% \end{lemma}

\begin{definition}
  Die Relation $\multimap$ ist wie folgt definiert: \\[5mm]
  \begin{tabular}{ll}
    \RN{App-Left}   & $(\app{e_1}{e_2},\eta) \multimap (e_1,\eta)$ \\[1mm]
    \RN{App-Right}  & $(\app{e_1}{e_2},\eta) \multimap (e_2,\eta)$ \\[1mm]
    \RN{Beta-V}     & $\regel{(e_1,\eta) \Downarrow (\abstr{\id'}{e'},\eta')
                              \quad (e_2,\eta) \Downarrow w'}
                             {(\app{e_1}{e_2},\eta) \multimap (e',\id':w';\eta')}$ \\[3mm]
    \RN{Let-Eval}   & $(\bli{\id}{e_1}{e_2},\eta) \multimap (e_1,\eta)$ \\[1mm]
    \RN{Let-Exec}   & $\regel{(e_1,\eta) \Downarrow w}
                             {(\bli{\id}{e_1}{e_2},\eta) \multimap (e_2,\id:w;\eta)}$ \\[3mm]
    \RN{Let-Rec}    & $(\blri{\id}{v}{e},\eta) \multimap (e,\id:v;\eta)$ \\[1mm]
    \RN{Cond-Eval}  & $(\bifte{e_0}{e_1}{e_2},\eta) \multimap (e_0,\eta)$ \\[1mm]
    \RN{Cond-True}  & $\regel{(e_0,\eta) \Downarrow \true}
                             {(\bifte{e_0}{e_1}{e_2},\eta) \multimap (e_1,\eta)}$ \\[3mm]
    \RN{Cond-False} & $\regel{(e_0,\eta) \Downarrow \false}
                             {(\bifte{e_0}{e_1}{e_2},\eta) \multimap (e_2,\eta)}$ \\[3mm]
    \RN{Tuple}      & $\regel{1 \le i \le n}
                             {((e_1,\ldots,e_n),\eta) \multimap (e_i,\eta)}$ \\[3mm]
  \end{tabular}
\end{definition}


\section{Ein allgemeines Typsystem}

\begin{definition}[Typen]
  Die Menge aller g"ultigen \emph{Typen} $\tau$ ist durch die folgende
  kontextfreie Grammatik definiert:
  \bgram
  \tau \is \bool \mid \z \mid \unit
  \al \tau_1 \to \tau_2 \mid (\tau_1,\ldots,\tau_n)
  \egram
\end{definition}

\begin{definition}[Typregeln]
  Ein Typurteil $\Tj{\Gamma}{e}{\tau}$ hei"st g"ultig, wenn es sich mit den folgenden Regeln herleiten l"asst: \\[5mm]
  \begin{tabular}{ll}
    \RN{Const}   & $\regel{\tj{c}{\tau}}
                          {\Tj{\Gamma}{c}{\tau}}$ \\[3mm]
    \RN{Id}      & $\regel{\Gamma(\id) = \tau}
                          {\Tj{\Gamma}{\id}{\tau}}$ \\[3mm]
    \RN{Abstr}   & $\regel{\Tj{(\id:\tau;\Gamma)}{e}{\beta}}
                          {\Tj{\Gamma}{\abstr{\id}{e}}{\tau \to \beta}}$ \\[3mm]
    \RN{App}     & $\regel{\Tj{\Gamma}{e_1}{\tau \to \tau'} \quad \Tj{\Gamma}{e_2}{\tau}}
                          {\Tj{\Gamma}{\app{e_1}{e_2}}{\tau'}}$ \\[3mm]
    \RN{Let}     & $\regel{\Tj{\Gamma}{e_1}{\tau} \quad \Tj{(\id:\tau;\Gamma)}{e_2}{\beta}}
                          {\Tj{\Gamma}{\bli{\id}{e_1}{e_2}}{\beta}}$ \\[3mm]
    \RN{Let-Rec} & $\regel{\Tj{(\id:\tau;\Gamma)}{v}{\tau} \quad \Tj{\id:\tau;\Gamma}{e}{\beta}}
                          {\Tj{\Gamma}{\blri{\id}{v}{e}}{\beta}}$ \\[3mm]
    \RN{Cond}    & $\regel{\Tj{\Gamma}{e_0}{\bool} \quad \Tj{\Gamma}{e_1}{\tau} \quad \Tj{\Gamma}{e_2}{\tau}}
                          {\Tj{\Gamma}{\bifte{e_0}{e_1}{e_2}}{\tau}}$ \\[3mm]
    \RN{Tuple}   & $\regel{\Tj{\Gamma}{e_1}{\tau_1} \quad \ldots \quad \Tj{\Gamma}{e_n}{\tau_n}}
                          {\Tj{\Gamma}{(e_1,\ldots,e_n)}{(\tau_1,\ldots,\tau_n)}}$ \\[3mm]
  \end{tabular} \\[2mm]
  Typurteile f"ur Umgebungen sind von der Form $\Tjh{\Gamma_w,\Gamma_\eta}{\eta}$, Typurteile f"ur Closures und
  Werte sind von der Form $\Tj{\Gamma}{\cl}{\tau}$ bzw. $\Tj{\Gamma}{w}{\tau}$.
  \[\begin{array}{cc}
    \Tjh{\Gamma,[\,]}{[\,]}
    & \regel{\Tj{(\Gamma_\eta;\Gamma)}{e}{\tau} \quad \Tjh{\Gamma,\Gamma_\eta}{\eta}}{\Tj{\Gamma}{(e,\eta)}{\tau}} \\
    \regel{\Tj{\Gamma_w}{w}{\tau} \quad \Tjh{\Gamma_w,\Gamma_\eta}{\eta}}{\Tjh{\Gamma_w,(\id:\tau;\Gamma_\eta)}{(\id:w;\eta)}}
    & \regel{\Tj{\Gamma}{w_1}{\tau_1} \quad \ldots \quad \Tj{\Gamma}{w_n}{\tau_n}}{\Tj{\Gamma}{(w_1,\ldots,w_n)}{(\tau_1,\ldots,\tau_n)}} \\
    \regel{\Tj{(\id:\tau;\Gamma_\eta;\Gamma_v)}{v}{\tau} \quad \Tjh{\Gamma_v,\Gamma_\eta}{\eta}}{\Tjh{\Gamma_v,(\id:\tau;\Gamma_\eta)}{(\id:v;\eta)}}
  \end{array}\]
\end{definition}

% \begin{lemma}
%   Sei $\Tj{\Gamma}{(e,\eta)}{\tau}$.
%   \begin{enumerate}
%   \item Wenn $(e,\eta) \multimap (e',\eta')$, dann ex. $\Gamma',\tau'$ mit $\Gamma \sqsubseteq \Gamma'$
%     und $\Tj{\Gamma'}{(e',\eta')}{\tau'}$.
%   \item Wenn $(e,\eta) \Downarrow w$, dann $\Tj{\Gamma}{w}{\tau}$.
%   \end{enumerate}
% \end{lemma}


\section{Stack Disziplin}

\begin{definition}
  Sei $w \in W$ und $\eta \in \Env$. Eine Umgebung $\eta'$ hei"st \emph{Teil von $\eta$ bzw. $w$},
  geschrieben $\eta' \sqsubseteq \eta$ bzw. $\eta' \sqsubseteq w$, wenn sich dies mit den folgenden
  Regeln herleiten l"asst: \\[5mm]
  \begin{tabular}{ccc}
    $\eta \sqsubseteq \eta$
    & $\regel{\eta' \sqsubseteq \eta}{\eta' \sqsubseteq (\id:\underline{\quad};\eta)}$
    & $\regel{\eta' \sqsubseteq w}{\eta' \sqsubseteq (\id:w;\eta)}$ \\[3mm]
    $\regel{\eta' \sqsubseteq \eta}{\eta' \sqsubseteq (\abstr{\id}{e},\eta)}$
    & $\regel{\eta \sqsubseteq w_i \quad 1 \le i \le n}{\eta \sqsubseteq (w_1,\ldots,w_n)}$ \\[3mm]
  \end{tabular} \\[2mm]
  $\eta'$ hei"st \emph{echter Teil von $\eta$}, wenn $\eta' \sqsubseteq \eta$ und $\eta' \ne \eta$.
\end{definition}

\begin{lemma}
  Die Relation $\sqsubseteq$ ist eine Halbordnung auf $\Env \times \Env$ mit kleinstem Element $[\,]$.
\end{lemma}

% \begin{definition}
%   Eine Umgebung $\eta$ bzw. ein Wert $w$ \emph{zeigt auf} eine Umgebung $\eta'$, geschrieben $\eta \leadsto \eta'$
%   bzw. $w \leadsto \eta'$, wenn sich dies mit den folgenden Regeln herleiten l"asst: \\[5mm]
%   \begin{tabular}{cccc}
%     $(\id:\underline{\quad};\eta) \leadsto \eta$
%     & $\regel{w \leadsto \eta'}{(\id:w;\eta) \leadsto \eta'}$ 
%     & $(\abstr{\id}{e},\eta) \leadsto \eta$
%     & $\regel{w_i \leadsto \eta \quad 1 \le i \le n}{(w_1,\ldots,w_n) \leadsto \eta}$ \\[3mm]
%   \end{tabular}
% \end{definition}

% \begin{lemma}
%   Wenn $\eta \leadsto \eta'$, dann $\eta' \sqsubset \eta$.
% \end{lemma}

\begin{definition}[Stack Disziplin]
  Eine Sprache erf"ullt die Stack-Disziplin, wenn f"ur alle Pfade
  Sei $(e_{n+1},\eta_{n+1}) \multimap \ldots \multimap (e_{1},\eta_{1})$  mit $n \ge 0$ gilt:
  \begin{itemize}
  \item \RN{Sd-1} Wenn $\eta \sqsubset \eta_1$, dann $\eta \in \{\eta_2,\ldots,\eta_{n+1}\}$
    oder $n\sqsubset\eta_{n+1}$.
  \item \RN{Sd-2} Wenn $(e_1,\eta_1) \Downarrow w$ und $\eta \sqsubseteq w$, dann ist
    $\eta \in \{\eta_2,\ldots,\eta_n\}$ oder $\eta \sqsubseteq \eta_{n+1}$.
  \end{itemize}
\end{definition}

\begin{lemma}
  Zum Nachweis der Stack-Disziplin gen"ugt es zu zeigen:
  \begin{itemize}
  \item \RN{Sd-1'} Wenn $(e_2,\eta_2) \multimap (e_1,\eta_1)$ und $\eta \sqsubset \eta_1$, dann
    $\eta \sqsubseteq \eta_2$.
  \item \RN{Sd-2'} Wenn $(e_2,\eta_2) \multimap (e_1,\eta_1) \Downarrow w$ und
    $\eta \sqsubseteq w$, dann $\eta \sqsubseteq \eta_2$.
  \end{itemize}
\end{lemma}

\begin{corollary} \label{corollary:SD-2}
  Wenn $(e_1,\eta_1) \Downarrow w$ und $\eta \sqsubseteq w$, dann $\eta \sqsubseteq \eta_1$.
\end{corollary}

\begin{proof}
  Hier muss man einen Trick anwenden, um \RN{Sd-2'} anwenden zu k"onnen, da \RN{Sd-2'} einen
  $\multimap$-Schritt verlangt. Man geht also von folgendem aus:
  $((\abstr{x}{x})\,e_1,\eta_1) \multimap (e_1,\eta_1) \Downarrow w$ und $\eta \sqsubseteq w$.
  Jetzt folgt die Aussage sofort mit \RN{Sd-2'}.
\end{proof}

\begin{theorem}
  Jede Sprache, die \RN{Sd-1'} und \RN{Sd-2'} gen"ugt, erf"ullt die Stack-Disziplin.
\end{theorem}

\begin{proof}
  \RN{Sd-1} und \RN{Sd-2} werden durch Induktion "uber $n$ bewiesen:
  \begin{itemize}
  \item \RN{Sd-1}
    \begin{itemize}
    \item $n = 0$

      Trivial.

    \item $n > 0$

      Nach I.V. ergibt sich f"ur $(e_n,\eta_n) \multimap \ldots \multimap (e_1,\eta_1)$, dass wenn
      $\eta \sqsubset \eta_1$, dann gilt $\eta \in \{\eta_2,\ldots,\eta_n\}$ oder $\eta \sqsubset \eta_n$.
      Jetzt ist \RN{Sd-1'} anwendbar, und es ergibt sich $\eta \in \{\eta_2,\ldots,\eta_n\}$ oder
      $\eta \sqsubseteq \eta_{n+1}$. Letzteres wiederum bedeutet $\eta \sqsubset \eta_{n+1}$ oder
      $\eta = \eta_{n+1}$. D.h. insgesamt erhalten wir $\eta \in \{\eta_2,\ldots,\eta_{n+1}\}$ oder
      $\eta \sqsubset \eta_{n+1}$, was zu zeigen war.
    \end{itemize}

  \item \RN{Sd-2}
    \begin{itemize}
    \item $n = 0$ 

      Folgt unmittelbar aus Korollar~\ref{corollary:SD-2}.

    \item $n = 1$

      Folgt unmittelbar aus \RN{Sd-2'}.

    \item $n > 1$

      Nach I.V. erhalten wir f"ur $(e_n,\eta_n) \multimap \ldots \multimap (e_1,\eta_1)$, dass wenn
      $\eta \sqsubseteq w$, dann $\eta \in \{\eta_2,\ldots,\eta_{n-1}\}$ oder $\eta \sqsubseteq \eta_n$.
      Daraus folgt unmittelbar, dass auch $\eta \in \{\eta_2,\ldots,\eta_n\}$ oder $\eta \sqsubset \eta_n$
      gilt, denn $n \ge 2$. Mit \RN{Sd-1'} folgt daraus unmittelbar, dass auch $\eta \in \{\eta_2,\ldots,\eta_n\}$
      oder $\eta \sqsubseteq \eta_{n+1}$ gilt, was zu zeigen war.
    \end{itemize}
  \end{itemize}
\end{proof}

\section{Zusammenhang}

\begin{lemma}
  Eine Sprache erf"ullt das Kriterium \RN{Sd-2}, wenn gilt:
  \begin{enumerate}
  \item Wenn $(e,\eta) \multimap (e',\eta')$ und $(e',\eta')$ wohlgetypt, aber nicht vom Basistyp, dann gilt
    $\eta = \eta'$.
  \item Wenn $(e',\eta') \Downarrow w$ und $\eta \sqsubseteq w$, dann $\eta \sqsubseteq \eta'$.
  \end{enumerate}
\end{lemma}

\begin{theorem}
  Die Sprache $\mathcal{L}_1$ mit pascal-artigem Typsystem erf"ullt die Stack-Disziplin.
\end{theorem}

\end{document}

% vi:set ts=2 sw=2 et:

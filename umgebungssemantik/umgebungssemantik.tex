\documentclass[12pt,fleqn]{article}
\usepackage{ngerman}
\usepackage{hyperref,german,amssymb,amstext,amsmath,amsthm,array,stmaryrd,color,latexsym}

% TP Macros
% General

\newcommand{\name}[1]{{\text{\it #1\/}}}
%\newcommand{\name}[1]{\mathit{#1}}

%\newcommand{\bfbox}[1]{\mathbf{#1}}

\newcommand{\I}{{\cal I}}

% Proof trees

\newcounter{tree}
\newcounter{node}[tree]

\newlength{\treeindent}
\newlength{\nodeindent}
\newlength{\nodesep}

\newcommand{\refnode}[1]
 {\ref{\thetree.#1}}

\newcommand{\contextcolor}{\color{blue}}

\definecolor{darkgreen}{rgb}{0.1,0.5,0}
\definecolor{redexcolor}{rgb}{0.7,1,0}

\newcommand{\resulttypecolor}{\color{darkgreen}}

\newcommand{\resultcolor}{\color{blue}}

\newcommand{\byrulecolor}{\color{red}}

\newcommand{\marked}[1]{\colorbox{redexcolor}{$#1$}}

\newcommand{\smallsteparrow}[1]{\stackrel{\mbox{\scriptsize\byrulecolor (#1)}}{\longrightarrow}}

\newcommand{\byrule}[1]{\hspace{-5mm}\byrulecolor\mbox{\scriptsize\ #1}}

\newif\ifarrows 
\arrowsfalse 

\newcommand{\arrow}[3]
  {\ifarrows
   \ncangle[angleA=-90,angleB=#1]{<-}{\thetree.#2}{\thetree.#3}
   \else
   \fi}

\newcommand{\node}[4]
 {\ifarrows
   \else \refstepcounter{node}
         \noindent\hspace{\treeindent}\hspace{#2\nodeindent}
         \rnode{\thetree.#1}{\makebox[6mm]{(\thenode)}}\label{\thetree.#1}
         $\blong 
          #3 \\ 
          \byrule{#4} 
          \elong$
         \vspace{\nodesep}
   \fi}

\newcommand{\dummyarrow}[3]
  {\arrow{#1}{#2}{#3dummy}}

\newcommand{\dummynode}[2]
 {\ifarrows 
  \else \noindent\hspace{\treeindent}\hspace{#2\nodeindent}
         \rnode{\thetree.#1dummy}{\makebox[6mm]{(\refnode{#1})}}\label{\thetree.#1dummy}
         $\ldots$
         \vspace{\nodesep}
   \fi}

\newcommand{\mktree}[1]
 {\stepcounter{tree} #1 \arrowstrue #1 \arrowsfalse}

\fboxrule=0mm

\newcommand{\cenv}[1]{\fbox{$\begin{array}{|ll|}\hline #1 \\\hline\end{array}$}}


% Special Symbols

\newcommand{\uminus}{\widetilde{\ }}
\renewcommand{\uminus}{-\!}
\newcommand{\nop}{()}

% Im X-Symbol-Manual empfohlen

\newcommand{\nsubset}{\not\subset}
%\newcommand{\textflorin}{\textit{f}}
\newcommand{\setB}{{\mathord{\mathbb B}}}
\newcommand{\setC}{{\mathord{\mathbb C}}}
\newcommand{\setN}{{\mathord{\mathbb N}}}
\newcommand{\setQ}{{\mathord{\mathbb Q}}}
\newcommand{\setR}{{\mathord{\mathbb R}}}
\newcommand{\setZ}{{\mathord{\mathbb Z}}}
\newcommand{\coloncolon}{\mathrel{::}}

% Eigene (k\"urzere) Befehle

\newcommand{\pfi}{\varphi}
\newcommand{\eps}{\varepsilon}
\newcommand{\eval}{\Downarrow}
\newcommand{\pto}{\hookrightarrow}
\newcommand{\emp}{\emptyset}
\newcommand{\sleq}{\subseteq}
\newcommand{\sgeq}{\supseteq}
\newcommand{\sqleq}{\sqsubseteq}
\newcommand{\sqgeq}{\sqsupseteq}
\newcommand{\lub}{\bigsqcup}
\newcommand{\glb}{\bigsqcap}
\newcommand{\lsem}{\llbracket}
\newcommand{\rsem}{\rrbracket}
\newcommand{\impl}{\models}
\newcommand{\step}{\vdash}
%\newcommand{\tr}{\triangleright}
\newcommand{\cc}{\coloncolon}


% Names (lower case italic)

\newcommand{\exn}{\name{exn}}
\newcommand{\id}{\name{id}}
\newcommand{\op}{\name{op}}

\newcommand{\true}{\name{true}}
\newcommand{\false}{\name{false}}
\newcommand{\Not}{\name{not}}

\newcommand{\sol}[3]{\name{solution}\,(#1,#2,#3)}
%\newcommand{\unify}{\name{unify}\,}

\newcommand{\Fst}{\name{fst}}
\newcommand{\Snd}{\name{snd}}

\newcommand{\Hd}{\name{hd}}
\newcommand{\Tl}{\name{tl}}
\newcommand{\Cons}{\name{cons}}
\newcommand{\Empty}{\name{is\_empty}}

\newcommand{\dom}[1]{\name{dom}(#1)}
\newcommand{\free}[1]{\name{free}\,(#1)}
\newcommand{\rank}[1]{\name{rank}\,(#1)}
\newcommand{\len}[1]{\name{len}\,(#1)}
\newcommand{\tr}[1]{\name{tr}(#1)}

% Names (upper case italic)

\newcommand{\Bool}{\name{Bool}}
\newcommand{\Btype}{\name{BType}}

\newcommand{\Conf}{\name{Conf}}
\newcommand{\Const}{\name{Const}}

\newcommand{\Dec}{\name{Dec}}

\newcommand{\Env}{\name{Env}}
\newcommand{\Exn}{\name{Exn}}
\newcommand{\EP}{\name{EP}}
\newcommand{\Exp}{\name{Exp}}

\newcommand{\Ncx}{\name{Ncx}}
\newcommand{\dbExp}{\name{dbExp}}
\newcommand{\dbVal}{\name{dbVal}}
\newcommand{\dbEnv}{\name{dbEnv}}
\newcommand{\dbCl}{\name{dbCl}}

\newcommand{\Id}{\name{Id}}
\newcommand{\Int}{\name{Int}}

\newcommand{\Lexp}{\name{LExp}}
\newcommand{\Loc}{\name{Loc}}

\newcommand{\Op}{\name{Op}}

\newcommand{\Type}{\name{Type}}
\newcommand{\Tvar}{\name{TVar}}
\newcommand{\teqns}[3]{\name{teqns}\,(#1,#2,#3)}
\newcommand{\tvar}[1]{\name{tvar}\,(#1)}
\newcommand{\unify}[1]{\name{unify}\,(#1)}

\newcommand{\Ptype}{\name{PType}}

\newcommand{\Unit}{\name{Unit}}

\newcommand{\Val}{\name{Val}}

% keywords

\newcommand{\z}{\mathbf{int}}
\newcommand{\bool}{\mathbf{bool}}
\newcommand{\unit}{\mathbf{unit}}
\newcommand{\blist}{\mathbf{list}}
\newcommand{\bref}{\mathbf{ref}}
\newcommand{\ltype}[1]{#1\,\blist}
\newcommand{\reftype}[1]{#1\,\bref}

\renewcommand{\div}{\mathbin{\mathbf{div}}}
\newcommand{\sel}{\mathbin{.}}
%\newcommand{\mod}{\mathbin{\mathbf{mod}}}
\renewcommand{\mod}{\text{mod}}

\newcommand{\bif}{\mathbf{if}}
\newcommand{\bthen}{\mathbf{then}}
\newcommand{\belse}{\mathbf{else}}

\newcommand{\blet}{\mathbf{let}}
\newcommand{\bin}{\mathbf{in}}
\newcommand{\bend}{\mathbf{end}}

\newcommand{\bval}{\mathbf{val}}
\newcommand{\brec}{\mathbf{rec}}
\newcommand{\bfix}{\mathbf{fix}}
\newcommand{\bfun}{\mathbf{fun}}
\newcommand{\band}{\mathbf{and}}
\newcommand{\btype}{\mathbf{type}}

%\newcommand{\andalso}[2]{#1\,\&\&\,#2}
%\newcommand{\orelse}[2]{#1\,\|\,#2}
\newcommand{\andalso}{\&\&}
\newcommand{\orelse}{\|}
\newcommand{\bandalso}{\mathbf{andalso}}
\newcommand{\borelse}{\mathbf{orelse}}

\newcommand{\bwhile}{\mathbf{while}}
\newcommand{\bdo}{\mathbf{do}}
\newcommand{\bfor}{\mathbf{for}}

\newcommand{\brepeat}{\mathbf{repeat}}
\newcommand{\buntil}{\mathbf{until}}

\newcommand{\barray}{\mathbf{array}}
\newcommand{\bof}{\mathbf{of}}

\newcommand{\bclass}{\mathbf{class}}

\newcommand{\app}[2]{#1\,#2}
\newcommand{\bift}[2]{\bif\ #1\ \bthen\ #2}
\newcommand{\bifte}[3]{\bif\ #1\ \bthen\ #2\ \belse\ #3}
\newcommand{\Bifte}[3]{\blong\bif\ #1\\\bthen\ #2\\\belse\ #3\elong}
\newcommand{\bwd}[2]{\bwhile\ #1\ \bdo\ #2}
\newcommand{\bru}[2]{\brepeat\ #1\ \buntil\ #2}


\newcommand{\blie}[2]{\blet\ #1\ \bin\ #2 \ \bend}
\newcommand{\vdec}[2]{#1 = #2}
\newcommand{\rec}[2]{\brec\ #1.\ #2}
\newcommand{\recdots}[2]{\brec\ #1.\ \ldots}
\newcommand{\abstr}[2]{\lambda #1.\,#2}
\newcommand{\appl}[2]{#1\,#2}

\newcommand{\Ref}{\name{ref}}
\newcommand{\Deref}{\,!\,}

\newcommand{\alloc}{\name{alloc}}
\newcommand{\Store}{\name{Store}}




% Typing rules 

\newcommand{\tj}[2]{#1\cc#2}
\newcommand{\ctj}[2]{\tj{#1}{{\resulttypecolor #2}}}
\newcommand{\Tj}[3]{#1 \, \triangleright \, #2\cc#3}
\newcommand{\cTj}[3]{\Tj{{\contextcolor #1}}{#2}{{\resulttypecolor #3}}}
\newcommand{\cbig}[2]{#1 \ \eval\ {\resultcolor #2}}
\newcommand{\cBig}[2]{#1 \\ \eval\ {\resultcolor #2}}
\newcommand{\Clos}[2]{\name{Closure}_{#1}(#2)}

\newcommand{\Tjl}[3]{#1 \,\triangleright_l\, #2\cc#3}
\newcommand{\Tjm}[3]{#1 \,\triangleright_m\, #2\cc#3}
\newcommand{\Tjh}[2]{#1 \triangleright  #2}

\newcommand{\brule}[1]{\begin{markiere}[#1]}
\newcommand{\erule}{\end{markiere}}

\newcommand{\regel}[2]{\ \begin{array}{@{}c@{}} #1 \\ \hline #2
 \end{array}\ }

\newcommand{\reason}[1]{\ \mbox{#1}}
\newcommand{\Reason}[1]{\vspace{1mm}\\ \mbox{ #1}}


% Program verification

\newcommand{\conj}{\,\land\,}
\newcommand{\Conj}{\bigwedge}
\newcommand{\disj}{\,\lor\,}
\newcommand{\Disj}{\bigvee\,}
\newcommand{\all}[1]{\forall{#1}.\,}
\newcommand{\ex}[1]{\exists{#1}.\,}

\newcommand{\power}[1]{\wp(#1)}

\newcommand{\disjoint}[2]{\name{disj}(#1,#2)}
\newcommand{\cont}[2]{#1 \mapsto #2}
\newcommand{\DEF}{\name{DEF}}

\newcommand{\ret}[2]{{\bf returns}\ #1.\, #2}
\newcommand{\tc}[2]{#1\,\{#2\}}
\newcommand{\triple}[3]{\{#1\}\,#2\,\{#3\}}

% Index

\newcommand{\define}[1]{{\em #1\/}\index{#1}}
\newcommand{\Define}[2]{{\em #1\/}\index{#2}}
\newcommand{\Index}[1]{\index{#1}}
\newcommand{\notation}[1]{#1\index{#1}}
\newcommand{\engl}[1]{(engl.: \define{#1})}
\newcommand{\Engl}[2]{(engl.: \Define{#1}{#2})}

% Theorems etc.

\newtheorem{theorem}{Satz}
\newtheorem{corollary}{Korollar}
\newtheorem{definition}{Definition:}
%\newtheorem{example}{Beispiel:}
%\newtheorem{examples}{Beispiele:}
\newtheorem{lemma}[theorem]{Lemma}

%\renewcommand{\thedefinition}{}
%\renewcommand{\theexample}{}
%\renewcommand{\theexamples}{}
\renewcommand{\theenumi}{\rm (\alph{enumi})}
\renewcommand{\labelenumi}{\theenumi}

%\newcommand{\enumarabic}{\renewcommand{\theenumi}{\rm (\arabic{enumi})}}

%\newcommand{\bcoro}[1]{\begin{corollary}\label{cor:#1}}
%\newcommand{\ecoro}{\end{corollary}}

\newcommand{\brdef}[1]{\begin{definition}\label{def:#1}\rm}
\newcommand{\erdef}{\end{definition}}

%\newcommand{\blemm}[1]{\begin{lemma}\label{lem:#1}}
%\newcommand{\bLemm}[2]{\begin{lemma}[#2]\label{lem:#1}\index{#2}}
%\newcommand{\elemm}{\end{lemma}}

%\newcommand{\btheo}[1]{\begin{theorem}\label{th:#1}}
%\newcommand{\bTheo}[2]{\begin{theorem}[#2]\label{th:#1}\index{#2}}
%\newcommand{\etheo}{\end{theorem}}

%\newcommand{\litem}[1]{\item\label{it:#1}}
%\newcommand{\ritem}[1]{\ref{it:#1}}


% Grammars

\newcommand{\bgram}{\[\begin{array}{rrlll}}
\newcommand{\egram}{\end{array}\]}

\newcommand{\is}{& ::= &}
\newcommand{\al}{\\ & \mid &}
\newcommand{\n}{\vspace{2mm}\\}




% Other Environments

\newcommand{\bcase}{\left\{\!\!\!\begin{array}{ll}}
\newcommand{\ecase}{\end{array}\right.}
\newcommand{\benum}{\begin{enumerate}}
\newcommand{\eenum}{\end{enumerate}}
\newcommand{\beqns}{\[\begin{array}{rcll}}
\newcommand{\eeqns}{\end{array}\]}
\newcommand{\bitem}{\begin{itemize}}
\newcommand{\eitem}{\end{itemize}}
\newcommand{\blong}{\!\!\begin{array}[t]{l}}
\newcommand{\elong}{\end{array}}
\newcommand{\btabl}{\begin{tabular}}
\newcommand{\etabl}{\end{tabular}}
\newcommand{\brexa}{\begin{example}\enumarabic\rm}
\newcommand{\erexa}{\end{example}}
\newcommand{\brexs}{\begin{examples}\enumarabic\rm}
\newcommand{\erexs}{\end{examples}}


% German abbreviations

\newcommand{\abk}[1]{#1.\ }
\newcommand{\bzw}{\abk{bzw}}
\newcommand{\bzgl}{\abk{bzgl}}
\newcommand{\das}{\abk{d.h}}
\newcommand{\evtl}{\abk{evtl}}
\newcommand{\usw}{\abk{usw}}
\newcommand{\vgl}{\abk{vgl}}
\newcommand{\zb}{\abk{z.B}}


\newcommand{\infix}[3]{#2\mathbin{#1}#3}

\newcommand{\bli}[3]{\blet\ \vdec{#1}{#2}\ \bin\ #3}

\newcommand{\Bli}[3]{\begin{array}[t]{@{}l}
                     \blet\ \vdec{#1}{#2}\\\bin\ #3
                     \end{array}}

\newcommand{\Vdec}[2]{\begin{array}[t]{@{}l}
                      #1 = \\
                      \ #2
                      \end{array}}

\newcommand{\BLI}[3]{\begin{array}[t]{@{}l}
                     \blet\ \Vdec{#1}{#2}\\\bin\ #3
                     \end{array}}

\newcommand{\Abstr}[2]{\begin{array}[t]{@{}l}
                       \lambda #1.\\\ \ #2
                       \end{array}}


\newcommand{\RN}[1]{\mbox{\textsc{(#1)}}}
\newcommand{\Cl}{\name{Cl}}
\newcommand{\cl}{\name{cl}}

\begin{document}

\section{Substitutionssemantik}

Voraussetzungen:
\begin{itemize}
  \item $\Id \cap \Val = \emptyset$

  \item Ausdr"ucke sind gleich modulo Umbenennung gebundener Namen.

        Geht nat"urlich auch ohne diese Konvention, aber macht Definitionen und Beweise h"asslich.
\end{itemize}

\begin{definition}[Big step Regeln]
Ein {\em big step} in der Substitutionssemantik ist eine Formel der Gestalt $e \Downarrow v$ mit $e\in\Exp$
und $v \in \Val$. Ein solcher big step hei"st {\em g"ultig}, wenn er sich mit den folgenden Regeln herleiten
l"asst: \\[5mm]
\begin{tabular}{ll}
  \RN{Val}        & $v \Downarrow v$ \\[3mm]
  \RN{Beta}       & $\regel{e_1 \Downarrow \abstr{\id}{e} \quad e[e_2/\id] \Downarrow v}
                           {\app{e_1}{e_2} \Downarrow v}$ \\[3mm]
  \RN{Op-1}       & $\regel{e_1 \Downarrow \op \quad e_2 \Downarrow n}
                           {\app{e_1}{e_2} \Downarrow \app{\op}{n}}$ \\[3mm]
  \RN{Op-2}       & $\regel{e_1 \Downarrow \app{\op}{n_1} \quad e_2 \Downarrow n_2 \quad \op^I(n_1,n_2) = n}
                           {\app{e_1}{e_2} \Downarrow n}$ \\[3mm]
  \RN{Cond-True}  & $\regel{e_0 \Downarrow \true \quad e_1 \Downarrow v}
                           {\bifte{e_0}{e_1}{e_2} \Downarrow v}$ \\[3mm]
  \RN{Cond-False} & $\regel{e_0 \Downarrow \false \quad e_2 \Downarrow v}
                           {\bifte{e_0}{e_1}{e_2} \Downarrow v}$ \\[3mm]
  \RN{Unfold}     & $\regel{e[\rec{\id}{e}/\id] \Downarrow v}
                           {\rec{\id}{e} \Downarrow v}$
\end{tabular}
\end{definition}


\section{Umgebungssemantik}

\begin{definition}[Closures und Umgebungen]
  Die Mengen $\Env$ aller {\em Umgebungen} und $\Cl$ aller {\em Closures} sind wie folgt definiert:
  \begin{enumerate}
    \item $\Env = \{\eta:\Id \pto \Cl\ |\ \dom{\eta} \text{ endlich}\}$
    \item $\Cl = \{(e,\eta)\in\Exp \times \Env\ |\ \free{e} \subseteq \dom{\eta}\}$
  \end{enumerate}
\end{definition}

\begin{definition}[Big step Regeln]
Ein {\em big step} in der Umgebungssemantik ist eine Formel der Gestalt $(e,\eta) \Downarrow (e',\eta')$,
wobei $e,e'\in\Exp$ und $\eta,\eta'\in\Env$. Ein derartiger big step hei"st {\em g"ultig}, wenn er sich mit den
folgenden Regeln herleiten l"asst: \\[5mm]
\begin{tabular}{ll}
  \RN{Val}        & $(v,\eta) \Downarrow (v,\eta)$ \\[3mm]
  \RN{Id}         & $\regel{\eta(\id) \Downarrow \cl}
                           {(\id,\eta) \Downarrow \cl}$ \\[3mm]
  \RN{Beta}       & $\regel{(e_1,\eta) \Downarrow (\abstr{\id}{e},\eta_1)
                            \quad (e,\eta_1[(e_2,\eta)/\id]) \Downarrow \cl}
                           {(\app{e_1}{e_2},\eta) \Downarrow \cl}$ \\[3mm]
  \RN{Op-1}       & $\regel{(e_1,\eta) \Downarrow (\op,\eta_1) \quad (e_2,\eta) \Downarrow (n,\eta_2)}
                           {(\app{e_1}{e_2},\eta) \Downarrow (\app{\op}{n},[])}$ \\[3mm]
  \RN{Op-2}       & $\regel{(e_1,\eta) \Downarrow (\app{\op}{n_1},\eta_1)
                            \quad (e_2,\eta) \Downarrow (n_2,\eta_2)
                            \quad \op^I(n_1,n_2) = n}
                           {(\app{e_1}{e_2},\eta) \Downarrow (n,[])}$ \\[3mm]
  \RN{Cond-True}  & $\regel{(e_0,\eta) \Downarrow (\true,\eta_0) \quad (e_1,\eta) \Downarrow \cl}
                           {(\bifte{e_0}{e_1}{e_2},\eta) \Downarrow \cl}$ \\[3mm]
  \RN{Cond-False} & $\regel{(e_0,\eta) \Downarrow (\false,\eta_0) \quad (e_2,\eta) \Downarrow \cl}
                           {(\bifte{e_0}{e_1}{e_2},\eta) \Downarrow \cl}$ \\[3mm]
  \RN{Unfold}     & $\regel{(e,\eta[(\rec{\id}{e},\eta)/\id]) \Downarrow \cl}
                           {(\rec{\id}{e},\eta) \Downarrow \cl}$
\end{tabular}
\end{definition}

\begin{theorem}[Wohldefiniertheit der Umgebungssemantik] \label{satz:Wohldefiniertheit}
  Seien $(e,\eta)\in \Cl$, $\hat{e} \in \Exp$ und $\hat{\eta}\in\Env$.
  Wenn $(e,\eta) \Downarrow (\hat{e},\hat{\eta})$, dann gilt $\hat{e}\in \Val$ und $(\hat{e},\hat{\eta}) \in \Cl$.
\end{theorem}

\begin{proof}
  Triviale Induktion.
\end{proof}

\begin{definition} \label{definition:Substitution}
  Sei $(e,\eta) \in \Cl$. Der Ausdruck $e\,\eta$ ist wie folgt definiert:
  \[\begin{array}{rcl}
    c\,\eta &=& c \\
    \id\,\eta &=& e'\,\eta', \text{ wobei } \eta(\id) = (e',\eta') \\
    (\app{e_1}{e_2})\,\eta &=& \app{(e_1\,\eta)}{(e_2\,\eta)} \\
    (\abstr{\id}{e})\,\eta &=& \abstr{\id'}{(e[\id'/\id]\,\eta)}, \text{ wobei } \id'\not\in\dom{\eta} \\
    (\rec{\id}{e})\,\eta &=& \rec{\id'}{(e[\id'/\id]\,\eta)}, \text{ wobei } \id'\not\in\dom{\eta} \\
    (\bifte{e_0}{e_1}{e_2})\,\eta &=& \bifte{(e_0\,\eta)}{(e_1\,\eta)}{(e_2\,\eta)}
  \end{array}\]
\end{definition}

\begin{lemma} \label{lemma:Substitution}
  Seien $(e,\eta), (e',\eta') \in \Cl$. Dann gilt:
  \begin{enumerate}
    \item $e\in\Val$ dann $(e\,\eta)\in\Val$
    \item $(e\,\eta)\in\Val$, dann entweder $e\in\Val$ oder
          $e=\id\in\Id$ und $\eta(\id) = (e',\eta')$ mit $(e'\,\eta')\in\Val$
    %\item $(e\,\eta) = (e\,\eta')$ wenn $\eta' =_{\free{e}} \eta$
    \item $e\,(\eta[(e',\eta')/\id]) = (e[\id'/\id]\,\eta)[(e'\,\eta')/\id']$ wenn $\id' \not\in \dom{\eta}$
  \end{enumerate}
\end{lemma}


\section{"Aquivalenz der Modelle}

\begin{theorem}[Korrektheit der Umgebungssemantik] \label{theorem:Korrektheit}
  Seien $(e,\eta) \in \Cl$, $v'\in\Val$ und $\eta'\in\Env$. Wenn $(e,\eta) \Downarrow (v',\eta')$, dann
  $(e\,\eta) \Downarrow (v'\,\eta')$.
\end{theorem}

\begin{proof}
  Induktion "uber die L"ange der Herleitung des big steps $(e,\eta) \Downarrow (v',\eta')$ mit Fallunterscheidung
  nach der zuletzt angewandten big step Regel:
  \begin{itemize}
    \item $(v,\eta) \Downarrow (v,\eta)$ mit \RN{Val}, dann ist $(v\,\eta) \in \Val$ und somit folgt
          $(v\,\eta) \Downarrow (v\,\eta)$ mit \RN{Val}.

    \item $(\id,\eta) \Downarrow (v',\eta')$ mit \RN{Id} bedingt $\eta(\id) = (\hat{e},\hat{\eta})$ und
          $(\hat{e},\hat{\eta}) \Downarrow (v',\eta')$. Nach IV muss also $(\hat{e}\,\hat{\eta})\Downarrow(v'\,\eta')$
          gelten, und somit folgt wegen $(\id\,\eta) = (\hat{e}\,\hat{\eta})$ die Behauptung.

    \item $(\app{e_1}{e_2},\eta) \Downarrow (v',\eta')$ mit Regel \RN{Beta} kann nur aus Pr"amissen der Form
          $(e_1,\eta) \Downarrow (\abstr{\id}{e},\hat{\eta})$ und $(e,\hat{\eta}[(e_2,\eta)/\id])\Downarrow(v',\eta')$
          folgen. Mit IV folgt daraus $(e_1\,\eta) \Downarrow (\abstr{\id}{e})\,\hat{\eta}$ und
          $e\,(\hat{\eta}[(e_2,\eta)/\id])\Downarrow(v'\,\eta')$. Nach Definition~\ref{definition:Substitution}
          gilt $(\abstr{\id}{e})\,\hat{\eta} = \abstr{\id'}{(e[\id'/\id]\,\hat{\eta})}$ wobei
          $\id'\not\in\dom{\hat{\eta}}$. Nach Lemma~\ref{lemma:Substitution} folgt dar"uberhinaus
          $e\,(\hat{\eta}[(e_2,\eta)/\id]) = (e[\id'/\id]\,\hat{\eta})[e_2\,\eta/\id]$, und somit
          $(\app{e_1}{e_2})\,\eta \Downarrow (v'\,\eta')$ mit Regel \RN{Beta}.

    \item $(\rec{\id}{e},\eta) \Downarrow (v',\eta')$ kann ausschliesslich mit \RN{Unfold} aus der Pr"amisse
          $(e,\eta[(\rec{\id}{e},\eta)/\id]) \Downarrow (v',\eta')$ folgen. Nach IV gilt dann ebenfalls
          $e\,(\eta[(\rec{\id}{e},\eta)/\id]) \Downarrow (v'\,\eta')$ und nach Definition~\ref{definition:Substitution}
          gilt $(\rec{\id}{e})\,\eta = \rec{\id'}{(e[\id'/\id]\,\eta)}$ mit $\id'\not\in\dom{\eta}$. Wegen
          Lemma~\ref{lemma:Substitution} folgt dann
          $e\,(\eta[(\rec{\id}{e},\eta)/\id]) = (e[\id'/\id]\,\eta)[\rec{\id'}{(e[\id'/\id]\,\eta)}/\id]$ und
          somit $(\rec{\id}{e})\,\eta \Downarrow (v'\,\eta')$ mit Regel \RN{Unfold}.
  \end{itemize}
  Die "ubrigen F"alle folgen analog.
\end{proof}

\begin{lemma}[Koinzidenz] \label{lemma:Koinzidenz}
  F"ur alle $\eta,\hat{\eta},\eta'\in\Env$, ...
  Wenn $(e[(e_2\,\eta)/\id],[\,]) \Downarrow (v',\eta')$, dann ex. $\eta''$, so dass
  $(e,\hat{\eta}[(e_2,\eta)/\id]) \Downarrow (v',\eta'')$.
\end{lemma}

\begin{proof}
  Sollte gelten, m"usste aber allgemeiner formuliert werden, um einfach beweisbar zu sein.
\end{proof}

\begin{theorem}[Vollst"andigkeit der Umgebungssemantik] \label{theorem:Vollstaendigkeit}
  Seien $(e,\eta)\in\Cl$ und $v \in \Val$.
  Wenn $(e\,\eta) \Downarrow v$, dann ex. $(v',\eta')\in\Cl$ mit $(v'\,\eta') = v$, so dass
  $(e,\eta) \Downarrow (v',\eta')$.
\end{theorem}

Man k"onnte hier vermutlich auch noch eine konkretere Aussage "uber $v'$ und $\eta'$ machen, so man das will.

\begin{proof}
  Offensichtlich gilt $\free{v} = \emptyset$. Die Behauptung folgt
  dann per Induktion "uber die L"ange der Herleitung des big steps $(e\,\eta) \Downarrow v$, wobei jeweils nach der
  zuletzt angewandten Regel unterschieden wird. Hierbei ist in jedem der F"alle zu unterscheiden, ob $e$ selbst
  schon von der f"ur die Regel passenden Form ist, oder ob $e$ ein Identifier ist. Ist $e$ ein Identifier, so
  folgt die Behauptung unmittelbar mit IV und big step Regel \RN{Id}; wir betrachten diesen Fall damit als erledigt
  und konzentrieren uns im Folgenden auf die F"alle f"ur $e\not\in\Id$.
  \begin{itemize}
    \item $v \Downarrow v$ mit \RN{Val}, d.h. $(e\,\eta) = v$. Sei o.B.d.A. $e \in \Val$, dann existiert ein
          big step $(e,\eta) \Downarrow (e,\eta)$ mit \RN{Val}.

    \item $(\rec{\id}{e})\,\eta \Downarrow v$ mit \RN{Unfold} bedingt o.B.d.A.
          $(e[\id'/\id]\,\eta)[(\rec{\id}{e})\,\eta/\id'] \Downarrow v$, wobei $\id' \not\in \dom{\eta}$.
          Nach Lemma~\ref{lemma:Substitution} folgt $e\,(\eta[(\rec{\id}{e})\,\eta/\id])\Downarrow v$ und nach IV
          existiert dann $(v',\eta') \in \Cl$ mit $(v'\,\eta') = v$, so dass
          $(e,\eta[(\rec{\id}{e},\eta)/\id]) \Downarrow (v',\eta')$. Die Behauptung folgt unmittelbar mit
          \RN{Unfold}.

    \item $(\app{e_1}{e_2})\,\eta \Downarrow v$ mit \RN{Beta} kann o.B.d.A. nur aus Pr"amissen der Form
          $(e_1\,\eta) \Downarrow \abstr{\id}{e}$ und $e[(e_2\,\eta)/\id]\Downarrow v$ folgen. Nach IV ex.
          $(\hat{v},\hat{\eta}),(v',\eta')\in\Cl$, so dass $(e_1,\eta) \Downarrow (\hat{v},\hat{\eta})$
          und $(e[(e_2\,\eta)/\id],[\,])\Downarrow (v',\eta')$. Nach Lemma~\ref{lemma:Koinzidenz} ex. $\eta''$
          mit $(e,\hat{\eta}[(e_2,\eta)/\id]) \Downarrow (v',\eta'')$. O.B.d.A. ex. $\hat{\id}\not\in\dom{\hat{\eta}}$,
          so dass $(e_1,\eta) \Downarrow (\abstr{\hat{\id}}{e[\hat{\id}/\id]},\hat{\eta})$ und
          $(e[\hat{\id}/\id],\hat{\eta}[(e_2,\eta)/\hat{\id}]) \Downarrow (v',\eta'')$. Die Behauptung folgt
          unmittelbar mit \RN{Beta}.
  \end{itemize}
  Die restlichen F"alle folgen "ahnlich einfach.
\end{proof}

\begin{theorem}["Aquivalenzsatz]
  Sei $e \in \Exp$, $v \in \Val$ mit $\free{e} = \emptyset$. Die folgenden Aussagen sind "aquivalent:
  \begin{enumerate}
    \item $e \Downarrow v$
    \item ex. $(v',\eta')\in\Cl$ mit $v = (v'\,\eta')$, so dass $(e,[\,]) \Downarrow (v',\eta')$
  \end{enumerate}
\end{theorem}

\begin{proof}
  Folgt leicht aus Satz~\ref{theorem:Korrektheit} und Satz~\ref{theorem:Vollstaendigkeit}.
\end{proof}

\begin{corollary}
  Sei $e \in \Exp$, $c \in \Const$ mit $\free{e} = \emptyset$. Die folgenden Aussagen sind "aquivalent:
  \begin{enumerate}
    \item $e \Downarrow c$
    \item ex. $\eta\in\Cl$, so dass $(e,[\,]) \Downarrow (c,\eta)$
  \end{enumerate}
\end{corollary}

\end{document}

% vi:set ts=2 sw=2 et:

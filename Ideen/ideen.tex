\documentclass[12pt,a4paper,bigheadings]{scrartcl}

\usepackage{amssymb}
\usepackage{amstext}
\usepackage{amsmath}
\usepackage{array}
\usepackage[ngerman]{babel}
\usepackage{color}
\usepackage{enumerate}
%\usepackage[T1]{fontenc}
\usepackage{german}
\usepackage[a4paper,%
            colorlinks=false,%
            final,%
            pdfkeywords={},%
            pdftitle={},%
            pdfauthor={Benedikt Meurer},%
            pdfsubject={},%
            pdfdisplaydoctitle=true]{hyperref}
\usepackage[latin1]{inputenc}
\usepackage{latexsym}
\usepackage[final]{listings}
\usepackage{makeidx}
%\usepackage{mathpartir}
\usepackage{ngerman}
%\usepackage[standard,thmmarks]{ntheorem}
\usepackage{scrpage2}
\usepackage{stmaryrd}
%\usepackage[DIV13,BCOR5mm]{typearea}
\usepackage{url}
\usepackage[all]{xy}

%% TP-Makros
% General

\newcommand{\name}[1]{{\text{\it #1\/}}}
%\newcommand{\name}[1]{\mathit{#1}}

%\newcommand{\bfbox}[1]{\mathbf{#1}}

\newcommand{\I}{{\cal I}}

% Proof trees

\newcounter{tree}
\newcounter{node}[tree]

\newlength{\treeindent}
\newlength{\nodeindent}
\newlength{\nodesep}

\newcommand{\refnode}[1]
 {\ref{\thetree.#1}}

\newcommand{\contextcolor}{\color{blue}}

\definecolor{darkgreen}{rgb}{0.1,0.5,0}
\definecolor{redexcolor}{rgb}{0.7,1,0}

\newcommand{\resulttypecolor}{\color{darkgreen}}

\newcommand{\resultcolor}{\color{blue}}

\newcommand{\byrulecolor}{\color{red}}

\newcommand{\marked}[1]{\colorbox{redexcolor}{$#1$}}

\newcommand{\smallsteparrow}[1]{\stackrel{\mbox{\scriptsize\byrulecolor (#1)}}{\longrightarrow}}

\newcommand{\byrule}[1]{\hspace{-5mm}\byrulecolor\mbox{\scriptsize\ #1}}

\newif\ifarrows 
\arrowsfalse 

\newcommand{\arrow}[3]
  {\ifarrows
   \ncangle[angleA=-90,angleB=#1]{<-}{\thetree.#2}{\thetree.#3}
   \else
   \fi}

\newcommand{\node}[4]
 {\ifarrows
   \else \refstepcounter{node}
         \noindent\hspace{\treeindent}\hspace{#2\nodeindent}
         \rnode{\thetree.#1}{\makebox[6mm]{(\thenode)}}\label{\thetree.#1}
         $\blong 
          #3 \\ 
          \byrule{#4} 
          \elong$
         \vspace{\nodesep}
   \fi}

\newcommand{\dummyarrow}[3]
  {\arrow{#1}{#2}{#3dummy}}

\newcommand{\dummynode}[2]
 {\ifarrows 
  \else \noindent\hspace{\treeindent}\hspace{#2\nodeindent}
         \rnode{\thetree.#1dummy}{\makebox[6mm]{(\refnode{#1})}}\label{\thetree.#1dummy}
         $\ldots$
         \vspace{\nodesep}
   \fi}

\newcommand{\mktree}[1]
 {\stepcounter{tree} #1 \arrowstrue #1 \arrowsfalse}

\fboxrule=0mm

\newcommand{\cenv}[1]{\fbox{$\begin{array}{|ll|}\hline #1 \\\hline\end{array}$}}


% Special Symbols

\newcommand{\uminus}{\widetilde{\ }}
\renewcommand{\uminus}{-\!}
\newcommand{\nop}{()}

% Im X-Symbol-Manual empfohlen

\newcommand{\nsubset}{\not\subset}
%\newcommand{\textflorin}{\textit{f}}
\newcommand{\setB}{{\mathord{\mathbb B}}}
\newcommand{\setC}{{\mathord{\mathbb C}}}
\newcommand{\setN}{{\mathord{\mathbb N}}}
\newcommand{\setQ}{{\mathord{\mathbb Q}}}
\newcommand{\setR}{{\mathord{\mathbb R}}}
\newcommand{\setZ}{{\mathord{\mathbb Z}}}
\newcommand{\coloncolon}{\mathrel{::}}

% Eigene (k\"urzere) Befehle

\newcommand{\pfi}{\varphi}
\newcommand{\eps}{\varepsilon}
\newcommand{\eval}{\Downarrow}
\newcommand{\pto}{\hookrightarrow}
\newcommand{\emp}{\emptyset}
\newcommand{\sleq}{\subseteq}
\newcommand{\sgeq}{\supseteq}
\newcommand{\sqleq}{\sqsubseteq}
\newcommand{\sqgeq}{\sqsupseteq}
\newcommand{\lub}{\bigsqcup}
\newcommand{\glb}{\bigsqcap}
\newcommand{\lsem}{\llbracket}
\newcommand{\rsem}{\rrbracket}
\newcommand{\sem}[1]{\lsem #1 \rsem}
\newcommand{\impl}{\models}
\newcommand{\step}{\vdash}
%\newcommand{\tr}{\triangleright}
\newcommand{\cc}{\coloncolon}


% Names (lower case italic)

\newcommand{\exn}{\name{exn}}
\newcommand{\id}{\name{id}}
\newcommand{\op}{\name{op}}

\newcommand{\true}{\name{true}}
\newcommand{\false}{\name{false}}
\newcommand{\Not}{\name{not}}

\newcommand{\sol}[3]{\name{solution}\,(#1,#2,#3)}
%\newcommand{\unify}{\name{unify}\,}

\newcommand{\Fst}{\name{fst}}
\newcommand{\Snd}{\name{snd}}

\newcommand{\Hd}{\name{hd}}
\newcommand{\Tl}{\name{tl}}
\newcommand{\Cons}{\name{cons}}
\newcommand{\Empty}{\name{is\_empty}}

\newcommand{\dom}[1]{\name{dom}(#1)}
\newcommand{\free}[1]{\name{free}\,(#1)}
\newcommand{\rank}[1]{\name{rank}\,(#1)}
\newcommand{\len}[1]{\name{len}\,(#1)}
\newcommand{\tr}[1]{\name{tr}(#1)}
\newcommand{\trdB}[1]{\name{tr}_{dB}(#1)}
\newcommand{\locns}[1]{\name{locns}\,(#1)}

% Names (upper case italic)

\newcommand{\Bool}{\name{Bool}}
\newcommand{\Btype}{\name{BType}}

\newcommand{\Conf}{\name{Conf}}
\newcommand{\Const}{\name{Const}}

\newcommand{\Dec}{\name{Dec}}

\newcommand{\Env}{\name{Env}}
\newcommand{\Exn}{\name{Exn}}
\newcommand{\EP}{\name{EP}}
\newcommand{\Exp}{\name{Exp}}

\newcommand{\Ncx}{\name{Ncx}}
\newcommand{\dbExp}{\name{dbExp}}
\newcommand{\dbVal}{\name{dbVal}}
\newcommand{\dbEnv}{\name{dbEnv}}
\newcommand{\dbCl}{\name{dbCl}}

\newcommand{\Id}{\name{Id}}
\newcommand{\Int}{\name{Int}}

\newcommand{\Lexp}{\name{LExp}}
\newcommand{\Loc}{\name{Loc}}

\newcommand{\Prog}{\name{Prog}}
\newcommand{\Instr}{\name{Instr}}
\newcommand{\Reg}{\name{Reg}}
\newcommand{\Stack}{\name{Stack}}
\newcommand{\State}{\name{State}}

\newcommand{\Op}{\name{Op}}

\newcommand{\Type}{\name{Type}}
\newcommand{\Tvar}{\name{TVar}}
\newcommand{\teqns}[3]{\name{teqns}\,(#1,#2,#3)}
\newcommand{\tvar}[1]{\name{tvar}\,(#1)}
\newcommand{\unify}[1]{\name{unify}\,(#1)}

\newcommand{\Ptype}{\name{PType}}

\newcommand{\Unit}{\name{Unit}}

\newcommand{\Val}{\name{Val}}

% keywords

\newcommand{\z}{\mathbf{int}}
\newcommand{\bool}{\mathbf{bool}}
\newcommand{\unit}{\mathbf{unit}}
\newcommand{\blist}{\mathbf{list}}
\newcommand{\bref}{\mathbf{ref}}
\newcommand{\ltype}[1]{#1\,\blist}
\newcommand{\reftype}[1]{#1\,\bref}

\renewcommand{\div}{\mathbin{\mathbf{div}}}
\newcommand{\sel}{\mathbin{.}}
%\newcommand{\mod}{\mathbin{\mathbf{mod}}}
\renewcommand{\mod}{\text{mod}}

\newcommand{\bif}{\mathbf{if}}
\newcommand{\bthen}{\mathbf{then}}
\newcommand{\belse}{\mathbf{else}}

\newcommand{\blet}{\mathbf{let}}
\newcommand{\bin}{\mathbf{in}}
\newcommand{\bend}{\mathbf{end}}

\newcommand{\bval}{\mathbf{val}}
\newcommand{\brec}{\mathbf{rec}}
\newcommand{\bfix}{\mathbf{fix}}
\newcommand{\bfun}{\mathbf{fun}}
\newcommand{\band}{\mathbf{and}}
\newcommand{\btype}{\mathbf{type}}

%\newcommand{\andalso}[2]{#1\,\&\&\,#2}
%\newcommand{\orelse}[2]{#1\,\|\,#2}
\newcommand{\andalso}{\&\&}
\newcommand{\orelse}{\|}
\newcommand{\bandalso}{\mathbf{andalso}}
\newcommand{\borelse}{\mathbf{orelse}}

\newcommand{\bwhile}{\mathbf{while}}
\newcommand{\bdo}{\mathbf{do}}
\newcommand{\bfor}{\mathbf{for}}

\newcommand{\brepeat}{\mathbf{repeat}}
\newcommand{\buntil}{\mathbf{until}}

\newcommand{\barray}{\mathbf{array}}
\newcommand{\bof}{\mathbf{of}}

\newcommand{\bclass}{\mathbf{class}}

\newcommand{\app}[2]{#1\,#2}
\newcommand{\bift}[2]{\bif\ #1\ \bthen\ #2}
\newcommand{\bifte}[3]{\bif\ #1\ \bthen\ #2\ \belse\ #3}
\newcommand{\Bifte}[3]{\blong\bif\ #1\\\bthen\ #2\\\belse\ #3\elong}
\newcommand{\bwd}[2]{\bwhile\ #1\ \bdo\ #2}
\newcommand{\bru}[2]{\brepeat\ #1\ \buntil\ #2}


\newcommand{\blie}[2]{\blet\ #1\ \bin\ #2 \ \bend}
\newcommand{\vdec}[2]{#1 = #2}
\newcommand{\rec}[2]{\brec\,#1.\,#2}
\newcommand{\recdots}[2]{\brec\ #1.\ \ldots}
\newcommand{\abstr}[2]{\lambda #1.\,#2}
\newcommand{\appl}[2]{#1\,#2}
\newcommand{\proj}[1]{\#_{#1}}

\newcommand{\Ref}{\name{ref}}
\newcommand{\Deref}{\,!\,}

\newcommand{\alloc}{\name{alloc}}
\newcommand{\Store}{\name{Store}}




% Typing rules 

\newcommand{\tj}[2]{#1\cc#2}
\newcommand{\ctj}[2]{\tj{#1}{{\resulttypecolor #2}}}
\newcommand{\Tj}[3]{#1 \, \triangleright \, #2\cc#3}
\newcommand{\cTj}[3]{\Tj{{\contextcolor #1}}{#2}{{\resulttypecolor #3}}}
\newcommand{\cbig}[2]{#1 \ \eval\ {\resultcolor #2}}
\newcommand{\cBig}[2]{#1 \\ \eval\ {\resultcolor #2}}
\newcommand{\Clos}[2]{\name{Closure}_{#1}(#2)}

\newcommand{\Tjl}[3]{#1 \,\triangleright_l\, #2\cc#3}
\newcommand{\Tjm}[3]{#1 \,\triangleright_m\, #2\cc#3}
\newcommand{\Tjh}[2]{#1 \triangleright  #2}

\newcommand{\Lj}[3]{#1\,\vdash\,#2\cc#3}

\newcommand{\brule}[1]{\begin{markiere}[#1]}
\newcommand{\erule}{\end{markiere}}

\newcommand{\regel}[2]{\ \begin{array}{@{}c@{}} #1 \\ \hline #2
 \end{array}\ }

\newcommand{\reason}[1]{\ \mbox{#1}}
\newcommand{\Reason}[1]{\vspace{1mm}\\ \mbox{ #1}}


% Program verification

\newcommand{\conj}{\,\land\,}
\newcommand{\Conj}{\bigwedge}
\newcommand{\disj}{\,\lor\,}
\newcommand{\Disj}{\bigvee\,}
\newcommand{\all}[1]{\forall{#1}.\,}
\newcommand{\ex}[1]{\exists{#1}.\,}

\newcommand{\power}[1]{\wp(#1)}

\newcommand{\disjoint}[2]{\name{disj}(#1,#2)}
\newcommand{\cont}[2]{#1 \mapsto #2}
\newcommand{\DEF}{\name{DEF}}

\newcommand{\ret}[2]{{\bf returns}\ #1.\, #2}
\newcommand{\tc}[2]{#1\,\{#2\}}
\newcommand{\triple}[3]{\{#1\}\,#2\,\{#3\}}

% Index

\newcommand{\define}[1]{{\em #1\/}\index{#1}}
\newcommand{\Define}[2]{{\em #1\/}\index{#2}}
\newcommand{\Index}[1]{\index{#1}}
\newcommand{\notation}[1]{#1\index{#1}}
\newcommand{\engl}[1]{(engl.: \define{#1})}
\newcommand{\Engl}[2]{(engl.: \Define{#1}{#2})}

% Theorems etc.

\newtheorem{theorem}{Satz}
\newtheorem{corollary}{Korollar}
\newtheorem{definition}{Definition:}
%\newtheorem{example}{Beispiel:}
%\newtheorem{examples}{Beispiele:}
\newtheorem{lemma}[theorem]{Lemma}
\newtheorem{proposition}[theorem]{Proposition}

%\renewcommand{\thedefinition}{}
%\renewcommand{\theexample}{}
%\renewcommand{\theexamples}{}
\renewcommand{\theenumi}{\rm (\alph{enumi})}
\renewcommand{\labelenumi}{\theenumi}

%\newcommand{\enumarabic}{\renewcommand{\theenumi}{\rm (\arabic{enumi})}}

%\newcommand{\bcoro}[1]{\begin{corollary}\label{cor:#1}}
%\newcommand{\ecoro}{\end{corollary}}

\newcommand{\brdef}[1]{\begin{definition}\label{def:#1}\rm}
\newcommand{\erdef}{\end{definition}}

%\newcommand{\blemm}[1]{\begin{lemma}\label{lem:#1}}
%\newcommand{\bLemm}[2]{\begin{lemma}[#2]\label{lem:#1}\index{#2}}
%\newcommand{\elemm}{\end{lemma}}

%\newcommand{\btheo}[1]{\begin{theorem}\label{th:#1}}
%\newcommand{\bTheo}[2]{\begin{theorem}[#2]\label{th:#1}\index{#2}}
%\newcommand{\etheo}{\end{theorem}}

%\newcommand{\litem}[1]{\item\label{it:#1}}
%\newcommand{\ritem}[1]{\ref{it:#1}}


% Grammars

\newcommand{\bgram}{\[\begin{array}{rrlll}}
\newcommand{\egram}{\end{array}\]}

\newcommand{\is}{& ::= &}
\newcommand{\al}{\\ & \mid &}
\newcommand{\n}{\vspace{2mm}\\}




% Other Environments

\newcommand{\bcase}{\left\{\!\!\!\begin{array}{ll}}
\newcommand{\ecase}{\end{array}\right.}
\newcommand{\benum}{\begin{enumerate}}
\newcommand{\eenum}{\end{enumerate}}
\newcommand{\beqns}{\[\begin{array}{rcll}}
\newcommand{\eeqns}{\end{array}\]}
\newcommand{\bitem}{\begin{itemize}}
\newcommand{\eitem}{\end{itemize}}
\newcommand{\blong}{\!\!\begin{array}[t]{l}}
\newcommand{\elong}{\end{array}}
\newcommand{\btabl}{\begin{tabular}}
\newcommand{\etabl}{\end{tabular}}
\newcommand{\brexa}{\begin{example}\enumarabic\rm}
\newcommand{\erexa}{\end{example}}
\newcommand{\brexs}{\begin{examples}\enumarabic\rm}
\newcommand{\erexs}{\end{examples}}


% German abbreviations

\newcommand{\abk}[1]{#1.\ }
\newcommand{\bzw}{\abk{bzw}}
\newcommand{\bzgl}{\abk{bzgl}}
\newcommand{\das}{\abk{d.h}}
\newcommand{\evtl}{\abk{evtl}}
\newcommand{\usw}{\abk{usw}}
\newcommand{\vgl}{\abk{vgl}}
\newcommand{\zb}{\abk{z.B}}


\newcommand{\infix}[3]{#2\mathbin{#1}#3}

\newcommand{\bli}[3]{\blet\ \vdec{#1}{#2}\ \bin\ #3}
\newcommand{\blidb}[2]{\blet\ {#1}\ \bin\ {#2}}
\newcommand{\blri}[3]{\blet\,\brec\ \vdec{#1}{#2}\ \bin\ #3}

\newcommand{\Bli}[3]{\begin{array}[t]{@{}l}
                     \blet\ \vdec{#1}{#2}\\\bin\ #3
                     \end{array}}

\newcommand{\Vdec}[2]{\begin{array}[t]{@{}l}
                      #1 = \\
                      \ #2
                      \end{array}}

\newcommand{\BLI}[3]{\begin{array}[t]{@{}l}
                     \blet\ \Vdec{#1}{#2}\\\bin\ #3
                     \end{array}}

\newcommand{\Abstr}[2]{\begin{array}[t]{@{}l}
                       \lambda #1.\\\ \ #2
                       \end{array}}


\newcommand{\semantic}[1]{\ensuremath{\llbracket#1\rrbracket}}
\newcommand{\assn}{\textbf{assn}}
\newcommand{\prop}{\textbf{prop}}
\newcommand{\expr}{\textbf{expr}}

\begin{document}

\section*{Ideensammlung}

\paragraph*{Vorgegeben:}
\begin{itemize}
  \item Menge $\Id$
  \item Menge $\Type$
  \item Mengen $\Loc^\tau$, $\Loc = \bigcup\limits_{\tau\in\Type} \Loc^\tau$
\end{itemize}

\paragraph*{Definitionen:}
\begin{itemize}
  \item Menge $\Exp$ der Ausdr"ucke
  \item Menge $\Val$ der Werte
  \item Typumgebungen $\Gamma$ wie "ublich
  \item $\Val^\tau = \{v\in\Val\,|\,\tj{v}{\tau}\}$, also alle abgeschlossenen
        Werte vom Typ $\tau$
  \item Umgebungen $\rho: \Id \pto \Val$
  \item $\rho$ {\em passt zu} $\Gamma$, geschrieben $\Gamma \models \rho$,
        wenn gilt:
        \begin{itemize}
          \item $\dom{\rho} = \dom{\Gamma}$
          \item $\forall \id\in\dom{\rho}: \rho(\id) \in \Val^{\Gamma(\id)}$
        \end{itemize}
\end{itemize}

\paragraph*{Semantische Bereiche:}
\begin{itemize}
  \item $\semantic{\tau\,\assn} = \Val^\tau \times \Store \pto \Bool$
  \item $\semantic{\unit\,\assn} = \Val^\unit \times \Store \pto \Bool \cong \Store \pto \Bool = \semantic{\assn}$
  \item $\semantic{\tau\,\expr} = \Store \pto \Val^\tau \times \Store$
\end{itemize}

\paragraph*{Totale Korrektheit:}
\begin{itemize}
  \item $\Tj{\Gamma}{\triple{p}{e}{\ret{\id:\tau}{q}}}{\prop}$
  \item $\semantic{\Tj{\Gamma}{p}{\assn}} \rho: \Store \pto \Bool$
  \item $\semantic{\Tj{\Gamma}{\ret{\id:\tau}{q}}{\tau\,\assn}}: \Val^\tau \times \Store \pto \Bool$
  \item $\semantic{\Tj{\Gamma}{e}{\tau\,\expr}}\rho: \Store \pto \Val^\tau \times \Store$
\end{itemize}

\noindent
Seien $f\in\semantic{\tau\,\expr}$, $\phi\in\semantic{\assn}$,
$\psi\in\semantic{\tau\,\assn}$. $f$ heisst genau dann {\em total korrekt} bzgl.
$\phi$ und $\psi$, wenn gilt:
\[
  \forall \sigma\in\dom{\phi}: \phi\,\sigma = \true
  \Rightarrow \exists (v,\sigma')\in\dom{\psi}: f\sigma = (v,\sigma') \wedge \psi(v,\sigma')=\true.
\]
{\em Totale Korrektheit} kann dann einfach als logisches Pr"adikat verwendet
werden.
\[\begin{array}{rl}
  & \semantic{\Tj{\Gamma}{\triple{p}{e}{\ret{\id:\tau}{q}}}{\prop}}\rho = \true
    \\
  \Leftrightarrow & \semantic{\Tj{\Gamma}{e}{\tau\,\expr}}\rho \\
  & \text{ total korrekt bzgl.} 
    \semantic{\Tj{\Gamma}{p}{\assn}}\rho \\
  & \text{ und } \semantic{\Tj{\Gamma}{\ret{\id:\tau}{q}}{\tau\,\assn}}\rho
\end{array}\]


\section*{29.02.2008}

\subsection*{Semantik der Logik}

Jedem logischen Typ $\pi \in \Ltype$ wird ein semantischer Bereich $\semantic{\pi}$ zugeordnet,
insb.:
\[\begin{array}{rcl}
  \semantic{\tau} & = & \Val^\tau \\
  \semantic{\assn} & = & \Store \pto \Bool
                    \quad \quad (= \semantic{\unit\,\assn} = \Val^\unit \times \Store \pto \Bool) \\
  \semantic{\tau\,\assn} & = & \Val^\tau \times \Store \pto \Bool \\
  \semantic{\tau\,\expr} & = & \Store \pto \Val^\tau \times \Store
\end{array}\]
Eine Funktion $f \in \semantic{\tau\,\expr}$ heisst {\em total korrekt} bzgl. $\phi \in \semantic{\assn}$
und $\psi\in\semantic{\tau\,\assn}$, wenn f"ur alle $\sigma \in \dom{\phi}$ gilt:
\begin{quote}
  Wenn $\phi(\sigma)=\true$, dann ist $\sigma\in\dom{f}$ und $\psi(\app{f}{\sigma}) = \true$.
\end{quote}
Eine {\em Umgebung} ist eine endliche partielle Funktion $\rho: \Id \pto \bigcup_{\pi\in\Ltype}\semantic{\pi}$.
$\rho$ {\em passt} zur Typumgebung $\Gamma$ -- geschrieben $\Gamma \models \rho$ -- wenn gilt:
\begin{itemize}
  \item $\dom{\rho} = \dom{\Gamma}$
  \item $\rho(\id) \in \semantic{\Gamma(\id)}$ f"ur alle $\id \in \dom{\rho}$
\end{itemize}


\section*{05.03.2008}

\subsection*{TCs in Pre-/Post-conditions}

Zweistufige Syntax f"ur Hoare-Formeln, wo totale Korrektheit nur auf oberster Ebene erlaubt ist,
also insbesondere nicht in Pre-/Post-conditions, wird nicht funktionieren. Beispielsweise l"asst
sich der Counter umschreiben zu
\[\begin{array}{rl}
  \exists c:\langle\ldots\rangle.\exists x:\reftype{\z}.
  & (\forall i:\z. \{!x=i\}\ c\#\name{inc}\ \{!x=i+1\}) \\
  \wedge & \{\true\}\ \name{new\_counter\_1}\,(\,)\ \{\ret{c_1}{c_1=c}\}
\end{array}\]
aber dann besagt die Spezifikation, dass $\name{new\_counter\_1}$ immer den gleichen counter
liefert. D.h. man wird wohl doch TCs in Pre-/Post-conditions zulassen m"ussen.

\subsection*{Neue Idee}

Dann k"onnte man eine einfache Syntax f"ur Assertions und Hoare-Formeln entwerfen. Dabei ist
eine Hoare-Formel nur ein Spezialfall einer Assertion, genauer eine Assertion von Typ $\bool$.

\[\bgram
p \is p_1 \wedge p_2
  \al \neg p_1
  \al \forall id:\hat{\pi}. p_1 \quad\quad \text{$\hat{\pi}$ ist eingeschr"anktes $\pi$}
  \al \app{p_1}{p_2}
  \al f \in F
  \al \id \in \Id
  \al v \in \bigcup_{\tau\in\Type} \Val^\tau
  \al p_1 = p_2
  \al \abstr{\id_1:\tau_1,\ldots,\id_n:\tau_n}{p_1} \quad (n \ge 1)
  \al \triple{p_1}{e}{p_2}
  \n
\egram\]

\noindent
Die TC wird durch $\triple{p_1}{e}{p_2}$ dargestellt. Hierbei wird, wie zuvor definiert, implizit
"uber alle Stores quantifiziert. D.h. f"ur alle (passenden) $\sigma$ gilt: Wenn $p_1$ f"ur $\sigma$
gilt, dann terminiert $e$ und es gilt $p_2$ (formal unsauber).

Die Menge $\Ltype$ der logischen Typen (f"ur Assertions) enth"alt dann zumindest:

\[\bgram
\pi \is \tau
    \al \pi_1 * \ldots * \pi_n \quad (n \ge 1)
    \al \pi_1 \xrightarrow{t} \pi_2
    \n
\egram\]

\noindent
Wobei in $\tau$ noch der Spezielle Typ $\textbf{store}$ enthalten sein muss, mglw.
als polymorpher Typ.

\[
  \regel{\Tj{\Gamma[\tau_1/\id_1,\ldots,\tau_n/\id_n]}{p_1}{\tau}}
        {\Tj{\Gamma}{\abstr{\id_i:\tau_i}{p_1}}{\tau_1*\ldots*\tau_n\xrightarrow{t}\tau}}
\]
\[
  \regel{\Tj{\Gamma}{p}{\textbf{store}\xrightarrow{t}\bool} \quad
         \Tj{\Gamma}{e}{\tau} \quad
         \Tj{\Gamma}{q}{\tau*\textbf{store}\xrightarrow{t}\bool}}
        {\Tj{\Gamma}{\triple{p}{e}{q}}{\bool}}
\]


\end{document}

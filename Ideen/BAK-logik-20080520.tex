\documentclass[12pt,a4paper,bigheadings]{scrartcl}

\usepackage{amssymb}
\usepackage{amstext}
\usepackage{amsmath}
\usepackage{array}
\usepackage[ngerman]{babel}
\usepackage{color}
\usepackage{enumerate}
%\usepackage[T1]{fontenc}
\usepackage{german}
\usepackage[a4paper,%
            colorlinks=false,%
            final,%
            pdfkeywords={},%
            pdftitle={},%
            pdfauthor={Benedikt Meurer},%
            pdfsubject={},%
            pdfdisplaydoctitle=true]{hyperref}
\usepackage[latin1]{inputenc}
\usepackage{latexsym}
\usepackage[final]{listings}
\usepackage{makeidx}
%\usepackage{mathpartir}
\usepackage{ngerman}
\usepackage[standard,thmmarks]{ntheorem}
\usepackage{scrpage2}
\usepackage{stmaryrd}
%\usepackage[DIV13,BCOR5mm]{typearea}
\usepackage{url}
\usepackage[all]{xy}

%% TP-Makros
% General

\newcommand{\name}[1]{{\text{\it #1\/}}}
%\newcommand{\name}[1]{\mathit{#1}}

%\newcommand{\bfbox}[1]{\mathbf{#1}}

\newcommand{\I}{{\cal I}}

% Proof trees

\newcounter{tree}
\newcounter{node}[tree]

\newlength{\treeindent}
\newlength{\nodeindent}
\newlength{\nodesep}

\newcommand{\refnode}[1]
 {\ref{\thetree.#1}}

\newcommand{\contextcolor}{\color{blue}}

\definecolor{darkgreen}{rgb}{0.1,0.5,0}
\definecolor{redexcolor}{rgb}{0.7,1,0}

\newcommand{\resulttypecolor}{\color{darkgreen}}

\newcommand{\resultcolor}{\color{blue}}

\newcommand{\byrulecolor}{\color{red}}

\newcommand{\marked}[1]{\colorbox{redexcolor}{$#1$}}

\newcommand{\smallsteparrow}[1]{\stackrel{\mbox{\scriptsize\byrulecolor (#1)}}{\longrightarrow}}

\newcommand{\byrule}[1]{\hspace{-5mm}\byrulecolor\mbox{\scriptsize\ #1}}

\newif\ifarrows 
\arrowsfalse 

\newcommand{\arrow}[3]
  {\ifarrows
   \ncangle[angleA=-90,angleB=#1]{<-}{\thetree.#2}{\thetree.#3}
   \else
   \fi}

\newcommand{\node}[4]
 {\ifarrows
   \else \refstepcounter{node}
         \noindent\hspace{\treeindent}\hspace{#2\nodeindent}
         \rnode{\thetree.#1}{\makebox[6mm]{(\thenode)}}\label{\thetree.#1}
         $\blong 
          #3 \\ 
          \byrule{#4} 
          \elong$
         \vspace{\nodesep}
   \fi}

\newcommand{\dummyarrow}[3]
  {\arrow{#1}{#2}{#3dummy}}

\newcommand{\dummynode}[2]
 {\ifarrows 
  \else \noindent\hspace{\treeindent}\hspace{#2\nodeindent}
         \rnode{\thetree.#1dummy}{\makebox[6mm]{(\refnode{#1})}}\label{\thetree.#1dummy}
         $\ldots$
         \vspace{\nodesep}
   \fi}

\newcommand{\mktree}[1]
 {\stepcounter{tree} #1 \arrowstrue #1 \arrowsfalse}

\fboxrule=0mm

\newcommand{\cenv}[1]{\fbox{$\begin{array}{|ll|}\hline #1 \\\hline\end{array}$}}


% Special Symbols

\newcommand{\uminus}{\widetilde{\ }}
\renewcommand{\uminus}{-\!}
\newcommand{\nop}{()}

% Im X-Symbol-Manual empfohlen

\newcommand{\nsubset}{\not\subset}
%\newcommand{\textflorin}{\textit{f}}
\newcommand{\setB}{{\mathord{\mathbb B}}}
\newcommand{\setC}{{\mathord{\mathbb C}}}
\newcommand{\setN}{{\mathord{\mathbb N}}}
\newcommand{\setQ}{{\mathord{\mathbb Q}}}
\newcommand{\setR}{{\mathord{\mathbb R}}}
\newcommand{\setZ}{{\mathord{\mathbb Z}}}
\newcommand{\coloncolon}{\mathrel{::}}

% Eigene (k\"urzere) Befehle

\newcommand{\pfi}{\varphi}
\newcommand{\eps}{\varepsilon}
\newcommand{\eval}{\Downarrow}
\newcommand{\pto}{\hookrightarrow}
\newcommand{\emp}{\emptyset}
\newcommand{\sleq}{\subseteq}
\newcommand{\sgeq}{\supseteq}
\newcommand{\sqleq}{\sqsubseteq}
\newcommand{\sqgeq}{\sqsupseteq}
\newcommand{\lub}{\bigsqcup}
\newcommand{\glb}{\bigsqcap}
\newcommand{\lsem}{\llbracket}
\newcommand{\rsem}{\rrbracket}
\newcommand{\impl}{\models}
\newcommand{\step}{\vdash}
%\newcommand{\tr}{\triangleright}
\newcommand{\cc}{\coloncolon}


% Names (lower case italic)

\newcommand{\exn}{\name{exn}}
\newcommand{\id}{\name{id}}
\newcommand{\op}{\name{op}}

\newcommand{\true}{\name{true}}
\newcommand{\false}{\name{false}}
\newcommand{\Not}{\name{not}}

\newcommand{\sol}[3]{\name{solution}\,(#1,#2,#3)}
%\newcommand{\unify}{\name{unify}\,}

\newcommand{\Fst}{\name{fst}}
\newcommand{\Snd}{\name{snd}}

\newcommand{\Hd}{\name{hd}}
\newcommand{\Tl}{\name{tl}}
\newcommand{\Cons}{\name{cons}}
\newcommand{\Empty}{\name{is\_empty}}

\newcommand{\dom}[1]{\name{dom}(#1)}
\newcommand{\free}[1]{\name{free}\,(#1)}
\newcommand{\rank}[1]{\name{rank}\,(#1)}
\newcommand{\len}[1]{\name{len}\,(#1)}
\newcommand{\tr}[1]{\name{tr}(#1)}

% Names (upper case italic)

\newcommand{\Bool}{\name{Bool}}
\newcommand{\Btype}{\name{BType}}

\newcommand{\Conf}{\name{Conf}}
\newcommand{\Const}{\name{Const}}

\newcommand{\Dec}{\name{Dec}}

\newcommand{\Env}{\name{Env}}
\newcommand{\Exn}{\name{Exn}}
\newcommand{\EP}{\name{EP}}
\newcommand{\Exp}{\name{Exp}}

\newcommand{\Ncx}{\name{Ncx}}
\newcommand{\dbExp}{\name{dbExp}}
\newcommand{\dbVal}{\name{dbVal}}
\newcommand{\dbEnv}{\name{dbEnv}}
\newcommand{\dbCl}{\name{dbCl}}

\newcommand{\Id}{\name{Id}}
\newcommand{\Int}{\name{Int}}

\newcommand{\Lexp}{\name{LExp}}
\newcommand{\Loc}{\name{Loc}}

\newcommand{\Op}{\name{Op}}

\newcommand{\Type}{\name{Type}}
\newcommand{\Tvar}{\name{TVar}}
\newcommand{\teqns}[3]{\name{teqns}\,(#1,#2,#3)}
\newcommand{\tvar}[1]{\name{tvar}\,(#1)}
\newcommand{\unify}[1]{\name{unify}\,(#1)}

\newcommand{\Ptype}{\name{PType}}

\newcommand{\Unit}{\name{Unit}}

\newcommand{\Val}{\name{Val}}

% keywords

\newcommand{\z}{\mathbf{int}}
\newcommand{\bool}{\mathbf{bool}}
\newcommand{\unit}{\mathbf{unit}}
\newcommand{\blist}{\mathbf{list}}
\newcommand{\bref}{\mathbf{ref}}
\newcommand{\ltype}[1]{#1\,\blist}
\newcommand{\reftype}[1]{#1\,\bref}

\renewcommand{\div}{\mathbin{\mathbf{div}}}
\newcommand{\sel}{\mathbin{.}}
%\newcommand{\mod}{\mathbin{\mathbf{mod}}}
\renewcommand{\mod}{\text{mod}}

\newcommand{\bif}{\mathbf{if}}
\newcommand{\bthen}{\mathbf{then}}
\newcommand{\belse}{\mathbf{else}}

\newcommand{\blet}{\mathbf{let}}
\newcommand{\bin}{\mathbf{in}}
\newcommand{\bend}{\mathbf{end}}

\newcommand{\bval}{\mathbf{val}}
\newcommand{\brec}{\mathbf{rec}}
\newcommand{\bfix}{\mathbf{fix}}
\newcommand{\bfun}{\mathbf{fun}}
\newcommand{\band}{\mathbf{and}}
\newcommand{\btype}{\mathbf{type}}

%\newcommand{\andalso}[2]{#1\,\&\&\,#2}
%\newcommand{\orelse}[2]{#1\,\|\,#2}
\newcommand{\andalso}{\&\&}
\newcommand{\orelse}{\|}
\newcommand{\bandalso}{\mathbf{andalso}}
\newcommand{\borelse}{\mathbf{orelse}}

\newcommand{\bwhile}{\mathbf{while}}
\newcommand{\bdo}{\mathbf{do}}
\newcommand{\bfor}{\mathbf{for}}

\newcommand{\brepeat}{\mathbf{repeat}}
\newcommand{\buntil}{\mathbf{until}}

\newcommand{\barray}{\mathbf{array}}
\newcommand{\bof}{\mathbf{of}}

\newcommand{\bclass}{\mathbf{class}}

\newcommand{\app}[2]{#1\,#2}
\newcommand{\bift}[2]{\bif\ #1\ \bthen\ #2}
\newcommand{\bifte}[3]{\bif\ #1\ \bthen\ #2\ \belse\ #3}
\newcommand{\Bifte}[3]{\blong\bif\ #1\\\bthen\ #2\\\belse\ #3\elong}
\newcommand{\bwd}[2]{\bwhile\ #1\ \bdo\ #2}
\newcommand{\bru}[2]{\brepeat\ #1\ \buntil\ #2}


\newcommand{\blie}[2]{\blet\ #1\ \bin\ #2 \ \bend}
\newcommand{\vdec}[2]{#1 = #2}
\newcommand{\rec}[2]{\brec\ #1.\ #2}
\newcommand{\recdots}[2]{\brec\ #1.\ \ldots}
\newcommand{\abstr}[2]{\lambda #1.\,#2}
\newcommand{\appl}[2]{#1\,#2}

\newcommand{\Ref}{\name{ref}}
\newcommand{\Deref}{\,!\,}

\newcommand{\alloc}{\name{alloc}}
\newcommand{\Store}{\name{Store}}




% Typing rules 

\newcommand{\tj}[2]{#1\cc#2}
\newcommand{\ctj}[2]{\tj{#1}{{\resulttypecolor #2}}}
\newcommand{\Tj}[3]{#1 \, \triangleright \, #2\cc#3}
\newcommand{\cTj}[3]{\Tj{{\contextcolor #1}}{#2}{{\resulttypecolor #3}}}
\newcommand{\cbig}[2]{#1 \ \eval\ {\resultcolor #2}}
\newcommand{\cBig}[2]{#1 \\ \eval\ {\resultcolor #2}}
\newcommand{\Clos}[2]{\name{Closure}_{#1}(#2)}

\newcommand{\Tjl}[3]{#1 \,\triangleright_l\, #2\cc#3}
\newcommand{\Tjm}[3]{#1 \,\triangleright_m\, #2\cc#3}
\newcommand{\Tjh}[2]{#1 \triangleright  #2}

\newcommand{\brule}[1]{\begin{markiere}[#1]}
\newcommand{\erule}{\end{markiere}}

\newcommand{\regel}[2]{\ \begin{array}{@{}c@{}} #1 \\ \hline #2
 \end{array}\ }

\newcommand{\reason}[1]{\ \mbox{#1}}
\newcommand{\Reason}[1]{\vspace{1mm}\\ \mbox{ #1}}


% Program verification

\newcommand{\conj}{\,\land\,}
\newcommand{\Conj}{\bigwedge}
\newcommand{\disj}{\,\lor\,}
\newcommand{\Disj}{\bigvee\,}
\newcommand{\all}[1]{\forall{#1}.\,}
\newcommand{\ex}[1]{\exists{#1}.\,}

\newcommand{\power}[1]{\wp(#1)}

\newcommand{\disjoint}[2]{\name{disj}(#1,#2)}
\newcommand{\cont}[2]{#1 \mapsto #2}
\newcommand{\DEF}{\name{DEF}}

\newcommand{\ret}[2]{{\bf returns}\ #1.\, #2}
\newcommand{\tc}[2]{#1\,\{#2\}}
\newcommand{\triple}[3]{\{#1\}\,#2\,\{#3\}}

% Index

\newcommand{\define}[1]{{\em #1\/}\index{#1}}
\newcommand{\Define}[2]{{\em #1\/}\index{#2}}
\newcommand{\Index}[1]{\index{#1}}
\newcommand{\notation}[1]{#1\index{#1}}
\newcommand{\engl}[1]{(engl.: \define{#1})}
\newcommand{\Engl}[2]{(engl.: \Define{#1}{#2})}

% Theorems etc.

\newtheorem{theorem}{Satz}
\newtheorem{corollary}{Korollar}
\newtheorem{definition}{Definition:}
%\newtheorem{example}{Beispiel:}
%\newtheorem{examples}{Beispiele:}
\newtheorem{lemma}[theorem]{Lemma}

%\renewcommand{\thedefinition}{}
%\renewcommand{\theexample}{}
%\renewcommand{\theexamples}{}
\renewcommand{\theenumi}{\rm (\alph{enumi})}
\renewcommand{\labelenumi}{\theenumi}

%\newcommand{\enumarabic}{\renewcommand{\theenumi}{\rm (\arabic{enumi})}}

%\newcommand{\bcoro}[1]{\begin{corollary}\label{cor:#1}}
%\newcommand{\ecoro}{\end{corollary}}

\newcommand{\brdef}[1]{\begin{definition}\label{def:#1}\rm}
\newcommand{\erdef}{\end{definition}}

%\newcommand{\blemm}[1]{\begin{lemma}\label{lem:#1}}
%\newcommand{\bLemm}[2]{\begin{lemma}[#2]\label{lem:#1}\index{#2}}
%\newcommand{\elemm}{\end{lemma}}

%\newcommand{\btheo}[1]{\begin{theorem}\label{th:#1}}
%\newcommand{\bTheo}[2]{\begin{theorem}[#2]\label{th:#1}\index{#2}}
%\newcommand{\etheo}{\end{theorem}}

%\newcommand{\litem}[1]{\item\label{it:#1}}
%\newcommand{\ritem}[1]{\ref{it:#1}}


% Grammars

\newcommand{\bgram}{\[\begin{array}{rrlll}}
\newcommand{\egram}{\end{array}\]}

\newcommand{\is}{& ::= &}
\newcommand{\al}{\\ & \mid &}
\newcommand{\n}{\vspace{2mm}\\}




% Other Environments

\newcommand{\bcase}{\left\{\!\!\!\begin{array}{ll}}
\newcommand{\ecase}{\end{array}\right.}
\newcommand{\benum}{\begin{enumerate}}
\newcommand{\eenum}{\end{enumerate}}
\newcommand{\beqns}{\[\begin{array}{rcll}}
\newcommand{\eeqns}{\end{array}\]}
\newcommand{\bitem}{\begin{itemize}}
\newcommand{\eitem}{\end{itemize}}
\newcommand{\blong}{\!\!\begin{array}[t]{l}}
\newcommand{\elong}{\end{array}}
\newcommand{\btabl}{\begin{tabular}}
\newcommand{\etabl}{\end{tabular}}
\newcommand{\brexa}{\begin{example}\enumarabic\rm}
\newcommand{\erexa}{\end{example}}
\newcommand{\brexs}{\begin{examples}\enumarabic\rm}
\newcommand{\erexs}{\end{examples}}


% German abbreviations

\newcommand{\abk}[1]{#1.\ }
\newcommand{\bzw}{\abk{bzw}}
\newcommand{\bzgl}{\abk{bzgl}}
\newcommand{\das}{\abk{d.h}}
\newcommand{\evtl}{\abk{evtl}}
\newcommand{\usw}{\abk{usw}}
\newcommand{\vgl}{\abk{vgl}}
\newcommand{\zb}{\abk{z.B}}


\newcommand{\infix}[3]{#2\mathbin{#1}#3}

\newcommand{\bli}[3]{\blet\ \vdec{#1}{#2}\ \bin\ #3}

\newcommand{\Bli}[3]{\begin{array}[t]{@{}l}
                     \blet\ \vdec{#1}{#2}\\\bin\ #3
                     \end{array}}

\newcommand{\Vdec}[2]{\begin{array}[t]{@{}l}
                      #1 = \\
                      \ #2
                      \end{array}}

\newcommand{\BLI}[3]{\begin{array}[t]{@{}l}
                     \blet\ \Vdec{#1}{#2}\\\bin\ #3
                     \end{array}}

\newcommand{\Abstr}[2]{\begin{array}[t]{@{}l}
                       \lambda #1.\\\ \ #2
                       \end{array}}


\newcommand{\semantic}[1]{\ensuremath{\llbracket#1\rrbracket}}
\newcommand{\assn}{\ensuremath{\mathbf{assn}}}
\newcommand{\atype}[1]{#1\,\assn}
\newcommand{\bexpr}{\ensuremath{\mathbf{expr}}}
\newcommand{\TEnv}{\name{TEnv}}
\newcommand{\Dtype}{\name{DType}}
\newcommand{\Atype}{\name{AType}}
\newcommand{\Assn}{\name{Assn}}
\newcommand{\Term}{\name{Term}}
\newcommand{\locns}{\name{locns}}
\newcommand{\reach}{\name{reach}}
\newcommand{\grph}[1]{\name{graph}(#1)}
\newcommand{\reachable}{\name{reachable}}
\newcommand{\supp}{\name{supp}}
\newcommand{\tto}{\ensuremath{\xrightarrow{t}}}
\renewcommand{\disjoint}[2]{\ensuremath{#2\not\hookrightarrow#1}}
\newcommand{\RN}[1]{\mbox{{\sc (#1)}}}
\newcommand{\Tje}[3]{#1 \,\tr_e\, #2\cc#3}
\newcommand{\TC}[4]{#1 \models \{#2\}\,#3\,\{#4\}}


\begin{document}


\section{Speicherzust"ande und Erreichbarkeit}

F"ur jeden Typ $\tau$ sei eine unendliche Menge $\Loc^\tau$ definiert, deren Elemente $X,Y,\ldots$
als {\em Speicherpl"atze} vom Typ $\tau$ bezeichnet werden. Es wird vorausgesetzt, dass die Mengen
$\Loc^\tau$ paarweise disjunkt sind und darauf ist definiert
\[
  \Loc = \bigcup_{\tau\in\Type} \Loc^\tau
\]
Wir schreiben $\locns{e}$ f"ur die Menge aller im Ausdruck $e$ (syntaktisch) vorkommenden Speicherpl"atze
und $\Val^\tau$ f"ur die Menge aller abgeschlossenen Werte vom Typ $\tau$ (wobei Abgeschlossenheit sich
hier nur auf die Bezeichner $\Id$ beschr"ankt) und $\Exp^\tau$ f"ur die Menge aller abgeschlossenen
Ausdr"ucke vom Typ $\tau$.

\begin{definition}[Speicherzustand] \
  \begin{enumerate}
    \item Eine partielle Funktion $\sigma: \Loc \pto \bigcup_{\tau\in\Type} \semantic{\tau}$ hei"st
          {\em Speicherzustand}, wenn gilt:
          \begin{itemize}
            \item $\sigma(\Loc^\tau) \subseteq \Val^\tau$ f"ur alle $\tau\in\Type$
            \item $\locns(\sigma(\Loc)) \subseteq \dom{\sigma}$
          \end{itemize}

    \item F"ur alle endlichen $L \subseteq \Loc$ definieren wir eine Menge aller {\em $L$-Speicherzust"ande}
          $\Store_L = \{\sigma\,|\,\text{$\sigma$ Speicherzustand mit $L\subseteq\dom{\sigma}$}\}$.

    \item Die Menge aller Speicherzust"ande ist dann $\Store = \bigcup_{L\subseteq\Loc,\text{ $L$ endl.}} \Store_L$.
  \end{enumerate}
\end{definition}
Demzufolge ist ein Zustand per Definition stets {\em wohlgetypt} und in sich abgeschlossen,
d.h. er enth"alt keine {\em dangling references}.

F"ur $\sigma\in\Store_L$ seien die Mengen $\reach_i(L,\sigma)\subseteq \Loc$ f"ur alle
$i\in\setN$ und die Menge $\reach(L,\sigma)\subseteq\Loc$ definiert durch
\[\begin{array}{rcl}
  \reach_0(L,\sigma) & = & L \\
  \reach_{i+1}(L,\sigma) & = & \reach_i(L,\sigma)
             \cup \bigcup_{X\in\reach_i(L,\sigma)} \locns(\sigma(X)) \\
  \reach(L,\sigma) & = & \bigcup_{i\in\setN} \reach_i(L,\sigma)
\end{array}\]
$\reach(L,\sigma)$ enth"alt alle Speicherpl"atze, die man von der Menge $L$ aus im
Zustand $\sigma$ erreichen kann. Da ein Zustand $\sigma\in\Store_L$ keine dangling references
enth"alt, gilt stets
\[
  \reach(L,\sigma) \subseteq \dom{\sigma}.
\]

\begin{lemma} \label{lemma:Store_L_und_reach}
  Seien $L,L'\subseteq\Loc$ endl., $L \subseteq L'$ und $\sigma,\sigma'\in\Store_{L'}$. Dann gilt:
  \begin{enumerate}
    \item $\Store_{L'} \subseteq \Store_{L}$
    \item $\reach(L,\sigma) \subseteq \reach(L',\sigma)$
    \item Wenn $\reach(L',\sigma) = \reach(L',\sigma')$ und $\sigma(X) = \sigma'(X)$ f.a.
          $X\in\reach(L',\sigma)$, dann $\reach(L,\sigma) = \reach(L,\sigma')$
  \end{enumerate}
\end{lemma}

\begin{beweis} \
  \begin{enumerate}
    \item Sei $\sigma''\in\Store_{L'}$, dann gilt $L' \subseteq \dom{\sigma''}$. Aus $L \subseteq L'$ folgt
          $L\subseteq\dom{\sigma''}$, also $\sigma''\in\Store_L$.
    \item Straight-forward induction.
    \item Straight-forward induction.
  \end{enumerate}
\end{beweis}

\begin{definition}[$L$-"Ahnlichkeit]
  Seien $\sigma,\sigma'\in\Store$ und $L \subseteq \dom{\sigma} \cap \dom{\sigma'}$. $\sigma$ und $\sigma'$
  heissen {\em $L$-"ahnlich}, geschrieben $\sigma \simeq_L \sigma'$, wenn $\sigma(X) = \sigma'(X)$ f"ur
  alle $X \in L$ gilt.
\end{definition}

\begin{lemma} \label{lemma:L_Aehnlichkeit}
  Seien $L,L'\subseteq\Loc$ mit $L \subseteq L'$ und $\sigma,\sigma'\in\Store$. Wenn $\sigma \simeq_{L'} \sigma$,
  dann $\sigma \simeq_L \sigma'$.
\end{lemma}

\begin{beweis}
  Trivial.
\end{beweis}

\begin{definition}[$L$-"Aquivalenz]
  Sei $L \subseteq \Loc$ endlich. Zwei Zust"ande $\sigma,\sigma'\in\Store_L$ hei"sen {\em $L$-"aquivalent}, geschrieben
  $\sigma \equiv_L \sigma'$, wenn gilt:
  \begin{enumerate}
    \item $\reach(L,\sigma) = \reach(L,\sigma')$
    \item $\sigma \simeq_{\reach(L,\sigma)} \sigma'$
  \end{enumerate}
\end{definition}

\begin{lemma} \label{lemma:L_Aequivalenz}
  Seien $L,L'\subseteq\Loc$ endl. mit $L \subseteq L'$ und $\sigma,\sigma'\in\Store_{L'}$. Wenn $\sigma \equiv_{L'} \sigma'$,
  dann $\sigma \equiv_L \sigma'$.
\end{lemma}

\begin{beweis}
  Folgt leicht mit Lemma~\ref{lemma:Store_L_und_reach} und Lemma~\ref{lemma:L_Aehnlichkeit}.
\end{beweis}

\begin{definition}[$L$-Definierbarkeit]
  Sei $L \subseteq \Loc$ endl. und $\phi:\Store\pto\Bool$.
  $\phi$ hei"st {\em $L$-definierbar}, wenn f"ur alle $\sigma,\sigma'\in\dom{\phi}$ mit
  $\sigma \equiv_L \sigma'$ gilt: $\phi\,\sigma = \phi\,\sigma'$.
\end{definition}

\begin{lemma} \label{lemma:L_Definierbarkeit}
  Seien $L,L'\subseteq\Loc$ mit $L \subseteq L'$ und sei $\phi:\Store \pto \Bool$. Ist $\phi$ $L$-definierbar,
  so ist $\phi$ ebenfalls $L'$-definierbar.
\end{lemma}

\begin{beweis}
  Folgt leicht mit Lemma~\ref{lemma:L_Aequivalenz}.
\end{beweis}


\section{Big step Semantik}

\begin{definition}[Konfiguration]
  Seien $e\in\Exp$ und $\sigma\in\Store$.
  Das Paar $(e,\sigma)$ hei"st {\em Konfiguration}, wenn $\locns(e) \subseteq \dom{\sigma}$.
\end{definition}

Die Speicherzust"ande $\sigma$ selbst sind schon per Definition abgeschlossen.

\begin{lemma}[Wohldefiniertheit der big step Semantik]
  Seien $e\in\Exp$, $v\in\Val$ und $\sigma,\sigma'\in\Store$. Ist $(e,\sigma)$ eine Konfiguration
  und gilt $(e,\sigma) \Downarrow (v,\sigma')$, so ist auch $(v,\sigma')$ eine Konfiguration.
\end{lemma}

\begin{lemma}
  Seien $\sigma_1,\sigma_2\in\Store$ mit $\grph{\sigma_1} \subseteq \grph{\sigma_2}$. Dann gilt:
  \begin{enumerate}
    \item Wenn $(e,\sigma_1) \Downarrow (v,\sigma_1')$, dann existiert ein $\sigma_2'\in\Store$ mit
          $(e,\sigma_2) \Downarrow (v,\sigma_2')$ und $\grph{\sigma_1'} \subseteq \grph{\sigma_2'}$.
    \item Wenn $(e,\sigma_2) \Downarrow (v,\sigma_2')$, dann existiert ein $\sigma_1'\in\Store$ mit
          $(e,\sigma_1) \Downarrow (v,\sigma_1')$ und $\grph{\sigma_1'} \subseteq \grph{\sigma_2'}$.
  \end{enumerate}
\end{lemma}

\begin{lemma} \label{lemma:big_steps_und_dom_sigma}
  Seien $e\in\Exp$, $v\in\Val$ und $\sigma,\sigma'\in\Store$. Wenn
  $(e,\sigma) \Downarrow (v,\sigma')$, dann gilt $\dom{\sigma} \subseteq \dom{\sigma'}$.
\end{lemma}


\section{Syntax der Logik}

\subsection{Typen}

Sei $\Type$ die Menge aller {\em Typen} $\tau$ der Programmiersprache. Darauf definieren wir 
\begin{itemize}
  \item die Menge $\Dtype$ aller {\em Datentypen} $\delta$,
  \item die Menge $\Atype$ aller {\em assertion types} $\theta$ und
  \item die Menge $\Ltype$ aller {\em logischen Typen} $\pi$
\end{itemize}
durch
\[\bgram
\delta  \is \tau
        \al \delta_1 \tto \delta_2
        \n
\theta  \is \assn
        \al \delta \tto \theta
        \n
\pi \is \delta
    \al \theta
    \n
\egram\]

\subsection{Terme und Formeln}

Sei $\Exp$ die Menge aller {\em Ausdr"ucke} $e$ und $\Val$ die Menge aller {\em Werte} $v$
der Programmiersprache (wobei angenommen wird, dass $\Id \subseteq \Val$). Darauf definieren wir 
\begin{itemize}
  \item die Menge $F$ aller {\em Funktionszeichen} $f$,
  \item die Menge \name{Term} aller {\em Terme} $t$,
  \item die Menge \name{Assn} aller {\em assertions} $p,q$ sowie
  \item die Menge \name{Formula} aller {\em (Hoare-)Formeln} $h$
\end{itemize}
durch
\[\bgram
f \is + \mid - \mid * \mid < \mid > \mid \le \mid \ge \mid \Cons \mid \ldots
  \n
t \is v
  \al f
  \al \app{t_1}{t_2}
  \al \abstr{\id:\delta}{t}
  \n
p,q \is t_1 \mapsto t_2
    \al \disjoint{p}{v}
    \al \app{p}{t}
    \al \abstr{\id:\delta}{p}
    \al \neg p
    \al p \wedge q
    \al \exists \id:\pi\in[v]. p
    \al h
    \n
h \is \triple{p}{e}{q}
  \al t_1 = t_2
  \al \neg h
  \al h_1 \wedge h_2
  \al \exists \id:\delta. h
\egram\]

\subsubsection{Syntaktischer Zucker}

Unter anderem folgende Konventionen:
\[\begin{array}{rcl}
  \triple{p}{e}{\ret{\id:\tau}{q}} &\equiv& \triple{p}{e}{\abstr{\id:\tau}{q}} \\
  \{p\} &\equiv& \triple{\true}{()}{\ret{u:\unit}{p}} \\
  && \text{mit $u \not \in \free{p}$}
\end{array}\]
Mglw. k"onnte man dar"uberhinaus:
\[\begin{array}{rcl}
  \exists \id:\delta.h &\equiv& \{\exists \id:\delta\in [\true]. h\}
\end{array}\]
W"are noch zu pr"ufen, z.B. indem man zun"achst die Hoare-Formel beibeh"alt und
dann anhand der Semantik zeigt, dass "aquivalent.


\section{Typregeln f"ur die Logik}

Der Begriff {\em Typumgebung} wird verallgemeinert f"ur die Logik. Eine {\em Typumgebung} ist
damit eine partielle Funktion
\[
  \Gamma: \Id \pto \Ltype
\]
mit endlichem Definitionsbereich. {\em Typurteile} f"ur die Logik sind von der Form
\[\begin{array}{l}
  \Tje{\Gamma}{e}{\tau} \\
  \Tj{\Gamma}{t}{\delta} \\
  \Tj{\Gamma}{p}{\theta} \\
  \Tjh{\Gamma}{h}
\end{array}\]
F"ur Typurteile $\Tje{\Gamma}{e}{\tau}$ der Programmiersprache gelten die "ublichen Typregeln.
F"ur die "ubrigen Typurteile werden neue Typregeln ben"otigt.

Die g"ultigen Typurteile f"ur assertions erhalten wir mit den Regeln \\[3mm]
\begin{tabular}{rl}
  \RN{P-App} & $\regel{\Tj{\Gamma}{p}{\delta \tto \theta} \quad \Tj{\Gamma}{t}{\delta}}
                      {\Tj{\Gamma}{\app{p}{t}}{\theta}}$ \\[1mm]
  \RN{P-Content} & $\regel{\Tj{\Gamma}{t_1}{\reftype{\tau}} \quad \Tj{\Gamma}{t_2}{\tau}}
                          {\Tj{\Gamma}{t_1 \mapsto t_2}{\assn}}$ \\[1mm]
  \RN{P-Abstr} & $\regel{\Tj{\Gamma[\delta/\id]}{p}{\theta}}
                        {\Tj{\Gamma}{\abstr{\id:\delta}{p}}{\delta \tto \theta}}$ \\[1mm]
  \RN{P-Disjoint} & $\regel{\Tj{\Gamma}{v}{\tau} \quad \Tj{\Gamma}{p}{\theta}}
                           {\Tj{\Gamma}{\disjoint{p}{v}}{\assn}}$ \\[1mm]
  \RN{P-Not} & $\regel{\Tj{\Gamma}{p}{\assn}}
                      {\Tj{\Gamma}{\neg p}{\assn}}$ \\[1mm]
  \RN{P-And} & $\regel{\Tj{\Gamma}{p_1}{\assn} \quad \Tj{\Gamma}{p_2}{\assn}}
                      {\Tj{\Gamma}{p_1 \wedge p_2}{\assn}}$ \\[1mm]
  \RN{P-Exists} & $\regel{\Tj{\Gamma}{v}{\tau} \quad \Tj{\Gamma[\pi/\id]}{p}{\assn}}
                         {\Tj{\Gamma}{\exists \id:\pi\in[v].p}{\assn}}$ \\[1mm]
  \RN{P-Hoare} & $\regel{\Tjh{\Gamma}{h}}
                        {\Tj{\Gamma}{h}{\assn}}$
\end{tabular} \\[3mm]
die g"ultigen Typurteile f"ur Terme erhalten wir mit den Regeln \\[3mm]
\begin{tabular}{rl}
  \RN{T-Val} & $\regel{\Tje{\Gamma}{v}{\tau} \quad \locns(v)=\emptyset}
                      {\Tj{\Gamma}{v}{\tau}}$ \\[1mm]
  \RN{T-Func} & $\regel{\tj{f}{\delta}}
                       {\Tj{\Gamma}{f}{\delta}}$ \\[1mm]
  \RN{T-App} & $\regel{\Tj{\Gamma}{t_1}{\delta \tto \delta'} \quad \Tj{\Gamma}{t_2}{\delta}}
                      {\Tj{\Gamma}{\app{t_1}{t_2}}{\delta'}}$ \\[1mm]
  \RN{T-Abstr} & $\regel{\Tj{\Gamma[\delta/\id]}{t}{\delta'}}
                        {\Tj{\Gamma}{\abstr{\id:\delta}{t}}{\delta \tto \delta'}}$ \\[1mm]
\end{tabular} \\[3mm]
und die g"ultigen Typurteile f"ur Hoare-Formeln mit \\[3mm]
\begin{tabular}{rl}
  \RN{H-Eq} & $\regel{\Tj{\Gamma}{t_1}{\delta} \quad \Tj{\Gamma}{t_2}{\delta}}
                     {\Tjh{\Gamma}{t_1 = t_2}}$ \\[1mm]
  \RN{H-Tc} & $\regel{\Tj{\Gamma}{p}{\assn} \quad \Tje{\Gamma}{e}{\tau} \quad \Tj{\Gamma}{q}{\tau} \tto \assn}
                     {\Tjh{\Gamma}{\triple{p}{e}{q}}}$ \\[1mm]
  \RN{H-Not} & $\regel{\Tjh{\Gamma}{h}}
                      {\Tjh{\Gamma}{\neg h}}$ \\[1mm]
  \RN{H-And} & $\regel{\Tjh{\Gamma}{h_1} \quad \Tjh{\Gamma}{h_2}}
                      {\Tjh{\Gamma}{h_1 \wedge h_2}}$ \\[1mm]
  \RN{H-Exists} & $\regel{\Tjh{\Gamma[\delta/\id]}{h}}
                         {\Tjh{\Gamma}{\exists \id:\delta.h}}$
\end{tabular}


\section{Semantik der Logik}

\subsection{Semantische Bereiche}

F"ur alle Typen $\pi$ und alle endl. $L \subseteq \Loc$ werden induktiv $L$-definierbare
sematische Bereiche $\semantic{\pi}_L$ und definiert. Daraus ergeben sich die eigentlichen
semantischen Bereiche $\semantic{\pi} = \bigcup\limits_{L \text{ endl.}} \semantic{\pi}_L$.
\[\begin{array}{rcl}
  \semantic{\tau}_L &:=& \{v \in \Val^\tau\,|\,\locns(v) = L\} \\
  \semantic{\assn}_L &:=& \{\phi:\Store\pto\Bool\,|\,\text{$\phi$ ist $L$-definierbar und $\dom{\phi} = \Store_L$}\} \\
  \semantic{\delta \tto \pi} &:=& \{f:\semantic{\delta}\to\semantic{\pi}\,|\,
                                       \text{$f(\semantic{\delta}_{L'}) \subseteq \semantic{\pi}_{L\cup L'}$ f.a. endl. $L'\subseteq\Loc$}\}
\end{array}\]

\begin{definition}[Support]
  F"ur alle $d \in \semantic{\pi}$ wird induktiv eine Menge $\supp(d) \subseteq \Loc$ definiert:
  \[\begin{array}{rcl}
    \supp(v) &:=& \locns(v) \\
    \supp(\phi) &:=& L \text{ mit } \dom{\phi} = \Store_L \\
    \supp(f) &:=& \bigcap_{d\in\semantic{\delta}} \supp(f\,d)
  \end{array}\]
\end{definition}

\begin{lemma}[Support]
  F"ur alle $\pi\in\Ltype$, $d\in\semantic{\pi}$ und $L \subseteq \Loc$ endl. gilt:
  Wenn $d \in \semantic{\pi}_L$, dann $L = \supp(d)$.
\end{lemma}

\begin{beweis}
  Beweis durch Induktion:
  \begin{enumerate}
    \item $v \in \semantic{\tau}_L$ gdw. $\locns(v) = L$ gdw. $\supp(v) = L$.

    \item $\phi \in \semantic{\assn}_L$ gdw. $\phi$ ist $L$-definierbar und $\dom{\phi} = \Store_L$.
          Letzteres bedeutet aber nichts anderes als $\supp(\phi) = L$.

    \item $f \in \semantic{\delta \tto \pi}_L$ gdw. $f(\semantic{\delta}_{L'}) \subseteq \semantic{\pi}_{L\cup L'}$
          f.a. endl. $L' \subseteq \Loc$. D.h. f.a. $d \in \semantic{\delta}_{L'}$ gilt $(f\,d) \in \semantic{\pi}_{L \cup L'}$.
          Nach I.V. folgt $L \cup L' = \supp(f\,d)$. Daraus folgt unmittelbar
          $L = \bigcap_{L'\text{ endl.}, d\in\semantic{\delta}_{L'}} \supp(f\,d)
             = \bigcap_{d\in\semantic{\delta}} \supp(f\,d) = \supp(f)$.
  \end{enumerate}
\end{beweis}

\begin{korollar}
  F"ur alle $\pi\in\Ltype$, $L,L' \subseteq \Loc$ endl. gilt:
  Wenn $\semantic{\pi}_L \cap \semantic{\pi}_{L'} \ne \emptyset$, dann $L = L'$.
\end{korollar}

\begin{beweis}
  Sei $d \in \semantic{\pi}$ mit $d \in \semantic{\pi}_L$ und $d \in \semantic{\pi}_{L'}$.
  Dann gilt $L = \supp(d) = L'$.
\end{beweis}


\subsection{Totale Korrektheit}

\begin{definition}[Totale Korrektheit]
  Seien $\tau\in\Type$, $e\in\Exp^\tau$,
  $\phi \in \semantic{\assn}$, $\psi \in \semantic{\tau \tto \assn}$
  und $L \subseteq \Loc$.
  $e$ hei"st {\em $L$-total korrekt} bzgl. $\phi$ und $\psi$, wenn f"ur alle
  $\sigma\in\Store_L$ mit $\phi\,\sigma=\true$ ein $v\in\Val^\tau$ und ein
  $\sigma'\in\Store_L$ existieren, so dass $(e,\sigma) \Downarrow (v,\sigma')$ und
  $\psi\,v\,\sigma'=\true$ gilt. Hierf"ur schreiben wir $\TC{L}{\phi}{e}{\psi}$.
\end{definition}

Statt {\em $L$-total korrekt} schreiben wir kurz {\em $L$-korrekt}.

\begin{lemma}[Totale Korrektheit]
  Seien $\tau\in\Type$, $e \in \Exp^\tau$, $\phi\in\semantic{\assn}$,
  $\psi \in \semantic{\tau\tto\assn}$ und $L \subseteq \Loc$ endl.
  mit $\TC{L}{\phi}{e}{\psi}$. Dann gilt:
  \begin{enumerate}
    \item $\TC{L'}{\phi}{e}{\psi}$ f"ur alle endl. $L' \subseteq \Loc$ mit $L \subseteq L'$.
    \item $\TC{L'}{\phi}{e}{\psi}$ mit $L' = \supp(\phi) \cup \locns(e) \cup \supp(\psi)$.
  \end{enumerate}
\end{lemma}

\begin{beweis}
  F"ur alle $\sigma \in \Store_L$ mit $\phi\,\sigma = \true$ existiert ein $v\in\Val^\tau$ und
  ein $\sigma'\in\Store_L$, so dass $(e,\sigma) \Downarrow (v,\sigma')$ gilt mit $\psi\,v\,\sigma'=\true$.
  \begin{enumerate}
    \item Da $L \subseteq L'$ folgt unmittelbar, dass erst recht f"ur alle $\sigma\in\Store_{L'}$ mit
          $\phi\,\sigma=\true$ solche $v\in\Val^\tau$ und $\sigma'\in\Store_L$ existieren.
          Nach Lemma~\ref{lemma:big_steps_und_dom_sigma} gilt $\dom{\sigma} \subseteq \dom{\sigma'}$, also
          insbesondere $L' \subseteq \dom{\sigma} \subseteq \dom{\sigma'}$, und somit folgt
          $\sigma'\in\Store_{L'}$.

    \item {\bf TODO}
  \end{enumerate}
\end{beweis}


\subsection{Semantik von Termen und Formeln}

Eine {\em Umgebung} ist eine partielle Abbildung $\rho:\Id\pto\bigcup_{\tau\in\Type} \Val^\tau$.
Eine Umgebung $\rho$ {\em passt} zu einer Typumgebung $\Gamma$, geschrieben $\Gamma \models \rho$,
wenn gilt:
\begin{itemize}
  \item $\dom{\rho} = \dom{\Gamma}$
  \item $\rho(\id) \in \Val^{\Gamma(\id)}$ f"ur alle $\id\in\dom{\rho}$
\end{itemize}
Die Menge aller zu $\Gamma$ passenden Umgebungen ist:
\[
  \Env_\Gamma := \{\rho\,|\,\Gamma \models \rho\}
\]

\noindent
Ein Zustand $\sigma$ {\em passt} zu einer Umgebung $\rho$, geschrieben $\rho \models \sigma$,
wenn gilt:
\begin{itemize}
  \item $\locns(\rho(\id)) \subseteq \dom{\sigma}$ f"ur alle $\id\in\dom{\rho}$
  \item $\locns(\sigma(X)) \subseteq \dom{\sigma}$ f"ur alle $X\in\dom{\sigma}$
\end{itemize}
Die Menge aller zu $\rho$ passenden Zust"ande $\Store_\rho$ ist def. durch:
\[
  \Store_\rho := \{\sigma\,|\,\rho\models\sigma\}
\]

\noindent
Den (g"ultigen) Typurteilen f"ur Ausdr"ucke, Terme und Formeln wird nun eine Semantik zugeordnet:
\begin{itemize}
  \item $\semantic{\Tj{\Gamma}{t}{\delta}}:\Env_\Gamma \to \semantic{\delta}$
  \item $\semantic{\Tj{\Gamma}{p}{\theta}}:\Env_\Gamma \to \semantic{\theta}$
  \item $\semantic{\Tjh{\Gamma}{h}}: \Env_\Gamma \to \Bool$
\end{itemize}

\pagebreak[3] \noindent
Die Semantik von Termen ist definiert durch:
\[\begin{array}{rcl}
  \semantic{\Tj{\Gamma}{v}{\tau}}\,\rho
  & := &
  v\,\rho
  \\
  \semantic{\Tj{\Gamma}{f}{\delta}}\,\rho
  & := &
  \semantic{\tj{f}{\delta}} \\
  && \text{wobei die Semantik hier noch zu def. ist}
  \\
  \semantic{\Tj{\Gamma}{\app{t_1}{t_2}}{\delta}}\,\rho
  & := &
  (\semantic{\Tj{\Gamma}{t_1}{\delta'\tto\delta}}\,\rho)\,(\semantic{\Tj{\Gamma}{t_2}{\delta'}}\,\rho)
  \\
  \semantic{\Tj{\Gamma}{\abstr{\id:\delta}{t}}{\delta\tto\delta'}}\,\rho\,d
  & := & 
  \semantic{\Tj{\Gamma[\delta/\id]}{t}{\delta'}}\,(\rho[d/\id])
\end{array}\]

\pagebreak[3] \noindent
Die Semantik von assertions ist wie folgt induktiv definiert:
\[\begin{array}{rcl}
  \semantic{\Tj{\Gamma}{h}{\assn}}\,\rho\,\sigma
  & := &
  \semantic{\Tjh{\Gamma}{h}}\,\rho
  \\
  \semantic{\Tj{\Gamma}{\app{p}{t}}{\theta}}\,\rho
  & := &
  (\semantic{\Tj{\Gamma}{p}{\delta \tto \theta}}\,\rho) (\semantic{\Tj{\Gamma}{t}{\delta}}\,\rho)
  \\
  \semantic{\Tj{\Gamma}{\abstr{\id:\delta}{p}}{\delta \tto \theta}}\,\rho\,d
  & := &
  \semantic{\Tj{\Gamma[\delta/\id]}{p}{\theta}}\,(\rho[d/\id]) 
  \\
  \\
  \semantic{\Tj{\Gamma}{t_1\mapsto t_2}{\assn}}\,\rho\,\sigma
  & := &
  \bcase
    \uparrow, & \text{falls } \sigma \not\in \Store_{\supp(v_1) \cup \supp(v_2)} \\
    \true, & \text{falls ex. $X\in\dom{\sigma}$ mit } X = v_1 \text{ und } \sigma(X) = v_2 \\
    \false, & \text{sonst}
  \ecase \\
  && \text{mit } v_1 = (\semantic{\Tj{\Gamma}{t_1}{\reftype{\tau}}}\,\rho), v_2 = (\semantic{\Tj{\Gamma}{t_2}{\tau}}\,\rho)
  \\
  \\
  \semantic{\Tj{\Gamma}{\disjoint{p}{v}}{\assn}}\,\rho\,\sigma
  & := &
  \bcase
    \uparrow, & \text{falls } \sigma \not\in \Store_{L \cup L'} \\
    \true, & \text{falls } \reach(L,\sigma) \cap \reach(L',\sigma) = \emptyset\\
    \false, & \text{sonst}
  \ecase \\
  && \text{mit } L = \supp(\semantic{\Tj{\Gamma}{v}{\tau}}\,\rho), L' = \supp(\semantic{\Tj{\Gamma}{p}{\theta}}\,\rho)
  \\
  \\
  \semantic{\Tj{\Gamma}{\neg p}{\assn}}\,\rho\,\sigma
  & := &
  \bcase
    \uparrow, & \text{falls } \sigma \not\in \Store_{\supp(\semantic{\Tj{\Gamma}{p}{\assn}}\,\rho)} \\
    \true, & \text{falls } (\semantic{\Tj{\Gamma}{p}{\assn}}\,\rho) = \false \\
    \false, & \text{sonst}
  \ecase
  \\
  \\
  \semantic{\Tj{\Gamma}{p_1 \wedge p_2}{\assn}}\,\rho\,\sigma
  & := &
  \bcase
    \uparrow, & \text{falls } \sigma \not\in \Store_{\supp(d_1) \cup \supp(d_2)} \\
    \true, & \text{falls } d_1 = \true \text{ und } d_2 = \true \\
    \false, & \text{sonst}
  \ecase \\
  && \text{mit } d_1 = (\semantic{\Tj{\Gamma}{p_1}{\assn}}\,\rho), d_2 = (\semantic{\Tj{\Gamma}{p_2}{\assn}}\,\rho)
  \\
  \\
  \semantic{\Tj{\Gamma}{\exists \id:\pi \in [v].p}{\assn}}\,\rho\,\sigma
  & := &
  \bcase
    \true, & \text{falls ex. $d\in\semantic{\pi}_{\locns(v)}$ mit} \\
           & (\semantic{\Tj{\Gamma[\pi/\id]}{p}{\assn}}\,(\rho[d/\id)\,\sigma = \true \\
    \false, & \text{falls f.a. $d\in\semantic{\pi}_{\locns(v)}$ gilt} \\
            & (\semantic{\Tj{\Gamma[\pi/\id]}{p}{\assn}}\,(\rho[d/\id)\,\sigma = \false \\
    \uparrow, & \text{sonst}
  \ecase
\end{array}\]

\noindent
Die Semantik von Hoare-Formeln ist dann definiert durch:
\[\begin{array}{rcl}
  \semantic{\Tjh{\Gamma}{t_1 = t_2}}\,\rho
  & := &
  \true, \text{ gdw. } \semantic{\Tj{\Gamma}{t_1}{\theta}}\,\rho = \semantic{\Tj{\Gamma}{t_2}{\theta}}\,\rho \\
  && \text{wobei die Gleichheit hier noch zu def. ist}
  \\
  \semantic{\Tjh{\Gamma}{\triple{p}{e}{q}}}\,\rho 
  & := &
  \true, \text{ gdw. } \TC{L}{\phi}{(e\,\rho)}{\psi}, \text{ wobei } \phi=\semantic{\Tj{\Gamma}{p}{\assn}}, \\
  && \psi=\semantic{\Tj{\Gamma}{q}{\tau\tto\assn}\,\rho} \text{ und }
  L = \supp(\psi) \cup \supp(e\,\rho) \cup \supp(\psi)

  \\
  \semantic{\Tjh{\Gamma}{\neg h}}\,\rho
  & := &
  \true, \text{ gdw. } \semantic{\Tjh{\Gamma}{h}}\,\rho = \false
  \\
  \semantic{\Tjh{\Gamma}{h_1 \wedge h_2}}\,\rho
  & := &
  \true, \text{ gdw. } \semantic{\Tjh{\Gamma}{h_1}}\,\rho = \true \text{ und }
  \semantic{\Tjh{\Gamma}{h_2}}\,\rho = \true
  \\
  \semantic{\Tjh{\Gamma}{\exists \id:\delta.h}}\,\rho
  & := &
  \true, \text{ gdw. ex. } d \in \semantic{\delta} \text{ so dass }
  \semantic{\Tjh{\Gamma[\delta/\id]}{h}}\,(\rho[d/\id])=\true
\end{array}\]

\begin{satz}[Wohldefiniertheit der Semantik]
  Sei $\Gamma \in \TEnv$ und $\rho \in \Env_\Gamma$. Dann gilt:
  \begin{enumerate}
    \item F"ur alle $t \in \Term$, $\delta \in \Dtype$ mit $\Tj{\Gamma}{t}{\delta}$ gilt
          $(\semantic{\Tj{\Gamma}{t}{\delta}}\,\rho)\in\semantic{\delta}$.
    \item F"ur alle $p \in \Assn$, $\theta \in \Atype$ mit $\Tj{\Gamma}{p}{\theta}$ gilt
          $(\semantic{\Tj{\Gamma}{p}{\theta}}\,\rho)\in\semantic{\theta}$.
  \end{enumerate}
\end{satz}

\begin{beweis}
\end{beweis}

\begin{definition}[Modell] \
  \begin{itemize}
    \item $\rho$ heisst {\em Modell} von $\Tjh{\Gamma}{h}$, wenn $\semantic{\Tjh{\Gamma}{h}}\,\rho = \true$
          gilt, geschrieben als $\rho \models \Tjh{\Gamma}{h}$.
    \item Gilt $\rho \models \Tjh{\Gamma}{h}$ f"ur alle $\rho\in\Env_\Gamma$, so schreiben wir
          $\models \Tjh{\Gamma}{h}$ und nennen  $\Tjh{\Gamma}{h}$ {\em g"ultig}.
    \item Ist $\Gamma = [\,]$, so schreiben wir $\models h$.
  \end{itemize}
\end{definition}


\subsection{Zusammenh"ange}

\begin{lemma}[Allokations-Invarianz]
  F"ur alle $\Gamma\in\TEnv$, $\tau\in\Type$, $y\in\semantic{\tau}$, $p \in\Assn$ mit
  $x \not \in \free{p}$ und $\Tjh{\Gamma}{\triple{p}{\bref\,y}{\ret{x:\reftype{\tau}}{p}}}$ gilt:
  \[
    \models \Tjh{\Gamma}{\triple{p}{\bref\,y}{\ret{x:\reftype{\tau}}{p}}}
  \]
\end{lemma}



\end{document}

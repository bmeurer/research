\documentclass[12pt,a4paper,bigheadings]{scrartcl}

\usepackage{amssymb}
\usepackage{amstext}
\usepackage{amsmath}
\usepackage{array}
\usepackage[ngerman]{babel}
\usepackage{color}
\usepackage{enumerate}
%\usepackage[T1]{fontenc}
\usepackage{german}
\usepackage[a4paper,%
            colorlinks=false,%
            final,%
            pdfkeywords={},%
            pdftitle={},%
            pdfauthor={Benedikt Meurer},%
            pdfsubject={},%
            pdfdisplaydoctitle=true]{hyperref}
\usepackage[latin1]{inputenc}
\usepackage{latexsym}
\usepackage[final]{listings}
\usepackage{makeidx}
%\usepackage{mathpartir}
\usepackage{ngerman}
%\usepackage[standard,thmmarks]{ntheorem}
\usepackage{scrpage2}
\usepackage{stmaryrd}
%\usepackage[DIV13,BCOR5mm]{typearea}
\usepackage{url}
\usepackage[all]{xy}

%% TP-Makros
% General

\newcommand{\name}[1]{{\text{\it #1\/}}}
%\newcommand{\name}[1]{\mathit{#1}}

%\newcommand{\bfbox}[1]{\mathbf{#1}}

\newcommand{\I}{{\cal I}}

% Proof trees

\newcounter{tree}
\newcounter{node}[tree]

\newlength{\treeindent}
\newlength{\nodeindent}
\newlength{\nodesep}

\newcommand{\refnode}[1]
 {\ref{\thetree.#1}}

\newcommand{\contextcolor}{\color{blue}}

\definecolor{darkgreen}{rgb}{0.1,0.5,0}
\definecolor{redexcolor}{rgb}{0.7,1,0}

\newcommand{\resulttypecolor}{\color{darkgreen}}

\newcommand{\resultcolor}{\color{blue}}

\newcommand{\byrulecolor}{\color{red}}

\newcommand{\marked}[1]{\colorbox{redexcolor}{$#1$}}

\newcommand{\smallsteparrow}[1]{\stackrel{\mbox{\scriptsize\byrulecolor (#1)}}{\longrightarrow}}

\newcommand{\byrule}[1]{\hspace{-5mm}\byrulecolor\mbox{\scriptsize\ #1}}

\newif\ifarrows 
\arrowsfalse 

\newcommand{\arrow}[3]
  {\ifarrows
   \ncangle[angleA=-90,angleB=#1]{<-}{\thetree.#2}{\thetree.#3}
   \else
   \fi}

\newcommand{\node}[4]
 {\ifarrows
   \else \refstepcounter{node}
         \noindent\hspace{\treeindent}\hspace{#2\nodeindent}
         \rnode{\thetree.#1}{\makebox[6mm]{(\thenode)}}\label{\thetree.#1}
         $\blong 
          #3 \\ 
          \byrule{#4} 
          \elong$
         \vspace{\nodesep}
   \fi}

\newcommand{\dummyarrow}[3]
  {\arrow{#1}{#2}{#3dummy}}

\newcommand{\dummynode}[2]
 {\ifarrows 
  \else \noindent\hspace{\treeindent}\hspace{#2\nodeindent}
         \rnode{\thetree.#1dummy}{\makebox[6mm]{(\refnode{#1})}}\label{\thetree.#1dummy}
         $\ldots$
         \vspace{\nodesep}
   \fi}

\newcommand{\mktree}[1]
 {\stepcounter{tree} #1 \arrowstrue #1 \arrowsfalse}

\fboxrule=0mm

\newcommand{\cenv}[1]{\fbox{$\begin{array}{|ll|}\hline #1 \\\hline\end{array}$}}


% Special Symbols

\newcommand{\uminus}{\widetilde{\ }}
\renewcommand{\uminus}{-\!}
\newcommand{\nop}{()}

% Im X-Symbol-Manual empfohlen

\newcommand{\nsubset}{\not\subset}
%\newcommand{\textflorin}{\textit{f}}
\newcommand{\setB}{{\mathord{\mathbb B}}}
\newcommand{\setC}{{\mathord{\mathbb C}}}
\newcommand{\setN}{{\mathord{\mathbb N}}}
\newcommand{\setQ}{{\mathord{\mathbb Q}}}
\newcommand{\setR}{{\mathord{\mathbb R}}}
\newcommand{\setZ}{{\mathord{\mathbb Z}}}
\newcommand{\coloncolon}{\mathrel{::}}

% Eigene (k\"urzere) Befehle

\newcommand{\pfi}{\varphi}
\newcommand{\eps}{\varepsilon}
\newcommand{\eval}{\Downarrow}
\newcommand{\pto}{\hookrightarrow}
\newcommand{\emp}{\emptyset}
\newcommand{\sleq}{\subseteq}
\newcommand{\sgeq}{\supseteq}
\newcommand{\sqleq}{\sqsubseteq}
\newcommand{\sqgeq}{\sqsupseteq}
\newcommand{\lub}{\bigsqcup}
\newcommand{\glb}{\bigsqcap}
\newcommand{\lsem}{\llbracket}
\newcommand{\rsem}{\rrbracket}
\newcommand{\sem}[1]{\lsem #1 \rsem}
\newcommand{\impl}{\models}
\newcommand{\step}{\vdash}
%\newcommand{\tr}{\triangleright}
\newcommand{\cc}{\coloncolon}


% Names (lower case italic)

\newcommand{\exn}{\name{exn}}
\newcommand{\id}{\name{id}}
\newcommand{\op}{\name{op}}

\newcommand{\true}{\name{true}}
\newcommand{\false}{\name{false}}
\newcommand{\Not}{\name{not}}

\newcommand{\sol}[3]{\name{solution}\,(#1,#2,#3)}
%\newcommand{\unify}{\name{unify}\,}

\newcommand{\Fst}{\name{fst}}
\newcommand{\Snd}{\name{snd}}

\newcommand{\Hd}{\name{hd}}
\newcommand{\Tl}{\name{tl}}
\newcommand{\Cons}{\name{cons}}
\newcommand{\Empty}{\name{is\_empty}}

\newcommand{\dom}[1]{\name{dom}(#1)}
\newcommand{\free}[1]{\name{free}\,(#1)}
\newcommand{\rank}[1]{\name{rank}\,(#1)}
\newcommand{\len}[1]{\name{len}\,(#1)}
\newcommand{\tr}[1]{\name{tr}(#1)}
\newcommand{\trdB}[1]{\name{tr}_{dB}(#1)}
\newcommand{\locns}[1]{\name{locns}\,(#1)}

% Names (upper case italic)

\newcommand{\Bool}{\name{Bool}}
\newcommand{\Btype}{\name{BType}}

\newcommand{\Conf}{\name{Conf}}
\newcommand{\Const}{\name{Const}}

\newcommand{\Dec}{\name{Dec}}

\newcommand{\Env}{\name{Env}}
\newcommand{\Exn}{\name{Exn}}
\newcommand{\EP}{\name{EP}}
\newcommand{\Exp}{\name{Exp}}

\newcommand{\Ncx}{\name{Ncx}}
\newcommand{\dbExp}{\name{dbExp}}
\newcommand{\dbVal}{\name{dbVal}}
\newcommand{\dbEnv}{\name{dbEnv}}
\newcommand{\dbCl}{\name{dbCl}}

\newcommand{\Id}{\name{Id}}
\newcommand{\Int}{\name{Int}}

\newcommand{\Lexp}{\name{LExp}}
\newcommand{\Loc}{\name{Loc}}

\newcommand{\Prog}{\name{Prog}}
\newcommand{\Instr}{\name{Instr}}
\newcommand{\Reg}{\name{Reg}}
\newcommand{\Stack}{\name{Stack}}
\newcommand{\State}{\name{State}}

\newcommand{\Op}{\name{Op}}

\newcommand{\Type}{\name{Type}}
\newcommand{\Tvar}{\name{TVar}}
\newcommand{\teqns}[3]{\name{teqns}\,(#1,#2,#3)}
\newcommand{\tvar}[1]{\name{tvar}\,(#1)}
\newcommand{\unify}[1]{\name{unify}\,(#1)}

\newcommand{\Ptype}{\name{PType}}

\newcommand{\Unit}{\name{Unit}}

\newcommand{\Val}{\name{Val}}

% keywords

\newcommand{\z}{\mathbf{int}}
\newcommand{\bool}{\mathbf{bool}}
\newcommand{\unit}{\mathbf{unit}}
\newcommand{\blist}{\mathbf{list}}
\newcommand{\bref}{\mathbf{ref}}
\newcommand{\ltype}[1]{#1\,\blist}
\newcommand{\reftype}[1]{#1\,\bref}

\renewcommand{\div}{\mathbin{\mathbf{div}}}
\newcommand{\sel}{\mathbin{.}}
%\newcommand{\mod}{\mathbin{\mathbf{mod}}}
\renewcommand{\mod}{\text{mod}}

\newcommand{\bif}{\mathbf{if}}
\newcommand{\bthen}{\mathbf{then}}
\newcommand{\belse}{\mathbf{else}}

\newcommand{\blet}{\mathbf{let}}
\newcommand{\bin}{\mathbf{in}}
\newcommand{\bend}{\mathbf{end}}

\newcommand{\bval}{\mathbf{val}}
\newcommand{\brec}{\mathbf{rec}}
\newcommand{\bfix}{\mathbf{fix}}
\newcommand{\bfun}{\mathbf{fun}}
\newcommand{\band}{\mathbf{and}}
\newcommand{\btype}{\mathbf{type}}

%\newcommand{\andalso}[2]{#1\,\&\&\,#2}
%\newcommand{\orelse}[2]{#1\,\|\,#2}
\newcommand{\andalso}{\&\&}
\newcommand{\orelse}{\|}
\newcommand{\bandalso}{\mathbf{andalso}}
\newcommand{\borelse}{\mathbf{orelse}}

\newcommand{\bwhile}{\mathbf{while}}
\newcommand{\bdo}{\mathbf{do}}
\newcommand{\bfor}{\mathbf{for}}

\newcommand{\brepeat}{\mathbf{repeat}}
\newcommand{\buntil}{\mathbf{until}}

\newcommand{\barray}{\mathbf{array}}
\newcommand{\bof}{\mathbf{of}}

\newcommand{\bclass}{\mathbf{class}}

\newcommand{\app}[2]{#1\,#2}
\newcommand{\bift}[2]{\bif\ #1\ \bthen\ #2}
\newcommand{\bifte}[3]{\bif\ #1\ \bthen\ #2\ \belse\ #3}
\newcommand{\Bifte}[3]{\blong\bif\ #1\\\bthen\ #2\\\belse\ #3\elong}
\newcommand{\bwd}[2]{\bwhile\ #1\ \bdo\ #2}
\newcommand{\bru}[2]{\brepeat\ #1\ \buntil\ #2}


\newcommand{\blie}[2]{\blet\ #1\ \bin\ #2 \ \bend}
\newcommand{\vdec}[2]{#1 = #2}
\newcommand{\rec}[2]{\brec\,#1.\,#2}
\newcommand{\recdots}[2]{\brec\ #1.\ \ldots}
\newcommand{\abstr}[2]{\lambda #1.\,#2}
\newcommand{\appl}[2]{#1\,#2}
\newcommand{\proj}[1]{\#_{#1}}

\newcommand{\Ref}{\name{ref}}
\newcommand{\Deref}{\,!\,}

\newcommand{\alloc}{\name{alloc}}
\newcommand{\Store}{\name{Store}}




% Typing rules 

\newcommand{\tj}[2]{#1\cc#2}
\newcommand{\ctj}[2]{\tj{#1}{{\resulttypecolor #2}}}
\newcommand{\Tj}[3]{#1 \, \triangleright \, #2\cc#3}
\newcommand{\cTj}[3]{\Tj{{\contextcolor #1}}{#2}{{\resulttypecolor #3}}}
\newcommand{\cbig}[2]{#1 \ \eval\ {\resultcolor #2}}
\newcommand{\cBig}[2]{#1 \\ \eval\ {\resultcolor #2}}
\newcommand{\Clos}[2]{\name{Closure}_{#1}(#2)}

\newcommand{\Tjl}[3]{#1 \,\triangleright_l\, #2\cc#3}
\newcommand{\Tjm}[3]{#1 \,\triangleright_m\, #2\cc#3}
\newcommand{\Tjh}[2]{#1 \triangleright  #2}

\newcommand{\Lj}[3]{#1\,\vdash\,#2\cc#3}

\newcommand{\brule}[1]{\begin{markiere}[#1]}
\newcommand{\erule}{\end{markiere}}

\newcommand{\regel}[2]{\ \begin{array}{@{}c@{}} #1 \\ \hline #2
 \end{array}\ }

\newcommand{\reason}[1]{\ \mbox{#1}}
\newcommand{\Reason}[1]{\vspace{1mm}\\ \mbox{ #1}}


% Program verification

\newcommand{\conj}{\,\land\,}
\newcommand{\Conj}{\bigwedge}
\newcommand{\disj}{\,\lor\,}
\newcommand{\Disj}{\bigvee\,}
\newcommand{\all}[1]{\forall{#1}.\,}
\newcommand{\ex}[1]{\exists{#1}.\,}

\newcommand{\power}[1]{\wp(#1)}

\newcommand{\disjoint}[2]{\name{disj}(#1,#2)}
\newcommand{\cont}[2]{#1 \mapsto #2}
\newcommand{\DEF}{\name{DEF}}

\newcommand{\ret}[2]{{\bf returns}\ #1.\, #2}
\newcommand{\tc}[2]{#1\,\{#2\}}
\newcommand{\triple}[3]{\{#1\}\,#2\,\{#3\}}

% Index

\newcommand{\define}[1]{{\em #1\/}\index{#1}}
\newcommand{\Define}[2]{{\em #1\/}\index{#2}}
\newcommand{\Index}[1]{\index{#1}}
\newcommand{\notation}[1]{#1\index{#1}}
\newcommand{\engl}[1]{(engl.: \define{#1})}
\newcommand{\Engl}[2]{(engl.: \Define{#1}{#2})}

% Theorems etc.

\newtheorem{theorem}{Satz}
\newtheorem{corollary}{Korollar}
\newtheorem{definition}{Definition:}
%\newtheorem{example}{Beispiel:}
%\newtheorem{examples}{Beispiele:}
\newtheorem{lemma}[theorem]{Lemma}
\newtheorem{proposition}[theorem]{Proposition}

%\renewcommand{\thedefinition}{}
%\renewcommand{\theexample}{}
%\renewcommand{\theexamples}{}
\renewcommand{\theenumi}{\rm (\alph{enumi})}
\renewcommand{\labelenumi}{\theenumi}

%\newcommand{\enumarabic}{\renewcommand{\theenumi}{\rm (\arabic{enumi})}}

%\newcommand{\bcoro}[1]{\begin{corollary}\label{cor:#1}}
%\newcommand{\ecoro}{\end{corollary}}

\newcommand{\brdef}[1]{\begin{definition}\label{def:#1}\rm}
\newcommand{\erdef}{\end{definition}}

%\newcommand{\blemm}[1]{\begin{lemma}\label{lem:#1}}
%\newcommand{\bLemm}[2]{\begin{lemma}[#2]\label{lem:#1}\index{#2}}
%\newcommand{\elemm}{\end{lemma}}

%\newcommand{\btheo}[1]{\begin{theorem}\label{th:#1}}
%\newcommand{\bTheo}[2]{\begin{theorem}[#2]\label{th:#1}\index{#2}}
%\newcommand{\etheo}{\end{theorem}}

%\newcommand{\litem}[1]{\item\label{it:#1}}
%\newcommand{\ritem}[1]{\ref{it:#1}}


% Grammars

\newcommand{\bgram}{\[\begin{array}{rrlll}}
\newcommand{\egram}{\end{array}\]}

\newcommand{\is}{& ::= &}
\newcommand{\al}{\\ & \mid &}
\newcommand{\n}{\vspace{2mm}\\}




% Other Environments

\newcommand{\bcase}{\left\{\!\!\!\begin{array}{ll}}
\newcommand{\ecase}{\end{array}\right.}
\newcommand{\benum}{\begin{enumerate}}
\newcommand{\eenum}{\end{enumerate}}
\newcommand{\beqns}{\[\begin{array}{rcll}}
\newcommand{\eeqns}{\end{array}\]}
\newcommand{\bitem}{\begin{itemize}}
\newcommand{\eitem}{\end{itemize}}
\newcommand{\blong}{\!\!\begin{array}[t]{l}}
\newcommand{\elong}{\end{array}}
\newcommand{\btabl}{\begin{tabular}}
\newcommand{\etabl}{\end{tabular}}
\newcommand{\brexa}{\begin{example}\enumarabic\rm}
\newcommand{\erexa}{\end{example}}
\newcommand{\brexs}{\begin{examples}\enumarabic\rm}
\newcommand{\erexs}{\end{examples}}


% German abbreviations

\newcommand{\abk}[1]{#1.\ }
\newcommand{\bzw}{\abk{bzw}}
\newcommand{\bzgl}{\abk{bzgl}}
\newcommand{\das}{\abk{d.h}}
\newcommand{\evtl}{\abk{evtl}}
\newcommand{\usw}{\abk{usw}}
\newcommand{\vgl}{\abk{vgl}}
\newcommand{\zb}{\abk{z.B}}


\newcommand{\infix}[3]{#2\mathbin{#1}#3}

\newcommand{\bli}[3]{\blet\ \vdec{#1}{#2}\ \bin\ #3}
\newcommand{\blidb}[2]{\blet\ {#1}\ \bin\ {#2}}
\newcommand{\blri}[3]{\blet\,\brec\ \vdec{#1}{#2}\ \bin\ #3}

\newcommand{\Bli}[3]{\begin{array}[t]{@{}l}
                     \blet\ \vdec{#1}{#2}\\\bin\ #3
                     \end{array}}

\newcommand{\Vdec}[2]{\begin{array}[t]{@{}l}
                      #1 = \\
                      \ #2
                      \end{array}}

\newcommand{\BLI}[3]{\begin{array}[t]{@{}l}
                     \blet\ \Vdec{#1}{#2}\\\bin\ #3
                     \end{array}}

\newcommand{\Abstr}[2]{\begin{array}[t]{@{}l}
                       \lambda #1.\\\ \ #2
                       \end{array}}


\newcommand{\semantic}[1]{\ensuremath{\llbracket#1\rrbracket}}
\newcommand{\assn}{\ensuremath{\mathbf{assn}}}
\newcommand{\atype}[1]{#1\,\assn}
\newcommand{\bexpr}{\ensuremath{\mathbf{expr}}}
\newcommand{\etype}[1]{#1\,\bexpr}
\newcommand{\Stype}{\name{SType}}
\newcommand{\locns}{\name{locns}}
\newcommand{\tto}{\ensuremath{\xrightarrow{t}}}
\renewcommand{\disjoint}[2]{\ensuremath{#2\not\hookrightarrow#1}}
\newcommand{\view}[1]{\ensuremath{\app{\name{view}}{#1}}}
\newcommand{\RN}[1]{\mbox{{\sc (#1)}}}


\begin{document}

\section{Syntax der Logik}

\subsection{Typen}

Sei $\Type$ die Menge aller {\em Typen} $\tau$ der Programmiersprache. Darauf definieren wir 
\begin{itemize}
  \item die Menge $\Stype$ aller {\em statischen Typen} $\theta$ und
  \item die Menge $\Ltype$ aller {\em logischen Typen} $\pi$
\end{itemize}
durch
\[\bgram
\theta  \is \tau
        \al \ltype{\theta}
        \al \theta_1 \tto \theta_2
        \al \theta_1 \times \ldots \times \theta_n \quad (n \ge 2)
        \n
\pi \is \theta
    \al \etype{\tau}
    \al \atype{\theta}
    \n
\egram\]
Statt $\atype{\unit}$ schreiben wir kurz $\assn$.

\subsection{Terme und Formeln}

Sei $\Exp$ die Menge aller {\em Ausdr"ucke} $e$ und $\Val$ die Menge aller {\em Werte} $v$
der Programmiersprache (wobei angenommen wird, dass $\Id \subseteq \Val$). Darauf definieren wir 
\begin{itemize}
  \item die Menge $F$ aller {\em Funktionszeichen} $f$,
  \item die Menge \name{Term} aller {\em Terme} $t$,
  \item die Menge \name{Assn} aller {\em assertions} $p,q,r,s$ sowie
  \item die Menge \name{Formula} aller {\em (Hoare-)Formeln} $h$
\end{itemize}
durch
\[\bgram
f \is + \mid - \mid * \mid < \mid > \mid \le \mid \ge \mid \Cons \mid \Hd \mid \Tl
  \n
t \is v
  \al f
  \al \app{t_1}{t_2}
  \al \abstr{\id:\theta}{t}
  \al (t_1,\ldots,t_n)  \quad (n \ge 2)
  \n
p,q,r,s \is t_1 = t_2
        \al t_1 \mapsto t_2
        \al \disjoint{p}{v}
        \al \app{p}{t}
        \al \abstr{\id:\theta}{p}
        \al \neg p
        \al p_1 \wedge p_2
        \al \exists \id:\theta. p
        \al h
        \n
h \is \triple{p}{e}{q}
  \al \view{p}
  \al \neg h
  \al h_1 \wedge h_2
  \al \exists \id:\theta. h
\egram\]


\section{Typregeln f"ur die Logik}

Der Begriff {\em Typumgebung} wird verallgemeinert f"ur die Logik. Eine {\em Typumgebung} ist
damit eine partielle Funktion
\[
  \Gamma: \Id \pto \Stype
\]
mit endlichem Definitionsbereich. {\em Typurteile} f"ur die Logik sind von der Form
\[\begin{array}{l}
  \Tj{\Gamma}{e}{\etype{\tau}} \\
  \Tj{\Gamma}{t}{\theta} \\
  \Tj{\Gamma}{p}{\atype{\theta}} \\
  \Tjh{\Gamma}{h}
\end{array}\]
Ein Typurteil $\Tj{\Gamma}{e}{\etype{\tau}}$ f"ur Ausdr"ucke der Programmiersprache hei"st
g"ultig, wenn das entsprechende Typurteil $\Tj{\Gamma}{e}{\tau}$ herleitbar ist. F"ur die
"ubrigen Typurteile werden neue Typregeln ben"otigt.

Die g"ultigen Typurteile f"ur assertions erhalten wir mit den Regeln \\[3mm]
\begin{tabular}{rl}
  \RN{A-Eq} & $\regel{\Tj{\Gamma}{t_1}{\theta} \quad \Tj{\Gamma}{t_2}{\theta}}
                     {\Tj{\Gamma}{t_1 = t_2}{\assn}}$ \\[1mm]
  \RN{A-App} & $\regel{\Tj{\Gamma}{p}{\atype{\theta}} \quad \Tj{\Gamma}{t}{\theta}}
                      {\Tj{\Gamma}{\app{p}{t}}{\assn}}$ \\[1mm]
  \RN{A-Content} & $\regel{\Tj{\Gamma}{t_1}{\reftype{\tau}} \quad \Tj{\Gamma}{t_2}{\tau}}
                          {\Tj{\Gamma}{t_1 \mapsto t_2}{\assn}}$ \\[1mm]
  \RN{A-Abstr} & $\regel{\Tj{\Gamma[\theta/\id]}{p}{\assn}}
                        {\Tj{\Gamma}{\abstr{\id:\theta}{p}}{\atype{\theta}}}$ \\[1mm]
  \RN{A-Disjoint} & $\regel{\Tj{\Gamma}{p}{\atype{\theta}} \quad \Tj{\Gamma}{v}{\tau}}
                           {\Tj{\Gamma}{\disjoint{p}{v}}{\assn}}$ \\[1mm]
  \RN{A-Not} & $\regel{\Tj{\Gamma}{p}{\assn}}
                      {\Tj{\Gamma}{\neg p}{\assn}}$ \\[1mm]
  \RN{A-And} & $\regel{\Tj{\Gamma}{p_1}{\assn} \quad \Tj{\Gamma}{p_2}{\assn}}
                      {\Tj{\Gamma}{p_1 \wedge p_2}{\assn}}$ \\[1mm]
  \RN{A-Exists} & $\regel{\Tj{\Gamma[\theta/\id]}{p}{\assn}}
                         {\Tj{\Gamma}{\exists \id:\theta.p}{\assn}}$ \\[1mm]
  \RN{A-Hoare} & $\regel{\Tjh{\Gamma}{h}}
                        {\Tj{\Gamma}{h}{\assn}}$
\end{tabular} \\[3mm]
die g"ultigen Typurteile f"ur Terme erhalten wir mit den Regeln \\[3mm]
\begin{tabular}{rl}
  \RN{T-Val} & $\regel{\Tj{\Gamma}{v}{\etype{\tau}}}
                      {\Tj{\Gamma}{v}{\tau}}$ \\[1mm]
  \RN{T-Func} & $\regel{\tj{f}{\theta}}
                       {\Tj{\Gamma}{f}{\theta}}$ \\[1mm]
  \RN{T-App} & $\regel{\Tj{\Gamma}{t_1}{\theta \tto \theta'} \quad \Tj{\Gamma}{t_2}{\theta}}
                      {\Tj{\Gamma}{\app{t_1}{t_2}}{\theta'}}$ \\[1mm]
  \RN{T-Abstr} & $\regel{\Tj{\Gamma[\theta/\id]}{t}{\theta'}}
                        {\Tj{\Gamma}{\abstr{\id:\theta}{t}}{\theta \tto \theta'}}$ \\[1mm]
  \RN{T-Tuple} & $\regel{\Tj{\Gamma}{t_i}{\theta_i} \text{ f"ur alle } i=1,\ldots,n}
                        {\Tj{\Gamma}{(t_1,\ldots,t_n)}{\theta_1\times\ldots\times\theta_n}}$
\end{tabular} \\[3mm]
und die g"ultigen Typurteile f"ur Hoare-Formeln mit \\[3mm]
\begin{tabular}{rl}
  \RN{H-Tc} & $\regel{\Tj{\Gamma}{p}{\assn} \quad \Tj{\Gamma}{e}{\etype{\tau}} \quad \Tj{\Gamma}{q}{\atype{\tau}}}
                     {\Tjh{\Gamma}{\triple{p}{e}{q}}}$ \\[1mm]
  \RN{H-View} & $\regel{\Tj{\Gamma}{p}{\atype{\theta}}}
                       {\Tjh{\Gamma}{\view{p}}}$ \\[1mm]
  \RN{H-Not} & $\regel{\Tjh{\Gamma}{h}}
                      {\Tjh{\Gamma}{\neg h}}$ \\[1mm]
  \RN{H-And} & $\regel{\Tjh{\Gamma}{h_1} \quad \Tjh{\Gamma}{h_2}}
                      {\Tjh{\Gamma}{h_1 \wedge h_2}}$ \\[1mm]
  \RN{H-Exists} & $\regel{\Tjh{\Gamma[\theta/\id]}{h}}
                         {\Tjh{\Gamma}{\exists \id:\theta.h}}$
\end{tabular}


\section{Semantik der Logik}

\subsection{Semantische Bereiche}

Es werden die folgenden {\em semantischen Bereiche} $\semantic{\pi}$ f"ur die logischen Typen $\pi$
definiert:
\[\begin{array}{rcl}
  \semantic{\tau} & = & \Val^\tau = \{v\in\Val\,|\,\tj{v}{\tau}\} \\
  \semantic{\ltype{\theta}} & = & ?????? \\
  \semantic{\theta_1 \tto \theta_2} & = & \semantic{\theta_1} \to \semantic{\theta_2} \\
  \semantic{\theta_1\times\ldots\times\theta_n}&=&\semantic{\theta_1}\times\ldots\times\semantic{\theta_n} \\
  \semantic{\etype{\tau}} & = & \Store \pto \semantic{\tau} \times \Store \\
  \semantic{\atype{\theta}} & = & \semantic{\theta} \times \Store \pto \Bool
\end{array}\]
Da $\semantic{\unit} = \Val^\unit = \{()\}$ und $\{()\}\times\Store$ isomorph zu
$\Store$ benutzen wir abk"urzend $\semantic{\assn}=\Store\pto\Bool$ und vernachl"assigen
an den entsprechenden Stellen das $\unit$-Element.

\begin{definition}[Totale Korrektheit]
  Seien $\tau\in\Type$, $f\in\semantic{\etype{\tau}}$, $\phi\in\semantic{\assn}$ und
  $\psi\in\semantic{\atype{\tau}}$. $f$ heisst {\em total korrekt} bzgl. $\phi$ und
  $\psi$ gdw.:
  \[\begin{array}{l}
    \forall \sigma\in\dom{\phi}.
    \bigl[
      \phi\,\sigma = \true 
      \Rightarrow \exists(v,\sigma')\in\dom{\psi}.f\,\sigma = (v,\sigma')\wedge \psi\,(v,\sigma')=\true
    \bigr]
  \end{array}\]
\end{definition}

\subsection{Semantik von Termen und Formeln}

Eine {\em Umgebung} ist eine partielle Abbildung $\rho:\Id\pto\bigcup_{\tau\in\Type} \Val^\tau$.
Eine Umgebung $\rho$ {\em passt} zu einer Typumgebung $\Gamma$, geschrieben $\Gamma \models \rho$,
wenn gilt:
\begin{itemize}
  \item $\dom{\rho} = \dom{\Gamma}$
  \item $\rho(\id) \in \Val^{\Gamma(\id)}$ f"ur alle $\id\in\dom{\rho}$
\end{itemize}
Die Menge aller zu $\Gamma$ passenden Umgebungen ist:
\[
  \Env_\Gamma := \{\rho\,|\,\Gamma \models \rho\}
\]

\noindent
Ein Zustand $\sigma$ {\em passt} zu einer Umgebung $\rho$, geschrieben $\rho \models \sigma$,
wenn gilt:
\begin{itemize}
  \item $\locns(\rho(\id)) \subseteq \dom{\sigma}$ f"ur alle $\id\in\dom{\rho}$
  \item $\locns(\sigma(X)) \subseteq \dom{\sigma}$ f"ur alle $X\in\dom{\sigma}$
\end{itemize}
Die Menge aller zu $\rho$ passenden Zust"ande $\Store_\rho$ ist def. durch:
\[
  \Store_\rho := \{\sigma\,|\,\rho\models\sigma\}
\]

\noindent
Den Typurteilen f"ur Ausdr"ucke, Terme und Formeln wird nun eine Semantik zugeordnet:
\begin{itemize}
  \item $\semantic{\Tj{\Gamma}{e}{\etype{\tau}}}:\Env_\Gamma \to \semantic{\etype{\tau}}$, \\
        mit $\dom{\semantic{\Tj{\Gamma}{e}{\etype{\tau}}}\,\rho} = \Store_\rho$
        f"ur alle $\rho\in\Env_\Gamma$
  \item $\semantic{\Tj{\Gamma}{t}{\theta}}:\Env_\Gamma \to \semantic{\theta}$
  \item $\semantic{\Tj{\Gamma}{p}{\atype{\theta}}}: \Env_\Gamma \to \semantic{\atype{\theta}}$, \\
        mit $\dom{\semantic{\Tj{\Gamma}{p}{\atype{\theta}}}\,\rho} = \semantic{\theta}\times\Store_\rho$
        f"ur alle $\rho\in\Env_\Gamma$
  \item $\semantic{\Tjh{\Gamma}{h}}: \Env_\Gamma \to \Bool$
\end{itemize}

\noindent
F"ur Ausdr"ucke kann die Semantik wahlweise "uber small oder big step Semantik definiert
werden, z.B.
\[\begin{array}{rcl}
  \semantic{\Tj{\Gamma}{e}{\etype{\tau}}}\,\rho\,\sigma = (v,\sigma')
  & :\Leftrightarrow &
  (e\rho,\sigma) \Downarrow (v,\sigma')
\end{array}\]
wobei $e \rho$ der Ausdruck ist, der aus $e$ entsteht, indem alle durch $\rho$ spezifizierten
Substitutionen ausgef"uhrt werden. Da $\Gamma \models \rho$, ist $e \rho$ somit ein abgeschlossener
Ausdruck vom Typ $\tau$.

\pagebreak[3] \noindent
Die Semantik von Termen ist definiert durch:
\[\begin{array}{rcl}
  \semantic{\Tj{\Gamma}{v}{\tau}}\,\rho
  & := &
  v\,\rho 
  \\
  \semantic{\Tj{\Gamma}{f}{\theta}}\,\rho
  & := &
  \semantic{\tj{f}{\theta}} \\
  && \text{wobei die Semantik hier noch zu def. ist}
  \\
  \semantic{\Tj{\Gamma}{\app{t_1}{t_2}}{\theta}}\,\rho
  & := &
  (\semantic{\Tj{\Gamma}{t_1}{\theta'\tto\theta}}\,\rho)\,(\semantic{\Tj{\Gamma}{t_2}{\theta'}}\,\rho)
  \\
  \semantic{\Tj{\Gamma}{\abstr{\id:\theta}{t}}{\theta\tto\theta'}}\,\rho
  & := &
  f: \semantic{\theta\tto\theta'},
  d \mapsto \semantic{\Tj{\Gamma[\theta/\id]}{t}{\theta'}}\,(\rho[d/\id])
  \\
  \semantic{\Tj{\Gamma}{(t_1,\ldots,t_n)}{\theta_1\times\ldots\times\theta_2}}\,\rho
  & := &
  (\semantic{\Tj{\Gamma}{t_1}{\theta_1}}\,\rho,\ldots,\semantic{\Tj{\Gamma}{t_2}{\theta_2}}\,\rho)
\end{array}\]

\pagebreak[3] \noindent
Die Semantik von assertions ist wie folgt induktiv definiert:
\[\begin{array}{rcl}
  \semantic{\Tj{\Gamma}{t_1 = t_2}{\assn}}\,\rho\,\sigma = \true
  & :\Leftrightarrow &
  \semantic{\Tj{\Gamma}{t_1}{\theta}}\,\rho = \semantic{\Tj{\Gamma}{t_2}{\theta}}\,\rho \\
  && \text{wobei die Gleichheit hier noch zu def. ist}
  \\
  \semantic{\Tj{\Gamma}{t_1\mapsto t_2}{\assn}}\,\rho\,\sigma = \true
  & :\Leftrightarrow &
  \text{ex. } X\in\dom{\sigma} \text{ mit } X = \semantic{\Tj{\Gamma}{t_1}{\reftype{\tau}}}\,\rho \\
  && \text{und } \sigma(X) = \semantic{\Tj{\Gamma}{t_2}{\tau}}\,\rho
  \\
  \semantic{\Tj{\Gamma}{\disjoint{p}{v}}{\assn}}\,\rho\,\sigma = \true
  & :\Leftrightarrow &
  \text{ex. endl. } L \subseteq \Loc \text{ so dass } \semantic{\Tj{\Gamma}{p}{\assn}}\,\rho \\
  && \text{$L$-definierbar ist, und } \\
  && \name{reach}(L,\sigma) \cap \name{reachable}(\locns(v))=\emptyset
  \\
  \semantic{\Tj{\Gamma}{\app{p}{t}}{\assn}}\,\rho\,\sigma = \true
  & :\Leftrightarrow &
  (\semantic{\Tj{\Gamma}{p}{\atype{\theta}}}\,\rho)\,(\semantic{\Tj{\Gamma}{t}{\theta}}\,\rho,\sigma)=\true
  \\
  \semantic{\Tj{\Gamma}{\abstr{\id:\theta}{p}}{\atype{\theta}}}\,\rho\,(d,\sigma) = \true
  & :\Leftrightarrow &
  \semantic{\Tj{\Gamma[\theta/\id]}{p}{\assn}}\,(\rho[d/\id])\,\sigma = \true
  \\
  \semantic{\Tj{\Gamma}{\neg p}{\assn}}\,\rho\,\sigma = \true
  & :\Leftrightarrow &
  \semantic{\Tj{\Gamma}{p}{\assn}}\,\rho\,\sigma = \false
  \\
  \semantic{\Tj{\Gamma}{p_1 \wedge p_2}{\assn}}\,\rho\,\sigma = \true
  & :\Leftrightarrow &
  \semantic{\Tj{\Gamma}{p_1}{\assn}}\,\rho\,\sigma = \true \text{ und } \\
  && \semantic{\Tj{\Gamma}{p_2}{\assn}}\,\rho\,\sigma = \true
  \\
  \semantic{\Tj{\Gamma}{\exists \id:\theta.p}{\assn}}\,\rho\,\sigma = \true
  & :\Leftrightarrow &
  \text{ex. } d\in\semantic{\theta} \text{ so dass } \\
  && \semantic{\Tj{\Gamma[\theta/\id]}{p}{\assn}}\,(\rho[d/\id])\,\sigma = \true
  \\
  \semantic{\Tj{\Gamma}{h}{\assn}}\,\rho\,\sigma = \true
  & :\Leftrightarrow &
  \semantic{\Tjh{\Gamma}{h}}\,\rho = \true
\end{array}\]

\noindent
Die Semantik von Hoare-Formeln ist dann definiert durch:
\[\begin{array}{rcl}
  \semantic{\Tjh{\Gamma}{\triple{p}{e}{q}}}\,\rho = \true
  & :\Leftrightarrow &
  \semantic{\Tj{\Gamma}{e}{\etype{\tau}}}\,\rho \text{ total korrekt bzgl. } \\
  && \semantic{\Tj{\Gamma}{p}{\assn}}\,\rho \text{ und } \semantic{\Tj{\Gamma}{q}{\atype{\tau}}}\,\rho
  \\
  \semantic{\Tjh{\Gamma}{\view{p}}}\,\rho = \true
  & :\Leftrightarrow &
  \text{ex. endl. } L \subseteq \Loc \text{ so dass } \semantic{\Tj{\Gamma}{p}{\assn}}\,\rho \\
  && \text{$L$-definierbar ist}
  \\
  \semantic{\Tjh{\Gamma}{\neg h}}\,\rho = \true
  & :\Leftrightarrow &
  \semantic{\Tjh{\Gamma}{h}}\,\rho = \false
  \\
  \semantic{\Tjh{\Gamma}{h_1 \wedge h_2}}\,\rho = \true
  & :\Leftrightarrow &
  \semantic{\Tjh{\Gamma}{h_1}}\,\rho = \true \text{ und }
  \semantic{\Tjh{\Gamma}{h_2}}\,\rho = \true
  \\
  \semantic{\Tjh{\Gamma}{\exists \id:\theta.h}}\,\rho = \true
  & :\Leftrightarrow &
  \text{ex. } d \in \semantic{\theta} \text{ so dass }
  \semantic{\Tjh{\Gamma[\theta/\id]}{h}}\,(\rho[d/\id])=\true
\end{array}\]

\pagebreak[3]
\begin{definition}[Modell] \
  \begin{itemize}
    \item $\rho$ heisst {\em Modell} von $\Tjh{\Gamma}{h}$, wenn $\semantic{\Tjh{\Gamma}{h}}\,\rho = \true$
          gilt, geschrieben als $\rho \models \Tjh{\Gamma}{h}$.
    \item Gilt $\rho \models \Tjh{\Gamma}{h}$ f"ur alle $\rho\in\Env_\Gamma$, so schreiben wir
          $\models \Tjh{\Gamma}{h}$ und nennen  $\Tjh{\Gamma}{h}$ {\em g"ultig}.
    \item Ist $\Gamma = [\,]$, so schreiben wir $\models h$.
  \end{itemize}
\end{definition}


\section{Kalk"ul}

Ableitbarkeit: $\vdash h$

\subsection{Regeln}


\end{document}

\documentclass[12pt,a4paper,bigheadings]{scrartcl}

\usepackage{amssymb}
\usepackage{amstext}
\usepackage{amsmath}
\usepackage{array}
\usepackage[ngerman]{babel}
\usepackage{color}
\usepackage{enumerate}
%\usepackage[T1]{fontenc}
\usepackage{german}
\usepackage[a4paper,%
            colorlinks=false,%
            final,%
            pdfkeywords={},%
            pdftitle={},%
            pdfauthor={Benedikt Meurer},%
            pdfsubject={},%
            pdfdisplaydoctitle=true]{hyperref}
\usepackage[latin1]{inputenc}
\usepackage{latexsym}
\usepackage[final]{listings}
\usepackage{makeidx}
%\usepackage{mathpartir}
\usepackage{ngerman}
\usepackage[standard,thmmarks]{ntheorem}
\usepackage{scrpage2}
\usepackage{stmaryrd}
%\usepackage[DIV13,BCOR5mm]{typearea}
\usepackage{url}
\usepackage[all]{xy}

%% BM-Makros
\include{BM}

%\newcommand{\semantic}[1]{\ensuremath{\llbracket#1\rrbracket}}
%\newcommand{\assn}{\ensuremath{\mathbf{assn}}}
%\newcommand{\atype}[1]{#1\,\assn}
%\newcommand{\bexpr}{\ensuremath{\mathbf{expr}}}
%\newcommand{\TEnv}{\name{TEnv}}
%\newcommand{\Dtype}{\name{DType}}
%\newcommand{\Atype}{\name{AType}}
%\newcommand{\Assn}{\name{Assn}}
%\newcommand{\Term}{\name{Term}}
%\newcommand{\locns}[1]{\name{locns}(#1)}
%\newcommand{\reach}[1]{\name{reach}(#1)}
%\newcommand{\grph}[1]{\name{graph}(#1)}
%\newcommand{\reachable}[1]{\name{reachable}(#1)}
%\newcommand{\supp}[1]{\name{supp}(#1)}
%\newcommand{\tto}{\ensuremath{\xrightarrow{t}}}
%\renewcommand{\disjoint}[2]{\ensuremath{#2\not\hookrightarrow#1}}
%\newcommand{\RN}[1]{\mbox{{\sc (#1)}}}
%\newcommand{\Tje}[3]{#1 \,\tr_e\, #2\cc#3}
%\newcommand{\TC}[4]{#1 \models \{#2\}\,#3\,\{#4\}}


\begin{document}


%%
%% Die Programmiersprache
%%

\section{Die Programmiersprache}

\subsection{Syntax}

\begin{definition}[Typen der Programmiersprache] \label{definition:Typen_der_Programmiersprache}
  Die Menge $\setType$ aller \Define{Typen}{Typ} $\tau$ ist durch die kontextfreie Grammatik
  \[\GRbeg
  \tau \GRis \typeBool \GRmid \typeInt \GRmid \typeUnit \GRmid \typeArrow{\tau_1}{\tau_2}
  \GRend\]
  definiert.
\end{definition}

\begin{definition}[Abstrakte Syntax der Programmiersprache] \label{definition:Abstrakte_Syntax_der_Programmiersprache}
  Zu jedem Typ $\tau\in\setType$ sei eine unendliche Menge $\setLoc^\tau$ von
  \Define{$\tau$-Speicherpl"atzen}{$\tau$-Speicherplatz}
  vorgegeben, wobei vorausgesetzt wird, dass die Mengen $\setLoc^\tau$ paarweise disjunkt
  sind. Die Menge $\setLoc$ aller \Define{Speicherpl"atze}{Speicherplatz} $\loc$ ist definiert durch
  \[
    \setLoc = \bigcup_{\tau\in\setType} \setLoc^\tau.
  \]
  Die Menge $\setOp$ aller Operatoren $\op$ ist definiert durch die kontextfreie Grammatik
  \[\begin{array}{rrcccccccccl}
  \op \GRis + & \mid & - & \mid &  *  &      &     &      &
      \GRal < & \mid & > & \mid & \le & \mid & \ge & \mid & =
      \GRal :=
  \end{array}\]
  die Menge $\setConst$ aller Konstanten $c$ durch
  \[\GRbeg
  c \GRis () \GRmid b \GRmid n \GRmid \op \GRmid \loc
  \GRend\]
  und die Menge $\setExp$ aller \Define{Ausdr\"ucke}{Ausdruck} $e$ von \Lf\ durch
  \[\GRbeg
  e \GRis c
    \GRal x
    \GRal \expApp{e_1}{e_2}
    \GRal \expAbstr{x}{e_1}
    \GRal \expRec{x}{e_1}
    \GRal \expLet{x}{e_1}{e_2}
    \GRal \expCond{e_0}{e_1}{e_2}
  \GRend\]
\end{definition}


\end{document}

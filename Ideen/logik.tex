\documentclass[12pt,a4paper,bigheadings]{scrartcl}

\usepackage{amssymb}
\usepackage{amstext}
\usepackage{amsmath}
\usepackage{array}
\usepackage[ngerman]{babel}
\usepackage{color}
\usepackage{enumerate}
%\usepackage[T1]{fontenc}
\usepackage{german}
\usepackage[a4paper,%
            colorlinks=false,%
            final,%
            pdfkeywords={},%
            pdftitle={},%
            pdfauthor={Benedikt Meurer},%
            pdfsubject={},%
            pdfdisplaydoctitle=true]{hyperref}
\usepackage[latin1]{inputenc}
\usepackage{latexsym}
\usepackage[final]{listings}
\usepackage{makeidx}
%\usepackage{mathpartir}
\usepackage{ngerman}
\usepackage[standard,thmmarks]{ntheorem}
\usepackage{scrpage2}
\usepackage{stmaryrd}
%\usepackage[DIV13,BCOR5mm]{typearea}
\usepackage{url}
\usepackage[all]{xy}

%% TP-Makros
% General

\newcommand{\name}[1]{{\text{\it #1\/}}}
%\newcommand{\name}[1]{\mathit{#1}}

%\newcommand{\bfbox}[1]{\mathbf{#1}}

\newcommand{\I}{{\cal I}}

% Proof trees

\newcounter{tree}
\newcounter{node}[tree]

\newlength{\treeindent}
\newlength{\nodeindent}
\newlength{\nodesep}

\newcommand{\refnode}[1]
 {\ref{\thetree.#1}}

\newcommand{\contextcolor}{\color{blue}}

\definecolor{darkgreen}{rgb}{0.1,0.5,0}
\definecolor{redexcolor}{rgb}{0.7,1,0}

\newcommand{\resulttypecolor}{\color{darkgreen}}

\newcommand{\resultcolor}{\color{blue}}

\newcommand{\byrulecolor}{\color{red}}

\newcommand{\marked}[1]{\colorbox{redexcolor}{$#1$}}

\newcommand{\smallsteparrow}[1]{\stackrel{\mbox{\scriptsize\byrulecolor (#1)}}{\longrightarrow}}

\newcommand{\byrule}[1]{\hspace{-5mm}\byrulecolor\mbox{\scriptsize\ #1}}

\newif\ifarrows 
\arrowsfalse 

\newcommand{\arrow}[3]
  {\ifarrows
   \ncangle[angleA=-90,angleB=#1]{<-}{\thetree.#2}{\thetree.#3}
   \else
   \fi}

\newcommand{\node}[4]
 {\ifarrows
   \else \refstepcounter{node}
         \noindent\hspace{\treeindent}\hspace{#2\nodeindent}
         \rnode{\thetree.#1}{\makebox[6mm]{(\thenode)}}\label{\thetree.#1}
         $\blong 
          #3 \\ 
          \byrule{#4} 
          \elong$
         \vspace{\nodesep}
   \fi}

\newcommand{\dummyarrow}[3]
  {\arrow{#1}{#2}{#3dummy}}

\newcommand{\dummynode}[2]
 {\ifarrows 
  \else \noindent\hspace{\treeindent}\hspace{#2\nodeindent}
         \rnode{\thetree.#1dummy}{\makebox[6mm]{(\refnode{#1})}}\label{\thetree.#1dummy}
         $\ldots$
         \vspace{\nodesep}
   \fi}

\newcommand{\mktree}[1]
 {\stepcounter{tree} #1 \arrowstrue #1 \arrowsfalse}

\fboxrule=0mm

\newcommand{\cenv}[1]{\fbox{$\begin{array}{|ll|}\hline #1 \\\hline\end{array}$}}


% Special Symbols

\newcommand{\uminus}{\widetilde{\ }}
\renewcommand{\uminus}{-\!}
\newcommand{\nop}{()}

% Im X-Symbol-Manual empfohlen

\newcommand{\nsubset}{\not\subset}
%\newcommand{\textflorin}{\textit{f}}
\newcommand{\setB}{{\mathord{\mathbb B}}}
\newcommand{\setC}{{\mathord{\mathbb C}}}
\newcommand{\setN}{{\mathord{\mathbb N}}}
\newcommand{\setQ}{{\mathord{\mathbb Q}}}
\newcommand{\setR}{{\mathord{\mathbb R}}}
\newcommand{\setZ}{{\mathord{\mathbb Z}}}
\newcommand{\coloncolon}{\mathrel{::}}

% Eigene (k\"urzere) Befehle

\newcommand{\pfi}{\varphi}
\newcommand{\eps}{\varepsilon}
\newcommand{\eval}{\Downarrow}
\newcommand{\pto}{\hookrightarrow}
\newcommand{\emp}{\emptyset}
\newcommand{\sleq}{\subseteq}
\newcommand{\sgeq}{\supseteq}
\newcommand{\sqleq}{\sqsubseteq}
\newcommand{\sqgeq}{\sqsupseteq}
\newcommand{\lub}{\bigsqcup}
\newcommand{\glb}{\bigsqcap}
\newcommand{\lsem}{\llbracket}
\newcommand{\rsem}{\rrbracket}
\newcommand{\sem}[1]{\lsem #1 \rsem}
\newcommand{\impl}{\models}
\newcommand{\step}{\vdash}
%\newcommand{\tr}{\triangleright}
\newcommand{\cc}{\coloncolon}


% Names (lower case italic)

\newcommand{\exn}{\name{exn}}
\newcommand{\id}{\name{id}}
\newcommand{\op}{\name{op}}

\newcommand{\true}{\name{true}}
\newcommand{\false}{\name{false}}
\newcommand{\Not}{\name{not}}

\newcommand{\sol}[3]{\name{solution}\,(#1,#2,#3)}
%\newcommand{\unify}{\name{unify}\,}

\newcommand{\Fst}{\name{fst}}
\newcommand{\Snd}{\name{snd}}

\newcommand{\Hd}{\name{hd}}
\newcommand{\Tl}{\name{tl}}
\newcommand{\Cons}{\name{cons}}
\newcommand{\Empty}{\name{is\_empty}}

\newcommand{\dom}[1]{\name{dom}(#1)}
\newcommand{\free}[1]{\name{free}\,(#1)}
\newcommand{\rank}[1]{\name{rank}\,(#1)}
\newcommand{\len}[1]{\name{len}\,(#1)}
\newcommand{\tr}[1]{\name{tr}(#1)}
\newcommand{\trdB}[1]{\name{tr}_{dB}(#1)}
\newcommand{\locns}[1]{\name{locns}\,(#1)}

% Names (upper case italic)

\newcommand{\Bool}{\name{Bool}}
\newcommand{\Btype}{\name{BType}}

\newcommand{\Conf}{\name{Conf}}
\newcommand{\Const}{\name{Const}}

\newcommand{\Dec}{\name{Dec}}

\newcommand{\Env}{\name{Env}}
\newcommand{\Exn}{\name{Exn}}
\newcommand{\EP}{\name{EP}}
\newcommand{\Exp}{\name{Exp}}

\newcommand{\Ncx}{\name{Ncx}}
\newcommand{\dbExp}{\name{dbExp}}
\newcommand{\dbVal}{\name{dbVal}}
\newcommand{\dbEnv}{\name{dbEnv}}
\newcommand{\dbCl}{\name{dbCl}}

\newcommand{\Id}{\name{Id}}
\newcommand{\Int}{\name{Int}}

\newcommand{\Lexp}{\name{LExp}}
\newcommand{\Loc}{\name{Loc}}

\newcommand{\Prog}{\name{Prog}}
\newcommand{\Instr}{\name{Instr}}
\newcommand{\Reg}{\name{Reg}}
\newcommand{\Stack}{\name{Stack}}
\newcommand{\State}{\name{State}}

\newcommand{\Op}{\name{Op}}

\newcommand{\Type}{\name{Type}}
\newcommand{\Tvar}{\name{TVar}}
\newcommand{\teqns}[3]{\name{teqns}\,(#1,#2,#3)}
\newcommand{\tvar}[1]{\name{tvar}\,(#1)}
\newcommand{\unify}[1]{\name{unify}\,(#1)}

\newcommand{\Ptype}{\name{PType}}

\newcommand{\Unit}{\name{Unit}}

\newcommand{\Val}{\name{Val}}

% keywords

\newcommand{\z}{\mathbf{int}}
\newcommand{\bool}{\mathbf{bool}}
\newcommand{\unit}{\mathbf{unit}}
\newcommand{\blist}{\mathbf{list}}
\newcommand{\bref}{\mathbf{ref}}
\newcommand{\ltype}[1]{#1\,\blist}
\newcommand{\reftype}[1]{#1\,\bref}

\renewcommand{\div}{\mathbin{\mathbf{div}}}
\newcommand{\sel}{\mathbin{.}}
%\newcommand{\mod}{\mathbin{\mathbf{mod}}}
\renewcommand{\mod}{\text{mod}}

\newcommand{\bif}{\mathbf{if}}
\newcommand{\bthen}{\mathbf{then}}
\newcommand{\belse}{\mathbf{else}}

\newcommand{\blet}{\mathbf{let}}
\newcommand{\bin}{\mathbf{in}}
\newcommand{\bend}{\mathbf{end}}

\newcommand{\bval}{\mathbf{val}}
\newcommand{\brec}{\mathbf{rec}}
\newcommand{\bfix}{\mathbf{fix}}
\newcommand{\bfun}{\mathbf{fun}}
\newcommand{\band}{\mathbf{and}}
\newcommand{\btype}{\mathbf{type}}

%\newcommand{\andalso}[2]{#1\,\&\&\,#2}
%\newcommand{\orelse}[2]{#1\,\|\,#2}
\newcommand{\andalso}{\&\&}
\newcommand{\orelse}{\|}
\newcommand{\bandalso}{\mathbf{andalso}}
\newcommand{\borelse}{\mathbf{orelse}}

\newcommand{\bwhile}{\mathbf{while}}
\newcommand{\bdo}{\mathbf{do}}
\newcommand{\bfor}{\mathbf{for}}

\newcommand{\brepeat}{\mathbf{repeat}}
\newcommand{\buntil}{\mathbf{until}}

\newcommand{\barray}{\mathbf{array}}
\newcommand{\bof}{\mathbf{of}}

\newcommand{\bclass}{\mathbf{class}}

\newcommand{\app}[2]{#1\,#2}
\newcommand{\bift}[2]{\bif\ #1\ \bthen\ #2}
\newcommand{\bifte}[3]{\bif\ #1\ \bthen\ #2\ \belse\ #3}
\newcommand{\Bifte}[3]{\blong\bif\ #1\\\bthen\ #2\\\belse\ #3\elong}
\newcommand{\bwd}[2]{\bwhile\ #1\ \bdo\ #2}
\newcommand{\bru}[2]{\brepeat\ #1\ \buntil\ #2}


\newcommand{\blie}[2]{\blet\ #1\ \bin\ #2 \ \bend}
\newcommand{\vdec}[2]{#1 = #2}
\newcommand{\rec}[2]{\brec\,#1.\,#2}
\newcommand{\recdots}[2]{\brec\ #1.\ \ldots}
\newcommand{\abstr}[2]{\lambda #1.\,#2}
\newcommand{\appl}[2]{#1\,#2}
\newcommand{\proj}[1]{\#_{#1}}

\newcommand{\Ref}{\name{ref}}
\newcommand{\Deref}{\,!\,}

\newcommand{\alloc}{\name{alloc}}
\newcommand{\Store}{\name{Store}}




% Typing rules 

\newcommand{\tj}[2]{#1\cc#2}
\newcommand{\ctj}[2]{\tj{#1}{{\resulttypecolor #2}}}
\newcommand{\Tj}[3]{#1 \, \triangleright \, #2\cc#3}
\newcommand{\cTj}[3]{\Tj{{\contextcolor #1}}{#2}{{\resulttypecolor #3}}}
\newcommand{\cbig}[2]{#1 \ \eval\ {\resultcolor #2}}
\newcommand{\cBig}[2]{#1 \\ \eval\ {\resultcolor #2}}
\newcommand{\Clos}[2]{\name{Closure}_{#1}(#2)}

\newcommand{\Tjl}[3]{#1 \,\triangleright_l\, #2\cc#3}
\newcommand{\Tjm}[3]{#1 \,\triangleright_m\, #2\cc#3}
\newcommand{\Tjh}[2]{#1 \triangleright  #2}

\newcommand{\Lj}[3]{#1\,\vdash\,#2\cc#3}

\newcommand{\brule}[1]{\begin{markiere}[#1]}
\newcommand{\erule}{\end{markiere}}

\newcommand{\regel}[2]{\ \begin{array}{@{}c@{}} #1 \\ \hline #2
 \end{array}\ }

\newcommand{\reason}[1]{\ \mbox{#1}}
\newcommand{\Reason}[1]{\vspace{1mm}\\ \mbox{ #1}}


% Program verification

\newcommand{\conj}{\,\land\,}
\newcommand{\Conj}{\bigwedge}
\newcommand{\disj}{\,\lor\,}
\newcommand{\Disj}{\bigvee\,}
\newcommand{\all}[1]{\forall{#1}.\,}
\newcommand{\ex}[1]{\exists{#1}.\,}

\newcommand{\power}[1]{\wp(#1)}

\newcommand{\disjoint}[2]{\name{disj}(#1,#2)}
\newcommand{\cont}[2]{#1 \mapsto #2}
\newcommand{\DEF}{\name{DEF}}

\newcommand{\ret}[2]{{\bf returns}\ #1.\, #2}
\newcommand{\tc}[2]{#1\,\{#2\}}
\newcommand{\triple}[3]{\{#1\}\,#2\,\{#3\}}

% Index

\newcommand{\define}[1]{{\em #1\/}\index{#1}}
\newcommand{\Define}[2]{{\em #1\/}\index{#2}}
\newcommand{\Index}[1]{\index{#1}}
\newcommand{\notation}[1]{#1\index{#1}}
\newcommand{\engl}[1]{(engl.: \define{#1})}
\newcommand{\Engl}[2]{(engl.: \Define{#1}{#2})}

% Theorems etc.

\newtheorem{theorem}{Satz}
\newtheorem{corollary}{Korollar}
\newtheorem{definition}{Definition:}
%\newtheorem{example}{Beispiel:}
%\newtheorem{examples}{Beispiele:}
\newtheorem{lemma}[theorem]{Lemma}
\newtheorem{proposition}[theorem]{Proposition}

%\renewcommand{\thedefinition}{}
%\renewcommand{\theexample}{}
%\renewcommand{\theexamples}{}
\renewcommand{\theenumi}{\rm (\alph{enumi})}
\renewcommand{\labelenumi}{\theenumi}

%\newcommand{\enumarabic}{\renewcommand{\theenumi}{\rm (\arabic{enumi})}}

%\newcommand{\bcoro}[1]{\begin{corollary}\label{cor:#1}}
%\newcommand{\ecoro}{\end{corollary}}

\newcommand{\brdef}[1]{\begin{definition}\label{def:#1}\rm}
\newcommand{\erdef}{\end{definition}}

%\newcommand{\blemm}[1]{\begin{lemma}\label{lem:#1}}
%\newcommand{\bLemm}[2]{\begin{lemma}[#2]\label{lem:#1}\index{#2}}
%\newcommand{\elemm}{\end{lemma}}

%\newcommand{\btheo}[1]{\begin{theorem}\label{th:#1}}
%\newcommand{\bTheo}[2]{\begin{theorem}[#2]\label{th:#1}\index{#2}}
%\newcommand{\etheo}{\end{theorem}}

%\newcommand{\litem}[1]{\item\label{it:#1}}
%\newcommand{\ritem}[1]{\ref{it:#1}}


% Grammars

\newcommand{\bgram}{\[\begin{array}{rrlll}}
\newcommand{\egram}{\end{array}\]}

\newcommand{\is}{& ::= &}
\newcommand{\al}{\\ & \mid &}
\newcommand{\n}{\vspace{2mm}\\}




% Other Environments

\newcommand{\bcase}{\left\{\!\!\!\begin{array}{ll}}
\newcommand{\ecase}{\end{array}\right.}
\newcommand{\benum}{\begin{enumerate}}
\newcommand{\eenum}{\end{enumerate}}
\newcommand{\beqns}{\[\begin{array}{rcll}}
\newcommand{\eeqns}{\end{array}\]}
\newcommand{\bitem}{\begin{itemize}}
\newcommand{\eitem}{\end{itemize}}
\newcommand{\blong}{\!\!\begin{array}[t]{l}}
\newcommand{\elong}{\end{array}}
\newcommand{\btabl}{\begin{tabular}}
\newcommand{\etabl}{\end{tabular}}
\newcommand{\brexa}{\begin{example}\enumarabic\rm}
\newcommand{\erexa}{\end{example}}
\newcommand{\brexs}{\begin{examples}\enumarabic\rm}
\newcommand{\erexs}{\end{examples}}


% German abbreviations

\newcommand{\abk}[1]{#1.\ }
\newcommand{\bzw}{\abk{bzw}}
\newcommand{\bzgl}{\abk{bzgl}}
\newcommand{\das}{\abk{d.h}}
\newcommand{\evtl}{\abk{evtl}}
\newcommand{\usw}{\abk{usw}}
\newcommand{\vgl}{\abk{vgl}}
\newcommand{\zb}{\abk{z.B}}


\newcommand{\infix}[3]{#2\mathbin{#1}#3}

\newcommand{\bli}[3]{\blet\ \vdec{#1}{#2}\ \bin\ #3}
\newcommand{\blidb}[2]{\blet\ {#1}\ \bin\ {#2}}
\newcommand{\blri}[3]{\blet\,\brec\ \vdec{#1}{#2}\ \bin\ #3}

\newcommand{\Bli}[3]{\begin{array}[t]{@{}l}
                     \blet\ \vdec{#1}{#2}\\\bin\ #3
                     \end{array}}

\newcommand{\Vdec}[2]{\begin{array}[t]{@{}l}
                      #1 = \\
                      \ #2
                      \end{array}}

\newcommand{\BLI}[3]{\begin{array}[t]{@{}l}
                     \blet\ \Vdec{#1}{#2}\\\bin\ #3
                     \end{array}}

\newcommand{\Abstr}[2]{\begin{array}[t]{@{}l}
                       \lambda #1.\\\ \ #2
                       \end{array}}


\newcommand{\semantic}[1]{\ensuremath{\llbracket#1\rrbracket}}
\newcommand{\assn}{\ensuremath{\mathbf{assn}}}
\newcommand{\atype}[1]{#1\,\assn}
\newcommand{\bexpr}{\ensuremath{\mathbf{expr}}}
\newcommand{\TEnv}{\name{TEnv}}
\newcommand{\Dtype}{\name{DType}}
\newcommand{\Atype}{\name{AType}}
\newcommand{\Assn}{\name{Assn}}
\newcommand{\Term}{\name{Term}}
\newcommand{\locns}{\name{locns}}
\newcommand{\reach}{\name{reach}}
\newcommand{\grph}[1]{\name{graph}(#1)}
\newcommand{\reachable}{\name{reachable}}
\newcommand{\supp}{\name{supp}}
\newcommand{\tto}{\ensuremath{\xrightarrow{t}}}
\renewcommand{\disjoint}[2]{\ensuremath{#2\not\hookrightarrow#1}}
\newcommand{\RN}[1]{\mbox{{\sc (#1)}}}
\newcommand{\Tje}[3]{#1 \,\tr_e\, #2\cc#3}
\newcommand{\TC}[4]{#1 \models \{#2\}\,#3\,\{#4\}}


\begin{document}


\section{Speicherzust"ande und Erreichbarkeit}

F"ur jeden Typ $\tau$ sei eine unendliche Menge $\Loc^\tau$ definiert, deren Elemente $X,Y,\ldots$
als {\em Speicherpl"atze} vom Typ $\tau$ bezeichnet werden. Es wird vorausgesetzt, dass die Mengen
$\Loc^\tau$ paarweise disjunkt sind und darauf ist definiert
\[
  \Loc = \bigcup_{\tau\in\Type} \Loc^\tau
\]
Wir schreiben $\locns{e}$ f"ur die Menge aller im Ausdruck $e$ (syntaktisch) vorkommenden Speicherpl"atze
und $\Val^\tau$ f"ur die Menge aller abgeschlossenen Werte vom Typ $\tau$ (wobei Abgeschlossenheit sich
hier nur auf die Bezeichner $\Id$ beschr"ankt) und $\Exp^\tau$ f"ur die Menge aller abgeschlossenen
Ausdr"ucke vom Typ $\tau$.

\begin{definition}[Speicherzustand] \
  \begin{enumerate}
    \item Eine partielle Funktion $\sigma: \Loc \pto \bigcup_{\tau\in\Type} \semantic{\tau}$ hei"st
          {\em Speicherzustand}, wenn gilt:
          \begin{itemize}
            \item $\sigma(\Loc^\tau) \subseteq \Val^\tau$ f"ur alle $\tau\in\Type$
            \item $\locns(\sigma(\Loc)) \subseteq \dom{\sigma}$
          \end{itemize}

    \item F"ur alle endlichen $L \subseteq \Loc$ definieren wir eine Menge aller {\em $L$-Speicherzust"ande}
          $\Store_L = \{\sigma\,|\,\text{$\sigma$ Speicherzustand mit $L\subseteq\dom{\sigma}$}\}$.

    \item Die Menge aller Speicherzust"ande ist dann $\Store = \bigcup_{L\subseteq\Loc,\text{ $L$ endl.}} \Store_L$.
  \end{enumerate}
\end{definition}
Demzufolge ist ein Zustand per Definition stets {\em wohlgetypt} und in sich abgeschlossen,
d.h. er enth"alt keine {\em dangling references}.

F"ur $\sigma\in\Store_L$ seien die Mengen $\reach_i(L,\sigma)\subseteq \Loc$ f"ur alle
$i\in\setN$ und die Menge $\reach(L,\sigma)\subseteq\Loc$ definiert durch
\[\begin{array}{rcl}
  \reach_0(L,\sigma) & = & L \\
  \reach_{i+1}(L,\sigma) & = & \reach_i(L,\sigma)
             \cup \bigcup_{X\in\reach_i(L,\sigma)} \locns(\sigma(X)) \\
  \reach(L,\sigma) & = & \bigcup_{i\in\setN} \reach_i(L,\sigma)
\end{array}\]
$\reach(L,\sigma)$ enth"alt alle Speicherpl"atze, die man von der Menge $L$ aus im
Zustand $\sigma$ erreichen kann. Da ein Zustand $\sigma\in\Store_L$ keine dangling references
enth"alt, gilt stets
\[
  \reach(L,\sigma) \subseteq \dom{\sigma}.
\]

\begin{lemma} \label{lemma:Store_L_und_reach}
  Seien $L,L'\subseteq\Loc$ endl., $L \subseteq L'$ und $\sigma,\sigma'\in\Store_{L'}$. Dann gilt:
  \begin{enumerate}
    \item $\Store_{L'} \subseteq \Store_{L}$
    \item $\reach(L,\sigma) \subseteq \reach(L',\sigma)$
    \item Wenn $\reach(L',\sigma) = \reach(L',\sigma')$ und $\sigma(X) = \sigma'(X)$ f.a.
          $X\in\reach(L',\sigma)$, dann $\reach(L,\sigma) = \reach(L,\sigma')$
  \end{enumerate}
\end{lemma}

\begin{beweis} \
  \begin{enumerate}
    \item Sei $\sigma''\in\Store_{L'}$, dann gilt $L' \subseteq \dom{\sigma''}$. Aus $L \subseteq L'$ folgt
          $L\subseteq\dom{\sigma''}$, also $\sigma''\in\Store_L$.
    \item Straight-forward induction.
    \item Straight-forward induction.
  \end{enumerate}
\end{beweis}

\begin{definition}[$L$-"Ahnlichkeit]
  Seien $\sigma,\sigma'\in\Store$ und $L \subseteq \dom{\sigma} \cap \dom{\sigma'}$. $\sigma$ und $\sigma'$
  heissen {\em $L$-"ahnlich}, geschrieben $\sigma \simeq_L \sigma'$, wenn $\sigma(X) = \sigma'(X)$ f"ur
  alle $X \in L$ gilt.
\end{definition}

\begin{lemma} \label{lemma:L_Aehnlichkeit}
  Seien $L,L'\subseteq\Loc$ mit $L \subseteq L'$ und $\sigma,\sigma'\in\Store$. Wenn $\sigma \simeq_{L'} \sigma$,
  dann $\sigma \simeq_L \sigma'$.
\end{lemma}

\begin{beweis}
  Trivial.
\end{beweis}

\begin{definition}[$L$-"Aquivalenz]
  Sei $L \subseteq \Loc$ endlich. Zwei Zust"ande $\sigma,\sigma'\in\Store_L$ hei"sen {\em $L$-"aquivalent}, geschrieben
  $\sigma \equiv_L \sigma'$, wenn gilt:
  \begin{enumerate}
    \item $\reach(L,\sigma) = \reach(L,\sigma')$
    \item $\sigma \simeq_{\reach(L,\sigma)} \sigma'$
  \end{enumerate}
\end{definition}

\begin{lemma} \label{lemma:L_Aequivalenz}
  Seien $L,L'\subseteq\Loc$ endl. mit $L \subseteq L'$ und $\sigma,\sigma'\in\Store_{L'}$. Wenn $\sigma \equiv_{L'} \sigma'$,
  dann $\sigma \equiv_L \sigma'$.
\end{lemma}

\begin{beweis}
  Folgt leicht mit Lemma~\ref{lemma:Store_L_und_reach} und Lemma~\ref{lemma:L_Aehnlichkeit}.
\end{beweis}

\begin{definition}[$L$-Definierbarkeit]
  Sei $L \subseteq \Loc$ endl. und $\phi:\Store\pto\Bool$.
  $\phi$ hei"st {\em $L$-definierbar}, wenn f"ur alle $\sigma,\sigma'\in\dom{\phi}$ mit
  $\sigma \equiv_L \sigma'$ gilt: $\phi\,\sigma = \phi\,\sigma'$.
\end{definition}

\begin{lemma} \label{lemma:L_Definierbarkeit}
  Seien $L,L'\subseteq\Loc$ mit $L \subseteq L'$ und sei $\phi:\Store \pto \Bool$. Ist $\phi$ $L$-definierbar,
  so ist $\phi$ ebenfalls $L'$-definierbar.
\end{lemma}

\begin{beweis}
  Folgt leicht mit Lemma~\ref{lemma:L_Aequivalenz}.
\end{beweis}


\section{Big step Semantik}

\begin{definition}[Konfiguration]
  Seien $e\in\Exp$ und $\sigma\in\Store$.
  Das Paar $(e,\sigma)$ hei"st {\em Konfiguration}, wenn $\locns(e) \subseteq \dom{\sigma}$.
\end{definition}

Die Speicherzust"ande $\sigma$ selbst sind schon per Definition abgeschlossen.

\begin{lemma}[Wohldefiniertheit der big step Semantik]
  Seien $e\in\Exp$, $v\in\Val$ und $\sigma,\sigma'\in\Store$. Ist $(e,\sigma)$ eine Konfiguration
  und gilt $(e,\sigma) \Downarrow (v,\sigma')$, so ist auch $(v,\sigma')$ eine Konfiguration.
\end{lemma}

\begin{lemma}
  Seien $\sigma_1,\sigma_2\in\Store$ mit $\grph{\sigma_1} \subseteq \grph{\sigma_2}$. Dann gilt:
  \begin{enumerate}
    \item Wenn $(e,\sigma_1) \Downarrow (v,\sigma_1')$, dann existiert ein $\sigma_2'\in\Store$ mit
          $(e,\sigma_2) \Downarrow (v,\sigma_2')$ und $\grph{\sigma_1'} \subseteq \grph{\sigma_2'}$.
    \item Wenn $(e,\sigma_2) \Downarrow (v,\sigma_2')$, dann existiert ein $\sigma_1'\in\Store$ mit
          $(e,\sigma_1) \Downarrow (v,\sigma_1')$ und $\grph{\sigma_1'} \subseteq \grph{\sigma_2'}$.
  \end{enumerate}
\end{lemma}

\end{document}

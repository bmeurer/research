\documentclass[12pt,a4paper]{article}

\usepackage{amsmath}
\usepackage{amssymb}
\usepackage{amstext}
\usepackage{array}
\usepackage[american]{babel}
\usepackage{color}
\usepackage{enumerate}
\usepackage[a4paper,%
            colorlinks=false,%
            final,%
            pdfkeywords={},%
            pdftitle={},%
            pdfauthor={Benedikt Meurer},%
            pdfsubject={},%
            pdfdisplaydoctitle=true]{hyperref}
\usepackage{ifthen}
\usepackage[latin1]{inputenc}
\usepackage{latexsym}
\usepackage[final]{listings}
\usepackage{makeidx}
\usepackage{ngerman}
\usepackage[standard,thmmarks]{ntheorem}
\usepackage{stmaryrd}

%% LaTeX macros
%%
%% macros.tex - Useful LaTeX macros
%%
%% Copyright (c) 2006-2011 Benedikt Meurer <benedikt.meurer@googlemail.com>
%% 


%%
%% Styles
%%

\newcommand{\nstyle}[1]{\ensuremath{\mathsf{#1}}}
\newcommand{\sstyle}[1]{\ensuremath{\mathit{#1}}}


%%
%% Misc
%%

\newcommand{\abort}{\ensuremath{\mathbf{abort}}}
\newcommand{\pto}{\rightharpoonup}
\newcommand{\step}{\ensuremath{\rightsquigarrow}}

\newcommand{\sem}[1]{\ensuremath{[\![#1]\!]}}


%%
%% Names
%%

\newcommand{\arity}{\ensuremath{\mathit{arity}}}
\newcommand{\cl}{\ensuremath{\mathit{cl}}}
\newcommand{\fr}{\ensuremath{\mathit{fr}}}
\newcommand{\free}{\ensuremath{\mathit{free}}}
\newcommand{\graph}{\ensuremath{\mathit{graph}}}
\newcommand{\id}{\ensuremath{\mathit{id}}}


%%
%% Sets
%%

\newcommand{\I}{\ensuremath{\mathcal I}}
\newcommand{\N}{\ensuremath{\mathbb N}}
\renewcommand{\O}{\ensuremath{\mathcal O}}
\newcommand{\Z}{\ensuremath{\mathbb Z}}
\newcommand{\Cl}{\sstyle{Cl}}
\newcommand{\Env}{\sstyle{Env}}
\newcommand{\Exp}{\sstyle{Exp}}
\newcommand{\Frame}{\sstyle{Frame}}
\newcommand{\Id}{\sstyle{Id}}
\newcommand{\Node}{\sstyle{Node}}
\newcommand{\Val}{\sstyle{Val}}


%%
%% Expressions
%%

\newcommand{\app}[2]{{#1}\,{#2}}
\newcommand{\abstr}[2]{\lambda{#1}.\,{#2}}
\newcommand{\ifte}[3]{\mathbf{if}\,{#1}\,\mathbf{then}\,{#2}\,\mathbf{else}\,{#3}}


%%
%% Values
%%

\newcommand{\clov}[2]{\langle{#1},{#2}\rangle}
\newcommand{\false}{\mathbf{false}}
\newcommand{\true}{\mathbf{true}}


%%
%% Grammars
%%

\newenvironment{grammar}{\begin{array}{rrlll}}{\end{array}}

\newcommand{\is}{& ::= &}
\newcommand{\al}{\\ & \mid &}
\newcommand{\nl}{\vspace{2mm}\\}


%%
%% Other environments
%%

\newenvironment{case}{\left\{\!\!\!\begin{array}{ll}}{\end{array}\right.}


%%% Local Variables: 
%%% mode: latex
%%% TeX-master: "compiler"
%%% End: 


\begin{document}

\title{Namenlose $\lambda$-Notationen}
\author{Benedikt Meurer}
\maketitle


\section{Umgebungssemantik}

Vorgegeben sei eine (unendliche) Menge $\Var$ von Namen $x$. Die abstrakte Syntax der Programmiersprache
ist durch die kontextfreie Grammatik
\[\begin{grammar}
  e \in \Exp
  \is c \mid x
  \al \abstr{x}{e}
  \al \app{e_1}{e_2}
  \al \ifte{e_0}{e_1}{e_2}
  \al \letin{x}{e_1}{e_2}
\end{grammar}\]
definiert.


\section{B"aume}

Die folgenden Definitionen basieren auf \emph{``Grundlagen der Programmiersprachen''} von Loeckx,
Mehlhorn und Willhelm.
\begin{definition}[Baumbereiche und B"aume] \
  \begin{enumerate}
  \item Die partiellen Funktionen $\mathit{parent}$ und $\mathit{brother}$ sind definiert durch
    \[\begin{array}{rl}
      \mathit{parent}: & \N^* \pto \N^*, \\
      & w \cdot n \mapsto w \\
      \\
      \mathit{brother}: & \N^* \pto \N^*, \\
      & w \cdot (n+1) \mapsto w \cdot n \\
    \end{array}\]
    Die reflexive, transitive H"ulle von $\mathit{parent}$ hei"st $\mathit{ancestor}$.
  \item Ein \emph{Baumbereich} $K$ ist eine endliche Teilmenge von $\N^*$, die abgeschlossen
    ist unter der Funktion $\mathit{parent}$, d.h. $\parent{K} \subseteq K$. Die Elemente des
    Baumbereichs hei"sen \emph{Knoten}.
  \item Sei $D$ eine Menge. Ein \emph{$D$-Baum} (kurz \emph{Baum}) ist ein Paar $(K,\ell)$,
    wobei $K$ ein Baumbereich ist, und $\ell$ eine totale Abbildung von $K$ nach $D$. Die
    Abbildung $\ell$ hei"st die \emph{Beschriftung} des Baumes und $K$ sein \emph{Bereich}.
    Die Menge aller $D$-B"aume bezeichnen wir mit $T_D$.
  \item Ein Baumbereich $K$ hei"st \emph{vollst"andig}, wenn $K$ abgeschlossen ist unter der
    Funktion $\mathit{brother}$, d.h. $\brother{K} \subseteq K$. Ein Baum $(K,\ell)$ hei"st
    \emph{vollst"andig}, wenn sein Baumbereich $K$ vollst"andig ist.
  \end{enumerate}
\end{definition}


\section{Umgebungssemantik}

Eine \emph{Umgebungssemantik} besteht aus Mengen $\Env$, $\Exp$, $\Ide$ und $\Val$ und beruht im
Wesentlichen auf den beiden Funktionen
\[\begin{array}{rl}
  \lookup: & \Env \to \Ide \to \Val_\bot \\
  \update: & \Env \to \Ide \to \Val \to \Env 
\end{array}\]
mit denen sich die beiden entscheidenden Regeln formulieren lassen: \\[3mm]
\begin{tabular}{rl}
  \RN{Lookup} & $\RULE{\lookup\,\eta\,\ide = v}{(\ide,\eta) \Downarrow v}$ \\[3mm]
  \RN{Beta-V} & $\RULE{(e_1,\eta) \Downarrow (\abstr{\ide'}{e'},\eta') \quad (e_2,\eta) \Downarrow v' \quad (e',\update\,n'\,\ide'\,v') \Downarrow w}{(\app{e_1}{e_2},\eta) \Downarrow w}$ \\[3mm]
\end{tabular} \\[3mm]
Zwischen den Funktionen $\lookup$ und $\update$ muss also offensichtlich folgender
Zusammenhang bestehen:
\[\begin{array}{rcl}
  \lookup\,(\update\,\eta\,\ide\,v)\,\ide'
  &=&
  \begin{case}
    v, & \text{falls $\ide = \ide'$} \\
    \lookup\,\eta\,\ide', & \text{sonst}
  \end{case}
\end{array}\]


\section{"Ubersetzung}

Im Folgenden geben wir ein Schema an, wie man die "Ubersetzung in eine namenlose Umgebungssemantik durchf"uhrt.
Dazu ben"otigt man
\begin{itemize}
\item eine Menge $\sEnv$ von \emph{statischen Umgebungen} $\Gamma$, die sog. \emph{"Ubersetzungskontexte},
\item eine Menge $\sVal$ von \emph{statischen Werten} $\iota$, die sog. \emph{Indizes},
\item eine Menge $\dEnv$ von \emph{dynamischen Umgebungen} $\rho$, und
\item eine Menge $\dVal$ von \emph{dynamischen Werten} $\omega$,
\end{itemize}
sowie die folgenden 4 Funktionen:
\[\begin{array}{rl}
  \slookup: & \sEnv \to \Ide \to \sVal_\bot \\
  \supdate: & \sEnv \to \Ide \to \sEnv \\
  \dlookup: & \dEnv \to \sVal \to \dVal_\bot \\
  \dupdate: & \dEnv \to \dVal \to \dEnv \\
\end{array}\]
Namenlose Ausdr"ucke $e^0 \in \Exp^0$ erh"alt man aus gew"ohnlichen Ausdr"ucken $e$, indem man bindende Vorkommen
von Namen wegl"asst und angewandte Vorkommen durch statische Werte ersetzt. Die "Ubersetzungsrelation
\[(\cdot \vdash \cdot \to \cdot) \subseteq \sEnv \times \Exp \times \Exp^0\]
ist die kleinste Relation f"ur die gilt:
\[\begin{array}{rcl}
  \RULE{\slookup\,\Gamma\,\ide = \iota}{\Gamma \vdash \ide \to \iota}
  &\quad\quad&
  \RULE{(\supdate\,\Gamma\,\ide) \vdash e \to e^0}{\Gamma \vdash \abstr{\ide}{e} \to \abstr{}{e^0}} \\
\end{array}\]
und \emph{``straight-forward''} f"ur alle "ubrigen Ausdr"ucke.

Die Semantik namenloser Ausdr"ucke wird mit Hilfe von dynamischen Umgebungen definiert, wobei die
beiden entscheidenden Regeln wie folgt aussehen: \\[3mm]
\begin{tabular}{rl}
  \RN{Lookup} & $\RULE{\dlookup\,\rho\,\iota = \omega}{(\iota,\rho) \Downarrow \omega}$ \\[3mm]
  \RN{Beta-V} & $\RULE{(e_1,\rho) \Downarrow (\abstr{}{e'},\rho') \quad (e_2,\rho) \Downarrow \omega' \quad (e',\dupdate\,\rho'\,\omega') \Downarrow \omega}{(\app{e_1}{e_2},\rho) \Downarrow \omega}$ \\[3mm]
\end{tabular} \\[3mm]

\end{document}
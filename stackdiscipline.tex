\documentclass[12pt,a4paper,fleqn]{article}

\usepackage{amsmath}
\usepackage{amssymb}
\usepackage{amstext}
\usepackage{array}
\usepackage[american]{babel}
\usepackage{color}
\usepackage{enumerate}
\usepackage[a4paper,%
            colorlinks=false,%
            final,%
            pdfkeywords={},%
            pdftitle={},%
            pdfauthor={Benedikt Meurer},%
            pdfsubject={},%
            pdfdisplaydoctitle=true]{hyperref}
\usepackage{ifthen}
\usepackage[latin1]{inputenc}
\usepackage{latexsym}
\usepackage[final]{listings}
\usepackage{makeidx}
\usepackage{ngerman}
\usepackage[standard,thmmarks]{ntheorem}
\usepackage{stmaryrd}

%% LaTeX macros
%%
%% macros.tex - Useful LaTeX macros
%%
%% Copyright (c) 2006-2011 Benedikt Meurer <benedikt.meurer@googlemail.com>
%% 


%%
%% Styles
%%

\newcommand{\nstyle}[1]{\ensuremath{\mathsf{#1}}}
\newcommand{\sstyle}[1]{\ensuremath{\mathit{#1}}}


%%
%% Misc
%%

\newcommand{\abort}{\ensuremath{\mathbf{abort}}}
\newcommand{\pto}{\rightharpoonup}
\newcommand{\step}{\ensuremath{\rightsquigarrow}}

\newcommand{\sem}[1]{\ensuremath{[\![#1]\!]}}


%%
%% Names
%%

\newcommand{\arity}{\ensuremath{\mathit{arity}}}
\newcommand{\cl}{\ensuremath{\mathit{cl}}}
\newcommand{\fr}{\ensuremath{\mathit{fr}}}
\newcommand{\free}{\ensuremath{\mathit{free}}}
\newcommand{\graph}{\ensuremath{\mathit{graph}}}
\newcommand{\id}{\ensuremath{\mathit{id}}}


%%
%% Sets
%%

\newcommand{\I}{\ensuremath{\mathcal I}}
\newcommand{\N}{\ensuremath{\mathbb N}}
\renewcommand{\O}{\ensuremath{\mathcal O}}
\newcommand{\Z}{\ensuremath{\mathbb Z}}
\newcommand{\Cl}{\sstyle{Cl}}
\newcommand{\Env}{\sstyle{Env}}
\newcommand{\Exp}{\sstyle{Exp}}
\newcommand{\Frame}{\sstyle{Frame}}
\newcommand{\Id}{\sstyle{Id}}
\newcommand{\Node}{\sstyle{Node}}
\newcommand{\Val}{\sstyle{Val}}


%%
%% Expressions
%%

\newcommand{\app}[2]{{#1}\,{#2}}
\newcommand{\abstr}[2]{\lambda{#1}.\,{#2}}
\newcommand{\ifte}[3]{\mathbf{if}\,{#1}\,\mathbf{then}\,{#2}\,\mathbf{else}\,{#3}}


%%
%% Values
%%

\newcommand{\clov}[2]{\langle{#1},{#2}\rangle}
\newcommand{\false}{\mathbf{false}}
\newcommand{\true}{\mathbf{true}}


%%
%% Grammars
%%

\newenvironment{grammar}{\begin{array}{rrlll}}{\end{array}}

\newcommand{\is}{& ::= &}
\newcommand{\al}{\\ & \mid &}
\newcommand{\nl}{\vspace{2mm}\\}


%%
%% Other environments
%%

\newenvironment{case}{\left\{\!\!\!\begin{array}{ll}}{\end{array}\right.}


%%% Local Variables: 
%%% mode: latex
%%% TeX-master: "compiler"
%%% End: 


\begin{document}

\title{Stackdisziplin}
\author{Benedikt Meurer}
\maketitle


\section{Namenlose Umgebungssemantik}

Vorgegeben sei eine Menge von \emph{Indizes} $\iota = \underline{0},\underline{1},\ldots$ und eine
Menge von Konstanten $c$. Die Menge der \emph{Ausdr"ucke} $e$ ist definiert durch folgende kontextfreie
Grammatik:
\[\begin{grammar}
  f \is \abstr{}{e}
  \nl
  e \is c \mid \iota \mid f \mid \app{e_1}{e_2}
\end{grammar}\]
Die Mengen der \emph{Werte} $v$ und \emph{Umgebungen} $\eta$ sind definiert durch:
\[\begin{grammar}
  v \is c \mid (f,\eta)
  \nl
  \eta \is \varepsilon \mid \eta,v
\end{grammar}\]
Ein \emph{big step} ist eine Formel der Gestalt $(e,\eta) \Downarrow v$ und hei"st \emph{g"ultig},
wenn er sich mit den folgenden Regeln sowie den nicht angegebenen Regeln f"ur Konstanten herleiten
l"asst: \\[5mm]
\begin{tabular}{rl}
  \RN{Const} & $(c,\eta) \Downarrow c$ \\[1mm]
  \RN{Closure} & $(f,\eta) \Downarrow (f,\eta)$ \\[1mm]
  \RN{Var} & $(\underline{0},\eta;v) \Downarrow v$ \\[1mm]
  \RN{Skip} & $\RULE{(\underline{i},\eta) \Downarrow v'}{(\underline{i+1},\eta;v) \Downarrow v'}$ \\[3mm]
  \RN{Beta-V} & $\RULE{(e_1,\eta) \Downarrow (\abstr{}{e'},\eta') \quad (e_2,\eta) \Downarrow v' \quad (e',\eta';v') \Downarrow c}{(\app{e_1}{e_2},\eta) \Downarrow c}$ \\[3mm]
\end{tabular} \\[5mm]
Wie man sieht, handelt es sich um eine eingeschr"ankte Semantik f"ur Stackdisziplin, da Funktionen
niemals als Ergebnis anderer Funktionen auftreten k"onnen und somit Bindungen niemals "uber ihren
Scope hinaus erhalten bleiben k"onnen.


\section{Stacksemantik}

Die Mengen der \emph{Stackwerte} $w$ und \emph{Stacks} $\sigma$ sind definiert durch:
\[\begin{grammar}
  w \is c \mid (f,n)
  \nl
  d \is [w,n]
  \nl
  \sigma \is \varepsilon \mid \sigma;d
\end{grammar}\]
Die Elemente $d$ aus denen ein Stack aufgebaut wird, bezeichnen wir als \emph{Stackeintr"age}.
Jeder Stackeintrage besteht aus einen Wert und einen (absoluten) Offset. Statt
$\sigma=\varepsilon;d_1;\ldots;d_n$ schreiben wir kurz $d_1;\ldots;d_n$ und f"ur $j \le n$ ist
$\sigma^j=d_1;\ldots;d_j$. Weiterhin bezeichnet $|\sigma|=n$ die \emph{L"ange} von $\sigma$.

Ein \emph{big step} ist eine Formel der Gestalt $(e,\sigma) \Downarrow w$ und hei"st \emph{g"ultig},
wenn er sich mit den folgenden Regeln sowie den wiederum nicht angegebenen Regeln f"ur Konstanten
herleiten l"asst: \\[5mm]
\begin{tabular}{rl}
  \RN{Const} & $(c,\sigma) \Downarrow c$ \\[1mm]
  \RN{Closure} & $(f,\sigma) \Downarrow (f,|\sigma|)$ \\[1mm]
  \RN{Var} & $(\underline{0},\sigma;[w,n]) \Downarrow w$ \\[1mm]
  \RN{Skip} & $\RULE{(\underline{i},\sigma^n) \Downarrow w'}{(\underline{i+1},\sigma;[w,n]) \Downarrow w'}$ \\[3mm]
  \RN{Beta-V} & $\RULE{(e_1,\sigma) \Downarrow (\abstr{}{e'},n') \quad (e_2,\sigma) \Downarrow w' \quad (e',\sigma;[w',n']) \Downarrow c}{(\app{e_1}{e_2},\sigma) \Downarrow c}$ \\[3mm]
\end{tabular} \\[5mm]


\section{Zusammenhang}

Wir definieren Relationen $\sigma \sim \eta$ und $\sigma \models w \sim v$ durch: \\[5mm]
\begin{tabular}{ccc}
  $\varepsilon \sim \varepsilon$
  &\quad\quad&
  $\RULE{\sigma \models w \sim v \quad \sigma^n \sim \eta}{\sigma;[w,n] \sim \eta;v}$ \\[3mm]
  $\sigma \models c \sim c$
  &&
  $\RULE{\sigma^n \sim \eta}{\sigma \models (f,n) \sim (f,\eta)}$ \\[3mm]
\end{tabular} \\[5mm]
Offensichtlich gilt folgender einfacher Zusammenhang:
\begin{lemma} \label{lemma1}
  Wenn $\sigma \models w \sim v$, dann $\sigma;d \models w \sim v$.
\end{lemma}

\begin{proof}
  Trivial, denn es gilt $\sigma^n = (\sigma;d)^n$ f"ur alle $\sigma$ mit $n \le |\sigma|$.
\end{proof}
Insgesamt gilt folgender Zusammenhang zwischen den beiden Semantiken:
\pagebreak[2]
\begin{theorem} \label{theorem1}
  Wenn $\sigma \sim \eta$, dann gilt:
  \begin{enumerate}
  \item Wenn $(e,\sigma) \Downarrow w$, dann existiert ein $v$ mit $(e,\eta) \Downarrow v$ und
    $\sigma \models w \sim v$.
  \item Wenn $(e,\eta) \Downarrow v$, dann existiert ein $w$ mit $(e,\sigma) \Downarrow w$ und
    $\sigma \models w \sim v$.
  \end{enumerate}
\end{theorem}

\begin{proof}
  Gelte also $\sigma \sim \eta$.
  \begin{enumerate}
  \item Induktion "uber die L"ange der Herleitung von $(e,\sigma) \Downarrow w$ mit Fallunterscheidung
    "uber die zuletzt angewandten Regel:
    \begin{itemize}
    \item[\RN{Const}] Trivial.

    \item[\RN{Closure}] Dann gilt $(f,\sigma) \Downarrow (f,|\sigma|)$. Andererseits
      existiert auch ein big step $(f,\eta) \Downarrow (f,\eta)$ mit \RN{Closure} in
      der Umgebungssemantik. Da $\sigma^{|\sigma|} = \sigma$ und nach Voraussetzung $\sigma \sim \eta$
      gilt, folgt unmittelbar $\sigma \models (f,|\sigma|) \sim (f,\eta)$.

    \item[\RN{Var}] Dann gilt $(\underline{0},\sigma;[w,n]) \Downarrow w$. Weiterhin m"ussen $v$ und $\eta$
      existieren mit $\sigma;[w,n] \sim \eta;v$ und $\sigma \models w \sim v$. Nach Lemma~\ref{lemma1}
      folgt $\sigma;[w,n] \models w \sim v$. Trivialerweise existiert auch ein big step
      $(\underline{0},\eta;v) \Downarrow v$ mit \RN{Var} in der Umgebungssemantik.

    \item[\RN{Skip}] Dann gilt $(\underline{i+1},\sigma;[w,n]) \Downarrow w'$, was nur aus
      $(\underline{i},\sigma^n) \Downarrow w'$ folgen kann. Weiterhin m"ussen $v$ und $\eta$ existieren
      mit $\sigma;[w,n] \sim \eta;v$, also insbesondere $\sigma \sim \eta$. Also existiert nach
      Induktionsvoraussetzung ein $v'$ mit $(\underline{i},\eta) \Downarrow v'$ und
      $\sigma^n \models w' \sim v'$. Nach Lemma~\ref{lemma1} folgt $\sigma;[w,n] \models w' \sim v'$
      und die Existenz des big steps $(\underline{i+1},\eta;v) \Downarrow v'$ folgt mit \RN{Skip} in
      der Umgebungssemantik.

    \item[\RN{Beta-V}] Dann gilt $(\app{e_1}{e_2},\sigma) \Downarrow c$, was nur aus Pr"amissen der
      Form
      \begin{enumerate}
      \item[(a)] $(e_1,\sigma) \Downarrow (\abstr{}{e},n)$,
      \item[(b)] $(e_2,\sigma) \Downarrow w$, und
      \item[(c)] $(e,\sigma;[w,n]) \Downarrow c$
      \end{enumerate}
      folgen kann. Nach Induktionsvoraussetzung existieren also $\hat{v},v'$ mit
      \begin{enumerate}
      \item[(a')] $(e_1,\eta) \Downarrow \hat{v}$ mit $\sigma \models (\abstr{}{e},n) \sim \hat{v}$, und
      \item[(b')] $(e_2,\eta) \Downarrow v'$ mit $\sigma \models w \sim v'$.
      \end{enumerate}
      Wegen (a') muss ein $\eta'$ existieren, so dass gilt $\hat{v} = (\abstr{}{e},\eta')$ und
      $\sigma^n \sim \eta'$. Also gilt insbesondere $\sigma;[w,n] \sim \eta';v'$ und nach
      Induktionsvoraussetzung folgt aus (c), dass ein $c'$ existiert mit
      \begin{enumerate}
      \item[(c')] $(e,\eta';v') \Downarrow c'$ und $c \sim c'$.
      \end{enumerate}
      Also existiert insgesamt ein big step $(\app{e_1}{e_2},\eta) \Downarrow c'$ mit
      \RN{Beta-V} in der Umgebungssemantik.
    \end{itemize}

  \item Induktion "uber die L"ange der Herleitung von $(e,\eta) \Downarrow v$ und Fallunterscheidung
    nach der zuletzt angewandten Regel:
    \begin{itemize}
    \item[\RN{Const}] Trivial.

    \item[\RN{Closure}] Dann gilt $(f,\eta) \Downarrow (f,\eta)$. Andererseits
      existiert auch ein big step $(f,\sigma) \Downarrow (f,|\sigma|)$ mit
      \RN{Closure} in der Stacksemantik. Da $\sigma^{|\sigma|} = \sigma$ und nach Voraussetzung
      $\sigma \sim \eta$ gilt, folgt $\sigma \models (f,|\sigma|) \sim (f,\eta)$.

    \item[\RN{Var}] Dann gilt $(\underline{0},\eta;v) \Downarrow v$. Weiterhin m"ussen $w$, $\sigma$
      und $n$ existieren, so dass $\sigma;[w,n] \sim \eta;v$ und $\sigma \models w \sim v$ gilt.
      Nach Lemma~\ref{lemma1} gilt ebenfalls $\sigma;[w,n] \models w \sim v$ und der big step
      $(\underline{0},\sigma;[w,n]) \Downarrow w$ existiert in der Stacksemantik wegen Regel \RN{Var}.

    \item[\RN{Skip}] Dann gilt $(\underline{i+1},\eta;v) \Downarrow v'$, was nur aus
      $(\underline{i},\eta) \Downarrow v'$ folgen kann. Weiterhin m"ussen $\sigma$, $w$ und $n$ existieren,
      so dass $\sigma;[w,n] \sim \eta;v$, also insbesondere $\sigma^n \sim \eta$ gilt. Nach Induktionsvoraussetzung
      existiert also ein $w'$ mit $(\underline{i},\sigma^n) \Downarrow w'$ und $\sigma^n \models w' \sim v'$.
      Nach Lemma~\ref{lemma1} folgt $\sigma \models w' \sim v'$, und der big step
      $(\underline{i+1},\sigma;[w,n]) \Downarrow w'$ existiert wegen \RN{Skip} in der Stacksemantik.

    \item[\RN{Beta-V}] Dann gilt $(\app{e_1}{e_2},\eta) \Downarrow c$, was ausschliesslich aus Pr"amissen
      der Form
      \begin{enumerate}
      \item[(a)] $(e_1,\eta) \Downarrow (\abstr{}{e},\eta')$,
      \item[(b)] $(e_2,\eta) \Downarrow v'$, und
      \item[(c)] $(e,\eta';v) \Downarrow c$
      \end{enumerate}
      folgen kann. Nach Induktionsvoraussetzung existieren also $\hat{w}, w'$ mit
      \begin{enumerate}
      \item[(a')] $(e_1,\sigma) \Downarrow \hat{w}$ mit $\sigma \models \hat{w} \sim (\abstr{}{e},\eta')$, und
      \item[(b')] $(e_2,\sigma) \Downarrow w'$ mit $\sigma \models w' \sim v'$.
      \end{enumerate}
      Wegen (a') muss ein $n$ existieren, so dass gilt $\hat{w} = (\abstr{}{e},n)$ und $\sigma^n \sim \eta'$.
      Also gilt insbesondere $\sigma;[w,n] \sim \eta';v'$ und nach Induktionsvoraussetzung folgt aus (c), dass
      ein $c'$ existiert mit
      \begin{enumerate}
      \item[(c')] $(e,\sigma;[w,n]) \Downarrow c'$ und $c' \sim c$.
      \end{enumerate}
      Also existiert insgesamt ein big step $(\app{e_1}{e_2},\sigma) \Downarrow c'$ mit \RN{Beta-V} 
      in der Stacksemantik.
    \end{itemize}
  \end{enumerate}
\end{proof}

\begin{corollary}
  Wenn $\tj{e}{\beta}$, dann gilt: $(e,\varepsilon) \Downarrow c$ in der Umgebungssemantik gdw.
  $(e,\varepsilon) \Downarrow c$ in der Stacksemantik.
\end{corollary}


\section{Erweiterungen}


\subsection{Paare}

Die Syntax der Programmiersprache wird um Paare und Projektionen erweitert:
\[\begin{grammar}
  c \is (c_1,c_2) \mid \fst \mid \snd
  \nl
  e \is (e_1,e_2)
\end{grammar}\]
Die Werte m"ussen um Paarwerte erweitert werden
\[\begin{grammar}
  v \is (v_1,v_2)
\end{grammar}\]
und folgende neue Regeln m"ussen zur Umgebungssemantik hinzugenommen werden: \\[5mm]
\begin{tabular}{rl}
  \RN{Fst} & $\RULE{(e_1,\eta) \Downarrow \fst \quad (e_2,\eta) \Downarrow (v_1,v_2)}{(\app{e_1}{e_2},\eta) \Downarrow v_1}$ \\[3mm]
  \RN{Snd} & $\RULE{(e_1,\eta) \Downarrow \snd \quad (e_2,\eta) \Downarrow (v_1,v_2)}{(\app{e_1}{e_2},\eta) \Downarrow v_2}$ \\[3mm]
  \RN{Pair} & $\RULE{(e_1,\eta) \Downarrow v_1 \quad (e_2,\eta) \Downarrow v_2}{((e_1,e_2),\eta) \Downarrow (v_1,v_2)}$ \\[3mm]
\end{tabular} \\[5mm]
In der Stacksemantik kommen ebenfalls neue Werte
\[\begin{grammar}
  w \is (w_1,w_2)
\end{grammar}\]
und neue Regeln hinzu: \\[5mm]
\begin{tabular}{rl}
  \RN{Fst} & $\RULE{(e_1,\sigma) \Downarrow \fst \quad (e_2,\sigma) \Downarrow (v_1,v_2)}{(\app{e_1}{e_2},\sigma) \Downarrow v_1}$ \\[3mm]
  \RN{Snd} & $\RULE{(e_1,\sigma) \Downarrow \snd \quad (e_2,\sigma) \Downarrow (v_1,v_2)}{(\app{e_1}{e_2},\sigma) \Downarrow v_2}$ \\[3mm]
  \RN{Pair} & $\RULE{(e_1,\sigma) \Downarrow v_1 \quad (e_2,\sigma) \Downarrow v_2}{((e_1,e_2),\sigma) \Downarrow (v_1,v_2)}$ \\[3mm]
\end{tabular}

\subsubsection{Zusammenhang}

Es ist trivial einzusehen, dass die Zusammenh"ange in Theorem~\ref{theorem1} nach wie vor gelten, da die
neuen Regeln das $\sigma$ bzw. $\eta$ unver"andert durchreichen.


\subsection{Rekursion}

Wir erweitern die Syntax der Programmiersprache durch ein neues Konstrukt f"ur Rekursion:
\[\begin{grammar}
  e \is \letrec{f}{e}
\end{grammar}\]
In den Umgebungen m"ussen nun neue Eintr"age f"ur rekursive Funktionen erg"anzt werden
\[\begin{grammar}
  \eta \is \eta;f
\end{grammar}\]
und folgende neue Regeln m"ussen zur Umgebungssemantik hinzgenommen werden: \\[5mm]
\begin{tabular}{rl}
  \RN{Var-Rec} & $(\underline{0},\eta;f) \Downarrow (f,\eta;f)$ \\[1mm]
  \RN{Skip-Rec} & $\RULE{(\underline{i},\eta) \Downarrow v}{(\underline{i+1},\eta;f) \Downarrow v}$ \\[3mm]
  \RN{Let-Rec} & $\RULE{(e,\eta;f) \Downarrow c}{(\letrec{f}{e},\eta) \Downarrow c}$ \\[3mm]
\end{tabular} \\[5mm]
Auf Seiten der Stacksemantik m"ussen ebenfalls Eintr"age f"ur rekursive Funktionen vorgesehen werden
\[\begin{grammar}
  d \is f
\end{grammar}\]
und folgende neue Regeln m"ussen analog zur Stacksemantik hinzugef"ugt werden: \\[5mm]
\begin{tabular}{rl}
  \RN{Var-Rec} & $(\underline{0},\sigma;f) \Downarrow (f,|\sigma|+1)$ \\[1mm]
  \RN{Skip-Rec} & $\RULE{(\underline{i},\sigma) \Downarrow w}{(\underline{i+1},\sigma;f) \Downarrow w}$ \\[3mm]
  \RN{Let-Rec} & $\RULE{(e,\sigma;f) \Downarrow c}{(\letrec{f}{e},\sigma) \Downarrow c}$ \\[3mm]
\end{tabular} \\[5mm]

\subsubsection{Zusammenhang}

Die Relation $\sigma \sim \eta$ wird wie folgt erweitert: \\[5mm]
\begin{tabular}{c}
  $\RULE{\sigma \sim \eta}{\sigma;f \sim \eta;f}$ \\[3mm]
\end{tabular} \\[5mm]
Offensichtlich gilt weiterhin Lemma~\ref{lemma1}. Im Folgenden skizzieren wir kurz die
notwendige Erweiterung des Beweises f"ur Theorem~\ref{theorem1}.

\begin{proof}
  Wir betrachten nur die neuen F"alle.
  \begin{enumerate}
  \item Wiederum durch Induktion "uber die L"ange der Herleitung des big steps $(e,\sigma) \Downarrow w$
    und Fallunterscheidung nach der zuletzt angewandten Regel:
    \begin{itemize}
    \item[\RN{Var-Rec}] Dann gilt $(\underline{0},\sigma;f) \Downarrow (f,|\sigma|+1)$. Weiterhin gilt
      $(\sigma;f)^{|\sigma|+1} = \sigma;f$, also insbesondere $\sigma;f \models (f,|\sigma|+1) \sim (f,\eta;f)$.
      Offensichtlich existiert auch ein big step $(\underline{0},\eta;f) \Downarrow (f,\eta;f)$ mit
      \RN{Var-Rec} in der Umgebungssemantik.
      
    \item[\RN{Skip-Rec}] Dann gilt $(\underline{i+1},\sigma;f) \Downarrow w$, was nur aus der Pr"amisse
      $(\underline{i},\sigma) \Downarrow w$ folgen kann. Weiterhin existiert ein $\eta$, so dass
      $\sigma;f \sim \eta;f$, also insbesondere $\sigma \sim \eta$. Nach Induktionsvoraussetzung
      existiert also ein $v$, so dass $(\underline{i},\eta) \Downarrow v$ mit $\sigma \models w \sim v$.
      Also existiert auch ein big step $(\underline{i+1},\eta;f) \Downarrow v$ mit \RN{Skip-Rec} in der
      Umgebungssemantik. Mit Lemma~\ref{lemma1} folgt dar"uberhinaus $\sigma;f \models w \sim v$.

    \item[\RN{Let-Rec}] Dann gilt $(\letrec{f}{e},\sigma) \Downarrow c$, was nur aus $(e,\sigma;f) \Downarrow c$
      folgen kann. Da $\sigma \sim \eta$ gilt insbesondere $\sigma;f \sim \eta;f$, also folgt mit
      Induktionsvoraussetzung, dass ein $v$ existiert, so dass $(e,\eta;f) \Downarrow v$ und
      $\sigma;f \models c \sim v$. Das wiederum impliziert $v = c$, also existiert auch ein big step
      $(\letrec{f}{e},\eta) \Downarrow c$ mit \RN{Let-Rec} in der Umgebungssemantik.
    \end{itemize}

  \item Diese Richtung verl"auft analog. Sie bleibt dem geneigten Leser als "Ubung "uberlassen.
  \end{enumerate}
\end{proof}


\subsection{Simultane Rekursion}

Simultane Rekursion erfordert eine etwas subtilere Vorgehensweise. Hierbei ist entscheidend, dass
mehrere Funktionen auf die gleiche (rekursive) Umgebung Bezug nehmen. Zun"achst wird die Syntax
verallgemeinert:
\[\begin{grammar}
  e \is \letrec{f_1,\ldots,f_n}{e}
\end{grammar}\]
In den Umgebungen muss jetzt festgehalten werden, dass diese $f_1,\ldots,f_n$ auf ``gleicher Ebene''
stehen. Entsprechend sehen die verallgemeinerten Eintr"age
\[\begin{grammar}
  \eta \is \eta;(f_1,\ldots,f_n)
\end{grammar}\]
und die verallgemeinerten Regeln aus: \\[5mm]
\begin{tabular}{rl}
  \RN{Var-Rec} & $\RULE{0 \le i \le n}{(\underline{i},\eta;(f_n,\ldots,f_0)) \Downarrow (f_i,\eta;(f_n,\ldots,f_0))}$ \\[3mm]
  \RN{Skip-Rec} & $\RULE{(\underline{i},\eta) \Downarrow v \quad i \ge 1}{(\underline{i+n},\eta;(f_1,\ldots,f_n)) \Downarrow v}$ \\[3mm]
  \RN{Let-Rec} & $\RULE{(e,\eta;(f_1,\ldots,f_n)) \Downarrow c}{(\letrec{f_1,\ldots,f_n}{e},\eta) \Downarrow c}$ \\[3mm]
\end{tabular} \\[5mm]
Der Trick besteht -- wie unschwer zu erkennen -- darin, alle simultan rekursiv definierten Funktionen
bzgl. der Umgebung als Einheit zu betrachten.

Auf analoge Weise l"asst sich jetzt die Stacksemantik verallgemeinern, durch neue Eintr"age
\[\begin{grammar}
  d \is (f_1,\ldots,f_n)
\end{grammar}\]
und neue Regeln: \\[5mm]
\begin{tabular}{rl}
  \RN{Var-Rec} & $\RULE{0 \le i \le n}{(\underline{i},\sigma;(f_n,\ldots,f_0)) \Downarrow (f_i,|\sigma|+1)}$ \\[3mm]
  \RN{Skip-Rec} & $\RULE{(\underline{i},\sigma) \Downarrow w \quad i \ge 1}{(\underline{i+n},\sigma;(f_1,\ldots,f_n)) \Downarrow w}$ \\[3mm]
  \RN{Let-Rec} & $\RULE{(e,\sigma;(f_1,\ldots,f_n)) \Downarrow c}{(\letrec{f_1,\ldots,f_n}{e},\sigma) \Downarrow c}$ \\[3mm]
\end{tabular} \\[5mm]

\subsubsection{Zusammenhang}

Der Beweis l"auft analog zur einfachen Rekursion. Die Relation $\sigma \sim \eta$ muss erweitert werden
durch folgende Regel: \\[5mm]
\begin{tabular}{c}
  $\RULE{\sigma \sim \eta}{\sigma;(f_1,\ldots,f_n) \sim \eta;(f_1,\ldots,f_n)}$ \\[3mm]
\end{tabular} \\[5mm]
Der Beweis von Theorem~\ref{theorem1} f"ur simultane Rekursion bleibt dem Leser "uberlassen.


\end{document}
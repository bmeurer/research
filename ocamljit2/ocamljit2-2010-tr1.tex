\documentclass[10pt,a4paper,twocolumn]{article}

\usepackage{amssymb}
\usepackage[a4paper,%
            colorlinks=false,%
            draft,%
            pdfkeywords={},%
            pdftitle={},%
            pdfauthor={Benedikt Meurer},%
            pdfsubject={},%
            pdfdisplaydoctitle=true]{hyperref}

\author{Benedikt Meurer}
\title{OCAMLJIT 2.0 - Faster Objective Caml}

\begin{document}

\maketitle

\begin{abstract}
  This paper presents the initial state of an ongoing research
  project to improve the performance of the \textsc{Ocaml}
  byte-code interpreter using just in time native code generation.

  TODO
\end{abstract}


%% Introduction
\section{Introduction}

\textsc{OCaml}\cite{Leroy10} TODO


%% Related work
\section{Related work}

We present the existing \textsc{OCaml} system and the previous \textsc{OCamlJit}\cite{Starynkevitch04}
implementation based on the \textsc{Gnu Lightning}\cite{Bonzini10} library. For the sake of completeness
the overview of the \textsc{OCaml} compiler and runtime given in \cite{Starynkevitch04} is repeated below.
Readers already familiar with the internals of the \textsc{Ocaml} implementation can skip to
\ref{subsection:OCamlJit}.

\subsection{OCaml compiler usage}

The \textsc{OCaml} system can be used either interactively via the \texttt{ocaml} command, which
provides a ``\textit{read, compile, eval}'' loop, or in batch mode through the \texttt{ocamlc}
command, which takes a set of \texttt{*.ml} or \texttt{*.mli} source or \texttt{*.cmi} or
\texttt{*.cmo} byte-code object files\footnote{Source files are either module interfaces
  \texttt{*.mli} or module implementations \texttt{*.ml}. Byte-code object files are either
  compiled module interfaces \texttt{*.cmi} or compiled module implementations \texttt{*.cmo}.}
and produces a byte-code object file (to be reused by a subsequent invocation of \texttt{ocamlc})
or a byte-code executable file. \texttt{ocamlc} also handles \texttt{*.c} sources and \texttt{*.so}
shared libraries (for external C functions invoked from \textsc{OCaml}) and also deals with
byte-code library files \texttt{*.cma}. The byte-code executable files produced by \texttt{ocamlc}
are interpreted by the \texttt{ocamlrun} program\footnote{Technically speeking, an \textsc{OCaml}
  byte-code executable file is a \texttt{\#!/usr/bin/ocamlrun} Unix script, so \texttt{execve}
  of an \textsc{OCaml} byte-code executables invokes \texttt{ocamlrun}.} upon execution.

The goal of the present work is to substitute the interpreter part of the \textsc{OCaml} runtime
\texttt{ocamlrun} with a Just-In-Time compiler, which incrementally compiles the byte-code to
native code and runs the generated native code, instead of interpreting the byte-code. Unlike
\textsc{OCamlJit}\cite{Starynkevitch04}, we do not aim to provide a separate \texttt{ocamljitrun}
runtime, since that way one would have to explicitly use the runtime replacement to achieve the
benefits of Just-In-Time compilation.

\subsection{Overview of the OCaml system}

This section presents a brief overview of the \textsc{OCaml} byte-code compiler and runtime.
Feel free to skip to \ref{subsection:OCamlJit} if you are already familiar with the details.

\subsubsection{Compiler phases and representations}

Compilation starts by parsing an \textsc{OCaml} source file (or a source region in interactive
mode) into an abstract syntax tree (AST, see file \texttt{parsing/parsedtree.mli} of the \textsc{OCaml}
source code). Figure~\ref{figure:Abstract_syntax_tree} illustrates the abstract syntax tree for the expression
\texttt{let x=1 in x+3}. Compilation then proceeds by computing the type annotations to produce
a typed syntax tree (see file \texttt{typing/typedtree.mli}) as shown in figure~\ref{figure:Typed_syntax_tree}.

\begin{figure}[ht]
  \centering
  TODO
  \caption{Abstract syntax tree}
  \label{figure:Abstract_syntax_tree}
\end{figure}

\begin{figure}[ht]
  \centering
  TODO
  \caption{Typed syntax tree}
  \label{figure:Typed_syntax_tree}
\end{figure}

From this typed syntax tree, the byte-code compiler generates a so called {\em lambda representation} (see
file \texttt{bytecomp/lambda.mli})\footnote{The native code compiler \texttt{ocamlopt} has a similar
  representation (in file \texttt{asmcomp/clambda.mli}), which adds explicit direct or indirect calls and
  closures.} as shown in Figure~\ref{figure:Lambda_representation}, inspired by
the untyped call-by-value $\lambda$-calculus. This intermediate representation is not directly related to the
source code, because source file positions and some names are lost at this stage, and does not contain any
kind of explicit type information.

\begin{figure}[ht]
  \centering
  \texttt{(let (x/1038 1) (+ x/1038 3))}
  \caption{Lambda representation}
  \label{figure:Lambda_representation}
\end{figure}

This lambda tree representation includes leaves for variables and constants as well as nodes for function
definitions and applications, \texttt{let} and \texttt{let rec}, primitive operations (like addition, etc.),
switches (used in compiled forms of \texttt{match} expressions), exception handling (\texttt{raise}
and \texttt{try$\ldots$with} nodes), imperative traits (sequences, \texttt{for} and \texttt{while}
loops), etc.

After several optimizations are applied\footnote{These are mostly peephole optimizations, transforming
  lambda trees into {\em better} or smaller lambda tress.}, the lambda tree representation is
transformed into a list of byte-code instructions as shown in figure~\ref{figure:Byte_code}.
This instruction list is optimized and afterwards written into the generated byte-code file
(or kept in a memory buffer for the interactive top-level).

\begin{figure}[ht]
  \centering
  \begin{tabular}{l}
    \texttt{const 1} \\
    \texttt{push} \\
    \texttt{acc 0} \\
    \texttt{offsetint 3}
  \end{tabular}
  \caption{Byte-code}
  \label{figure:Byte_code}
\end{figure}

The \texttt{ocaml} top-level and the \texttt{ocamlc} compiler provide two undocumented command
line options \texttt{-dlambda} and \texttt{-dinstr} to display the internal representations
mentioned above.

\subsubsection{The OCaml virtual machine}

The \textsc{OCaml} virtual machine (see file \texttt{byterun/interp.c}) is an interpreter for the
byte-code\footnote{Since each token of the byte-code is a 32-bit word, the byte-code is actually
  a \emph{word-code}.} produced by \texttt{ocamlc} (as described in the previous section). It operates on a 
stack of \textsc{OCaml} values and six \emph{virtual registers}: the stack pointer \texttt{sp},
the accumulator \texttt{accu}, the environment pointer \texttt{env} (pointing to the current
closure, which contains the values for the free variables), the trap frame pointer \texttt{trapsp},
the extra arguments counter \texttt{extra\_args} and the byte-code program counter \texttt{pc}.
The virtual machine also deals with byte-code segments (byte-code sequences to be interpreted)
and a global data array \texttt{caml\_global\_data} of \textsc{OCaml} values. There is usually
only a single byte-code segment, the byte-code sequence in the executable file produced by
\texttt{ocamlc}, which also contains the marshalled representation of the global data.

\textsc{OCaml} values are either pointers (usually pointing to blocks in the garbage collected
heap) or tagged integers. Each block starts with a header containing a tag, the size of the block
and the color (used by the garbage collector). Some tags describe special blocks like strings,
closures or floating point arrays, but most of them are used for representing sum types.
\textsc{OCaml} distinguishes pointers and tagged integers using the least significant bit: if
the least significant bit is $1$, the value is a tagged integer, with the integer value stored
in the remaining $31$ or $63$ bits, otherwise the value is a pointer.

The \textsc{OCaml} stack, which is -- in contrast to the native code compiler -- separate from
the native C stack, contains values, byte-code return addresses and extra argument counts,
organized into so-called \emph{call frames}. The byte-code is pointed to from either return
addresses stored on the stack or closures stored in the heap. The first slot of every closure
contains the byte-code address of the closure's function code; the remaining slots contain
the values of the free variables, thereby forming the environment of the closure.

The byte-code interpreter \texttt{ocamlrun} starts by unmarshalling the global data, loading
the byte-code sequence into memory and processing dynamically linked libraries containing
external C function primitives (referenced from the byte-code file). Once everything is in
place, the function \texttt{caml\_interprete} (see file \texttt{byterun/interp.c}) is invoked
to interpret the initial byte-code segment (using threaded code \cite{Bell73,ErtlGregg03}),
starting from its first byte-code, with an empty stack, and a default environment and accumulator.

\subsubsection{The byte-code instruction set}

Every byte-code instruction is represented by one or more consecutive 32-bit words. The
first word of each byte-code includes the operation (see file \texttt{byterun/instruct.c})
and the remaining words include the operands, which are constant integer arguments (usually
offsets). Most byte-codes operate on the accumulator and the top-most stack elements and
produce a result in the accumulator. Some byte-codes include offsets (denoted $p,q,\ldots$
or byte subscripts), which are either encoded within the operation or following the
operation. All byte-code offsets referencing other byte-codes are interpreted relative to their
respective positions, so the byte-code forms a \emph{position-independent code}.

Byte-code instructions are classified as follows:
\begin{itemize}
\item Stack manipulation instructions: $\mathtt{ACC}_p$ loads \texttt{accu} with the value $p$-th topmost stack
  cell; $\mathtt{PUSH}$ pushes the accumulator value onto the stack; $\mathtt{POP}_p$ pops the topmost $p$ elements
  off the stack; $\mathtt{ASSIGN}_p$ places the accumulator value into the $p$-th topmost stack cell.
\item Loading instructions: $\mathtt{CONSTINT}_p$ loads \texttt{accu} with the tagged integer $p$;
  $\mathtt{ATOM}_p$ loads \texttt{accu} with the $p$-tagged atom\footnote{Atoms are preallocated $0$-sized
    blocks}.
\item Primitive operations: unary $\mathtt{NEGINT}$ (negation) and $\mathtt{BOOLNOT}$ (logical negation)
  operate on \texttt{accu}; binary operations $\mathtt{ADDINT}$, $\mathtt{SUBINT}$, $\mathtt{MULINT}$,
  $\mathtt{DIVINT}$, $\mathtt{MODINT}$, $\mathtt{ANDINT}$, etc. take the first operand in \texttt{accu}
  and the second one popped off the stack, and place the result into the accumulator, where division and
  modulus also check for $0$ and may raise an exception; comparisons $\mathtt{EQ}$, $\mathtt{NEQ}$,
  $\mathtt{LTINT}$, $\mathtt{LEINT}$, etc. similarly place their boolean result into \texttt{accu};
  $\texttt{ISINT}$ tests whether the accumulator value is a tagged integer.
\item Environment operations: $\mathtt{ENVACC}_p$ loads \texttt{accu} with the $p$-th slot of
  the current environment \texttt{env}.
\item Apply operations are divided into two categories: $\mathtt{APPLY}_p$ creates a new call frame on
  the \textsc{OCaml} stack and jump to the called code, whereas $\mathtt{APPTERM}_{p,q}$ ($q$ denotes the
  height of the current stack frame) performs a tail-call to the called code. In either case the accumulator
  contains the closure which is applied to the $p$ topmost arguments on the \textsc{OCaml} stack and becomes
  the new \texttt{env}. Call frames contain arguments, byte-code return address and previous arity
  \texttt{extra\_args}; when adding call frames, the stack may grow if necessary.
\item Function return instructions: $\mathtt{RETURN}_p$ pops the topmost $p$ values off the stack and
  returns to the caller; $\mathtt{RESTART}$ and $\mathtt{GRAB}_p$ handle partial application (creating
  appropriate closures as needed).
\item Closure allocation instructions: $\mathtt{CLOSURE}_{p,q}$ allocates a single closure of $p$ variable slots and
  byte-code at offset $q$; $\mathtt{CLOSUREREC}_{p,q,k_1,\ldots,k_p}$ allocates a mutually recursive 
  closure with $p$ functions $k_1,\ldots,k_p$ and $q$ variable slots.
\item Allocation instructions: $\mathtt{MAKEBLOCK}_{p,q}$ creates a $p$-tagged block of $q$ values (first
  in \texttt{accu}, remaining popped off the stack); $\mathtt{MAKEFLOATBLOCK}_p$ creates an array of floats.
  Every allocation may trigger a garbage collection. The accumulator is loaded with the pointer to the
  newly allocated block.
\item Field access and modify instructions: $\mathtt{GETGLOBALFIELD}_{p,q}$ loads \texttt{accu} with the
  $q$-th field of the $p$-th global value; $\mathtt{GETFIELD}_p$ loads the accumulator with the $p$-th
  field of the block pointed to by \texttt{accu}; symmetrically $\mathtt{SETFIELD}_p$ sets the $p$-th
  field of \texttt{accu} to the value popped off the stack; similarly $\mathtt{GETFLOATFIELD}_p$ and
  $\mathtt{SETFLOATFIELD}_p$ handle fields in floating point blocks; $\mathtt{GETGLOBAL}_p$ and
  $\mathtt{SETGLOBAL}_p$ handle global fields in \texttt{caml\_global\_data}; $\mathtt{GETSTRINGCHAR}$
  and $\mathtt{SETSTRINGCHAR}$ are used to access and update characters within strings (the index
  being popped off the stack). All modifying operations have to cooperate with the garbage collector.
\item Control instructions include -- in addition to the instructions for function application --
  conditional $\mathtt{BRANCHIF}_p$ and $\mathtt{BRANCHIFNOT}_p$ (depending upon boolean value in
  \texttt{accu}), comparing $\mathtt{BEQ}_{p,q}$, $\mathtt{BNEG}_{p,q}, \ldots$ (comparing $q$ with
  the integer value in \texttt{accu}) and unconditional $\mathtt{BRANCH}_p$ jumps. There is also
  a $\mathtt{SWITCH}_{p,q,k_1,\ldots,k_p,k_1',\ldots,k_q'}$ instruction (used to compile the \textsc{OCaml}
  \texttt{match} construct), which tests \texttt{accu} and jumps to offset $k_{i-1}$ if \texttt{accu}
  is the tagged integer $i$ or $k_{j-1}'$ if \texttt{accu} points to a $j$-tagged block.
  Exception and signal handling instructions $\mathtt{PUSHTRAP}_p$, $\mathtt{POPTRAP}$, $\mathtt{RAISE}$
  $\mathtt{CHECK\_SIGNALS}$ are also control instructions, as is the halting $\mathtt{STOP}$ instruction,
  which is the last byte-code of a segment.
\item Calling external C primitives: $\mathtt{C\_CALL}_{p,q}$ calls the $p$-th C primitive with
  $q$ arguments (first taken from \texttt{accu}, remaining popped off the stack). The result
  is stored into \texttt{accu}. Primitives are used for several basic operations, including
  operations on floating point values.
\item Object-oriented operations: $\mathtt{GETMETHOD}$ retrieves an object method by its index;
  $\mathtt{GETPUBMET}$ and $\mathtt{GETDYNMET}$ fetch the method for a given method tag.
\item Debugger related instructions: the \textsc{OCaml} debugger places breakpoints by
  overwriting byte-code instructions with $\mathtt{BREAK}$; these are unsupported by
  \textsc{OCamlJit2}.
\end{itemize}

\subsection{OCamlJit} \label{subsection:OCamlJit}

In this section we briefly describe the design and implementation of the \textsc{Gnu Lightning}\cite{Bonzini10}
based \textsc{OCamlJit}\cite{Starynkevitch04},
which inspired many aspects of our present work. The main goal of \textsc{OCamlJit} is maximal compatibility
with the \textsc{OCaml} byte-code interpreter, its runtime system (including the garbage collector), and its
behavior. That means that a program running under \texttt{ocamljitrun} sees the same virtual machine as
\texttt{ocamlrun}, so utilizing \textsc{OCamlJit} is a matter of exchanging \texttt{ocamlrun} with
\texttt{ocamljitrun} when executing \textsc{OCaml} programs.

In order to meet these constraint, \textsc{OCamlJit} has to mimic as much as possible the byte-code
interpreter.

\subsubsection{Implementation}

Since \textsc{Lightning} provides only a few registers, the most commonly used \textsc{OCaml} virtual
registers (\texttt{accu}, \texttt{sp}, \texttt{env}) are mapped to \textsc{Lightning} registers (hence
into machine registers), while other less common \textsc{OCaml} registers are grouped into a \emph{state
record} pointed to by a \textsc{Lightning} register.

The core of \textsc{OCamlJit} is the \texttt{caml\_jit\_translate} C function, which scans a byte-code
sequence and emits equivalent native code. It loops around a big \texttt{switch} statement, with one
\texttt{case} for every byte-code instruction. For example, figure~\ref{figure:caml_jit_translate_andint} shows
the case for the \texttt{ANDINT} instruction, which pops a tagged integer off the \textsc{OCaml} stack
and does a bitwise ``and'' with \texttt{accu}.

\begin{figure}[ht]
  \centering
  \begin{verbatim}
 case ANDINT:
   /* tmp1 = *sp; ++sp; accu &= tmp1; */
   jit_ldr_p(JML_REG_TMP1, JML_REG_SP);
   jit_addi_p(JML_REG_SP, JML_REG_SP,
              WORDSIZE);
   jit_andr_l(JML_REG_ACCU, JML_REG_ACCU,
              JML_REG_TMP1);
   break;
\end{verbatim}
  \caption{\texttt{ANDINT} case in \texttt{caml\_jit\_translate}}
  \label{figure:caml_jit_translate_andint}
\end{figure}

\paragraph{Byte-code and native-code addresses}

Total compatibility demands the use of byte-code addresses, in particular in \textsc{OCaml}
closures. This means that closure application -- a very common operation in \textsc{OCaml} -- has
to retrieve the byte-code from the closure, find the corresponding native machine code, and jump to
it. The byte-code address to native machine code address mapping is accomplished by a sparse 
translation hash-table. An entry is added to this table for every compiled byte-code instruction.

Every byte-code segment has its own native code block, which references a linked list of native
code chunks. Such a code chunk is a page-aligned executable memory segment, allocated via the
\texttt{mmap} system call with read, write and execute permissions.

The byte-code is incrementally translated to native machine code. The native code generation is
interrupted once the current native code chunk is filled, or when a \texttt{STOP} instruction is
reached, when a configurable number of byte-code instructions has been translated, or when the
currently translated byte-code address is already known in the translation hash-table
(because a byte-code sequence containing it was previously translated).

\paragraph{Interaction with the runtime}

The generated native machine code has to interact with the runtime. A common interaction is
invoking the garbage collector when space in the minor heap is exhausted during allocations
(i.e. \texttt{MAKEBLOCK}, \texttt{CLOSURE}, etc.). Most of the commonly used runtime interactions
are inlined in the generated native code.

Less common interactions, which are too complex to be inlined, are done by saving the current
state of the virtual machine, returning from the virtual machine with a particular
\texttt{MLJSTATE\_*} (passing the required data in the state structure), and letting the
\texttt{caml\_interprete} function do the processing. For example \texttt{MLJSTATE\_MODIF} is
used to modify a heap cell while \texttt{MLJSTATE\_RAISE} is used to raise an exception (via C).

\paragraph{Performance}

On x86 machines, \texttt{ocamljitrun} typically gives significant
speedups w.r.t. \texttt{ocamlrun} of a factor above two. While this is already a quite
interesting achievement, there are however some shortcomings to be noted:
\begin{enumerate}
\item The compilation overhead is high even with this naive compilation scheme. This is
  especially noticable in short running programs. For example, building the \textsc{OCaml}
  standard library is nearly three times slower with \texttt{ocamljitrun} than with
  \texttt{ocamlrun}. There are various points that add to this slow translation speed; for
  example, adding every (byte-code, native-code) address pair to the translation
  hash-table takes a significant amount of the translation time -- actually more than a fourth
  of it\cite{Starynkevitch04}.
\item The limited register set provided by \textsc{Gnu Lightning} prevents effecient register
  usage for modern targets like x86\_64, ARM or PowerPC, which offer more than just 8 general
  purpose registers. In addition, the \emph{portability at the lowest level} approach of
  \textsc{Gnu Lightning} prevents several interesting platform specific optimizations.
\item The naive compilation scheme (with peephole optimizations) is limited in efficiency. There
  is almost no way to achieve performance close to the optimizing native code compiler \texttt{ocamlopt}.
\end{enumerate}
The present work is part of a research project to overcome these problems. Up until now we have
already addressed the first two of the shortcomings listed above.


%% Earlier work
% \section{Earlier work}

% We started the project with an early prototype of an \textsc{OCaml} JIT engine based on the
% \textsc{LLVM} compiler framework\cite{Lattner02,Lattner04}. \textsc{LLVM}\footnote{\url{http://llvm.org/}}
% provides an internal code representation (IR), which describes a program using an abstract RISC-like instruction
% set, and offers C/C++ APIs for code generation and Just-In-Time compilation, which currently supports
% x86, x86\_64 and PowerPC.

% Our prototype did a straight-forward translation of byte-code to the \textsc{LLVM} IR, using tail-calls
% to implement function application and return. To keep the prototype simple, we chose to compile one
% byte-code segment at a time.

% Even though our approach was everything but smart, the quality of the x86/x86\_64 native code generated by
% the \textsc{LLVM} JIT compiler looked really promising. For programs running longer than 5 seconds,
% this naive JIT engine was around twice as fast as the byte-code interpreter. However, for short running
% programs, the JIT compilation overhead was more than noticable, causing a slowdown of up to a factor of
% five compared to the byte-code interpreter.

% Similar results were observed by other projects using \textsc{LLVM} to JIT-compile byte-code at runtime.
% For example, the Mono project\footnote{\url{http://www.mono-project.com/}}, which aims to provide an open source
% implementation of the .NET platform, reported impressive speedups for longer running, computationally intensive
% applications using \textsc{LLVM}, but at the cost of increased compilation time and memory usage.

% While this may make sense for a .NET implementation, it doesn't make sense for an \textsc{OCaml}
% byte-code runtime. Long running, computationally intensive \textsc{OCaml} applications should be
% compiled with the optimizing native code compiler \texttt{ocamlopt}. The primary use cases for the
% byte-code runtime are: driving the interactive top-level, executing the various tools from the
% \textsc{OCaml} distribution (i.e. \texttt{ocamlc}, \texttt{ocamlopt}, etc.), and running other small
% to mid-sized \textsc{OCaml} programs. Hence short compilation times are more beneficial than advanced
% numerical optimizations.

% We therefore canceled all work on the \textsc{LLVM} based \textsc{OCaml} JIT engine, and started from
% scratch with a custom JIT engine, that is described in the next section.


%% Design and implementation
\section{Design and implementation} \label{section:Design_and_implementation}

We aim to provide almost the same amount of compatibility with the \textsc{OCaml} byte-code interpreter
as provided by \textsc{OCamlJit}, although not at all costs. At each point where we have to either
adjust the runtime or add costly work-arounds within the JIT engine, we choose to adjust the
runtime. Nevertheless, \textsc{OCamlJit2} should be able to handle any byte-code sequence, and the
heap and stack should be the same as with the \textsc{OCaml} byte-code interpreter.

Our design goals are therefore quite similar to those of \textsc{OCamlJit}:
\begin{itemize}
\item The runtime (i.e. the garbage collector and the required C primitives) should be 
  (roughly) the same as in the byte-code interpreter.
\item The heap is the same, in particular all \textsc{OCaml} values keep exactly the same representation
  (i.e. the code pointer inside a closure is still a byte-code pointer, not a native code pointer). The
  relevant parts of the byte-code runtime (i.e. the serialization mechanism) are reused\footnote{Especially
    hashing and serializing closures or objects (with their classes) give the same result
    in \textsc{OCamlJit2} than in \textsc{OCaml} byte-code.}.
\item C programs embedding the \textsc{OCaml} byte-code interpreter should work with \textsc{OCamlJit2}
  as well, without the need to recompile the C program\footnote{We therefore need to ensure that the
    API of \textsc{OCamlJit2}'s \texttt{libcamlrun\_shared.so} match exactly the API of \textsc{OCaml}'s
    \texttt{libcamlrun\_shared.so}.}.
\item \textsc{OCamlJit2} should execute programs at least as fast as the byte-code interpreter,
  even for short running programs. In particular, the compilation time must not be noticably
  longer when evaluating phrases within the \texttt{ocaml} top-level.
\end{itemize}
A program running with \textsc{OCamlJit2} should therefore see the same virtual machine as seen when
running with the byte-code interpreter. It should not be able to tell whether it is being interpreted
or JIT compiled, except by measuring the execution speed.

\subsection{Address mapping}

As previously described, \textsc{OCamlJit} uses a sparse hash-table to map byte-code addresses to
native machine code addresses.
Whenever the virtual machine needs to jump to a byte-code pointer (i.e. during closure application),
it needs to find the corresponding native machine code in the hash-table, and jump to it (falling back
to the JIT compiler if no native code address is recorded).

Since closure application is a somewhat common operation in \textsc{OCaml}, this indirection via a
hash-table does not only complicate the implementation, but also decreases the performance of the
virtual machine. In addition, constructing the hash-table during JIT compilation takes a significant
amount of the translation time. To make matters worse, \textsc{OCamlJit} adds an entry for every
translated (byte-code, native-code) address pair to the hash-table, even though the vast majority
of these entries is never used.

Using a hash-table for the address mapping was therefore not an option for \textsc{OCamlJit2}. While
looking for another way to associate native code with byte-code addresses, we found that the threaded
byte-code interpreter is faced with a similar problem, whose solution may also be applicable in the
context of a JIT engine. When threading is enabled for the byte-code interpreter\footnote{This requires
the byte-code interpreter to be compiled with \texttt{gcc} at the time of this writing.}, prior to
execution, every byte-code segment is preprocessed by the C function \texttt{caml\_thread\_code},
which replaces every instruction opcode in the byte-code sequence with the offset of the C code
implementing the instruction relative to some well-defined base address. Executing a byte-code
instruction is then a matter of determining the C code address from the offset in the instruction
opcode and the well-defined base address\footnote{On 32bit targets the base address is $0$, hence the
offset is already the address, while on 64bit targets the offset is added to the base address.},
and jumping to that address.

\subsection{TODO}

\begin{enumerate}
\item Byte-code instruction replaced with native code offset; positive value is (not yet compiled)
  byte-code instruction (or (not yet compiled) branch target), negative value is offset of native code
  for this byte-code relative to \texttt{caml\_jit\_code\_end}
\item Incremental compilation; on function basis (with few exceptions, i.e. trap handlers might not be
  compiled immediately); driven by \texttt{caml\_jit\_compile}
\item Compilation triggered by compile trampoline; jump to a code pointer \texttt{bcp} is done
  by \texttt{jmp caml\_jit\_code\_end+*bcp}, a negative \texttt{*bcp} resolves to the already available
  native code for the byte-code, positive \texttt{*bcp} (must be a byte-code instruction then) causes
  a jump to the compile trampoline (via NOP indirections)
\end{enumerate}


%% Future work
\section{Further work}

TODO


%% Conclusion
\section{Conclusion}

TODO


%% References
\bibliographystyle{plain}
\bibliography{citations}

\end{document}
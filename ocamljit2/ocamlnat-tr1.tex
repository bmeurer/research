\documentclass[10pt,a4paper,draft,twocolumn]{article}

\usepackage{amssymb}
\usepackage[english]{babel}
\usepackage{color}
\usepackage[colorlinks=false,%
            pdfkeywords={OCaml, Just In Time compilation, machine code generation},%
            pdftitle={Towards a usable native toplevel for OCaml},%
            pdfauthor={Marcell Fischbach, Benedikt Meurer},%
            pdfsubject={},%
            pdfdisplaydoctitle=true]{hyperref}
\usepackage{tikz}
\usepackage{varwidth}

\usetikzlibrary{arrows}
\usetikzlibrary{trees}
\usetikzlibrary{arrows,decorations.pathmorphing,backgrounds,positioning,fit,petri}

\begin{document}

\title{%
  Towards a usable native toplevel for OCaml
}
\author{%
  Marcell Fischbach\thanks{
    Universit\"at Siegen,
    D-57068 Siegen,
    Germany,
    \url{marcellfischbach@googlemail.com}
  }
  \and
  Benedikt Meurer\thanks{
    Compilerbau und Softwareanalyse,
    Naturwissenschaftl.-Technische Fakult\"at,
    Universit\"at Siegen,
    D-57068 Siegen,
    Germany,
    \url{meurer@informatik.uni-siegen.de}
  }
}
\date{}

\maketitle

\begin{abstract}
  TODO
\end{abstract}


%% Introduction
\section{Introduction}

The OCaml \cite{Leroy11,Remy02} system is the main implementation of the Caml
language \cite{Caml11}, featuring a powerful module system
combined with a full-fledged object-oriented layer. It comes with an optimizing native
code compiler \texttt{ocamlopt}, for high performance; a byte-code compiler \texttt{ocamlc}
and interpreter \texttt{ocamlrun}, for increased portability; and an interactive top-level
\texttt{ocaml} based on the byte-code runtime, for interactive use of OCaml through a
read-eval-print loop.

\texttt{ocamlc} and \texttt{ocaml} translate the source code into a sequence of byte-code
instructions for the OCaml virtual machine \texttt{ocamlrun}, which is based on the ZINC
machine \cite{Leroy90} originally developed for Caml Light \cite{Leroy02}. The optimizing
native code compiler \texttt{ocamlopt} produces fast machine code for the supported targets
(at the time of this writing, these are Alpha, ARM, Itanum, Motorola 68k, MIPS, PA-RISC, PowerPC,
Sparc, and x86/x86-64), but is currently only applicable to \emph{static program compilation}.
For example, it cannot yet be used with multi-stage programming in MetaOCaml \cite{Taha03,Taha06},
the Coq proof assistant \cite{Bertot04,Coq10}, or the interactive toplevel \texttt{ocaml}.

This paper presents our work on a new native OCaml toplevel, called \texttt{ocamlnat}, which is
based on the native runtime and the compilation engine of the optimizing native code compiler.

TODO


%% Existing systems
\section{Existing systems} \label{section:Existing_systems}

\tikzset{
  phase/.style={
      rectangle,
      inner sep=2mm,
      minimum size=6mm,
      thick,
      draw=blue!50!black!50,
      top color=white,
      bottom color=blue!50!black!20
  }
}

\begin{figure}[htb]
  \centering
  \begin{tikzpicture}[node distance=8mm,text height=1.5ex,text depth=.25ex]
    \node (start) {};
    \node[phase] (parsing) [below=of start] {Parsing} edge[<-] node[right] {\it source program} (start);
    \node[phase] (typing) [below=of parsing] {Typing} edge[<-] node[right] {\tt Parsetree} (parsing);
    \node[phase] (transl) [below=of typing] {Translate} edge[<-] node[right] {\tt Typedtree} (typing);
    \node[phase] (simplif) [below=of transl] {Simplify} edge[<-] node[right] {\tt Lambda} (transl);
    \node (end) [below=of simplif] {} edge[<-] node[right] {\tt Lambda} (simplif);
  \end{tikzpicture}
  \caption{Shared compiler frontend}
  \label{fig:Shared_compiler_frontend}
\end{figure}

TODO


%% Conclusion
\section{Conclusion} \label{section:Conclusion}

TODO


%% Acknowlegdements
\section*{Acknowledgements}

TODO


%% References
\bibliographystyle{abbrv}
\bibliography{citations}

\end{document}

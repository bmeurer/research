\documentclass[12pt,a4paper]{article}

\usepackage{amsmath}
\usepackage{amssymb}
\usepackage{amsthm}
\usepackage{mathpartir}

\theoremstyle{definition}
\newtheorem{definition}{Definition}
\theoremstyle{plain}
\newtheorem{lemma}{Lemma}
\newtheorem{theorem}{Theorem}

\newcommand{\abort}{\ensuremath{\mathsf{abort}}}
\newcommand{\abstr}[2]{\ensuremath{\lambda{#1}.\,{#2}}}
\newcommand{\app}[2]{\ensuremath{{#1}\,{#2}}}
\newcommand{\Nat}{\ensuremath{\mathsf{Nat}}}

\DeclareMathOperator{\dom}{dom}

\begin{document}

\author{Benedikt Meurer}
\date{\today}
\title{Foundational safety proofs using\\big-step operational semantics}
\maketitle


Foundational proofs rely only on the operational semantics of the language and the axioms of
the higher-order logic (or any other suitable logic that may be used to encode the proofs)


\section{The language and its big-step semantics}
\label{sec:The_language_and_its_big_step_semantics}

The language we consider in this section is the pure $\lambda$-calculus extended with constants,
the simplest functional language that exhibits run-time errors (closed terms that ``go wrong'').
Its syntax is as follows: \\[3mm]
\begin{tabular}{lrcl}
  Variables: & \multicolumn{3}{l}{$x,y,z,\ldots$} \\
  Constants: & $c$ & $::=$ & $\mathsf{0} \mid \mathsf{1} \mid \ldots$ \\
  Terms: & $a,b$ & $::=$ & $c \mid x \mid \abstr{x}{a} \mid \app{a}{b}$ \\
\end{tabular} \\[3mm]
We write $a[x \mapsto b]$ for the capture-avoiding substitution of $b$ for all unbound occurrences
of $x$ in $a$. A term $v$ is a \emph{value} if it is a constant $c$ or a closed term of the form
$\abstr{x}{a}$. A \emph{result} $r$ is either a value $v$ or the special token $\abort$, which is
used to indicate abortion due to a run-time error.

\begin{figure}[htb]
  \centering
  \begin{mathpar}
    \inferrule[(Val)]{%
    }{%
      v \Rightarrow v
    }%
    \and
    \inferrule[(App)]{%
      a_1 \Rightarrow \abstr{x}{b} \\
      a_2 \Rightarrow v_2 \\
      b[x \mapsto v_2] \Rightarrow v
    }{%
      \app{a_1}{a_2} \Rightarrow v
    }%
    \and
    \inferrule[(App-Abort)]{%
      a_1 \Rightarrow c
    }{%
      \app{a_1}{a_2} \Rightarrow \abort
    }%
    \and
    \inferrule[(App-Abort-1)]{%
      a_1 \Rightarrow \abort
    }{%
      \app{a_1}{a_2} \Rightarrow \abort
    }%
    \and
    \inferrule[(App-Abort-2)]{%
      a_1 \Rightarrow \abstr{x}{b} \\
      a_2 \Rightarrow \abort
    }{%
      \app{a_1}{a_2} \Rightarrow \abort
    }%
    \and
    \inferrule[(App-Abort-3)]{%
      a_1 \Rightarrow \abstr{x}{b} \\
      a_2 \Rightarrow v_2 \\
      b[x \mapsto v_2] \Rightarrow \abort
    }{%
      \app{a_1}{a_2} \Rightarrow \abort
    }%
  \end{mathpar}
  \caption{Big-step inference rules}
  \label{fig:Big_step_inference_rules}
\end{figure}

The standard call-by-value semantics with explicit abortion in big-step style for this language
is defined by the inference rules in figure~\ref{fig:Big_step_inference_rules}, interpreted inductively.
More precisely, the relation $a \Rightarrow r$ (read: ``$a$ evaluates to $r$'') holds iff
it is the conclusion of a finite derivation tree built from the given rules. A conclusion
$a \Rightarrow \abort$ indicates that the evaluation got stuck due to a run-time error.


\section{Semantic types}
\label{sec:Semantic_types}

\begin{definition} \label{def:Types}
  A \emph{type} is a set $\tau$ of values. A term $a$ is of type $\tau$, denoted by $a : \tau$,
  if $a$ is closed and $a \Downarrow r$ implies $r \in \tau$ for all results $r$.
\end{definition}

Intuitively, $a : \tau$ means that $a$ diverges or evaluates to a value of type $\tau$. Note that
for a value $v$ the statements $v : \tau$ and $v \in \tau$ are equivalent.
We can now specify the semantic types.
\[\begin{array}{rcl}
  \bot &\equiv& \emptyset \\
  \top &\equiv& \{ v \mid \text{$v$ is a value} \} \\
  \Nat &\equiv& \{ \mathsf{0},\mathsf{1},\ldots \} \\
  \tau \to \tau' &\equiv& \{ \abstr{x}{a} \mid \text{$a[v/x] : \tau'$ for every $v \in \tau$} \} \\
\end{array}\]

\begin{definition} \label{def:Typing}
  A \emph{type environment} is a mapping $\Gamma$ from variables to types. A \emph{value environment}
  (ground substitution) is a mapping $\gamma$ from variables to values. We write $\gamma : \Gamma$ if
  $\dom(\gamma) = \dom(\Gamma)$ and $\gamma(x) : \Gamma(x)$ for every $x \in \dom(\gamma)$.
  We write $\Gamma \models a : \tau$ if $\gamma(a) : \tau$ for every $\gamma : \Gamma$, where $\gamma(a)$
  is the result of replacing the free variables of $a$ with their values under $\gamma$.
\end{definition}

Given these definitions, we can now prove the usual ``typability implies safety'' theorem.
Using our semantic types, this is almost trivial.

\begin{theorem}[Safety]
  If $\emptyset \models a : \tau$, then $a \not\Rightarrow \abort$.
\end{theorem}

\begin{proof}
  Immediate consequence of definitions \ref{def:Types} and \ref{def:Typing}.
\end{proof}

\begin{figure}[htb]
  \centering
  \begin{mathpar}
    \inferrule[(T-Var)]{%
    }{%
      \Gamma \models x : \Gamma(x)
    }%
    \and
    \inferrule[(T-Const)]{%
    }{%
      \Gamma \models c : \Nat
    }%
    \and
    \inferrule[(T-Abs)]{%
      \Gamma[x \mapsto \tau] \models a : \tau'
    }{%
      \Gamma \models \abstr{x}{a} : \tau \to \tau'
    }%
    \and
    \inferrule[(T-App)]{%
      \Gamma \models a : \tau \to \tau' \\
      \Gamma \models b : \tau
    }{%
      \Gamma \models \app{a}{b} : \tau'
    }%
  \end{mathpar}
  \caption{Semantic typing lemmata}
  \label{fig:Semantic_typing_lemmata}
\end{figure}

The remainder of this section is devoted to proving the semantic typing lemmata in
figure~\ref{fig:Semantic_typing_lemmata}. Each of these lemmata states that if
certain instances of the relation $\models$ hold, then certain other instances hold.
Note that $\models$ can be viewed as a three place relation where $\Gamma \models a : \tau$
means that the relation $\models$ holds on the type environment $\Gamma$, the term $a$, and
the type $\tau$. Once we have proved these lemmata, they can be used in the same manner as
the standard typing rules to prove statements of the form $\Gamma \models a : \tau$.

We now consider each of the typing lemmata in figure~\ref{fig:Semantic_typing_lemmata}. Note
that there is a typing lemma for every syntactic case in the grammar of terms. The lemma
\textsc{(T-Var)}, stating that $\Gamma \models x : \Gamma(x)$ holds, follows immediately from
the definition of $\models$. The fact that $\Nat$ is a type, and the lemma \textsc{(T-Const)},
stating that $\Gamma \models c : \Nat$, both follow directly from the definition of $\Nat$.
It remains for us to show that the lemmata for applications and lambda terms hold.
First we observe that $\tau \to \tau'$ is always a type.

\begin{lemma}[Abstraction]
  Let $\Gamma$ be a type environment, let $\tau$ and $\tau'$ be types, and let
  $\Gamma[x \mapsto \tau]$ be the type environment that is identical to
  $\Gamma$ except that it maps $x$ to $\tau$. If $\Gamma[x \mapsto \tau] \models a : \tau'$,
  then $\Gamma \models \abstr{x}{a} : \tau \to \tau'$.
\end{lemma}

\begin{proof}
  Let $\gamma : \Gamma$. $\gamma(\abstr{x}{a}) \Rightarrow r$ implies $r = \gamma(\abstr{x}{a})$.
  $\Gamma[x \mapsto \tau] \models a : \tau'$ implies $(\gamma[x \mapsto v])(a) \in \tau'$ for
  every $v \in \tau$. Hence we have $\gamma(\abstr{x}{a}) \in \tau \to \tau'$ by definition of $\to$.
\end{proof}

\begin{lemma}[Application]
  Let $\Gamma$ be a type environment, and let $\tau$ and $\tau'$ be types.
  If $\Gamma \models a_1 : \tau \to \tau'$ and $\Gamma \models a_2 : \tau$,
  then $\Gamma \models \app{a_1}{a_2} : \tau'$.
\end{lemma}

\begin{proof}
  Let $\gamma : \Gamma$ and $\gamma(\app{a_1}{a_2}) \Rightarrow r$. Then there has to be some $r_1$
  such that $\gamma(a_1) \Rightarrow r_1$. Since $\gamma(a_1) : \tau \to \tau'$ we have
  $r_1 = \abstr{x}{b} \in \tau \to \tau'$. Then there is also some $r_2$ such that $\gamma(a_2) \Rightarrow r_2$,
  with $r_2 \in \tau$ since $\gamma(a_2) : \tau$. This requires $b[x \mapsto r_2] \Rightarrow r$,
  and since $b[x \mapsto r_2] : \tau'$ by definition of $\to$, we conclude $r \in \tau'$.
\end{proof}

\end{document}
